\section{Grupa splotu. Prezentacja Wirtingera} % (fold)
\label{sec:group_wirtinger}

Ponieważ dopełnienie dowolnego splotu, zarówno w przestrzeni $\R^3$ jak i $S^3$, jest łukowo spójne, jego grupa podstawowa nie zależy od wyboru punktu bazowego.
Dzięki temu poniższa definicja ma sens:

\begin{definition}
    \label{def:knot_group}
    \index{grupa!węzła}
    Niech $L$ będzie splotem.
    Grupę podstawową jego dopełnienia, $\pi_1(\R^3 \setminus L)$, nazywamy grupą splotu.
\end{definition}

Kiedy mówimy o~grupie węzła, zazwyczaj mamy na myśli obiekt opisany powyżej, a nie grupę kolorującą z~definicji \ref{colgrp_def}.
Nie należy ich mylić, grupa węzła ma bowiem dużo większe znaczenie.

Podamy teraz kilka przykładów węzłów oraz ich grup.

\begin{example}
    Niewęzeł: $\Z$.
\end{example}

\begin{example}
    Trójlistnik: grupa warkoczowa $B_3 \cong \langle x, y \mid x^2 = y^3\rangle$.
\end{example}

\begin{proof}
    Wynika to z równości
    % https://en.wikipedia.org/wiki/Tietze_transformations
    \begin{align}
        \pi_1(S^3 \setminus 3_1) & = \langle x, y, z \mid xz = yx, zy = xz, yx = zy \rangle \\
                                 & = \langle x, y \mid xyx = yxy \rangle \\
                                 & = \langle x, y, a, b \mid xyx = yxy, a = yx, b = xyx \rangle \\
                                 & = \langle x, a, b \mid xa = a^2x^{-1}, b = xa \rangle \\
                                 & = \langle a, b \mid b = a^2(ba^{-1})^{-1} \rangle \\
                                 & = \langle a, b \mid a^3 = b^2 \rangle,
    \end{align}
    prawdziwych na mocy transformacji Tietzego.
\end{proof}

\begin{example}
    Węzeł $(p,q)$-torusowy: $\langle x, y \mid x^p = y^q \rangle$.
\end{example}

\begin{example}
    Węzeł ósemkowy: $\langle x, y \mid yxy^{{-1}}xy=xyx^{{-1}}yx \rangle$.
\end{example}

\begin{proposition}
    \label{prop:knot_group_invariant}
    Grupa węzła jest niezmiennikiem węzłów.
\end{proposition}

\begin{proof}
    Gdy dwa węzły są równoważne, istnieje izotopijny z~identycznością homeomorfizm $\R^3 \to \R^3$, który posyła pierwszy węzeł na drugi.
    Obcięty do dopełnień węzłów indukuje izomorfizm grup podstawowych.
\end{proof}

\begin{proposition}
    Grupa węzła jest niezmiennikiem mocniejszym od genusu, a~w~przypadku węzłów złożonych, także od indeksu mostowego.
\end{proposition}

\begin{proof}[Niedowód]
    Jest to wniosek 3 z~pracy \cite{feustel78}.
\end{proof}

\begin{proposition}
    Niech $K_1, K_2$ będą węzłami pierwszymi.
    Jeżeli ich grupy są izomorficzne, to same węzły są równoważne.
\end{proposition}

\begin{proof}
    Jak piszą Gordon, Luecke w \cite{gordon89}, jest to bezpośredni wniosek z ich twierdzenia 2: nietrywialna chirurgia Dehna na nietrywialnym węźle nigdy nie daje $S^3$.
\end{proof}

Wcześniej Whitten wiedział tylko, że jeśli węzły pierwsze mają izomorficzne grupy, to dopełnienia tych węzłów są homeomorficzne.
Jak sam wspomina w \cite{whitten87}: ,,\emph{The group of a prime knot does not, however, necessarily determine the topological type of the exterior. Dehn hips on certain “essential” solid tori in the exteriors of torus knots and of cable knots produce Haken manifolds that are homotopically equivalent but not homeomorphic to the original exteriors and that, in fact, cannot be imbedded in $S^3$.}''.

Na przykładzie grupy $\langle x,y,z \mid xyx=yxy,xzx=zxz\rangle$, która odpowiada zarówno sumie prostej różno-, jak i~jednoskrętnych trójlistników, widać że założenia o pierwszości nie można pominąć.
Prawdziwe jest ogólniejsze stwierdzenie:

\begin{proposition}
    \label{prop:knot_group_sum}
    Niech $K_1, K_2$ będą zorientowanymi węzłami.
    Wtedy węzłom $K_1 \shrap K_2$, $K_1 \shrap mr K_2$ odpowiadają izomorficzne grupy.
\end{proposition}

\index{prezentacja Wirtingera}
Wiemy więc już trochę o~nowym niezmienniku, ale nie umiemy go jeszcze wyznaczać.
Jak zauważył Wilhelm Wirtinger około roku 1905, a więc jeszcze przed narodzinami teorii węzłów, grupa węzła zawsze posiada pewną specjalną prezentację, nazwaną na jego cześć prezentacją Wirtingera.
Jest to skończona prezentacja, w~której wszystkie relacje są postaci $w g_i w^{-1} = g_j$, gdzie $w$ to pewne słowo na generatorach, $g_1, \ldots, g_k$.
Przedstawimy ją zaraz ze względu na użyteczność w~rachunkach, dowodząc jednocześnie jej istnienia.

\begin{proposition}
    \label{prop:wirtinger}
    Grupa każdego węzła posiada prezentację Wirtingera.
\end{proposition}

\begin{proof}
    Oto zarys konstruktywnego dowodu.
    Przedstawiony algorytm jest bardzo wygodnym sposobem na wyznaczenie grupy węzła.
    Niech $K$ będzie węzłem z~diagramem o~$n$ łukach i~$m$ skrzyżowaniach.
    Wtedy
    \begin{equation}
        \pi_1(K) \cong \langle a_1, \ldots, a_n \mid r_1, \ldots, r_m\rangle,
    \end{equation}
    gdzie $a_i$ to włókna diagramu, zaś $r_x$ to relacje Wirtingera: $a_ia_ja_i^{-1}a_k^{-1}=1$,
\begin{comment}
    \[
    \begin{tikzpicture}[baseline=-0.65ex,scale=0.15]
    \begin{knot}[clip width=15]
        \strand[semithick,-Latex] (-5, -5) to (5, 5);
        \strand[semithick,-Latex] (-5, 5) to (5, -5);
        \node[darkblue] at (5, 5)[below right] {$a_i$};
        \node[darkblue] at (5, -5)[above right] {$a_k$};
        \node[darkblue] at (-5, 5)[below left] {$a_j$};
    \end{knot}
    \end{tikzpicture}
    \quad\quad
    \begin{tikzpicture}[baseline=-0.65ex,scale=0.15]
    \begin{knot}[clip width=15, flip crossing/.list={1}]
        \strand[semithick,-Latex] (-5, -5) to (5, 5);
        \strand[semithick,-Latex] (-5, 5) to (5, -5);
        \node[darkblue] at (5, 5)[below right] {$a_j$};
        \node[darkblue] at (-5, -5)[above left] {$a_k$};
        \node[darkblue] at (-5, 5)[below left] {$a_i$};
    \end{knot}
    \end{tikzpicture}
    \]
\end{comment}
    w~których łuk $a_i$ biegnie górą, zaś $a_j$ leży po jego lewej stronie.
\end{proof}

\begin{figure}[H]
    \begin{minipage}[b]{.48\linewidth}
        \[
            \begin{tikzpicture}[baseline=-0.65ex, scale=0.2]
                \useasboundingbox (-5, -5) rectangle (5,5);
                \begin{knot}[clip width=3.5, end tolerance=1pt, flip crossing/.list={1}]
                    \strand[thick, Latex-] (-5,5) to (5,-5);
                    \strand[thick, -Latex] (-5,-5) to (5,5);
                    % top left
                    \strand[thick, Latex-, darkblue] (-5, 1) to (-1, 5);
                    % bottom left
                    \strand[thick, Latex-, darkblue] (-5, -1) to (-1, -5);
                    % bottom right
                    \strand[thick, -Latex, darkblue] (5, -1) to (1, -5);
                    % top right
                    \strand[thick, -Latex, darkblue] (5, 1) to (1, 5);
                    \node[darkblue] at (-7, -2) {$x_k$};
                    \node[darkblue] at (-7, 2) {$x_{j+1}$};
                    \node[darkblue] at (7, -2) {$x_j$};
                    \node[darkblue] at (7, 2) {$x_k$};
                \end{knot}
            \end{tikzpicture}
        \]
        \subcaption{skrzyżowanie dodatnie: $x_j = x_k x_{j+1} x_k^{-1}$}
    \end{minipage}
    \begin{minipage}[b]{.48\linewidth}
        \[
            \begin{tikzpicture}[baseline=-0.65ex, scale=0.2]
                \useasboundingbox (-5, -5) rectangle (5,5);
                \begin{knot}[clip width=3.5, end tolerance=1pt, flip crossing/.list={1}]
                    \strand[thick, Latex-] (-5,5) to (5,-5);
                    \strand[thick, Latex-] (-5,-5) to (5,5);
                    % top left
                    \strand[thick, Latex-, darkblue] (-5, 1) to (-1, 5);
                    % bottom left
                    \strand[thick, -Latex, darkblue] (-5, -1) to (-1, -5);
                    % bottom right
                    \strand[thick, -Latex, darkblue] (5, -1) to (1, -5);
                    % top right
                    \strand[thick, Latex-, darkblue] (5, 1) to (1, 5);
                    \node[darkblue] at (-7, -2) {$x_k$};
                    \node[darkblue] at (-7, 2) {$x_{j+1}$};
                    \node[darkblue] at (7, -2) {$x_j$};
                    \node[darkblue] at (7, 2) {$x_k$};
                \end{knot}
            \end{tikzpicture}
        \]
        \subcaption{skrzyżowanie ujemne: $x_j = x_k^{-1} x_{j+1} x_k$}
    \end{minipage}
\end{figure}

\begin{corollary}
    \label{prop:knot_group_abelianisation}
    Niech $G$ będzie grupą węzła.
    Wtedy jej abelianizacją jest $G^{\operatorname{ab}} = \Z$.
\end{corollary}

\begin{proof}
    Relacja $a_ia_ja_i^{-1}a_k^{-1}=1$ po przejściu do abelianizacji przyjmuje postać $a_j = a_k$.
    Oznacza to, że etykieta łuku nie zmienia się podczas przejścia pod każdym skrzyżowaniem, zatem wszystkie etykiety są takie same.

    Można też zauważyć, że abelianizacją grupy podstawowej węzła jest pierwsza grupa homologii okręgu, czyli $\Z$.
\end{proof}

Istnieje alternatywna prezentacja grupy węzła, która pochodzi od Dehna, gdzie zamiast etykietować łuki, przypisuje się różne litery czterem częściom płaszczyzny, które są rozcinane przez skrzyżowanie.
Pomijamy tę prezentację dla oszczędności miejsca.
Klasycznie, jak na przykład w~\cite{crowell63}, macierz, a~co za tym idzie, także wielomian Alexandera wprowadza się przy użyciu prezentacji Wirtingera i~pochodnej Foxa.
Oryginalna praca Alexandera była jednak bliższa duchem pomysłom Dehna.

\begin{definition}[pochodna Foxa]
    \index{pochodna Foxa}
    Niech $G$ będzie wolną grupą generowaną przez (niekoniecznie skończony) podzbiór $\{g_i\}_{i \in I}$.
    Odwzorowanie $\partial/\partial g_i \colon G \to \Z G$ spełniające trzy aksjomaty:
    \begin{align}
        \frac{\partial}{\partial g_i} (e) & = 0 \\
        \frac{\partial}{\partial g_i} (g_j) & = \delta_{ij} \\
        \forall u, v \in G : \frac{\partial}{\partial g_i} (uv) & = \frac{\partial}{\partial g_i}(u) + u \frac{\partial}{\partial g_i} (w),
    \end{align}
    gdzie $\delta_{ij}$ oznacza deltę Kroneckera, nazywamy pochodną cząstkową Foxa.
\end{definition}

Ustalmy prezentację grupy węzła z $n$ relacjami (słowami) $w_1, \ldots, w_n$ nad $n$-literowym alfabetem $x_1, \ldots, x_n$.
Zdefiniujmy następnie macierz Jacobiego wymiaru $n \times n$, elementami której są pochodne Foxa słów $w_i$ względem zmiennych $x_j$:
\begin{equation}
    J = \left(\frac{\partial w_i}{\partial x_j}\right).
\end{equation}

Wykreślmy z macierzy $J$ najpiew jedną kolumnę oraz jeden wiersz z tej macierzy, po czym podstawmy za wszystkie litery zmienną $t$ i policzmy wyznacznik.
Otrzymaliśmy znowu wielomian Alexandera.
Fox napisał cykl pięciu artykułów \cite{fox53}, \cite{fox54}, \cite{fox56}, \cite{fox58}, \cite{fox60} poświęcony wolnemu rachunkowi różniczkowemu, powyższa definicja jest tylko małym wycinkiem tego cyklu opublikowanego w Annals of Mathematics.

Dwa następne stwierdzenia są już trudniejsze w~dowodzie,
na przykład uzasadnienie pierwszego może wymagać:
twierdzenia o~sferze, o~pętli oraz hipotezy Knesera.

\begin{proposition}
    \label{prop:knot_group_split}
    Niech $L \subseteq S^3$ będzie splotem.
    Następujące warunki są równoważne:
    \begin{enumerate}
        \item grupa podstawowa splotu $L$ nie jest produktem wolnym,
        \item splot $L$ nie jest rozszczepialny,
        \item splot $L$ jest rozmaitością Hakena o~nieściśliwym brzegu.
    \end{enumerate}
\end{proposition}

\begin{proof}[Niedowód]
    Kawauchi w \cite{kawauchi96}, patrz twierdzenie 6.1.4.
\end{proof}

\begin{proposition}
    \label{prop:knot_group_free}
    Niech $L \subseteq S^3$ będzie splotem.
    Następujące warunki są równoważne:
    \begin{enumerate}
        \item grupa podstawowa splotu $L$ jest wolna, rangi $n$,
        \item splot $L$ jest trywialny, złożony z $n$ ogniw.
    \end{enumerate}
\end{proposition}

\begin{proof}[Niedowód]
    Kawauchi w \cite{kawauchi96}, patrz wniosek 6.1.5.
\end{proof}

Twierdzenie Dehna z~1915 mówi, że jedynym węzłem, którego grupą są liczby całkowite $\mathbb Z$, jest niewęzeł.
Wynik ten został później istotnie uogólniony.
Michael Kervaire pokazał w~1966 roku (w \cite{kervaire65}) jakie warunki musi spełniać grupa $G$, by istniał pewien węzeł, którego grupą jest właśnie $G$.
Patrz też twierdzenie 14.1.1 w \cite{kawauchi96}.

\begin{proposition}
    Niech $G$ będzie grupą węzła $S^n \subseteq S^{n+2}$.
    Wtedy:
    \begin{enumerate}[leftmargin=*]
        \itemsep0em
        \item grupa $G$ jest skończenie prezentowana,
        \item abelianizacja $G/G'$ jest nieskończoną grupą cykliczną,
        \item druga grupa homologii $H_2(G) = 0$ jest trywialna,
        \item istnieje element $x \in G$ zwany południkiem taki, że $G$ jest najmniejszą podgrupą normalną $G$, która zawiera $x$.
    \end{enumerate}
\end{proposition}

Wyżej wymienione warunki konieczne są także wystarczające, jeżeli $n \ge 3$, jednakże problem pełnej charakteryzacji w~czwartym wymiarze jest otwarty.
Warunki 2. i 3. wynikają z~dualności Alexandera, zaś 1. i 4. stanowią przeformułowanie prezentacji Wirtingera.

% Koniec sekcji Grupa węzła. Prezentacja Wirtingera
