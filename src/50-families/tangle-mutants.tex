\subsection{Mutanty i mutacje}
\label{sec:mutant}
Na zakończenie wspomnimy o~mutacjach.

\begin{definition}[mutacja]
    \label{def:mutacja}
    \index{mutacja}
    Półobrót supła względem osi poziomej, pionowej albo też prostopadłej do płaszczyzny, w~jakiej leży diagram, nazywamy mutacją.
    W razie potrzeby zmieniamy orientację supła na przeciwną.
\end{definition}

\begin{tobedone}
rysunek z pracy ,,Tabulating and distinguishing mutants''
\end{tobedone}

\begin{definition}[mutant]
    \label{def:mutant}
    \index{mutant}
    Niech $K$ będzie węzłem.
    Węzeł, który powstaje przez wykonanie ciągu mutacji na węźle $K$, nazywamy mutantem węzła $K$.
\end{definition}

Mutacja węzła o~co najwyżej dziesięciu skrzyżowaniach nie zmienia jego klasy abstrakcji.
Najprostszą, a zarazem najsłynniejszą parą różnych od siebie mutantów stanowią węzeł Conwaya $11n_{34}$ oraz Kinoshity-Terasakiego $11n_{42}$.
\index{węzeł!Conwaya}
\index{węzeł!Kinoshity-Terasakiego}

\begin{tobedone}
rysunek
\end{tobedone}

Conway zauważył podczas klasyfikacji niealternujących węzłów, że tylko one posiadają trywialny wielomian Alexandera.
Mają też taki sam wielomian Jonesa,
\begin{equation}
    \jones(t) = t^{6} -2t^5 +2t^4 -2t^3 +t^2 +2t^{-1} -2t^{-2} +2t^{-3} -t^{-4}.
\end{equation}
Kinoshita, Terasaki zdefiniowali nieskończoną rodzinę węzłów o trywialnym wielomianie Alexandera, której pierwszym wyrazem jest węzeł $11n_{42}$ w~\cite{kinoshita57}.
Dowód tego, że $11n_{34}$ oraz $11n_{42}$ są różne, jako pierwszy podał prawdopodobnie Riley w~1971 roku \cite{riley71}: wykorzystał on homomorfizmy z~grupy węzła w~$PSL(2, 7)$.
Genusy, odpowiednio: $3$ i~$2$, wyznaczył Gabai piętnaście lat później w~\cite{gabai86}, używał foliacji.

Mutanty nie dają się łatwo odróżniać niezmiennikami.

\begin{proposition}
    Mutacja węzłów nie zmienia następujących niezmienników:
    kablowego wielomianu Jonesa, % menasco91
    2-kablowego wielomianu HOMFLY, % przytycki89
    kablowego wielomianu Kauffmana, % lipson87
    \index{wielomian!kablowy}
    sygnatury Tristrama-Levine'a, % cooper99
    \index{sygnatura}
    symplicjalnej objętości Gromowa, % ruberman87: Ruberman [42] showed that mutants have equal volume in all hyperbolic pieces of the JSJ decomposition.
    \index{objętość!Gromowa}
    instanton homologii Floera, % ruberman99
    \index{homologia!Floera}
    niezmienników Wittena % rong94
    \index{niezmiennik!Wittena}
    ani Cassona. % kirk89
    \index{niezmiennik!Cassona}
    % wielomian Alexandera
\end{proposition}

\begin{proof}
    Prace \cite{menasco91}, \cite{przytycki89}, \cite{lipson87}, \cite{cooper99}, \cite{ruberman87}, \cite{ruberman99}, \cite{rong94} oraz \cite{kirk89}.
\end{proof}

Poziom zaawansowania tej książki nie pozwala przedstawić szczegółów dowodu.
Możemy jedynie przytoczyć obserwację... 

\begin{tobedone}
The following observation gives evidence of difficulty in distinguishing between a link and its Conway mutant (cf. [Viro 1977]):
Proposition 3.8.2 If L' is a Conway mutant of a link L, then the double covering spaces over 53 with branch sets Land L' are orientation-preservingly homeomorphic.
We also observe here another fact in [Cooper 1982] which you can understand after you know the facts in Chapter 5. Namely, the Seifert matrices of any knot and its Conway mutant are S-equivalent, so that their Alexander polynomials and their signatures are equal, respectively.
\end{tobedone}

Niedawno Stojmenow podjął się systematycznie szukania mutantów wśród węzłów o~mniej niż 19 skrzyżowaniach (praca \cite{stoimenow10} z~2010 roku).
Początkowo pracował sam, badając pewne subtelne przykłady postanowił uwikłać w swój projekt Toshifumiego Tanakę, a później także Daniela Mateię.
Praca \cite{stoimenow10} jest kontynuacją artykułu, który napisali wspólnymi siłami.

I tak na stronie 531 można przeczytać, że ,,niezmienniki Wasiljewa stopnia co najwyżej 8. nie rozróżniają mutantów węzłów \cite{chmutov94}, ja tego nie widzę.
\index{niezmiennik!Wasiljewa}
Mniej więcej sześć lat później wynik poprawił J. Murakami (nie mylić z H. Murakamim!) do 10. stopnia w~niezindeksowanej pracy \cite{murakami99}.
W międzyczasie Cromwell, Morton znaleźli niezmiennik stopnia 11., który odróżnia węzły Conwaya oraz Kinoshity-Terasakiego; patrz \cite{cromwell96}.
% czy Murakami potwierdził wynik Cromwella, Mortona?

Podsumowanie poszukiwań mutantów stanowi poniższa tabela.

\begin{table}[h]
\centering
\begin{tabular}{llllll} \toprule
skrzyżowania & 11 & 12 & 13  & 14   & 15    \\ \midrule
pary         & 16 & 70 & 703 & 3917 & 24884 \\
trójki       &    & 5  & 38  & 233  & 1000  \\
czwórki      &    &    & 32  & 262  & 2909  \\
szóstki      &    &    & 1   & 17   & 172   \\
ósemki       &    &    &     & 6    & 84    \\
łącznie      & 16 & 75 & 774 & 4435 & 29049 \\
\bottomrule
\hline
\end{tabular}
\caption{Liczba grup mutantów wśród pierwszych węzłów do 15 skrzyżowań}
\end{table}

Mutant węzła złożonego także jest złożony, co więcej istnieje bijekcja między węzłami pierwszymi, z których są zbudowane \cite{ruberman87}.
Wystarczy zatem ograniczyć się do węzłów pierwszych, niestety wciąż nie jest znana ogólna procedura pozwalająca wyliczyć wszystkie mutanty danego węzła.

Zbiór problemów niskowymiarowej topologii opublikowany przez Kirby'iego zawiera następujące pytanie:
\begin{conjecture}
    Niech $K$ będzie prostym\footnote{simple} węzłem bez orientacji.
    Czy istnieją węzły niebędące mutantami $K$, których nie można odróżnić od $K$ wielomianem Jonesa oraz wszystkimi jego satelitami?
\end{conjecture}

Stojmenow pisze, że tak: pierwszą chronologicznie parą jest $14_{41721}$, $14_{42125}$.
Dowód korzysta ze wzoru fuzyjnego pokazanego przez Masbauma oraz Vogela \cite{masbaum94}.
% fusion formula
Wzór ten zastosowany do konkretnej pary węzłów sprawia trudności rachunkowe, jednak jest wystarczającym narzędziem, by rozszerzyć konstrukcję do ogólnego wyniku:

\begin{proposition}
    Istnieje nieskończenie wiele par prostych węzłów hiperbolicznych o tych samych kolorowych wielomianach Jonesa, które nie są swoimi mutantami.
\end{proposition}

\begin{proof}
    \cite{toshifumi09}.
\end{proof}

Millett, Ewing napisali w połowie lat osiemdziesiątych w języku C programy liczące wyjątkowo szybko wielomian HOMFLY oraz wielomian Kauffmana danego węzła.
Programy te pozwalają na obliczenia tam, gdzie nie dają sobie rady inne narzędzia.
Autorzy nie wiedzieli wtedy, że ktoś jeszcze będzie z nich korzystać w przyszłości, dlatego poczynili w kodzie liczne optymalizacje dla stacji roboczej Sun, jaką wtedy dysponowali.
Okazuje się, że dla węzłów o mniej więcej 50 skrzyżowaniach program kończy swoje działanie zrzutem pamięci, wpada w pętlę bez wyjścia, albo zwraca niepoprawny wynik (składniki wielomianu Kauffmana są postaci $a^m z^n$, gdzie $m + n$ jest nieparzyste).

Stoimenow korzystał z tych programów podczas tablicowania mutantów.
W swojej pracy poruszył jeszcze temat symetrycznych mutantów, my nie poświęcimy temu zagadnieniu miejsca.

%%% to już nie jest stoimenow10
Jeśli wyjściowy diagram był alternujący, to mutant też jest alternujący.
Istnieje podejrzenie, że mutacja nie zmienia liczby gordyjskiej.
Gordon i~Luecke w~2006 pokazali to dla klasy węzłów $1$-gordyjskich (\cite{gordon06}), dużo wcześniej wiedzieliśmy tylko, że jedynym mutantem niewęzła jest niewęzeł (Rolfsen w~\cite{rolfsen93}?).

\begin{proposition}
    Niech $m, n$ będą nieujemnymi liczbami całkowitymi.
    Wtedy istnieje węzeł $K$ o genusie plastrowym równym $m$, którego pewien mutant ma genus plastrowy równy $n$.
\end{proposition}

\begin{proof}
    Kim, Livingston w \cite{kim05}.
    Wcześniej Livingston pokazał istnienie mutantów o~różnym genusie plastrowym (\cite{livingston83}).
\end{proof}

