\section{Węzły Lissajous} % (fold)
\label{sec:lissajous}

\begin{definition}
    \index{węzeł!Lissajous}
    Węzłem Lissajous nazywamy węzeł zadany parametrycznie:
    \[
        x = \cos(n_xt + \varphi_x) \quad
        y = \cos(n_yt + \varphi_y) \quad
        z = \cos(n_zt + \varphi_z),
    \]
    gdzie $n_x, n_y, n_z$ to stałe całkowite, zaś $\varphi_x, \varphi_y, \varphi_z$ rzeczywiste.
\end{definition}

Węzeł nie może posiadać samoprzecięć, dlatego żadna z~wielkości $n_i\varphi_j-n_j\varphi_i$, dla różnych indeksów $i, j$ nie może być krotnością $\pi$.
Bez straty ogólności możemy założyć, że $\varphi_z = 0$.
Dodatkowo stałe $n_x, n_y, n_z$ muszą być parami względnie pierwsze.

Przykładem są $5_2$ (dla $n_x = 3$, $n_y = 2$, $n_z = 7$, $\varphi_x = 7/10$, $\varphi_y = 2/10$), $6_1$, $7_4$, $8_{15}$, $10_1$, $10_{35}$, $10_{58}$, suma prawego i~lewego trójlistnika, suma dwóch kopii $5_2$.
Istnieje nieskończenie wiele węzłów Lissajous (\cite{lamm97}).
Każdy skręcony węzeł z~zerowym niezmiennikiem Arfa jest Lissajous.

Węzły Lissajous są bardzo symetryczne.
Jeśli wszystkie stałe $n_x, n_y, n_z$ są nieparzyste, węzeł jest silnie dodatnio achiralny.
Jeśli jedna z~nich, na przykład $n_x$, jest parzysta, to półobrót wokół osi $x$ odwzorowuje węzeł w~siebie.
To nakłada ograniczenia na wielomian Alexandera.
W przypadku nieparzystym (parzystym), $\Delta(t)$ jest kwadratem w~pierścieniu $\Z[t]$ (w $\Z/2[t]$).
Dowód ostatniego stwierdzenia zawierają prace \cite{hartley79} oraz \cite{murasugi71}.

\begin{corollary}
    Trójlistnik, ósemka, węzły torusowe oraz dwumostowe węzły włókniste nie są węzłami Lissajous.
\end{corollary}

Węzły bilardowe to zamknięte trajektorie kuli, która zostaje wystrzelona z jednej ze ścian sześcianu, odbija się pod takim samym kątem, pod jakim pada na ściany.
Jones, Przytycki pokazali w \cite{jones98}, że węzły bilardowe to dokładnie węzły Lissajous i zadali pytanie, czy każdy węzeł można zrealizować jako trajektorię kuli w jakimś wielościanie.

Odpowiedź jest pozytywna.
Koseleff, Pecker w \cite{koseleff14} korzystając z twierdzenia Manturowa (każdy splot jest domknięciem kwasitorycznego warkocza) pokazują, że każdy węzeł ma diagram, który jest wielokątem gwiaździstym.
Użyte zostaje także twierdzenie Kroneckera z 1884 roku: jeśli liczby $\theta_0 = 1, \theta_1, \ldots, \theta_k$ są liniowo niezależne nad ciałem $\Q$, to zbiór punkty $(\lfloor n\theta_i \rfloor_{i=0}^k)_{n=0}^\infty$ leżą gęsto w kostce jednostkowej.

Lamm, Obermeyer dowiedli w 1999, że węzły bilardowe wewnątrz walca są taśmowe albo okresowe, więc w walcu nie można zrealizować każdego węzła.
Lamm postawił hipotezę, że jest to możliwe w eliptycznym walcu.
Pozytywnej odpowiedzi ponownie udzielił niedawno Pecker w \cite{pecker12}.

% Koniec sekcji Węzły Lissajous