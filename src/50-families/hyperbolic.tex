\section{Węzły hiperboliczne} % (fold)
\label{sec:hyperbolic}
\begin{definition}
    Węzeł nazywamy hiperbolicznym, jeżeli na jego dopełnieniu można zadać metrykę o~stałej krzywiźnie $-1$.
    \index{węzeł!hiperboliczny}
\end{definition}

\begin{theorem}[trychotomia Thurstona, 1978]
    \index{twierdzenie!Thurstona}
    Każdy węzeł należy do dokładnie jednej z trzech nieskończonych rodzin: węzłów torusowych, satelitarnych albo hiperbolicznych.
\end{theorem}

% Stwierdzenie to nazywa się czasem trychotomią Thurstona, gdyż dzieli węzły na trzy rodzaje.
Z twierdzenia o~sztywności (\emph{rigidity theorem}) Mostowa i~Prasada (1973), jeśli na dopełnieniu węzła można zadać strukturę hiperboliczną, to w~tylko jeden sposób.
\index{twierdzenie!o sztywności}
Co więcej, Mostow pokazał, że jeśli istnieje izomorfizm grup podstawowych domkniętych, hiperbolicznych 3-rozmaitości, to są one izometryczne.
Wiedzę o~węzłach hiperbolicznych można czerpać z~prac: \cite{weeks05} (poprawiona wersja dostępna w~serwisie ArXiv), "Hyperbolic Knot Theory" J. Purcell.

\begin{proposition}[kryterium Thurstona]
    Niech $L$ będzie splotem z dopełnieniem $X$ i grupą podstawową $\pi = \pi_1(X)$.
    Jeżeli spełnione są następujące warunki:
    \begin{enumerate}
    \item $L$ nie rozszczepia się,
    \item $L$ nie jest niewęzłem,
    \item żadna składowa $L$ nie jest niezakłóconym węzłem satelitarnym (?),
    \item $L$ nie jest węzłem torusowym,
    \end{enumerate}
    to $L$ jest splotem hiperbolicznym.
    Warunki podane wyżej mają swoje odpowiedniki dla przestrzeni $X$ ($X$ nie zawiera właściwej 2-sfery, właściwego dysku, właściwego torusa, właściwego pierścienia) oraz grupy $\pi$ ($\pi$ nie jest wolnym produktem, nie jest cykliczna oraz nie zawiera kopii $\Z^2$).
\end{proposition}

\begin{proposition}
    Żaden węzeł nie ma mniejszej objętości hiperbolicznej od ósemki.
\end{proposition}

Dowód tego faktu podał Cao z~Meyerhoffem w~2001 roku.
Opierali się oni na działaniu komputerowego programu, który wyeliminował inne możliwości.

\begin{proposition}
    Objętość hiperboliczna nie odróżnia hiperbolicznych mutantów.
\end{proposition}

Nie jestem w~stanie podać odnośnika do dowodu w~literaturze -- przypominam, że chodzi o~mutanty z~definicji \ref{def:mutant}.
Stwierdzenie to można znaleźć na przykład w~książce Adamsa (strona 124).

\begin{proposition}
    Grupa symetrii węzła hiperbolicznego jest skończona: cykliczna lub diedralna.
\end{proposition}

\begin{proof}
    Praca \cite{kodama92}.
\end{proof}

Objętość hiperboliczna bardzo dobrze odróżnia od siebie węzły.
Pewien węzeł o~dwunastu skrzyżowaniach i~$5_2$ mają jednak tę samą objętość.

\begin{proposition}
    Każdy węzeł hiperboliczny jest pierwszy.
\end{proposition}

Prawie każdy węzeł pierwszy o~mniej niż 17 skrzyżowaniach jest hiperboliczny, na 32 wyjątki składa się 12 węzłów torusowych oraz 20 satelitów trójlistnika.
Te ostatnie mają co najmniej 11 skrzyżowań.
Baza ciągów liczb całkowitych OEIS zawiera informacje na temat liczności poszczególnych typów węzłów.
Analizując ciągi A051764, A051765 oraz A052408 można dojść do wniosku, że wraz ze wzrostem liczby skrzyżowań, stosunek liczby węzłów hiperbolicznych do wszystkich węzłów dąży do $1$:

\renewcommand*{\arraystretch}{1.4}
\footnotesize
\begin{longtable}{lcccccccccccccc}
\hline
    \textbf{rodzaj} & 3 & 4 & 5 & 6 & 7 & 8  & 9  & 10  & 11  & 12   & 13   & 14    & 15     \\ \hline \endhead
    torusowe        & 1 & 0 & 1 & 0 & 1 & 1  & 1  & 1   & 1   & 0    & 1    & 1     & 2      \\
    satelitarne     & 0 & 0 & 0 & 0 & 0 & 0  & 0  & 0   & 0   & 0    & 2    & 2     & 6      \\
    hiperboliczne   & 0 & 1 & 1 & 3 & 6 & 20 & 48 & 164 & 551 & 2176 & 9985 & 46969 & 253285 \\
    \hline
\end{longtable}
\normalsize

W pracy \cite{malyutin16} A. Malyutin pokazał jednak, że to przypuszczenie jest sprzeczne z~wieloma innymi starymi hipotezami teorii węzłów: \ref{malyutin1} -- \ref{malyutin4}.

\begin{conjecture}
    \label{malyutin1}
    Indeks skrzyżowaniowy jest addytywny względem sumy spójnej.
\end{conjecture}

(To jest powtórzenie hipotezy \ref{cnj:crossing_additive}).
Murasugi dowiódł prawdziwości hipotezy dla węzłów alternujących, w~pracy \cite{murasugi87} jest to wniosek z~dowodu hipotezy Taita.
Krótko po tym Lickorish, Thistlethwaite powtórzyli to dla węzłów adekwatnych w \cite{lickorish88}.
Na początku XX wieku Diao \cite{diao04} oraz Gruber \cite{gruber03} niezależnie udowodnili hipotezę \ref{malyutin1} dla pewnej szerokiej klasy węzłów, obejmującej wszystkie węzły torusowe oraz wiele węzłów alternujących oraz pewne inne obiekty, których nie chcemy opisywać.

\begin{conjecture}
    Satelita ma większy (w słabszej wersji: nie mniejszy) indeks skrzyżowaniowy niż jego towarzysze.
\end{conjecture}

Lackenby pokazał w~\cite{lackenby14}, że jeśli $K$ jest satelitą z towarzyszem $L$, to $\operatorname{cr} K \ge 10^{-13} \operatorname{cr} L$.

\begin{conjecture}
    Węzeł złożony ma większy (w słabszej wersji: nie mniejszy) indeks skrzyżowaniowy niż jego faktory.
\end{conjecture}

Mówimy, że węzeł pierwszy $P$ jest $\lambda$-regularny, jeśli $cr K \ge \lambda \cdot cr P$ za każdym razem, gdy węzeł $P$ jest faktorem węzła $K$.
Zatem hipotezę można wysłowić krótko ,,węzły pierwsze są $1$-regularne''.
Na podstawie prac Murasugiego, Kauffmana i~Thistlethwaite'a z~końca lat 80. wiemy, że zachodzi dla węzłów alternujących.
Diao pokazał w \cite[tw. 3.8]{diao04}, że węzły torusowe także są $1$-regularne, natomiast Lackenby przedstawił w~\cite{lackenby09} rozumowanie, dlaczego wszystkie węzły są $1/152$-regularne.
Pisaliśmy o tym w podsekcji \ref{sub:crossing_number}.

\begin{conjecture}
    \label{malyutin4}
    Węzły pierwsze są $2/3$-regularne.
\end{conjecture}

Rozwiązanie zagadki przyniosła praca samego Malyutina \cite{malyutin19} opublikowana latem 2019 roku, przynajmniej dla splotów.
Pokazał w~niej, że jeśli oznaczymy liczbę splotów pierwszych i~nierozszczepialnych o~$n$ lub mniej skrzyżowaniach przez $P_n$, zaś liczbę hiperbolicznych splotów, także o~$n$ lub mniej skrzyżowaniach, przez $H_n$, prawdziwe będzie oszacowanie
\begin{equation}
    \liminf_{n \to \infty} \frac{H_n}{P_n} < 1 - 10^{-13}.
\end{equation}

% Every non-split, prime, alternating link that is not a~torus link is hyperbolic by a~result of William Menasco.



% https://arxiv.org/abs/math/0309466
% https://arxiv.org/abs/math/0311380
% Koniec sekcji Węzły hiperboliczne
