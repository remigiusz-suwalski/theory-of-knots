\section{Węzły hiperboliczne} % (fold)
\label{sec:hyperbolic}
\begin{definition}
    Węzeł nazywamy hiperbolicznym, jeżeli na jego dopełnieniu można zadać metrykę o~stałej krzywiźnie $-1$.
    \index{węzeł!hiperboliczny}
\end{definition}

\begin{theorem}[trychotomia Thurstona, 1978]
    \index{twierdzenie!Thurstona}
    Każdy węzeł należy do dokładnie jednej z trzech nieskończonych rodzin: węzłów torusowych, satelitarnych albo hiperbolicznych.
\end{theorem}

% Stwierdzenie to nazywa się czasem trychotomią Thurstona, gdyż dzieli węzły na trzy rodzaje.
Z twierdzenia o~sztywności (\emph{rigidity theorem}) Mostowa i~Prasada (1973), jeśli na dopełnieniu węzła można zadać strukturę hiperboliczną, to w~tylko jeden sposób.
\index{twierdzenie!o sztywności}
Co więcej, Mostow pokazał, że jeśli istnieje izomorfizm grup podstawowych domkniętych, hiperbolicznych 3-rozmaitości, to są one izometryczne.
Wiedzę o~węzłach hiperbolicznych można czerpać z~prac: \cite{weeks05} (poprawiona wersja dostępna w~serwisie ArXiv), "Hyperbolic Knot Theory" J. Purcell.

\begin{proposition}[kryterium Thurstona]
    Niech $L$ będzie splotem z dopełnieniem $X$ i grupą podstawową $\pi = \pi_1(X)$.
    Jeżeli spełnione są następujące warunki:
    \begin{enumerate}
    \item $L$ nie rozszczepia się,
    \item $L$ nie jest niewęzłem,
    \item żadna składowa $L$ nie jest niezakłóconym węzłem satelitarnym (?),
    \item $L$ nie jest węzłem torusowym,
    \end{enumerate}
    to $L$ jest splotem hiperbolicznym.
    Warunki podane wyżej mają swoje odpowiedniki dla przestrzeni $X$ ($X$ nie zawiera właściwej 2-sfery, właściwego dysku, właściwego torusa, właściwego pierścienia) oraz grupy $\pi$ ($\pi$ nie jest wolnym produktem, nie jest cykliczna oraz nie zawiera kopii $\Z^2$).
\end{proposition}

\begin{proposition}
    Żaden węzeł nie ma mniejszej objętości hiperbolicznej od ósemki.
\end{proposition}

Dowód tego faktu podał Cao z~Meyerhoffem w~2001 roku.
Opierali się oni na działaniu komputerowego programu, który wyeliminował inne możliwości.

\begin{proposition}
    Objętość hiperboliczna nie odróżnia hiperbolicznych mutantów.
\end{proposition}

Nie jestem w~stanie podać odnośnika do dowodu w~literaturze -- przypominam, że chodzi o~mutanty z~definicji \ref{def:mutant}.
Stwierdzenie to można znaleźć na przykład w~książce Adamsa (strona 124).

\begin{proposition}
    Grupa symetrii węzła hiperbolicznego jest skończona: cykliczna lub diedralna.
\end{proposition}

\begin{proof}
    Praca \cite{kodama92}.
\end{proof}

Objętość hiperboliczna bardzo dobrze odróżnia od siebie węzły.
Pewien węzeł o~dwunastu skrzyżowaniach i~$5_2$ mają jednak tę samą objętość.

\begin{proposition}
    Każdy węzeł hiperboliczny jest pierwszy.
\end{proposition}

Spośród wszystkich węzłów pierwszych o~mniej niż 17 skrzyżowaniach, prawie wszystkie są hiperboliczne: 12 z~nich to węzły torusowe, 20 to satelity trójlistnika (te ostatnie mają ponad 10 skrzyżowań).
Jak pokazuje poniższa tabela (oparta na bazie ciągów OEIS, numery 51764, 51765, 52408), węzły pierwsze nie przypominają licznością liczb pierwszych.

\renewcommand*{\arraystretch}{1.4}
\footnotesize
\begin{longtable}{lcccccccccccccc}
\hline
    \textbf{rodzaj} & 3 & 4 & 5 & 6 & 7 & 8  & 9  & 10  & 11  & 12   & 13   & 14    & 15     \\ \hline \endhead
    torusowe        & 1 & 0 & 1 & 0 & 1 & 1  & 1  & 1   & 1   & 0    & 1    & 1     & 2      \\
    satelitarne     & 0 & 0 & 0 & 0 & 0 & 0  & 0  & 0   & 0   & 0    & 2    & 2     & 6      \\
    hiperboliczne   & 0 & 1 & 1 & 3 & 6 & 20 & 48 & 164 & 551 & 2176 & 9985 & 46969 & 253285 \\
    \hline
\end{longtable}
\normalsize

Powysza tabela nasuwa hipotezę,
że stosunek liczby węzłów hiperbolicznych do liczby wszystkich
(o $n$ skrzyżowaniach) dąży do jedynki przy $n \to \infty$.
W pracy ,,On the question of genericity of hyperbolic knots'' (dostępnej w~serwisie \href{https://arxiv.org/abs/1612.03368v1}{ArxiV}) A. Malyutin pokazał jednak, że to przypuszczenie jest sprzeczne z~wieloma innymi starymi hipotezami teorii węzłów.

\begin{conjecture}
    Indeks skrzyżowaniowy jest addytywny względem sumy spójnej.
\end{conjecture}

(To jest powtórzenie hipotezy \ref{cnj:crossing_additive}).
Murasugi pokazał prawdziwość hipotezy dla węzłów alternujących.
Diao i~Gruber na początku XXI wieku udowodnili ją dla węzłów torusowych.

\begin{conjecture}
    Satelita ma większy
    (w słabszej wersji: nie mniejszy)
    indeks skrzyżowaniowy niż jego towarzysze.
\end{conjecture}

\begin{conjecture}
    Węzeł złożony ma większy
    (w słabszej wersji: nie mniejszy)
    indeks skrzyżowaniowy niż jego faktory.
\end{conjecture}

Mówimy, że węzeł pierwszy $P$ jest $\lambda$-regularny,
jeśli $cr K \ge \lambda \cdot cr P$ za każdym razem,
gdy węzeł $P$ jest faktorem węzła $K$.
Węzły torusowe lub alternujące są $1$-regularne.
Lackenby pokazał w~\cite{lackenby09}, że wszystkie węzły są $1/152$-regularne, pisaliśmy o tym w podsekcji \ref{sub:crossing_number}.

\begin{conjecture}
    Węzły pierwsze są $2/3$-regularne.
\end{conjecture}

% Every non-split, prime, alternating link that is not a~torus link is hyperbolic by a~result of William Menasco.



% https://arxiv.org/abs/math/0309466
% https://arxiv.org/abs/math/0311380
% Koniec sekcji Węzły hiperboliczne