\section{Węzły satelitarne} % (fold)
\label{sec:satellite}
Wyobraźmy sobie węzeł $K'$ leżący wewnątrz trywialnego torusa $V$ tak, by nie zawierał się w~żadnej mieszczącej się w~tym torusie 3-kuli.
Zawiążmy następnie torus $V$: ustalmy włożenie $f \colon V \to S^3$.
Splot $K = f(K')$ orbituje wokół swojego kompana, tj. obrazu rdzenia torusa przez włożenie $f$ i~nie opuszcza jego małego rurowego otoczenia.
Dość nieprzypadkowo splot $K$ nazywamy satelitarnym, zaś obraz rdzenia -- węzłem towarzyszącym.
Konstrukcja satelity jest, w~porównaniu z~sumą spójną, dość zawiła.
Pożądanym byłoby mieć do dyspozycji więcej operacji rozkładających węzły na prostsze, i~badać nierozkładalne obiekty.
Ale ich nie ma.

Torus $f(\partial V)$ nie jest równoległy do brzegu ani ściśliwy.
Odwrotnie, jeśli w~dopełnieniu węzła mamy torus, którego południk lub równoleżnik ogranicza dysk, to węzeł nie biegnie wzdłuż torusa lub ten jest niezawęźlony.
Przypadek, gdy torus jest otoczeniem rurowym węzła, też nie jest ciekawy.
To motywuje następującą definicję.

\begin{definition}
    \index{węzeł!satelitarny}
    Węzeł nazywamy satelitarnym, jeśli zawiera nieściśliwy, nierównoległy do brzegu torus we własnym dopełnieniu.
\end{definition}

Klasa węzłów satelitarnych obejmuje węzły złożone.
W ich przypadku można wskazać pewien szczególny torus nieściśliwy -- połykający pierwszy składnik, a~potem podążający za drugim.
Świetnie przedstawione jest to na stronie 82 książki \cite{cromwell04} Cromwella.
Niektóre węzły przedstawiają się jako satelity w~dokładnie jeden sposób, inne nie są jednoznaczne.
W \cite{jaco79} ulepszono definicję satelitarności do tzw. \emph{splicing}u i~opisano jednoznaczny rozkład Jaco-Shalena-Johannsona, czego prawdziwość przypuszczał wcześniej Waldhausen.
Żaden węzeł torusowy ani trywialny nie jest satelitą, ale są nimi kable oraz duble Whiteheada.

\begin{definition}
    \index{dubel Whiteheada}
    Jeżeli $K' \subseteq V$ jest skręconym jednokrotnie niewęzłem, to węzeł $K$ nazywamy dublem Whiteheada.
\end{definition}

Każdy węzeł posiada nieskończenie wiele dubli Whiteheada: wystarczy rozciąć torus $V$, skręcić jedną końcówkę i~ponownie zszyć, żaden z~nich nie jest odróżniany od niewęzła przez wielomian Alexandera.

\begin{definition}
    \index{węzeł!kablowy}
    Jeżeli $K' \subseteq \partial V$ jest węzłem torusowym,    to $K$ nazywamy węzłem kablowym.
\end{definition}

Satelita, którego indeks zawijający przekracza dwa, ma co najmniej 27 skrzyżowań.
Jeśli jego kompan to ósemka -- co najmniej siedemnaście.
Oprócz tego istnieją cztery satelity o~ponad dwunastu skrzyżowaniach.
Podejrzewa się, że satelita, który wykonuje $m$ pełnych obrotów wokół kompana o~indeksie skrzyżowaniowym $k$, nie posiada diagramu o~mniej niż $km^2$ skrzyżowaniach.

\begin{proposition}
    Każdy kabel wyznacza jednoznacznie węzeł, z~którego powstał.
\end{proposition}

\begin{proof}
    Wniosek 2 z~pracy \cite{feustel78} Feustela, Whittena pokazuje, że na podstawie kabla można wyznaczyć parametry węzła torusowego $K'_{p,q}$ oraz topologię dopełnienia oryginalnego węzła.
    Wiemy jednak z~twierdzenia Gordona-Lueckego, że różne węzły mają różne dopełnienia.
\end{proof}

Schubert pokazał, że zorientowane klasy izotopii węzłów w~$S^3$ tworzą wolny przemienny monoid na przeliczalnie wielu generatorach.
Krótko po tym odkrył, że może podać nowy dowód tego twierdzenia przez uważną analizę nieściśliwych torusów obecnych w~dopełnieniu sumy spójnej.
To doprowadziło go do definicji węzłów satelitarnych i~towarzyszących w~przełomowej pracy \cite{schubert53} oraz zunifikowało teorię 3-rozmaitości oraz węzłów.
Patrz też \cite{motegi97}.

% Koniec sekcji Węzły satelitarne
