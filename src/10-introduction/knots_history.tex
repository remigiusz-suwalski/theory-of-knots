\subsection{Historia tablic węzłów}
Pierwszą osobą, która podjęła się szukania węzłów, był Peter Guthrie Tait, szkocki fizyk.
Razem z Thomsonem (lordem Kelvinem) wierzyli, że węzły są kluczem do zrozumienia widma spektroskopowego różnych pierwiastków: na przykład atom sodu mógł być splotem Hopfa ze względu na dwie linie emisyjne.
Eksperyment Michelsona-Morleya z 1887 roku zabił ich ,,wirową teorię atomu'', ale nie miało to znaczenia dla teorii węzłów jako działu matematyki.

Używana po dziś dzień strategia, którą przyjął Tait, jest stosunkowa prosta: narysować wszysktie możliwe diagramy o~zadanym indeksie skrzyżowaniowym, po czym połączyć ze sobą te, które przedstawiają jeden węzeł.
Na potrzeby pierwszego etapu Tait wymyślił schemat kodowania diagramów.
Wiele lat wcześniej, Gauss wraz ze swoim uczniem Listingiem badał węzły i~opracował (niezależnie!) podobną notację.
My przytoczymy opis dalszego ulepszenia tej metody, zwanego notacją Dowkera-Thistletwaite’a.

Tait wykorzystując swoją notację podał w~1876 pierwszą tablicę piętnastu węzłów o~mniej niż ośmiu skrzyżowaniach.
Nie należy traktować tego jako skromny wynik: nie miał on do dyspozycji żadnych twierdzeń topologicznych do odróżniania węzłów.
Onieśmielony przez liczbę możliwych ciągów dla kolejnych indeksów skrzyżowaniowych, powstrzymał się przed rozszerzaniem swojej tablicy.
To właśnie grupowanie diagramów przedstawiających ten sam węzeł, a~nie samo szukanie wszystkich możliwych diagramów, sprawia trudność.

Aby sobie pomóc, Tait znalazł lokalną modyfikację diagramu, która nie zmienia indeksu skrzyżowaniowego, znaną obecnie jako flype.
\index{ruchy!flype}
Flype to stary szkocki czasownik oznaczający ,,wykręcać na drugą stronę''.

\begin{comment}
\[
\begin{tikzpicture}[baseline=-0.65ex, scale=0.1]
\begin{knot}[clip width=5, end tolerance=1pt, flip crossing/.list={1}]
    \strand[semithick] (-21, -5) [in=180, out=0] to (-7, 5);
    \strand[semithick] (-21, 5) [in=180, out=0] to (-7, -5);
    \draw (-7, -7) rectangle (7, 7);
    \node at (0, 0) {\Huge {$T$}};
    \draw[semithick] (7, -5) to (21, -5);
    \draw[semithick] (7, 5) to (21, 5);
\end{knot}
\end{tikzpicture}
\quad \cong_{\mathrm{flype}} \quad
\begin{tikzpicture}[baseline=-0.65ex, scale=0.1]
\begin{knot}[clip width=5, end tolerance=1pt]
    \strand[semithick] (21, -5) [in=0, out=180] to (7, 5);
    \strand[semithick] (21, 5) [in=0, out=180] to (7, -5);
    \draw (-7, -7) rectangle (7, 7);
    \node at (0, 0) {\rotatebox[origin=c]{-180}{\Huge $T$}};
    \draw[semithick] (-7, -5) to (-21, -5);
    \draw[semithick] (-7, 5) to (-21, 5);
\end{knot}
\end{tikzpicture}
\]
\end{comment}

Inną taktykę szukania węzłów przyjał wielebny Thomas Kirkman: zaczynał od małego zbioru "nieredukowalnych" rzutów, do których systematycznie dokładał skrzyżowania.
Tait przeczytał pracę Kirkmana, po czym w~latach 1884/1885 opracował listę węzłów alternujących o~mniej niż 11 skrzyżowaniach.
% Kirkman miał wtedy 78 lat!
Tuż przed oddaniem jej do druku odkrył inny spis węzłów stworzony przez amerykańskiego naukowca Charlesa Little'a.
Znalazł wtedy jeden duplikat u~siebie, natomiast u Little'a jeden duplikat i~jedno pominięcie.

Zachęcony przez Taita, Little zabrał się za alternujące węzły o~11 skrzyżowaniach i~za trudniejsze zadanie, stablicowanie węzłów niealternujących, czyli takich, które nie posiadają alternującego diagramu.
Jak wynika z~pierwszej pracy Taita, początkowo nie wierzono, że takie w~ogóle istnieją.
Dowód znaleziono wiele lat później, niealternujące są $8_{19}$, $8_{20}$, $8_{21}$, ale nie pierwsze węzły o mniejszej liczbie skrzyżowań.
Patrz twierdzenie \ref{prp:bankwitz}.
Little pracował przez sześć lat (1893 -- 1899) i~znalazł 43 niealternujące węzły o~10 skrzyżowaniach.
Żadnego nie pominął, ale trafił mu się jeden duplikat.

W kolejnych dziesięcioleciach nie nastąpił znaczący postęp, zarówno w~rozszerzaniu tablic jak i~sprawdzaniu tych już istniejących.
Haseman w~1918 roku znalazła achiralne węzły o~12 i~14 skrzyżowaniach \cite{haseman18}.
W 1927 roku Alexander z~Briggsem przy użyciu pierwszej grupy homologii rozgałęzionego nakrycia cyklicznego (!) potrafili odróżnić od siebie dowolne dwa węzły (z~pominięciem 3 par) o~co najwyżej 9 skrzyżowaniach \cite{briggs27}.
Reidemeister poradził sobie z~tymi wyjątkami w~1932 roku, korzystając z~indeksu zaczepienia i~homomorfizmów z~grupy węzła na grupy diedralne \cite{reidemeister32}.
% branch curves in irregular covers associated to homomorphisms of the knot group onto dihedral groups

%%%%% Tait, Little wyprodukowali prawie bezbłędną tablicę węzłów o~co najwyżej 11 skrzyżowaniach przy użyciu grafów.

Dopiero Conway w~latach sześćdziesiątych minionego wieku znalazł pierwsze węzły o~mniej niż 12 skrzyżowaniach oraz wszystkie sploty o~mniej niż 11 skrzyżowaniach w~oparciu o~pomysły Kirkmana.
% An enumeration of knots and links, 1970.
Zajęło mu to jedynie kilka godzin!
Conway znalazł 1 duplikat oraz 11 pominięć w~tablicach Little'a, ale sam popełnił 4 pominięcia.
Przeoczył między innymi słynny duplikat w~niealternującej tablicy Little'a, parę Perko.
% 1974?
Przyczyną było prawdopodobnie to, że dwa diagramy miały różny spin:
Little błędnie twierdził, że spin minimalnego diagramu jest niezmiennikiem, gdyż błędnie założył, że flype oraz 2-przejścia wystarczają do zmiany dowolnego minimalnego diagramu w~inny.

Pominęcia w~tablicy Conwaya znalazł Caudron około 1980 roku \cite{caudron82}.
Rękopis \cite{bonahon80} Bonahona, Siebenmanna klasyfikuje węzły algebraiczne.
Z~nielicznymi niealgebraicznymi węzłami do 11 skrzyżowań poradził sobie Perko w ,,Invariants of 11-crossing knots'' i~\cite{perko82}, co było kresem ery ręcznych obliczeń.

Na początku lat osiemdziesiątych Dowker i~Thistlethwaite \cite{dowker83} stabularyzowali z~pomocą komputera węzły do 13 skrzyżowań.
Przez blisko dekadę nic się nie działo, aż wreszcie grupa studentów wygrała dostęp do superkomputera Cray.
Razem z~Hoste znaleźli alternujące węzły do 14 skrzyżowań, jednocześnie sprawdzając istniejące tabele Thistlethwaite'a.
Około roku 1998 Hoste z~Weeksem (oraz niezależnie Thistlethwaite) znaleźli w~\cite{thistlethwaite98} 1 701 936 pierwszych węzłów do 16 skrzyżowań.
Spośród nich, tylko 32 nie jest węzłami hiperbolicznymi, wszystkie pozostałe poddają się maszynerii geometrii hiperbolicznej.