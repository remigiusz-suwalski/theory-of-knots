Teoria węzłów to gałąź topologii,
która powstała z inspiracji węzłami,
jakie pojawiają się w~codziennym życiu: przy wiązaniu butów albo cumowaniu statków.
Zajmuje się ona badaniem przede wszystkim węzłów,
czyli pewnych włożeń okręgu $S^1$ w trójwymiarową przestrzeń euklidesową $\R^3$ lub sferę $S^3$,
ale także splotów (zaplątanych w sobie węzłów), warkoczy, supłów oraz podobnych obiektów.
Matematyczne węzły różnią się tym od zwykłych, że ich końce są ze sobą połączone.

Oto kilka przykładów.
Węzeł (a) nazywamy niewęzłem (jest to kalka angielskiego \emph{unknot}).
Następne w kolejce widoczne są trójlistnik (b,~\emph{trefoil}), ósemka (c,~\emph{figure-eight}), pięciolistnik (d,~\emph{cinquefoil}) oraz słynna para Perko (e,~f~wg oryginalnej numeracji Rolfsena).

\begin{figure}[H]
	\centering
	\begin{minipage}[b]{.14\linewidth}
		\centering
		$\begin{tikzpicture}[baseline=-0.65ex, scale=0.5] \begin{knot}[clip width=5, end tolerance=1pt] \strand[semithick] (0,0) circle (\linewidth); \end{knot}
\end{tikzpicture}$
		\subcaption{}
	\end{minipage}
	\begin{minipage}[b]{.14\linewidth}
		\centering
		\includegraphics[width=\linewidth]{../data/3_1.png}
		\subcaption{$3_1$}
	\end{minipage}
	\begin{minipage}[b]{.14\linewidth}
		\centering
		\includegraphics[width=\linewidth]{../data/4_1.png}
		\subcaption{$4_1$}
	\end{minipage}
	\begin{minipage}[b]{.14\linewidth}
		\centering
		\includegraphics[width=\linewidth]{../data/5_1.png}
		\subcaption{$5_1$}
	\end{minipage}
	\begin{minipage}[b]{.14\linewidth}
		\centering
		\includegraphics[width=\linewidth]{../data/perko1.png}
		\subcaption{$10_{161}$}
	\end{minipage}
	\begin{minipage}[b]{.14\linewidth}
		\centering
		\includegraphics[width=\linewidth]{../data/perko2.png}
		\subcaption{$10_{162}$}
	\end{minipage}
\end{figure}

Początkowo celem teorii węzłów była klasyfikacja wszystkich węzłów.
Od XIX wieku, kiedy teoria węzłów wyodrębniła się jako osobny dział matematyki,
zdążyliśmy skatalogować ponad sześć miliardów tych obiektów.
Pozornie tak samo wyglądające węzły mogą się od siebie różnić.
Do wykrywania tych subtelnych różnic używa się przede wszystkim niezmienników topologicznych takich jak grupy, wielomiany bądź liczby.
Poznamy je w dalszych rozdziałach.

Matematycy uogólnili pojęcie węzła:
można rozpatrywać je w wyższych wymiarach albo zastąpić okrąg inną przestrzenią topologiczną.
Będziemy starać się unikać tych uogólnień.

\section{Węzły i sploty}
Największą różnicą między węzłami matematycznymi oraz tymi z prawdziwego jest życia jest to, że te pierwsze nie mają luźnych końców.
Można przyjąć nieidealną, naiwną definicję:

\begin{definition}
	Ciągłe oraz różnowartościowe odwzorowanie $S^1 \to \R^3$ to \textbf{węzeł}.
\end{definition}

Zastanówmy się, jakim formalizmem opisać manipulowanie fizycznym sznurkiem.
Nie można użyć izotopii
(dwa węzły są izotopijne, jeśli istnieje ciągła funkcja $F \colon S^1 \times [0, 1] \to \R^3$ taka, że $F(-, 0)$ jest pierwszym, zaś $F(-,1)$ drugim węzłem).
Zauważmy, że każde splątanie ściaga się do punktu.
Dowolny węzeł jest równoważny z trywialnym, zatem istnieje tylko jedna klasa abstrakcji.
\begin{figure}[H]
	\begin{minipage}[b]{.23\linewidth}
		\centering
		\includegraphics[width=\linewidth]{../data/missing.jpg}
		\subcaption{węzeł}
	\end{minipage}
	\begin{minipage}[b]{.23\linewidth}
		\centering
		\includegraphics[width=\linewidth]{../data/missing.jpg}
		\subcaption{prostszy węzeł}
	\end{minipage}
	\begin{minipage}[b]{.23\linewidth}
		\centering
		\includegraphics[width=\linewidth]{../data/missing.jpg}
		\subcaption{niewęzeł}
	\end{minipage}
\end{figure}

Trzeba uwzględnić to, jak węzeł leży w przestrzeni.
Właściwym narzędziem jest więc izotopia otaczająca.
Intuicyjnie: dwa węzły uznajemy za równoważne,
jeśli można przejść od jednego do drugiego przy użyciu deformacji całej przestrzeni $\R^3$.

\begin{definition}[izotopia otaczająca] \label{def_ambient_isotopy}
	Ciągłe odwzorowanie $F \colon \R^3 \times [0,1] \to \R^3$,
	które staje się homeomorfizmem po ustaleniu drugiego argumentu i takie,
	że $F(-, 0)$ jest funkcją tożsamościową,
	zaś $F(-, 1)$ złożona z pierwszym węzłem daje drugi węzeł,
	nazywamy izotopią otaczającą.
\end{definition}

\begin{definition}[węzeł] \label{def:knot}
	Gładkie włożenie $S^1 \to \R^3$ izotopijne otaczająco z zamkniętą łamaną bez samoprzecięć nazywamy węzłem poskromionym.
\end{definition}

Wykluczamy w ten sposób patologiczne z kombinatorycznego punktu widzenia węzły dzikie.
Przez prawie całą książkę interesować nas będą jedynie węzły poskromione,
dlatego jeśli nie zaznaczono inaczej, przez węzeł rozumiemy węzeł poskromiony.
\todo{Wstawić przykład węzła dzikiego.}
Dość często będziemy utożsamiać węzeł z jego obrazem.

Jednocześnie homeomorfizmy $F(-,t)$ zastępujemy przez dyfeomorfizmy zachowujące orientację.
Chwila namysłu wystarcza do przekonania się, że definicja \ref{def_ambient_isotopy} obejmuje zwykłą izotopię,
a przy tym nie pozwala na rozwiązanie nietrywialnych węzłów przez ściągnięcie zaplątania do punktu.
Istnieje jeszcze jedna, konkurencyjna definicja węzłów równoważnych:

\begin{definition}
	Dwa węzły są równoważne, gdy jeden z nich jest obrazem drugiego przez zachowujący orientację homeomorfizm $\R^3 \to \R^3$.
\end{definition}

Stwierdzenie to przestaje być prawdziwe po zastąpieniu przestrzeni $\R^m$ przez $S^m$.

\begin{proof}
	Podany niżej dowód pochodzi z książki ,,Topology from the differentiable viewpoint'' Johna Millnora.
	Musimy pokazać, że dyfeomorfizm $f \colon \R^m \to \R^m$ jest gładko izotopijny z identycznością.
	Translacje są izotopiami, więc bez straty ogólności zakładamy, że $f(0) = 0$.
	Pochodna $f$ w zerze jest dana wzorem $\mathrm{d}f_0(x) = \lim_{t \to 0} f(tx) /t$,
	naturalną definicją	izotopii $F \colon \R^m \times [0, 1] \to \R^m$ jest więc
	\[
		F(x, t) = \begin{cases}
			f(tx) / t & 0 < t \le 1 \\
			\mathrm{d}f_0(x) & t = 0
		\end{cases} .
	\]

	Funkcja $F$ jest gładka,
	gdyż na mocy lematu Hadamarda funkcja $f$ zapisuje się jako suma $x_1 g_1(x) + \ldots + x_mg_m(x)$,
	gdzie funkcje $g_i$ są gładkie, co jakoś kończy dowód.
\end{proof}

\begin{definition}[splot] \label{def_link}
	Sumę rozłączną skończenie wielu węzłów nazywamy splotem.
\end{definition}

\begin{example}
	Whitehead w 1934 odkrył kontrprzykład do nieudanego dowodu hipotezy Poincarego.
	Był nim splot o dwóch składowych przedstawiony na poniższym rysunku.
	Splot Hopfa to najprostszy splot nietrywialny, którym w 1931 r. zajmował się Heinz Hopf,
	topolog niemiecki, w ramach badań nad tzw. rozwłóknieniem (Hopf fibration).

	\begin{figure}[H]
		\begin{minipage}[b]{.32\linewidth}
			\centering
			\includegraphics[width=\linewidth]{../data/missing.jpg}
			\subcaption{splot Whiteheada}
		\end{minipage}
		\begin{minipage}[b]{.32\linewidth}
			\centering
			\includegraphics[width=\linewidth]{../data/missing.jpg}
			\subcaption{splot Heada}
		\end{minipage}
		\begin{minipage}[b]{.32\linewidth}
			\centering
			\includegraphics[width=\linewidth]{../data/missing.jpg}
			\subcaption{splot jakiś inny}
		\end{minipage}
	\end{figure}
\end{example}

\begin{definition}[rozszczepialność]
	Splot, który jest niespójną sumą niepustych splotów, nazywamy rozszczepialnym.
\end{definition}

Jeśli dwa węzły są równoważne, to ich dopełnienia są oczywiście homeomorficzne.
Pytanie o prawdziwość implikacji odwrotnej jako pierwszy zadał najprawdopodobniej Tietze\footnote{Praca \emph{Über die topologischen Invarianten mehrdimensionaler Mannigfaltigkeiten}.} w 1908 roku.
Odpowiedź jest pozytywna: każdy węzeł jest wyznaczony jednoznacznie przez swoje dopełnienie.

\begin{theorem}[Gordon, Luecke, 1989] \label{thm_gordon_luecke}
	Poskromione węzły o homeomorficznych (z zachowaniem orientacji) dopełnieniach są wzajemnie izotopijne.
\end{theorem}

Wcześniej wiedziano tylko, że istnieją co najwyżej dwa węzły o zadanym dopełnieniu.

\begin{proof}[Niedowód]
	Wynika to z ogólniejszego stwierdzenia:
	nietrywialna chirurgia Dehna na węźle w~3-sferze nigdy nie daje 3-sfery.
	Pełny dowód zawiera praca \cite{gordon89}.
\end{proof}

Istnieje nieskończenie wiele (jakich?) kontrprzykładów do odpowiednika twierdzenia Gordona-Lueckego dla splotów,
których nie da się, patrząc na samo dopełnienie, odróżnić od splotu Whiteheada.

\section{Diagramy}
Chociaż w świetle definicji \ref{def:knot} węzły są pewnymi regularnymi podzbiorami przestrzeni $\R^3$,
z kombinatorycznego punktu widzenia wygodniej jest rysować je na  płaszczyźnie.

\begin{definition} [diagram] \label{def_diagrams}
	Cień to rzut węzła $K \subseteq \R^3$ na płaszczyznę.
	Diagram to cień bez katastrof
	(potrójnych przecięć, stycznych i dziobów)
	razem z informacją o tym, jak przebiegają skrzyżowania.
\end{definition}

\begin{definition} [włókno]
	Fragment diagramu, który biegnie między dwoma kolejnymi tunelami (podskrzyżowaniami) nazywamy włóknem.
\end{definition}

\begin{proposition}
	Każdy splot posiada diagram -- zbiór diagramów jest otwarty i gęsty w zbiorze wszystkich.
\end{proposition}

\begin{proof}
	Rzut splotu na równoległe płaszyczyzny jest taki sam,
	a te można sparametryzować prostymi przechodzącymi przez początek układu współrzędnych,
	które tworzą przestrzeń rzutową $\R \mathbb P^2$.
	Niech $S$ będzie zbiorem prostych, które dają złe rzuty.
	Wystarczy pokazać jego nigdziegęstość.
	Okazuje się, że $S$ jest też jednowymiarowy.
	(Dowód za \cite{crowell63}).
\end{proof}

\begin{definition}
	Diagram jest alternujący,
	gdy podczas poruszania się wzdłuż splotu
	mijamy jego skrzyżowania na zmianę z góry oraz z dołu.
	Splot jest alternujący, gdy posiada taki diagram.
\end{definition}

%Niestety pomimo upływu czasus, nikt nie napisał komputerowego programu realizującego ten algorytm (stan na 1994).
%Może podejmie się tego Czytelnik?
%Inne algorytmy istnieją, jednak wszystkie działają w wykładniczym czasie.

W 1961 roku W. Haken \cite{haken61} podał niezawodny przepis na wykrycie diagramu niewęzła,
częściowo rozwiązując jeden z ważniejszych problemów teorii węzłów.
Przez wiele lat nikt nie podjął się implementacji tego algorytmu,
udało się to na przykład B. Burtonowi, R. Budneyowi oraz W. Petterssonowi w komputerowym programie Regina
\footnote{Dostępny pod adresem \url{https://regina-normal.github.io/}.} na przełomie tysiącleci.
Burton, Rubinstein i Tillman pokazali w pracy \cite{burton12}, jak sprawdzać,
czy powierzchnia normalna na striangulowanej 3-rozmaitości jest (nie)ściśliwa w czasie wykładniczym.
To okazało się być wystarczającym do udzielenia negatywnej odpowiedzi na pytanie Thurstona:
,,czy przestrzeń Seiferta-Webera jest rozmaitością Hakena?'',
a zatem wykraczającego poza poziom tej pracy.
Patrz także {\url{http://geometrygames.org/SnapPea/index.html}.

Przykładami trudnych w rozpoznaniu niewęzłów są: niewęzeł Goritza, Freedmana.

\begin{figure}[H]
	\begin{minipage}[b]{.32\linewidth}
		\centering
		\includegraphics[width=\linewidth]{../data/missing.jpg}
		\subcaption{normalny}
	\end{minipage}
	\begin{minipage}[b]{.32\linewidth}
		\centering
		\includegraphics[width=\linewidth]{../data/missing.jpg}
		\subcaption{Goritza}
	\end{minipage}
	\begin{minipage}[b]{.32\linewidth}
		\centering
		\includegraphics[width=\linewidth]{../data/missing.jpg}
		\subcaption{Freedmana}
	\end{minipage}
\end{figure}

Więcej trudnych niewęzłów zawiera świeża praca
,,Algorithmic simplification of knot diagrams: new moves and experiments'
\footnote{\url{https://arxiv.org/pdf/1508.03226.pdf}}
C. Petronio, A. Zanellatiego.

Innym narzędziem wykrywającym niewęzły jest homologia Chowanowa (opisana później),
jak pokazał Kronheimer z Mrówką \cite{kronheimer11}.
Bar-Natan, topolog izraelski, napisał program liczący te homologie szybko \cite{barnatan07},
zapewne w czasie $O(\exp(c \sqrt n))$, dla diagramu o $n$ skrzyżowaniach.
Nie możemy liczyć na istotne przyspieszenie:
znalezienie przybliżenia wielomianu Jonesa jest problemem \#P-trudnym (\cite{kuperberg15}, \cite{vertigan05}),
a przy znanych homologiach -- wręcz trywialnym.
Patrz też \ref{jones_sharp_p_hard}.

\subsection{Metody kodowania węzłów i splotów}
Tait, Little wyprodukowali prawie bezbłędną tablicę węzłów o co najwyżej 11 skrzyżowaniach przy użyciu grafów.

\subsubsection{Notacja Conwaya}
Wprowadzona przez Conwaya w pracy ,,An Enumeration of Knots and Links, and Some of Their Algebraic Properties''.

\subsubsection{Notacja Dowkera--Thistlethwaite'a}
Poprawia notację Taita.

\subsubsection{Notacja Alexandera-Briggsa}
Najbardziej tradycyjna, wprowadzona w 1927 roku, rozszerzona później przez Rolfsena.