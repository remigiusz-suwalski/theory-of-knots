Teoria węzłów to gałąź topologii,
która powstała z~inspiracji węzłami,
jakie pojawiają się w~codziennym życiu: przy wiązaniu butów albo cumowaniu statków.
Zajmuje się ona badaniem przede wszystkim węzłów,
czyli pewnych włożeń okręgu $S^1$ w~trójwymiarową przestrzeń euklidesową $\R^3$ lub sferę $S^3$,
ale także splotów (zaplątanych w~sobie węzłów), warkoczy, supłów oraz podobnych obiektów.
Matematyczne węzły różnią się tym od zwykłych, że ich końce są ze sobą połączone.

Oto kilka przykładów.
Węzeł (a) nazywamy niewęzłem, jest to kalka angielskiego \emph{unknot}.
Następne w~kolejce widoczne są trójlistnik (b,~\emph{trefoil}), ósemka (c,~\emph{figure-eight}), pięciolistnik (d,~\emph{cinquefoil}) oraz słynna para Perko (e,~f~wg oryginalnej numeracji Rolfsena).
Pod diagramami umieściliśmy notację Alexandera-Briggsa, jeszcze do niej wrócimy.

\begin{comment}
\begin{figure}[H]
    \centering
    \begin{minipage}[b]{.14\linewidth}
        \centering
        $\begin{tikzpicture}[baseline=-0.65ex, scale=0.5] \begin{knot}[clip width=5, end tolerance=1pt] \strand[semithick] (0,0) circle (\linewidth); \end{knot}
\end{tikzpicture}$
        \subcaption{}
    \end{minipage}
    \begin{minipage}[b]{.14\linewidth}
        \centering
        \includegraphics[width=\linewidth]{../data/3_1.png}
        \subcaption{$3_1$}
    \end{minipage}
    \begin{minipage}[b]{.14\linewidth}
        \centering
        \includegraphics[width=\linewidth]{../data/4_1.png}
        \subcaption{$4_1$}
    \end{minipage}
    \begin{minipage}[b]{.14\linewidth}
        \centering
        \includegraphics[width=\linewidth]{../data/5_1.png}
        \subcaption{$5_1$}
    \end{minipage}
    \begin{minipage}[b]{.14\linewidth}
        \centering
        \includegraphics[width=\linewidth]{../data/perko1.png}
        \subcaption{$10_{161}$}
    \end{minipage}
    \begin{minipage}[b]{.14\linewidth}
        \centering
        \includegraphics[width=\linewidth]{../data/perko2.png}
        \subcaption{$10_{162}$}
    \end{minipage}
\end{figure}
\end{comment}

Początkowo celem teorii węzłów była klasyfikacja wszystkich węzłów.
Od XIX wieku, kiedy teoria węzłów wyodrębniła się jako osobny dział matematyki,
zdążyliśmy skatalogować ponad sześć miliardów tych obiektów.
Pozornie tak samo wyglądające węzły mogą się od siebie różnić.
Do wykrywania tych subtelnych różnic używa się przede wszystkim niezmienników topologicznych takich jak grupy, wielomiany bądź liczby.
Poznamy je w~dalszych rozdziałach.

Matematycy uogólnili pojęcie węzła:
można rozpatrywać je w~wyższych wymiarach albo zastąpić okrąg inną przestrzenią topologiczną.
Będziemy starać się unikać tych uogólnień.

\section{Węzły i~sploty}
Największą różnicą między węzłami matematycznymi oraz tymi z~prawdziwego jest życia jest to, że te pierwsze nie mają luźnych końców.
Można przyjąć nieidealną, naiwną definicję:

\begin{definition}[węzeł]
    Ciągłe oraz różnowartościowe odwzorowanie $S^1 \to \R^3$ nazywamy węzłem.
\end{definition}

Niestety, dopuszcza ona patologiczne z~kombinatorycznego punktu widzenia węzły dzikie, jak ten z~rysunku \ref{wild_knot}:

\begin{comment}
\begin{figure}
    \centering
    \label{wild_knot}
    \includegraphics[width=0.5\linewidth]{wild_knot.png}
    \caption{Węzeł dziki}
\end{figure}
\end{comment}

Zastanówmy się, jakim formalizmem opisać manipulowanie fizycznym sznurkiem, by wykluczyć węzły dzikie z~naszych rozważań.
Nie można użyć izotopii (dwa węzły są izotopijne, jeśli istnieje ciągła funkcja $F \colon S^1 \times [0, 1] \to \R^3$ taka, że $F(-, 0)$ jest pierwszym, zaś $F(-,1)$ drugim węzłem), gdyż każdy węzeł jest izotopijny z punktem:

% TODO: Tu brakuje obrazka.

W podobny sposób moglibyśmy przekształcić dowolny węzeł w~niewęzeł.
Teoria, w~której wszystkie obiekty są takie same, nie jest zbyt ciekawa.
Zwykła izotopia nie oddaje dobrze tego, czym jest równoważność węzłów wykonanych z~prawdziwego sznurka.
Trzeba od niej wymagać dodatkowo, by była gładka albo lokalnie płaska.
Z twierdzenia o rozszerzaniu izotopii wynika, że można ją wtedy podnieść do izotopii otaczającej.
Ta ostatnia uwzględnia, jak węzeł leży w~przestrzeni i okazuje się być właściwym pojęciem równości dla teorii węzłów:

\begin{definition}[izotopia otaczająca]

    Niech $N, M$ będą rozmaitościami, zaś $K_1, K_2 \colon N \to M$ włożeniami.
    Ciągłe odwzorowanie $F \colon M \times [0,1] \to M$ spełniające następujące warunki:
    \begin{enumerate}
        \item funkcja $F(-, 0)$ jest odwzorowaniem tożsamościowym,
        \item każda z funkcji $F(-, t)$ jest homeomorfizmem,
        \item złożenie $F(-, 1)$ z pierwszym włożeniem $K_1$ daje drugie włożenie $K_2$
    \end{enumerate}
    nazywamy izotopią otaczającą przenoszącą $K_1$ na $K_2$.
\end{definition}

W topologii rozważa się włożenia dowolnych rozmaitości, nam wystarczy jeden szczególny przypadek $N = S^1$ oraz $M = \R^3$.
Intuicyjnie, funkcja $F$ zniekształca przestrzeń $\R^3$ tak, że w~chwili początkowej $t = 0$ widzimy pierwszy, zaś w~chwili końcowej $t = 1$ drugi węzeł.
Izotopia otaczająca nie pozwala na ściąganie zaplątanych fragmentów do punktu.

% TODO: Homeomorfizmy $F_t$ można zastąpić przez dyfeomorfizmy zachowujące orientację.

\begin{definition}[węzeł]
    \label{def:knot}
    \index{węzeł}
    Gładkie włożenie $S^1 \to \R^3$ otaczająco izotopijne z~zamkniętą łamaną bez samoprzecięć nazywamy węzłem poskromionym.
    Dwa węzły są równoważne, jeśli istnieje pomiędzy nimi izotopia otaczająca.
\end{definition}

Przez prawie całą książkę interesować nas będą jedynie węzły poskromione, czyli takie które nie są dzikie,
dlatego jeśli nie zaznaczono inaczej, od teraz pisząc węzeł mamy na myśli węzeł poskromiony.
Istnieje jeszcze jedna, konkurencyjna definicja węzłów równoważnych:

\begin{proposition}
    \label{equivalent_knots_2}
    Dwa węzły są równoważne, gdy jeden z~nich jest obrazem drugiego przez zachowujący orientację homeomorfizm $\R^3 \to \R^3$.
\end{proposition}

Stwierdzenie to przestaje być prawdziwe po zastąpieniu przestrzeni $\R^m$ przez $S^m$.

\begin{proof}
    Podany niżej dowód pochodzi z~książki ,,Topology from the differentiable viewpoint'' Johna Milnora.
    Musimy pokazać, że dyfeomorfizm $f \colon \R^m \to \R^m$ jest gładko izotopijny z~identycznością.
    Translacje są izotopiami, więc bez straty ogólności zakładamy, że $f(0) = 0$.
    Pochodna $f$ w~zerze jest dana wzorem $\mathrm{d}f_0(x) = \lim_{t \to 0} f(tx) /t$,
    naturalną definicję izotopii $F \colon \R^m \times [0, 1] \to \R^m$ stanowi więc
    \[
        F(x, t) = \begin{cases}
            \mathrm{d}f_0(x) & t = 0 \\
            f(tx) / t & 0 < t \le 1
        \end{cases} .
    \]

    Funkcja $F$ jest gładka, gdyż na mocy lematu Hadamarda funkcja $f$ zapisuje się jako suma $x_1 g_1(x) + \ldots + x_mg_m(x)$, gdzie funkcje $g_i$ są gładkie, co jakoś kończy dowód.
\end{proof}

Formalnie węzły to pewne odwzorowania, więc prawidłowym sposobem na zapisanie, że są izotopijne (czyli dla nas: równe), jest $K_1 \simeq K_2$.
Ponieważ nie prowadzi to do problemów, będziemy jednak stosować zapis $K_1 = K_2$.
Jednocześnie często węzeł (jako odwzorowanie) nie będzie odróżniany od obrazu tego odwzorowania.

\begin{definition}[splot, ogniwo]

    \index{splot}
    Sumę parami rozłącznych węzłów $K_1, K_2, \ldots, K_n$ nazywamy splotem.
    Składniki sumy nazywamy ogniwami.
\end{definition}

Przez analogię do węzłów mówimy, że dwa sploty są takie same, jeśli jeden jest obrazem drugiego przez zachowujący orientację homeomorfizm $\R^3 \to \R^3$.
W~takiej sytuacji obydwa sploty mają tyle samo ogniw.

\begin{example}
    \index{splot!Hopfa}
    Splot Hopfa to najprostszy splot nietrywialny, którym w~1931 r. zajmował się Heinz Hopf, topolog niemiecki, w~ramach badań nad tzw. rozwłóknieniem (Hopf fibration).
\end{example}

\begin{example}
    \index{splot!Whiteheada}
    Whitehead w~1934 odkrył kontrprzykład do nieudanego dowodu hipotezy Poincarego.
    Był nim splot o~dwóch składowych przedstawiony na poniższym rysunku.
\end{example}

\begin{comment}
    \begin{figure}[H]
        \begin{minipage}[b]{.48\linewidth}
            \centering
            \includegraphics[width=0.5\linewidth]{../data/mixed/L2a1.png}
            \subcaption{splot Hopfa}
        \end{minipage}
        \begin{minipage}[b]{.48\linewidth}
            \centering
            \includegraphics[width=0.5\linewidth]{../data/mixed/L5a1.png}
            \subcaption{splot Whiteheada}
        \end{minipage}
    \end{figure}
\end{comment}

Jeśli dwa węzły są równoważne, to ich dopełnienia są oczywiście homeomorficzne.
Pytanie o~prawdziwość implikacji odwrotnej jako pierwszy zadał najprawdopodobniej w~1908 roku Tietze (,,Über die topologischen Invarianten mehrdimensionaler Mannigfaltigkeiten'').
W roku 1987 pokazano, że istnieją co najwyżej dwa węzły o~zadanym dopełnieniu (\cite{culler87}).
Dwa lata później poznaliśmy pozytywną odpowiedź na pytanie Tietzego: każdy węzeł jest wyznaczony jednoznacznie przez swoje dopełnienie.

\begin{theorem}[Gordon, Luecke, 1989]
    \label{thm_gordon_luecke}
    \index{twierdzenie!Gordona-Lueckego}
    Poskromione węzły o~homeomorficznych (z zachowaniem orientacji) dopełnieniach są wzajemnie izotopijne.
\end{theorem}

\begin{proof}[Niedowód]
    Wynika to z~ogólniejszego stwierdzenia:
    nietrywialna chirurgia Dehna na węźle w~3-sferze nigdy nie daje 3-sfery.
    Pełny dowód zawiera praca \cite{gordon89}.
\end{proof}

Twierdzenie to zamienia problem lokalny (czy dwa węzły w kuli $S^3$ są równoważne?) w~problem globalny (czy dwie przestrzenie topologiczne są homeomorficzne?).
Whitehead w~pracy \cite{whitehead37} z~1937 roku podał nieskończenie wiele splotów, których dopełnienia wyglądają jak dopełnienia splotu Whiteheada.
Odpowiednik twierdzenia \ref{thm_gordon_luecke} dla splotów jest więc fałszywy.

% TODO: przesunąć to gdzieś dalej, za definicję skrzyżowania
Poniższa definicja nie jest nam jeszcze potrzebna, ale wygodnie przytoczyć ją już teraz.

\begin{definition}[rozszczepialność]
    \index{splot!rozszczepialny}
    Jeżeli splot $L$ można zanurzyć w przestrzeni $\R^3$ tak, że niektóre jego ogniwa będą leżeć nad pewną rozłączną ze splotem płaszczyzną, zaś pozostałe pod nią, to powiemy, że splot $L$ jest rozszczepialny.
\end{definition}

Liczbę nierozszczepialnych splotów, pierwszych lub złożonych, zebrano w tabeli.
Źródło: baza danych OEIS, ciąg \href{https://oeis.org/A086825}{A086825}.

\renewcommand*{\arraystretch}{1.4}
\footnotesize
\begin{longtable}{lccccccccc}
    \hline
    \textbf{skrzyżowania}  &  0  &  1  &  2  &  3  &  4  &  5  &  6   &  7   &  8   \\  \hline  \endhead
    sploty                 &  1  &  0  &  1  &  1  &  3  &  4  &  15  &  24  &  82  \\
    \hline
\end{longtable}
\normalsize


\section{Diagramy. Ruchy Reidemeistera}
Chociaż w~świetle definicji \ref{def:knot} węzły są pewnymi regularnymi podzbiorami przestrzeni $\R^3$,
z kombinatorycznego punktu widzenia wygodniej jest rysować je na płaszczyźnie.

\begin{definition}[orientacja]
    \index{splot!zorientowany}
    Węzeł, w~którym wybrano kierunek, w~którym należy się po nim poruszać, nazywamy zorientowanym.
    Splot nazywamy zorientowanym, jeśli wszystkie jego ogniwa traktowane jako węzły są zorientowane.
\end{definition}

\begin{definition}
    Rzut węzła $K \subseteq \R^3$ na płaszczyznę nazywamy cieniem.
\end{definition}

\begin{definition}[skrzyżowanie]
    Podwójny punkt w cieniu nazywamy skrzyżowaniem.
\end{definition}

\begin{definition}[diagram]
    \index{diagram}
    Cień razem z~informacją o~tym, jak przebiegają skrzyżowania i pozbawiony katastrof: potrójnych przecięć, stycznych czy dziobów nazywamy diagramem.
    % TODO: Narysować katastrofy
\end{definition}

Orientację na diagramie zaznaczamy małą strzałką wskazującą kierunek poruszania się.

\begin{definition}[włókno]
    \index{włókno}
    Fragment diagramu, który biegnie między dwoma kolejnymi tunelami, czyli podskrzyżowaniami, nazywamy włóknem.
\end{definition}

\begin{definition}[nić]
    \index{nić}
    Fragment diagramu, który biegnie między dwoma kolejnymi skrzyżowaniami, nazywamy nicią.
\end{definition}

Nici powstają z włókien przez rozcięcie ich przy każdym nadskrzyżowaniu.

\begin{proposition}
    Niech $L$ będzie splotem.
    Jego diagramy tworzą otwarty i~gęsty podzbiór wszystkich rzutów.
\end{proposition}

Kawauchi \cite[s. 7]{kawauchi96} wspomina w tym miejscu podręcznik Crowella, Foxa \cite[s. 7]{crowell63}.

\begin{proof}
    Rzut splotu na równoległe płaszczyzny jest taki sam, a te można sparametryzować prostymi przechodzącymi przez początek układu współrzędnych, które tworzą przestrzeń rzutową $\R \mathbb P^2$.
    Niech $S$ będzie zbiorem prostych, które dają złe rzuty.
    Wystarczy pokazać jego nigdziegęstość.
    Okazuje się, że $S$ jest też jednowymiarowy.
\end{proof}

\begin{corollary}
    Każdy splot posiada diagram.
\end{corollary}

Wynika stąd, że każdy węzeł ma wiele diagramów.
Mając dane dwa różne diagramy chcielibyśmy wiedzieć, czy reprezentują ten sam węzeł.
Na szczęście Reidemeister w latach 20. XX wieku podał proste kryterium rozstrzygające ten problem.
Najpierw zdefiniujmy trzy lokalne operacje na diagramach.

\begin{definition}[ruchy Reidemeistera]
    \index{ruchy!Reidemeistera}
    Trzy gatunki lokalnych deformacji diagramu splotu:
\begin{comment}
    \[
        \underbrace{\begin{tikzpicture}[baseline=-0.65ex,scale=0.1]
        \begin{knot}[clip width=5]
            \strand[thick] (-5, 10) to [in=left, out=down] (2, -5);
            \strand[thick] (5, 0) to [in=right, out=down] (2, -5);
            \strand[thick] (5, 0) to [in=right, out=up] (2, 5);
            \strand[thick] (-5, -10) to [in=left, out=up] (2, 5);
        \end{knot}
        \end{tikzpicture}
        \, \cong \,
        \begin{tikzpicture}[baseline=-0.65ex,scale=0.1]
        \begin{knot}[clip width=5]
            \strand[thick] (0,10) to (0,-10);
        \end{knot}
        \end{tikzpicture}}_{R_1}
        %%%
        \quad \quad \quad
        \underbrace{\begin{tikzpicture}[baseline=-0.65ex,scale=0.1]
        \begin{knot}[clip width=5]
            \strand[thick] (-5, 10) to [in=up, out=down] (5, 0);
            \strand[thick] (-5, -10) to [in=down, out=up] (5, 0);
            \strand[thick] (5, 10) to [in=up, out=down] (-5, 0);
            \strand[thick] (5, -10) to [in=down, out=up] (-5, 0);
        \end{knot}
        \end{tikzpicture}
        \, \cong \,
        \begin{tikzpicture}[baseline=-0.65ex,scale=0.1]
        \begin{knot}[clip width=5]
            \strand[thick] (-5, 10) to [in=up, out=down] (-2, 0);
            \strand[thick] (-5, -10) to [in=down, out=up] (-2, 0);
            \strand[thick] (5, 10) to [in=up, out=down] (2, 0);
            \strand[thick] (5, -10) to [in=down, out=up] (2, 0);
        \end{knot}
        \end{tikzpicture}}_{R_2}
        %%%
        \quad \quad \quad
        \underbrace{\begin{tikzpicture}[baseline=-0.65ex,scale=0.1]
        \begin{knot}[clip width=5, flip crossing/.list={1,2,3}]
            \strand[thick] (-10, -10) -- (10, 10);
            \strand[thick] (-10, 10) -- (10, -10);
            \strand[thick] (-10, 0) to [in=left, out=right] (0, 10);
            \strand[thick] (10, 0) to [in=right, out=left] (0, 10);
        \end{knot}
        \end{tikzpicture}
        \, \cong \,
        \begin{tikzpicture}[baseline=-0.65ex,scale=0.1]
        \begin{knot}[clip width=5, flip crossing/.list={1,2,3}]
            \strand[thick] (-10, -10) -- (10, 10);
            \strand[thick] (-10, 10) -- (10, -10);
            \strand[thick] (-10, 0) to [in=left, out=right] (0, -10);
            \strand[thick] (10, 0) to [in=right, out=left] (0, -10);
        \end{knot}
        \end{tikzpicture}}_{R_3}
    \]
\end{comment}
    skręcenie lub rozkręcenie ($R_1$), wsunięcie lub rozsunięcie ($R_2$) lub przesunięcie łuku przez skrzyżowanie ($R_3$) nazywamy ruchami Reidemeistera.
\end{definition}

Ruch $R_i$ operuje więc na $i$ łukach diagramu.
Reidemeister w~swojej pierwszej pracy przyjął inną kolejność,
jego drugi ruch jest naszym pierwszym.

\begin{theorem}[Reidemeister, 1927]
    \label{thm:reidemeister}
    \index{twierdzenie!Reidemeistera}
    Niech $D_1, D_2$ będą diagramami dwóch splotów $L_1, L_2$.
    Sploty $L_1, L_2$ są takie same wtedy i tylko wtedy, gdy diagram $D_2$ można otrzymać z $D_1$ wykonując skończony ciąg ruchów Reidemeistera oraz gładkich deformacji łuków, bez zmiany biegu skrzyżowań.
\end{theorem}

Dowód podali niezależnie Reidemeister \cite{reidemeister27} oraz Alexander, Briggs \cite{briggs27}.
Twierdzenie Reidemeistera jest prawdziwe także dla splotów zorientowanych, wtedy jednak w każdym ruchu trzeba uwzględnić wszystkie możliwe orientacje łuków.

\begin{proof}
    Szkielet dowodu można znaleźć w~książce Burdego i~Zieschanga.
    Kluczowe pomysły zawiera ,,Knots, links, braids and $3$-manifolds''
    Prasołowa i~Sosińskiego.
    Innym przystępnym źródłem jest podręcznik \cite{murasugi96} Murasugiego ,,Knot theory and its applications''.
\end{proof}

\begin{tobedone}
    Trace (1983) showed that two knot diagrams for the same knot are related by using only type II and III moves if and only if they have the same writhe and winding number.
    Trace, Bruce (1983), "On the Reidemeister moves of a classical knot", Proceedings of the American Mathematical Society, 89 (4): 722–724, doi:10.2307/2044613, MR 0719004

    Wspomnieć o węzłach obramowanych (framed knots), które są równoważne przy użyciu II, III i podwójnego I ruchu Reidemeistera.
\end{tobedone}

\begin{tobedone}
    Furthermore, combined work of Östlund (2001), Manturov (2004), and Hagge (2006) shows that for every knot type there are a pair of knot diagrams so that every sequence of Reidemeister moves taking one to the other must use all three types of moves.
    Östlund, Olof-Petter (2001), "Invariants of knot diagrams and relations among Reidemeister moves", J. Knot Theory Ramifications, 10 (8): 1215–1227, arXiv:math/0005108, doi:10.1142/S0218216501001402, MR 1871226
    Manturov, Vassily Olegovich (2004), Knot theory, Boca Raton, FL: Chapman et Hall/CRC, doi:10.1201/9780203402849, ISBN 0-415-31001-6, MR 2068425
    Hagge, Tobias (2006), "Every Reidemeister move is needed for each knot type", Proc. Amer. Math. Soc., 134 (1): 295–301, doi:10.1090/S0002-9939-05-07935-9, MR 2170571
\end{tobedone}

\begin{tobedone}
    Alexander Coward demonstrated that for link diagrams representing equivalent links, there is a sequence of moves ordered by type: first type I moves, then type II moves, type III, and then type II. The moves before the type III moves increase crossing number while those after decrease crossing number.
    Coward, Alexander; Lackenby, Marc (2014), "An upper bound on Reidemeister moves", American Journal of Mathematics, 136 (4): 1023–1066, arXiv:1104.1882, doi:10.1353/ajm.2014.0027, MR 3245186?
\end{tobedone}

W praktyce twierdzenia \ref{thm:reidemeister} nie stosuje się bezpośrednio do diagramów splotów.
Mając dane dwa spójne diagramy tego samego splotu trudno znaleźć jest ciąg ruchów przekształcający jeden z nich w drugi.
Załóżmy, że widać na nich odpowiednio $n_1, n_2$ skrzyżowań.
Jak piszą Coward, Lackenby w \cite{coward11}, istnieje funkcja $f \colon \N \times \N \to \N$ taka, że między dwoma diagramami można przejść wykonując co najwyżej $f(n_1, n_2)$ ruchów.
Wynika to z oczywistego faktu, że istnieje skończenie wiele spójnych diagramów o danej liczbie skrzyżowań oraz twierdzenia Reidemeistera.
Okazuje się jednak, że od funkcji $f$ można żądać, by była obliczalna i faktycznie, główny wynik \cite{coward11} orzeka, że
\begin{equation}
    f(n_1, n_2) = 2^{2^{\ldots^{2^{n_1 + n_2}}}}
\end{equation}
jest taką funkcją.
Piętrowa potęga liczy sobie aż $10^{1000000 (n_1 + n_2)}$ warstw, ale przynajmniej jest jawnie zdefiniowana.
Natomiast jeżeli $n_2 = 0$, czyli drugi diagram przedstawia niewęzeł, wystarcza $(236n_1)^{11}$ ruchów, to świeższy wynik samego Lackenby'a \cite{lackenby15}.

\begin{tobedone}
    Przedstawić rozumowanie (piramidka z węzłami), dlaczego to nie jest takie oczywiste.
\end{tobedone}

Zamiast tego definiuje się niezmienniki, czyli funkcje ze zbioru wszystkich diagramów, które nie zmieniają swojej wartości podczas wykonywania ruchów Reidemeistera.
Kiedy pewien niezmiennik przyjmuje różne wartości na dwóch diagramach, te przedstawiają dwa istotnie różne sploty.
Gdy wartości są te same, nie dostajemy żadnej informacji.
Sploty mogą być równoważne albo nie.
Niezmienniki będą nam stale towarzyszyć w~wędrówce po krainie węzłów.

% koniec sekcji Ruchy Reidemeistera

%Niestety pomimo upływu czasus, nikt nie napisał komputerowego programu realizującego ten algorytm (stan na 1994).
%Może podejmie się tego Czytelnik?
%Inne algorytmy istnieją, jednak wszystkie działają w~wykładniczym czasie.

W 1961 roku W. Haken \cite{haken61} podał niezawodny przepis na wykrycie diagramu niewęzła,
częściowo rozwiązując jeden z~ważniejszych problemów teorii węzłów.
Przez wiele lat nikt nie podjął się implementacji tego algorytmu,
udało się to niedawno Burtonowi, Budneyowi oraz Petterssonowi w~komputerowym programie Regina\footnote{Dostępny pod adresem \url{https://regina-normal.github.io/}.} na przełomie tysiącleci.
Burton, Rubinstein i~Tillman pokazali w~pracy \cite{burton12}, jak sprawdzać,
czy powierzchnia normalna na striangulowanej 3-rozmaitości jest (nie)ściśliwa w~czasie wykładniczym.
To okazało się być wystarczającym do udzielenia negatywnej odpowiedzi na pytanie Thurstona:
,,czy przestrzeń Seiferta-Webera jest rozmaitością Hakena?'',
a zatem wykraczającego poza poziom tej pracy.

SnapPea\footnote{Dostępny pod adresem \url{http://geometrygames.org/SnapPea/index.html}.} to inny popularny wśród niskowymiarowych topologów program pozwalający badać hiperboliczne 3-rozmaitości, patrz sekcja \ref{sec:hyperbolic}.

\begin{tobedone}
    Dowiązać tutaj wszystkie wykrywacze niewęzła opisane w książce.
\end{tobedone}

Przykładami trudnych w~rozpoznaniu niewęzłów są: niewęzeł Goritza, Freedmana.
Więcej trudnych niewęzłów zawiera praca \cite{zanellati16} autorstwa C. Petronio oraz A. Zanellatiego.

\index{niewęzeł}
\index{niewęzeł!Goritza}
\index{niewęzeł!Freedmana}
\begin{comment}
\begin{figure}[H]
    \begin{minipage}[b]{.32\linewidth}
        \centering
        \includegraphics[width=\linewidth]{../data/missing.jpg}
        \subcaption{normalny}
    \end{minipage}
    \begin{minipage}[b]{.32\linewidth}
        \centering
        \includegraphics[width=\linewidth]{../data/missing.jpg}
        \subcaption{Goritza}
    \end{minipage}
    \begin{minipage}[b]{.32\linewidth}
        \centering
        \includegraphics[width=\linewidth]{../data/missing.jpg}
        \subcaption{Freedmana}
    \end{minipage}
\end{figure}
\end{comment}

Zanim opowiemy, jak dotąd przebiegała klasyfikacja węzłów o małej liczbie skrzyżowań, zdefiniujemy klasę splotów ze specjalnymi diagramami.

\begin{definition}[alternacja]
    \index{splot!alternujący}
    Diagram splotu, gdzie podczas poruszania się wzdłuż każdego ogniwa nad- oraz podskrzyżowania mijane są naprzemiennie, nazywamy alternującym.
    Splot jest alternujący, jeśli posiada alternujący diagram.
\end{definition}

Około 1961 roku Fox zapytał ,,What is an alternating knot?''.
Szukano takiej definicji węzła alternującego, która nie odnosi się bezpośrednio do diagramów, aż w~2015 roku Joshua Greene podał geometryczną charakteryzację: nierozdzielczy splot w $S^3$ jest alternujący wtedy i tylko wtedy, gdy ogranicza dodatnią oraz ujemną określoną powierzchnię rozpinającą \cite{greene17}.
% definite spanning surface

Sundberg oraz Thistlethwaite pokazali w 1998 roku, że liczba splotów alternujących rośnie wykładniczo (\cite{sundberg98}):

\begin{proposition}
    Niech $a_n$ oznacza liczbę pierwszych, alternujących supłów o~$n$ skrzyżowaniach.
    Wtedy
    \begin{equation}
        a_n \sim (3c_1/4\sqrt{\pi})n^{-5/2}\lambda^{n-3/2},
    \end{equation}
    gdzie zarówno $c_1$, pierwszy współczynnik rozwinięcia Taylora funkcji $\Phi(\eta)$ zdefiniowanej w \cite{sundberg98}, jak i $\lambda$ są jawnie znanymi stałymi:
    \begin{align}
        c_1 & = \sqrt{\frac{5^7 \cdot (21001 + 371 \sqrt{21001})^3}{2 \cdot 3^{10} \cdot (17 + 3\sqrt{21001})^5}} \\
        \lambda & = \frac {1}{40} (101 + \sqrt{21001})
    \end{align}
    Niech $A_n$ oznacza liczbę pierwszych, alternujących splotów o $n$ skrzyżowaniach.
    Wtedy $A_n \approx \lambda^n$, dokładniej: jeśli $n \ge 3$, to
    \begin{equation}
        \frac{a_{n-1}}{16n - 24} \le A \le \frac{a_n - 1}{2}.
    \end{equation}
\end{proposition}

Czasami będziemy używać słów przed ich zdefiniowaniem, tak jak uczyniliśmy tutaj: węzły pierwsze i~supły pojawiają się odpowiednio w definicjach \ref{def:prime_knot}, \ref{def:tangle}.
Książkę trzeba więc przeczytać co~najmniej dwa razy.

\begin{proposition}
    Niech $a_n$ oznacza liczbę pierwszych, alternujących supłów o~$n$ skrzyżowaniach.
    Wtedy funkcja tworząca $f(z) = \sum_n a_n z^n$ spełnia równanie
    \begin{equation}
    f(1+z) - f(z)^2 - (1+f(z))q(f(z)) -z - \frac{2z^2}{1-z} = 0,
    \end{equation}
    gdzie $q(z)$ jest pomocniczą funkcją
    \begin{equation}
        q(z) = \frac{2z^2 - 10z - 1 + \sqrt{(1-4z)^3}} {2(z+2)^3} - \frac{2}{1+z} -z + 2.
    \end{equation}
\end{proposition}

Powyższa ciekawostka także pochodzi z cytowanej wcześniej pracy \cite{sundberg98}.
