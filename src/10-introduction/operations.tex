\section{Operacje na węzłach} % (fold)
\label{sec:knot_operations}
W tej sekcji poznamy sposoby otrzymywania nowych obiektów z już istniejących (rewers i lustro splotu).
Rodzina węzłów wyposażona w sumę spójną tworzy przemienny monoid z jednoznacznością rozkładu.
Znacznie później (w sekcji \ref{section_tangle}) określimy jeszcze sumę oraz iloczyn supłów.

\subsection{Operacje na pojedynczym splocie} % (fold)
\label{sub:single_operations}
\begin{definition}
    Niech $L$ będzie zorientowanym splotem.
    Przez \textbf{rewers} $L$, $rL$,
    rozumiemy ten sam splot z przeciwną orientacją.
    \textbf{Lustrem} $L$, $mL$,
    będziemy nazywać odbicia $L$ względem płaszczyzny.
    \begin{figure}[H]
        \begin{minipage}[b]{.32\linewidth}
            \centering 
            \includegraphics[width=\linewidth]{../data/link_mirror.png} 
            \subcaption{lustro $mL$}
        \end{minipage}
        \begin{minipage}[b]{.32\linewidth}
            \centering 
            \includegraphics[width=\linewidth]{../data/link.png} 
            \subcaption{przykładowy splot $L$}
        \end{minipage}
        \begin{minipage}[b]{.32\linewidth}
            \centering 
            \includegraphics[width=\linewidth]{../data/link_reverse.png} 
            \subcaption{rewers $rL$}
        \end{minipage}
    \end{figure}
\end{definition}

Na lewym obrazku odbiliśmy diagram względem poziomej prostej,
ale równie dobrze można po prostu zamienić każde nadskrzyżowanie na podskrzyżowanie.
Zauważmy, że wykonując powyższe operacje na węźle możemy otrzymać mniej niż czterech różne obiekty
($L$, $mL$, $rL$, $mrL$) -- na przykład trójlistnik jest własnym rewersem, ale nie lustrem.

\begin{definition}
    Istnieje pięć typów symetrii węzłów:
    całkowicie chiralny albo skrętny (węzły $K$, $rK$, $mK$ są parami nierównoważne),
    odwracalny ($K \cong rK$),
    zwierciadlany ujemnie ($K \cong mrK$),
    zwierciadlany dodatnio ($K \cong mK$) oraz
    całkowicie zwierciadlany ($rK \cong K \cong mK$).
\end{definition}

Węzeł $9_{32}$ jest całkowicie skrętny. 
Ósemka jest zwierciadlana, trójlistnik jest odwracalny,
zaś $8_{17}$ to najprostszy przykład węzła nieodwracalnego.
Ten ostatni fakt jest jednak daleki od oczywistego -- 
sześćdziesiąt lat temu nie było pewne,
czy węzły nieodwracalne w ogóle istnieją.
W roku 1962 Ralph Fox wskazał kilku kandydatów do tego tytułu.
Hale Trotter odkrył rok później nieskończoną rodzinę nieodwracalnych precli (patrz \ref{def:pretzel}).
Obecnie wiadomo, że prawie wszystkie węzły są nieodwracalne (\cite[s.~46]{murasugi96}).
Wszystkie węzły torusowe są skrętne.

\begin{proposition}[Trotter, 1963] \label{trotter}
    Precel $(p, q, r)$, gdzie $p, q, r$ sa parami różnymi liczbami całkowitymi, nie jest odwracalny.
\end{proposition}

\begin{proof}
    Praca \cite{trotter63} sprowadza w elementarny sposób problem do pytania, czy pewna grupa posiada inwersję.
\end{proof}

Tait odnosił wrażenie, że zwierciadlane węzły mają parzysty indeks skrzyżowań,
ale Hoste (Thistlethwaite?) znalazł w 1998 kontrprzykład o piętnastu skrzyżowaniach.
Jest on jedynym znanym nam dzisiaj.
Hipoteza Taita jest prawdziwa dla węzłów pierwszych, alternujących.

Poniższa tabela oparta jest (kolejno) o ciągi 
\href{https://oeis.org/A051766}{51766}, 
\href{https://oeis.org/A051769}{51769}, 
\href{https://oeis.org/A051768}{51768}, 
\href{https://oeis.org/A051767}{51767}, 
\href{https://oeis.org/A052400}{52400}, 
z bazy danych ``The On-Line Encyclopedia of Integer Sequences'' (OEIS).

\begin{table}[h]
    \centering
    \begin{tabular}{@{}*{20}l@{}} \toprule
        skrzyżowania & 3 & 4 & 5 & 6 & 7 & 8 & 9 & 10 & 11 & 12 & 13 & 14 \\ \midrule
        całkowicie skrętne & 0 & 0 & 0 & 0 & 0 & 0 & 2 & 27 & 187 & 1103 & 6919 & 37885 \\
        odwracalne & 1 & 0 & 2 & 2 & 7 & 16 & 47 & 125 & 365 & 1015 & 3069 & 8813 \\
        $-$ zwierciadlane & 0 & 0 & 0 & 0 & 0 & 1 & 0 & 6 & 0 & 40 & 0 & 227 \\
        $+$ zwierciadlane & 0 & 0 & 0 & 0 & 0 & 0 & 0 & 0 & 0 & 1 & 0 & 6 \\
        zwierciadlane & 0 & 1 & 0 & 1 & 0 & 4 & 0 & 7 & 0 & 17 & 0 & 41 \\
        \bottomrule
        \hline
    \end{tabular}
    \caption{Liczba węzłów o poszczególnych typach symetrii}
    \label{tablica_wezlow}
\end{table}

% Koniec podsekcji Operacje na pojedynczym splocie

\subsection{Sumy wielu węzłów} % (fold)
\label{sub:knot_sum}
Suma spójna węzłów to szczególny przypadek sklejenia dwóch rozmaitości wzdłuż brzegu.

\begin{definition}
    Niech $L_1, L_2$ będą splotami.
    Ich \textbf{sumą niespójną}, $L_1 \sqcup L_2$,
    jest teoriomnogościowa suma rozłączna splotów
    $L_1$ i $L_2$ leżących po różnych stronach pewnej płaszczyzny.
\end{definition}

\begin{definition}
    Nacinając dwa zorientowane węzły w dwóch bliskich punktach i 
    ponownie zszywając je dwoma łukami nieprzecinającymi już
    istniejących otrzymujemy \textbf{sumę spójną}, $K_1 \shrap K_2$.
    \[
        \begin{tikzpicture}[baseline=-0.65ex,scale=0.1]
        \begin{knot}[clip width=5, flip crossing/.list={5}, ignore endpoint intersections=false,]
            \strand[thick] (-3.5, -3.5) [in=down, out=up] to (3.5, 3.5);
            \strand[thick] (3.5, 3.5) [in=right, out=up] to (-4.5, 10);
            \strand[thick] (-4.5, 10) [in=up, out=left] to (-10, 3.5);
            \strand[thick] (-10, 3.5) to (-10, -3.5);
            \strand[thick] (-10, -3.5) [in=left, out=down] to (-4.5, -10);
            \strand[thick] (-4.5, -10) [in=down, out=right] to (3.5, -3.5);
            \strand[thick] (3.5, -3.5) [in=down, out=up] to (-3.5, 3.5);
            \strand[thick] (-3.5, 3.5) [in=left, out=up] to (4.5, 10);
            \strand[thick] (4.5, 10) [in=up, out=right] to (10, 3.5);
            \strand[thick, -Latex] (10, 3.5) to (10, -3.5);
            \strand[thick] (10, -3.5) [in=right, out=down] to (4.5, -10);
            \strand[thick] (4.5, -10) [in=down, out=left] to (-3.5, -3.5);
            \node at (0, -15) {$K_1$};
        \end{knot}
        \end{tikzpicture}
        \shrap
        \begin{tikzpicture}[baseline=-0.65ex,scale=0.1]
        \begin{knot}[clip width=5, flip crossing/.list={6}, ignore endpoint intersections=false,]
            \strand[thick] (-3.5, -3.5) [in=down, out=up] to (3.5, 3.5);
            %\strand[thick] (3.5, 3.5) [in=right, out=up] to (-4.5, 10);
            %\strand[thick] (-4.5, 10) [in=up, out=left] to (-10, 3.5);
            \strand[thick] (-10, -3.5) [in=left, out=up] to (0, 6.5);
            \strand[thick, Latex-] (-10, -3.5) [in=left, out=down] to (-4.5, -10);
            \strand[thick] (-4.5, -10) [in=down, out=right] to (3.5, -3.5);
            \strand[thick] (3.5, -3.5) [in=down, out=up] to (-3.5, 3.5);
            %\strand[thick] (-3.5, 3.5) [in=left, out=up] to (4.5, 10);
            %\strand[thick] (4.5, 10) [in=up, out=right] to (10, 3.5);
            \strand[thick] (10, -3.5) [in=right, out=up] to (0, 6.5);
            \strand[thick] (10, -3.5) [in=right, out=down] to (4.5, -10);
            \strand[thick] (4.5, -10) [in=down, out=left] to (-3.5, -3.5);
            %
            \strand[thick] (-3.5, 3.5) [in=left, out=up] to (0, 10);
            \strand[thick] (3.5, 3.5) [in=right, out=up] to (0, 10);
            \node at (0, -15) {$K_2$};
        \end{knot}
        \end{tikzpicture}
        =
        \begin{tikzpicture}[baseline=-0.65ex,scale=0.1]
        \begin{knot}[clip width=5, flip crossing/.list={5, 22, 23}, ignore endpoint intersections=false,]
            \strand[thick] (-18.5, -3.5) [in=down, out=up] to (-11.5, 3.5);
            \strand[thick] (-11.5, 3.5) [in=right, out=up] to (-19.5, 10);
            \strand[thick] (-19.5, 10) [in=up, out=left] to (-25, 3.5);
            \strand[thick] (-25, 3.5) to (-25, -3.5);
            \strand[thick] (-25, -3.5) [in=left, out=down] to (-19.5, -10);
            \strand[thick] (-19.5, -10) [in=down, out=right] to (-11.5, -3.5);
            \strand[thick] (-11.5, -3.5) [in=down, out=up] to (-18.5, 3.5);
            \strand[thick] (-18.5, 3.5) [in=left, out=up] to (-10.5, 10);
            \strand[thick] (-10.5, 10) [in=left, out=right] to (-5, 2);
            \strand[thick, -Latex] (-5, 2) to (-5+6, 2);
            \strand[thick] (5, 2) to (-5+6, 2);
            \strand[thick] (3, -2) to [in=left, out=right] (10.5, -10);
            \strand[thick, -Latex] (3, -2) to (0, -2);
            \strand[thick] (-5, -2) to (0, -2);
            \strand[thick] (-5, -2) [in=right, out=left] to (-10.5, -10);
            \strand[thick] (-10.5, -10) [in=down, out=left] to (-18.5, -3.5);
            %%%
            \strand[thick] (11.5, -3.5) [in=down, out=up] to (18.5, 3.5);
            \strand[thick] (-10 +15, 2) [in=left, out=right] to (15, 6.5);
            \strand[thick] (10.5, -10) [in=down, out=right] to (18.5, -3.5);
            \strand[thick] (18.5, -3.5) [in=down, out=up] to (11.5, 3.5);
            \strand[thick] (25, -3.5) [in=right, out=up] to (15, 6.5);
            \strand[thick] (25, -3.5) [in=right, out=down] to (19.5, -10);
            \strand[thick] (19.5, -10) [in=down, out=left] to (11.5, -3.5);
            \strand[thick] (11.5, 3.5) [in=left, out=up] to (15, 10);
            \strand[thick] (18.5, 3.5) [in=right, out=up] to (15, 10);
            %%%
            \node at (0, -15) {$K_1 \shrap K_2$};
        \end{knot}
        \end{tikzpicture}
        \]
\end{definition}

Uogólnieniem tego działania oraz 
(nieopisanej w naszej pracy) operacji \emph{plumbing}
jest suma Murasugiego, dobrze wyjaśniona w czwartym rozdziale \cite{kawauchi96}.

Ważna jest orientacja składników:
suma dwóch trójlistników może być, w zależności od ich orientacji,
węzłem babskim\todo{nazewnictwo pochodzi od żeglarzy, nie matematyków} lub prostym.
Fox pokazał w 1952 roku, że dopełnienia tych dwóch węzłów nie są homeomorficzne.
Suma przeciwnie skręconych trójlistników jest plastrowa\todo{definicja \ref{slice_def}},
natomiast tak samo skręconych -- nie.

\begin{proposition}
    Suma spójna jest dobrze określonym działaniem.
\end{proposition}

Suma spójna nie jest dobrze określona dla splotów:
nie istnieje kanoniczny wybór, które składowe łączyć ze sobą.

\begin{proof}
    Niech dane będą węzły $K_1$ oraz $K_2$
    oraz dwa różne łuki $\gamma_1$, $\gamma_2$,
    których można użyć do konstrukcji sumy spójnej.
    Skurczmy $K_1$, przeciągnijmy najpierw przez łuk $\gamma_1$, a następnie wzdłuż węzła $K_2$.
    Teraz wystarczy odwrócić ten proces z $\gamma_2$ w miejscu $\gamma_1$.
\end{proof}

\begin{proposition}
    Suma spójna zorientowanych węzłów posiada następujące własności:
    \begin{enumerate}[leftmargin=*]
    \itemsep0em
        \item jest łączna:
        $(K_1 \shrap K_2) \shrap K_3 \simeq K_1 \shrap (K_2 \shrap K_3)$,
        \item jest przemienna:
        $K_1 \shrap K_2 \simeq K_2 \shrap K_1$,
        \item posiada element neutralny:
        $K_1 \shrap \MalyNieWezel \simeq \MalyNieWezel \shrap K_1 \simeq K_1$.
    \end{enumerate}
\end{proposition}

Prosty dowód tego faktu pozostawiamy Czytelnikowi.
Pokażemy za to, iż węzły z sumą spójną nie tworzą grupy -- brakuje elementów przeciwnych.

\begin{proposition}
    Jeśli $K \shrap L = \NieWezel$, to $K = L = \NieWezel$.
\end{proposition}

\begin{proof}[Niedowód]
    Technika ta zwana jest szwindlem Mazura.
    Załóżmy, że $K \shrap L = \NieWezel$ i dopuśćmy wyjątkowo węzły dzikie.
    Skonstruujmy sumę $K \shrap L \shrap K \shrap \ldots$,
    przy czym kolejne składniki powinny zmniejszać się,
    aby ich suma nadal była węzłem.
    Wtedy
    \begin{align*}
        K & \simeq K \shrap [(L \shrap K) \shrap (L \shrap K) \ldots] \\
         & \simeq (K \shrap L) \shrap (K \shrap L) \shrap \ldots
         \simeq \NieWezel \shrap \NieWezel \shrap \ldots
         \simeq \NieWezel.
    \end{align*}
    Analogicznie pokazujemy, że $L \simeq \NieWezel$.
\end{proof}

Później poznamy prawdziwy, oparty na topologii algebraicznej, dowód.
Połączymy fakty \ref{genus_one} oraz \ref{genus_sum}.

Półgrupę węzłów z operacją sumy spójnej można ulepszyć do grupy na dwa sposoby:
albo poprzez zmianę działania, w jakie jest wyposażona,
albo osłabiając definicję węzłów równoważnych.
Drugi pomysł jest dużo lepszy niż pierwszy.
Na początku lat pięćdziesiątych J. Millnor wprowadził do matematyki pojęcie zgodności
(z angielskiego \emph{concordance}), które zastąpiło zwykłą równoważność.
Element neutralny nowej grupy to węzły plastrowe, ich opis leży w sekcji \ref{sec:slice}.
Zagadnienia te zakorzenione są w czterowymiarowej topologii.

% Koniec podsekcji Sumy wielu węzłów

% Koniec sekcji Operacje na węzłach
