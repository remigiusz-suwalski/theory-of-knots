\section{Niezmienniki liczbowe}
Jak wspomnieliśmy na początku rozdziału, sprawdzenie,
czy dwa diagramy przedstawiają sploty równoważne,
jest uciążliwym i~czasochłonnym zadaniem.
Aby je uprościć, podamy opis kilku prostych niezmienników o~całkowitych wartościach.
Zachodzą implikacje:
sploty równoważne $\Rightarrow$ ta sama wartość niezmiennika
oraz różne wartości niezmiennika $\Rightarrow$ różne sploty.

Tutaj przedstawiamy jedynie te niezmieniki, które nie wymagają mocnej znajomości reszty książki.
Później poznamy jeszcze sygnaturę (w~podsekcji \ref{sub:signature}) oraz liczbę warkoczową (w~podsekcji \ref{sub:braid_number}).

\subsection{Indeks skrzyżowaniowy} % (fold)
\label{sub:crossing_number}
\index{indeks!skrzyżowaniowy}
Z angielskiego \emph{crossing number}.

\begin{definition}
    Niech $L$ będzie splotem.
    Minimalną liczbę skrzyżowań widocznych na diagramie, który przedstawia splot $L$, nazywamy indeksem skrzyżowaniowym i~oznaczamy $\operatorname{cr}(L)$.
\end{definition}

Pytanie, czy indeks skrzyżowaniowy jest addytywny, to jeden z najstarszych problemów teorii węzłów.

\begin{conjecture}
    \label{cnj:crossing_additive}
    Niech $K$ oraz $L$ będą węzłami.
    Wtedy $\operatorname{cr}(K) + \operatorname{cr}(L) = \operatorname{cr}(K \shrap L)$.
\end{conjecture}

Oto częściowe odpowiedzi.
Jeśli $K_1, K_2$ są alternującymi węzłami o~odpowiednio $c_1, c_2$ skrzyżowaniach, to istnieje alternujący diagram ich sumy $K_1 \shrap K_2$ o~$c_1 + c_2$ skrzyżowaniach.
Kauffman, Murasugi oraz Thistlethwaite pokazali niezależnie, że diagram ten jest minimalny (patrz na przykład \cite{murasugi87}, wniosek 6).
Thistlethwaite rozszerzył wynik do tak zwanych węzłów adekwatnych w \cite{thistlethwaite88}.
Wreszcie Gruber w \cite{gruber03} udowodnił hipotezę \ref{cnj:crossing_additive} dla węzłów torusowych.
Lackenby w~pracy \cite{lackenby09} pokazał, że dla pewnej stałej $N \le 152$ zachodzi
\begin{equation}
    \frac 1 N \sum_{i=1}^n \operatorname{cr} K_i \le \operatorname{cr} \left(\bigshrap_{i=1}^n K_i\right) \le \sum_{i=1}^n \operatorname{cr} K_i.
\end{equation}
(Tylko pierwsza nierówność jest ciekawa).
Jego argumentu wykorzystującego powierzchnie normalne nie można poprawić tak, by otrzymać stałą $N = 1$.

% Koniec podsekcji Indeks skrzyżowaniowy

\subsection{Liczba gordyjska} % (fold)
\label{sub:unknotting_number}
\index{liczba!gordyjska}
Z angielskiego \emph{unknotting number}.

\begin{definition}
    Niech $L$ będzie splotem.
    Minimalną liczbę skrzyżowań, które trzeba odwrócić na pewnym jego diagramie, by dostać niewęzeł, nazywamy liczbą gordyjską i~oznaczamy $\operatorname{u}(L)$.
\end{definition}

Zgodnie z ,,klasyczną'' definicją, między odwracaniem kolejnych skrzyżowań mamy prawo wykonać izotopie otaczające; natomiast zgodnie ze ,,standardową'' definicją, takie izotopie są zabronione.
Obie definicje są równoważne: tłumaczy to książka Adamsa \cite[s. 58]{adams94}.

W pracy \cite{shimizu14} Shimizu rozpatruje różne operacje, które rozwiązują węzły lub sploty.
Nie będziemy się nimi zajmować, podamy tylko przykład: zamiana pod- i nadskrzyżowań wokół obszaru na diagramie rozwiązuje węzły, ale nie sploty; kontrprzykładem jest splot Hopfa.

Jeśli odwrócenie pewnych skrzyżowań daje niewęzeł, to odwrócenie pozostałych także.
To daje proste liczby gordyjskiej: $2 \operatorname{u} (K) \le \operatorname{cr} (K)$.
Nie jest zbyt pomocne, równość zachodzi pięć razy dla pierwszych węzłów do 12 skrzyżowań: $3_{1}$, $5_{1}$, $7_{1}$, $9_{1}$, $11a_{367}$.
Dokładna wartość liczby gordyjskiej jest znana tylko dla niektórych klas węzłów, na przykład torusowych (fakt \ref{torus_unknotting}) albo skręconych (definicja \ref{twist_knots}).

Dla każdego nietrywialnego splotu istnieje diagram wymagający odwrócenia dowolnie wielu skrzyżowań.
Dowód zawiera praca \cite{taniyama09} Taniyamy.
Pokazany jest tam jeszcze jeden godny uwagi fakt.
Jeśli liczba gordyjska diagramu $D$ wynosi $\frac 12 (\operatorname{cr} D - 1)$, co jest maksymalną możliwą wartością zgodnie z~naszym prostym ograniczeniem,
to węzeł jest $(2,p)$-torusowy albo wygląda jak diagram niewęzła po pierwszym ruchu Reidemeistera.

Bleiler odkrył w~\cite{bleiler84} fascynujący przykład wymiernego węzła $10_8$, który jest $2$-gordyjski, ale świadkiem tego nie może być żaden diagram mininalny, ponieważ, co jeszcze bardziej fascynujące, węzeł ten posiada tylko jeden diagram o~dziesięciu skrzyżowaniach oraz liczbie gordyjskiej 3.
Wynika stąd, że liczba $u$ nie musi być osiągana przez diagram minimalny, wbrew powszechnym przypuszczeniom obecnym jeszcze w latach 70.
Praca \cite{bernhard94} opisuje nieskończoną rodzinę węzłów $C_k$, gdzie $C_2 = 10_8$ jest węzłem Bleilera.

Przykład Bleilera pokazuje, że do szukania liczby gordyjskiej potrzeba wyrafinowanego algorytmu.
Ponieważ odwrócenie jednego ze skrzyżowań na minimalnym diagramie węzła $10_8$ daje $1$-gordyjski węzeł $4_1, 5_1, 6_1$ lub $6_2$, możemy liczyć, że każdy diagram minimalny ma skrzyżowanie, którego odwrócenie zmniejsza liczbę gordyjską.
Dlatego jeszcze w~latach 90. postawiono hipotezę:

\begin{conjecture}[Bernharda-Jablana, \cite{bernhard94}, \cite{jablan98}]
    \label{bernhard_jablan}
    Każdy węzeł $K$ posiada diagram $D$ realizujący liczbę gordyjską oraz skrzyżowanie, którego odwrócenie daje nowy węzeł $K'$ z diagramem $D'$ o mniejszej liczbie gordyjskiej: $u(D') < u(D)$.
\end{conjecture}

Zakładając prawdziwość hipotezy \ref{bernhard_jablan}, mamy prosty sposób na wyznaczenie liczby $u(K)$: weźmy skończenie wiele diagramów minimalnych dla węzła $K$, na każdym z~nich odwracajmy skrzyżowania i rekursywnie szukajmy liczb gordyjskich prostszych węzłów.
Najmniejsza spośród nich różni się wtedy o~jeden od liczby $u(K)$.

Brittenham, Hermiller w artykule \cite{brittenham17} twierdzą, że hipoteza jest fałszywa, ten nie został jednak jeszcze zrecenzowany.
Prawdziwość sprawdzono natomiast dla węzłów do jedenastu skrzyżowań oraz splotów o dwóch ogniwach do dziewięciu skrzyżowań (Kohn w \cite{kohn93}?).

\begin{example}[Brittenham, Hermiller]
    Hipoteza Bernharda-Jablana jest fałszywa dla co najmniej jednego spośród czterech węzłów: $12n_{288}$, $12n_{491}$, $12n_{501}$, $13n_{3370}$.
\end{example}

Bleiler postawił w~\cite{bleiler84} problem: czy jeden węzeł może mieć kilka diagramów minimalnych, z~których tylko niektóre są świadkiem $1$-gordyjskości?
Rozwiązanie przyszło wkrótce z Japonii: według \cite{kanenobumurakami86} dzieje się tak m.in. dla węzła $8_{14}$.
Stojemenow w~pracy \cite{stoimenow01} pełnej różnych przykładów wskazał dodatkowo węzły $14_{36750}$ oraz $14_{36760}$.

Sploty o liczbie gordyjskiej 1 zasługują na szczególną uwagę.

\begin{proposition}
    Niech $L$ będzie wymiernym splotem 1-gordyjskim.
    Wtedy na minimalnym diagramie $L$ jedno ze skrzyżowań jest rozwiązujące.
\end{proposition}

\begin{proof}
    Kanenobu, Murakami dla węzłów \cite{kanenobumurakami86}, wkrótce po tym Kohn dla splotów \cite{kohn91}.
\end{proof}

Z pracy \cite{kanenobumurakami86} wynika dodatkowo, że liczba gordyjska węzłów $8_{3}$, $8_{4}$, $8_{6}$, $8_{8}$, $8_{12}$, $9_{5}$, $9_{8}$, $9_{15}$, $9_{17}$, $9_{31}$ wynosi dokładnie 2, wcześniej wiedzieliśmy, że jest równa co najwyżej 2. 

Coward, Lackenby dowiedli w~\cite{coward11}, że jeśli $K$ jest 1-gordyjski i~o genusie 1, to z~dokładnością do pewnej relacji równoważności, tylko jedna zmiana skrzyżowania rozwiązuje go; chyba że $K$ jest ósemką -- wtedy takie zmiany są dwie.

\begin{proposition}
    \label{unknotting_one_prime}
    Węzły $1$-gordyjskie są pierwsze.
\end{proposition}

Podejrzewał to H. Wendt w~1937 roku, kiedy policzył liczbę gordyjską węzła babskiego używając homologii rozgałęzionego nakrycia cyklicznego.

\begin{proof}[Niedowód]
    W pracy \cite{scharleman85} z~1985 roku M. Scharleman podał dość zawiłe uzasadnienie, w~które zamieszane były grafy planarne.
    Obecnie znamy prostsze dowody, patrz \cite{lackenby97} albo \cite{zhang91}.
\end{proof}

Scharlemann pokazał w \cite[wniosek 1.6]{scharlemann98}, że liczba gordyjska jest podaddytywna, to znaczy zachodzi $u(K_1 \shrap K_2) \le u(K_1) + u(K_2)$.
Stąd oraz z faktu \ref{unknotting_one_prime} wynika, że suma dwóch $1$-gordyjskich węzłów jest $2$-gordyjska, ale od początku teorii węzłów podejrzewano dużo więcej:

\begin{conjecture}
    Niech $K, J$ będą węzłami.
    Wtedy $u(K \shrap J) = u(K) + u(J)$, czyli liczba gordyjska jest addytywna.
\end{conjecture}

Dotychczas wyznaczono liczbę gordyjską dla prawie wszystkich węzłów pierwszych o~co najwyżej dziesięciu skrzyżowaniach,
Cha, Livingston \cite{cha05} podają następującą listę wyjątków:
$10_{11}$, $10_{47}$, $10_{51}$, $10_{54}$, $10_{61}$, $10_{76}$, $10_{77}$, $10_{79}$, $10_{100}$ (stan na rok 2018).

Borodzik oraz Friedl podali niedawno całkiem mocne ograniczenia na liczbę gordyjską w~pracach \cite{borodzik14} i~\cite{borodzik15}.
Ich narzędziem jest parowanie Blanchfielda.
Poprawiają tam starsze estymaty wynikające z~sygnatury Levine'a-Tristrama, indeksu Nakanishiego oraz przeszkody Lickorisha.
Wśród węzłów o~co najwyżej dwunastu skrzyżowaniach 25 ma liczbę gordyjską równą co najmniej trzy, co trudno pokazać innymi metodami.

Liczbę gordyjską można uogólnić w naturalny sposób do metryki.
Mianowicie mając dane dwa węzły $K_0, K_1$, rozpatrzmy wszystkie homotopie $f : [0,1] \times S^1 \to \R^3$ takie, że wszystkie funkcje $f_t$ są zanurzeniami z co najwyżej jednym punktem podwójnym.
Zażądajmy dodatkowo, by styczne do krótkich łuków, które przecinają się w tym punkcie, były od siebie różne.
Odległością gordyjską między węzłami $K_0, K_1$ jest minimalna liczba podwójnych punktów, jakie posiada homotopia $f$.
Twierdzenie C~z~pracy \cite{gambaudo05} głosi, że zawiera ona prawie idealną kopię przestrzeni euklidesowej dowolnego wymiaru.
Dokładniej:

\begin{proposition}
    Dla każdej liczby całkowitej $n \ge 1$ istnieje funkcja $\xi: \Z^n \to \mathcal{K}$, dodatnie stałe $A, B, C$ oraz norma $\|\cdot\|$ na przestrzeni $\R^n$ takie, że spełniona jest podwójna nierówność
    \begin{equation}
        A\|x-y\|  - B \le d(\xi(x), \xi(y)) \le C\|x-y\|.
    \end{equation}
\end{proposition}

\begin{proof}
    Dowód korzysta z grup warkoczowych, które poznamy w sekcji \ref{sec:braid}.
\end{proof}

% Koniec podsekcji Liczba gordyjska

\subsection{Liczba mostowa} % (fold)
\label{sub:bridge_index}
\index{liczba!mostowa}
Z angielskiego \emph{bridge number}.
Wprowadzona w~1954 przez Schuberta.
\begin{definition}
    Liczba mostowa $\operatorname{br}(K)$ to minimalna liczba mostów:
    długich łuków, które biegną przez nadskrzyżowania.
\end{definition}

Można pokazać, że $n$-mostowe węzły rozkładają się na sumę dwóch trywialnych $n$-supłów.

\begin{proposition}
    \label{bridge_additive}
    Niech $K_1, K_2$ będą węzłami.
    Wtedy $\operatorname{br} (K_1) + \operatorname{br}(K_2) = \operatorname{br}(K_1 \# K_2) + 1$.
\end{proposition}

\begin{proof}[Nieedowód]
    Schubert pokazał to blisko pół wieku temu w~\cite{schubert54}.
    Nowszy dowód pochodzi od Schultens, w~artykule \cite{schultens03} skorzystała z~foliacji na brzegu węzła towarzyszącego sateltarnemu.
    Dokładniejszy opis powyższych prac wykraczałby poza zakres tego opracowania, zostanie więc pominięty.
\end{proof}

Tylko jeden węzeł jest jednomostowy, to niewęzeł.
Kolejne w~hierarchii skomplikowania, czyli dwumostowe, to domknięcia wymiernych supłów.
Węzły trójmostowe pozostają nie do końca zbadane, Japończycy pokazali w~\cite{fukuhama99}, że trzymostowe węzły genusu jeden są preclami.
Praca \cite{hilden12} zawiera klasyfikację wszystkich węzłów trzymostowych przy użyciu tak zwanej reprezentacji motylkowej, podobną do wyniku Schuberta opisanego w~sekcji \ref{sub:twobridge}.

Murasugi wspomina w rozdziale 4.3 podręcznika \cite{murasugi96} następującą hipotezę, nie podaje jednak wcale, skąd się wzięła:

\begin{conjecture}
    Jeśli $K$ jest węzłem, to $\crossing K \ge 3 \bridge K - 3$, przy czym równość zachodzi dokładnie dla niewęzła, trójlistnika i~sumy spójnej trójlistników.
\end{conjecture}

Należy więc uzupełnić brakujące informacje.
Murasugi w pracy \cite{murasugi88} przypuszcza, że dla splotów o $\mu$ ogniwach zachodzi nierównosć $\crossing L + \mu - 1 \ge 3 \bridge L - 3$, przedstawia jednocześnie dowód jej szczególnego przypadku, dla alternujących splotów algebraicznych.
Hipoteza Murasugiego stanowi uogólnienie dużo starszego problemu pochodzącego od Foxa \cite{fox50}, który zapytał, czy nierówność jest prawdziwa dla węzłów, gdy $\mu = 1$.

Nie istnieje związek między liczbą mostową oraz gordyjską.
Po pierwsze, węzły torusowe $T_{2,n}$ są dwumostowe, a~ich liczba gordyjska nieograniczona.
Po drugie, podwojenie węzła (poza specjalnymi przypadkami, jak pokazał Schubert) zwiększa liczbę mostową dwukrotnie; liczba gordyjska takiego podwojenia wynosi $1$.

Podobnie nie ma zależności między liczbą mostową oraz genusem.

% Koniec podsekcji Liczba mostowa


\subsection{Indeks zaczepienia} % (fold)
\label{sub:linking_number}
\begin{definition} \label{sign_def}
    Na diagramie zorientowanego splotu, każdemu skrzyżowaniu przypisujemy \textbf{znak} równy $\pm 1$.
    Skrzyżowania dodatnie nazywamy praworęcznymi, ujemne zaś: leworęcznymi.
\begin{comment}
    \[
        \sign \Bigl(\,\,\begin{tikzpicture}[baseline=-0.65ex,scale=0.07]
        \begin{knot}[clip width=5]
        \strand[semithick,-Latex] (-5,-5) -- (5,5);
        \strand[semithick,Latex-] (-5,5) -- (5,-5);
        \end{knot}\end{tikzpicture}\,\,\Bigr) = +1 \quad
        \sign \Bigl(\,\,\begin{tikzpicture}[baseline=-0.65ex,scale=0.07]
        \begin{knot}[clip width=5, flip crossing/.list={1}]
        \strand[semithick,-Latex] (-5,-5) -- (5,5);
        \strand[semithick,Latex-] (-5,5) -- (5,-5);
        \end{knot}\end{tikzpicture}\,\,\Bigr) = -1
    \]
\end{comment}
\end{definition}

\begin{definition}
    \index{indeks!zaczepienia}
    Niech $L = K_1 \sqcup K_2$ będzie splotem o dwóch ogniwach.
    Wielkość
    \[
        \operatorname{lk}(K_1, K_2) = \frac 12 \sum_i \sign c_i,
    \]
    gdzie sumowanie rozciąga się na wszystkie skrzyżowania, gdzie spotykają się łuki z różnych ogniw, nazywamy \textbf{indeksem zaczepienia} węzłów $K_1, K_2$.
    Ogólniej, jeśli $L = K_1 \sqcup \ldots \sqcup K_n$ jest splotem o $n$ ogniwach, to jego indeks zaczepienia wyznacza wzór $\operatorname{lk}(L) = \sum_{i < j} \operatorname{lk}(K_i, K_j)$.
\end{definition}

Zauważmy, że indeks zaczepienia splotu Hopfa wynosi $1$, natomiast splotu Whiteheada $0$.
Są zatem istotnie różne.
W obydwu przypadkach indeks zaczepienia jest liczbą całkowitą, nie stanowi to przypadku.
Na mocy twierdzenia Jordana $\operatorname{lk}$ jest funkcją o całkowitych wartościach.

\begin{proposition}
    Indeks zaczepienia jest dobrze określonym niezmiennikiem zorientowanych splotów.
\end{proposition}

\begin{proof}
    Sprawdźmy wpływ ruchów Reidemeistera na wartość
    $\operatorname{lk}(L)$:
\begin{comment}
    \[
        \fbox{
        \begin{tikzpicture}[baseline=-0.65ex,scale=0.07]
        \begin{knot}[clip width=5]
        \strand[semithick] (-10,10) .. controls (-10,2) and (-10,2) .. (-6,-2);
        \strand[semithick] (-10,-10) .. controls (-10,-2) and (-10,-1) .. (-9,0);
        \strand[semithick] (-7,1) -- (-6,2);
        \strand[semithick] (-6,2) .. controls (2,9) and (2,-9) .. (-6,-2);
        \end{knot}
        \end{tikzpicture}
        $\stackrel{R_1}{\cong} \,\,$
        \begin{tikzpicture}[baseline=-0.65ex,scale=0.07]
        \begin{knot}[clip width=5]
        \strand[semithick] (0,10) -- (0,-10);
        \end{knot}
        \end{tikzpicture}}
        %%%
        \quad \fbox{
        \begin{tikzpicture}[baseline=-0.65ex,scale=0.07]
        \begin{knot}[clip width=5]
        \strand[semithick] (4,-10) .. controls (4,-4) and (-4,-4) .. (-4,0);
        \strand[semithick] (4,10) .. controls (4, 4) and (-4, 4) .. (-4,0);
        \strand[semithick] (-4,-10) .. controls (-4,-4) and (4,-4) .. (4,0);
        \strand[semithick] (-4,10) .. controls (-4, 4) and (4,4) .. (4,0);
        \node[blue] at (-4,4)[left] {$a$};
        \node[blue] at (-4,-4)[left] {$-a$};
        \end{knot}
        \end{tikzpicture}
        $\stackrel{R_2}{\cong} \,\,$
        \begin{tikzpicture}[baseline=-0.65ex,scale=0.07]
        \begin{knot}[clip width=5]
        \strand[semithick] (4,-10) .. controls (4,-4) and (1,-4) .. (1,0);
        \strand[semithick] (4,10) .. controls (4, 4) and (1, 4) .. (1,0);
        \strand[semithick] (-4,-10) .. controls (-4,-4) and (-1,-4) .. (-1,0);
        \strand[semithick] (-4,10) .. controls (-4, 4) and (-1,4) .. (-1,0);
        \end{knot}
        \end{tikzpicture}}
        %%%
        \quad \fbox{
        \begin{tikzpicture}[baseline=-0.65ex,scale=0.07]
        \begin{knot}[clip width=5, flip crossing/.list={1,2,3}]
        \strand[semithick] (-10,-10) -- (10,10);
        \strand[semithick] (-10,10) -- (10,-10);
        \strand[semithick] (-10,-2) .. controls (-4, -2) and (-4,8) .. (0,8);
        \strand[semithick] (10,-2) .. controls (4, -2) and (4,8) .. (0,8);
        \node[blue] at (-6,4)[left] {$a$};
        \node[blue] at (6,4)[right] {$b$};
        \node[blue] at (0,-2)[below] {$c$};
        \end{knot}
        \end{tikzpicture}
        $\stackrel{R_3}{\cong} \,\,$
        \begin{tikzpicture}[baseline=-0.65ex,scale=0.07]
        \begin{knot}[clip width=5, flip crossing/.list={1,2,3}]
        \strand[semithick] (-10,-10) -- (10,10);
        \strand[semithick] (-10,10) -- (10,-10);
        \strand[semithick] (-10,2) .. controls (-4, 2) and (-4,-8) .. (0,-8);
        \strand[semithick] (10,2) .. controls (4, 2) and (4,-8) .. (0,-8);
        \node[blue] at (-6,-4)[left] {$a$};
        \node[blue] at (6,-4)[right] {$b$};
        \node[blue] at (0,2)[above] {$c$};
        \end{knot}
        \end{tikzpicture}}
    \]
\end{comment}
    Na mocy twierdzenia Reidemeistera dowód został zakończony.
\end{proof}


\subsection{Spin} % (fold)
\label{sub:writhe}
Przypomnijmy, że znak skrzyżowania na diagramie to liczba $1$ lub $-1$ (definicja \ref{sign_def}).

\begin{definition}[spin]
    \index{spin}
    Wielkość
    \begin{equation}
        w(D) = \sum_c \operatorname{sign} c,
    \end{equation}
    gdzie sumowanie przebiega po wszystkich skrzyżowaniach diagramu $D$ zorientowanego splotu lub węzła, nazywamy spinem.
    Z angielskiego \emph{writhe}.
\end{definition}

Co ważne, spin nie jest niezmiennikiem splotów ani węzłów.
Para Perko przedstawia ten sam węzeł z~minimalną liczbą skrzyżowań i~spinem równym siedem lub dziewięć.
Dzięki temu przez wiele lat nie została dostrzeżona.
Spin jest za to niezmiennikiem węzłów alternujących, mówi o~tym druga hipoteza Taita.

\begin{lemma}
    \label{writhe_not_invariant}
    Spin nie zależy od orientacji.
    Tylko I ruch Reidemeistera zmienia spin: $w(\MalyreidemeisterIa) = w(\MalyreidemeisterIb )-1$, pozostałe ruchy nie mają na niego wpływu.
\end{lemma}

% Koniec sekcji Spin


\subsection{Liczba patykowa} % (fold)
\label{sub:stick_index}
\index{liczba!patykowa}
Z angielskiego \emph{stick number}.

\begin{definition}
	Minimalną liczbę odcinków w~łamanej, która przedstawia węzeł $K$, nazywamy jego liczbą patykową i~oznaczamy $\operatorname{s}(K)$.
\end{definition}

Wielkość tę wprowadził do matematyki Randell w 1988 i~znalazł dokładną jej wartość dla niewęzła (3), trójlistnika (6) oraz ósemki (7).
Negami trzy lata później w~\cite{negami91} pokazał przy użyciu teorii grafów, że dla nietrywialnych węzłów prawdziwe są nierówności
\begin{equation}
    \frac{5+\sqrt{9 + 8 \operatorname{cr} K}}{2} \le \operatorname{s} K \le 2 \operatorname{cr} K.
\end{equation}

Trójlistnik to jedyny węzeł realizujący górne ograniczenie.
Z~pracy Elrifaia wynika, że dolne ograniczenie nie jest osiągane przez żaden węzeł o co najwyżej 26 skrzyżowaniach (\cite{elrifai06}).

Jin oraz Kim w 1993 ograniczyli liczby patykowe dla węzłów torusowych korzystając z~liczby supermostowej.
Wkrótce wynik został poprawiony przez samego Jina, w pracy \cite{jin97} znalazł dokładne wartości dla niektórych węzłów.
I~tak, jeśli $2 \le p < q \le 2p$, to $\operatorname{s} T_{p,q} = 2q$ oraz $\operatorname{s} T_{p, p-1} = 2$.
Ten sam wynik, choć dla węższego zakresu parametrów, odkryto w~\cite{greilsheimer97}.
Autorzy niezależnie od siebie znaleźli proste oszacowanie z góry dla liczby patykowej sumy spójnej:
\begin{equation}
	\operatorname{s}(K_1 \shrap K_2) \le \operatorname{s}(K_1) + \operatorname{s}(K_2) - 3.
\end{equation}

Koniec dekady przyniósł jeszcze jedną pracę McCabe'a z nierównością $\operatorname{s}(K) \le 3 + \operatorname{cr} (K)$ dla węzłów dwumostowych (\cite{mccabe98}) oraz odkrycie Calvo: jeśli ograniczymy się do łamanych o co najwyżej siedmiu odcinkach, ósemka przestaje być odwracalna.

Na początku XX wieku nierówności Negamiego poprawiono, z dołu dokonał tego Calvo w~\cite{calvo01}, z góry natomiast Huh, Oh w \cite{huh11}.
Górne ograniczenie można poprawić o $3/2$, jeżeli $K$ jest niealternującym węzłem pierwszym.
\begin{equation}
    \frac{7+\sqrt{1 + 8 \operatorname{cr} K}}{2} \le \operatorname{s} K \le \frac{3}{2} (1 + \operatorname{cr} K).
\end{equation}

% Koniec podsekcji Liczba patykowa


\subsection{Długość sznurowa} % (fold)
\label{sub:ropelength}
Długość sznurowa, z~angielskiego \emph{ropelength}, pochodzi z~fizycznej teorii węzłów, która bierze pod uwagę obiekty wykonane z~nieelastycznych materiałów.

\begin{definition}
	Niech $L$ będzie splotem o długości $l$ oraz grubości $\tau$: posiada rurowe otoczenie bez samoprzecięć z~przekrojem poprzecznym o~promieniu $\tau$.
	Iloraz
	\begin{equation}	
		\operatorname{len} L = \frac l \tau
	\end{equation}
	nazywamy długością sznurową splotu.
\end{definition}

Przez wiele lat zastanawiano się, czy można zawiązać węzeł ze sznura o~długości jednej stopy i~promieniu jednego cala lub równoważnie, czy $\operatorname{len} K \le 12$ dla pewnego węzła $K$.
Na początku XXI wieku wiedzieliśmy z \cite{cantarella02}, że najkrótszy węzeł ma długość $10.726$, potem Diao udzielił negatywnej odpowiedzi na to pytanie w \cite{diao03}.
Wreszcie rozumowanie \cite{denne06} oparte o~czterosieczne pokazuje, że długość sznurowa nietrywialnego węzła wynosi co najmniej $15.66$.
Ponieważ eksperymenty komputerowe pokazują, że długość trójlistnika nie przekracza $16.372$, oszacowanie to jest więc dość ostre.

% Węzeł realizujący długość sznurową jest klasy $C^1,1$.

Prowadzono obszerne poszukiwania na temat zależności między długością sznurową i~innymi niezmiennikami.
Mamy na przykład:

\begin{proposition}
	$\operatorname{len} K = \Omega (\operatorname{cr}^{3/4} K)$.
\end{proposition}

Ograniczenie to realizowane jest przez pewne węzły torusowe oraz sploty Hopfa.

\begin{proposition}
	$\operatorname{len} K = O(\operatorname{cr} K \cdot \log^5(\operatorname{cr} K)).$
\end{proposition}

\begin{proof}
	Świeży wynik z \cite{diao19}, którego dowód wykorzystuje kraty liczbowe.
\end{proof}

Wcześniej znaliśmy słabszą równość $\operatorname{len} K = O(\operatorname{cr}^{3/2} K)$ dzięki cyklom Hamiltona w~grafach zanurzonych właśnie w~kratach liczbowych \cite{yu04}.

% Koniec podsekcji Ropelength


% Koniec sekcji Niezmienniki liczbowe
