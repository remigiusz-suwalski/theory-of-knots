\section{Niezmienniki liczbowe}
Jak wspomnieliśmy na początku rozdziału, sprawdzenie,
czy dwa diagramy przedstawiają sploty równoważne,
jest uciążliwym i~czasochłonnym zadaniem.
Aby je uprościć, podamy opis kilku prostych niezmienników o~całkowitych wartościach.
Zachodzą implikacje:
sploty równoważne $\Rightarrow$ ta sama wartość niezmiennika
oraz różne wartości niezmiennika $\Rightarrow$ różne sploty.

\subsection{Indeks skrzyżowaniowy} % (fold)
\label{sub:crossing_number}
\index{indeks!skrzyżowaniowy}
Z angielskiego \emph{crossing number}.

\begin{definition}
    Indeks skrzyżowaniowy $\operatorname{cr}(L)$ splotu $L$ to
    minimalna liczba skrzyżowań widocznych na diagramie,
    który przedstawia splot $L$.
\end{definition}

Jeśli $K_1, K_2$ są alternującymi węzłami o~$c_1, c_2$ skrzyżowaniach, to istnieje alternujący diagram ich sumy $K_1 \shrap K_2$ o~$c_1 + c_2$ skrzyżowaniach.
Lackenby w~pracy \cite{lackenby09} pokazał, że dla pewnej stałej $N \le 152$ zachodzi
\[
    \frac 1 N \sum_{i=1}^n \operatorname{cr} K_i \le \operatorname{cr} \left(\bigshrap_{i=1}^n K_i\right) \le \sum_{i=1}^n \operatorname{cr} K_i.
\]
(Tylko pierwsza nierówność jest ciekawa).
Jego argumentu (wykorzystującego powierzchnie normalne) nie można poprawić tak, by otrzymać stałą $N = 1$.
Wiadomo jednak, że indeks skrzyżowaniowy jest addytywny dla niektórych klas węzłów: alternujących (\cite{kauffman88}), adekwatnych\footnote{Węzły adekwatne nie pojawiają się na żadnej innej stronie tej książki.} czy torusowych (\cite{gruber03}).

\begin{theorem}[Bankwitz, 1930] \label{thm:bankwitz}
    Wyznacznik węzła alternującego jest nie mniejszy od jego indeksu skrzyżowaniowego.
\end{theorem}

\begin{proof}
    Pierwszy dowód podał Bankwitz w~pracy \cite{bankwitz30}.
    Inne rozumowanie przedstawił Crowell w~(łatwo dostępnym) artykule o~niespodziewanym tytule Nonalternating links z~1957 r.
\end{proof}

% Koniec podsekcji Indeks skrzyżowaniowy

\subsection{Liczba gordyjska} % (fold)
\label{sub:unknotting_number}
\index{liczba!gordyjska}
Z angielskiego \emph{unknotting number}.

\begin{definition}
    Liczba gordyjska $\operatorname{u}(K)$ węzła $K$ to minimalna liczba skrzyżowań,
    które trzeba odwrócić na pewnym jego diagramie, by dostać niewęzeł.
    Liczba $u$ nie musi być osiągana przez diagram o~minimalnej liczbie skrzyżowań.
\end{definition}

Dotychczas wyznaczono liczbę gordyjską dla prawie wszystkich węzłów pierwszych o~co najwyżej dziesięciu skrzyżowaniach,
Cha i~Livingston podają następującą listę wyjątków:
$10_{11}$, $10_{47}$, $10_{51}$, $10_{54}$, $10_{61}$, $10_{76}$, $10_{77}$, $10_{79}$, $10_{100}$ (stan na rok 2008).
S. Bleiler odkrył w~\cite{bleiler84} fascynujący przykład wymiernego węzła $2$-gordyjskiego,
czego świadkiem nie może być diagram o~minimalnej liczbie skrzyżowań
(ponieważ, co jeszcze bardziej fascynujące, węzeł ten posiada tylko jeden diagram o~dziesięciu skrzyżowaniach z~trzema do odwrócenia).
Koduje go liczba $29/6$, w~tabeli węzłów figuruje jako $10_8$.
Stojmenow w~\cite{stoimenow01} pokazuje, że jeden węzeł może mieć kilka diagramów minimalnych,
z których tylko niektóre są świadkiem $1$-gordyjskości (są to między innymi $14_{36750}$ oraz $14_{36760}$.)

Coward, Lackenby dowiedli w~\cite{coward11}, że jeśli $K$ jest 1-gordyjski i~o genusie 1, to z~dokładnością do pewnej relacji równoważności, tylko jedna zmiana skrzyżowania rozwiązuje go; chyba że $K$ jest ósemką -- wtedy takie zmiany są dwie.
Kanenobu, Murakami oraz Kohn dwadzieścia lat wcześniej wiedzieli, że wymierne sploty 1-gordyjskie posiadają rozwiązujące skrzyżowanie na minimalnym diagramie (\cite{kanenobu86}, \cite{kohn91}).

Jeśli odwrócenie pewnych skrzyżowań daje niewęzeł, to odwrócenie pozostałych także.
To daje proste, choć niezbyt pomocne oszacowanie liczby gordyjskiej: $2 \operatorname{u} (K) \le \operatorname{cr} (K)$.
Borodzik oraz Friedl podali niedawno całkiem mocne ograniczenia na liczbę gordyjską w~pracach \cite{borodzik14} i~\cite{borodzik15} opierając się o~parowanie Blanchfielda
(poprawiając ograniczenia z~sygnatury Levine'a-Tristrama, indeksu Nakanishiego oraz przeszkodą Lickorisha).
Dwadzieścia pięć węzłów o~co najwyżej dwunastu skrzyżowaniach ma liczbę gordyjską równą co najmniej trzy, co trudno pokazać innymi metodami.

Dla każdego nietrywialnego splotu istnieje diagram wymagający odwrócenia dowolnie wielu skrzyżowań (dowód zawiera praca \cite{taniyama09} Taniyamy).
Pokazany jest tam jeszcze jeden godny uwagi fakt.
Jeśli liczba gordyjska diagramu $D$ wynosi $\frac 12 (c(D) - 1)$,
co jest maksymalną możliwą wartością zgodnie z~naszym prostym ograniczeniem,
to węzeł jest $(2,p)$-torusowy albo wygląda jak diagram niewęzła po pierwszym ruchu Reidemeistera.

% Liczba gordyjska nietrywialnego węzła skręconego (definicja \ref{twist_knots}) to jeden, wystarczy bowiem rozwiązać jego splecione końce.
Patrz także stwierdzenie \ref{torus_unknotting}.

Suma dwóch węzłów $1$-gordyjskch jest $2$-gordyjska, pokazał to Scharleman.
Od początku teorii węzłów podejrzewano dużo więcej:

\begin{conjecture}
    Niech $K, J$ będą węzłami.
    Wtedy $u(K \shrap J) = u(K) + u(J)$, czyli liczba gordyjska jest addytywna.
\end{conjecture}

Z tej nieudowodnionej do dzisiaj (stan na 2018) hipotezy można wyciągnąć wniosek,
że jeśli do rozwiązania węzła wystarcza odwrócenie skrzyżowania, to jest pierwszy.
Podejrzewał to H. Wendt w~1937 roku,
kiedy policzył liczbę gordyjską węzła babskiego używając homologii rozgałęzionego nakrycia cyklicznego.

\begin{proposition}
    Węzły $1$-gordyjskie są pierwsze.
\end{proposition}

\begin{proof}[Niedowód]
    W pracy \cite{scharleman85} z~1985 roku M. Scharleman podał dość zawiłe uzasadnienie, w~które zamieszane były grafy planarne.
\end{proof}

W pracy \cite{shimizu14} Ayaka Shimizu pokazuje przykład innej operacji rozwiązującej węzły, ale nie wszystkie sploty:
zmianę skrzyżowań wokół obszaru na diagramie.

Liczbę gordyjską można uogólnić w naturalny sposób do metryki.
Mianowicie mając dane dwa węzły $K_0, K_1$, rozpatrzmy wszystkie homotopie $f : [0,1] \times S^1 \to \R^3$ takie, że wszystkie funkcje $f_t$ są zanurzeniami z co najwyżej jednym punktem podwójnym.
Zażądajmy dodatkowo, by styczne do krótkich łuków, które przecinają się w tym punkcie, były od siebie różne.
Odległością gordyjską między węzłami $K_0, K_1$ jest minimalna liczba podwójnych punktów, jakie posiada homotopia $f$.

Przestrzeń węzłów z~tą metryką bywa sprzeczna z~intuicją: twierdzenie C~z~pracy \cite{gambaudo05} głosi, że zawiera ona prawie idealną kopię przestrzeni euklidesowej dowolnego wymiaru.
Dokładniej:

\begin{proposition}
    Dla każdej liczby całkowitej $d \ge 1$ istnieje funkcja $\xi: \Z^d \to \mathcal{K}$, dodatnie stałe $A, B, C$ oraz norma $\|\cdot\|$ na przestrzeni $\R^d$ takie, że spełniona jest podwójna nierówność
    \[
        A\|x-y\|  - B \le d(\xi(x), \xi(y)) \le C\|x-y\|.
    \]
\end{proposition}

Dowód korzysta z grup warkoczowych, które poznamy w sekcji \ref{sec:braid}.

\begin{conjecture}[Bernhard 1994, Jablan 1998] \label{bernhard_jablan}
    Każdy węzeł $K$ posiada diagram $D$ realizujący liczbę gordyjską oraz skrzyżowanie, którego odwrócenie daje nowy węzeł $K'$ z diagramem $D'$ o mniejszej liczbie gordyjskiej: $u(D') < u(D)$.
\end{conjecture}

Gdyby hipoteza była prawdziwa, mielibyśmy dość prosty algorytm do wyznaczania liczby gordyjskiej $u(K)$: wystarczy skonstruować skończenie wiele diagramów minimalnych dla węzła $K$, odwracać kolejne skrzyżowania i szukać rekursywnie liczb gordyjskich nowych węzłów.
Hipoteza jest prawdziwa dla węzłów do jedenastu skrzyżowań (patrz baza danych KnotInfo C. Livingstona) i splotów o dwóch składowych do dziewięciu skrzyżowań, pokazał to Kohn w pracy \cite{kohn93} z 1993 (!) roku.
Brittenham, Hermiller twierdzą, że hipoteza jest fałszywa.

% Koniec podsekcji Liczba gordyjska

\subsection{Liczba mostowa} % (fold)
\label{sub:bridge_index}
\index{liczba!mostowa}
Z angielskiego \emph{bridge number}.
Wprowadzona w~1954 przez Schuberta.
\begin{definition}
    Liczba mostowa $\operatorname{br}(K)$ to minimalna liczba mostów:
    długich łuków, które biegną przez nadskrzyżowania.
\end{definition}

Można pokazać, że $n$-mostowe węzły rozkładają się na sumę dwóch trywialnych $n$-supłów.

\begin{proposition}
    Niech $K_1, K_2$ będą węzłami.
    Wtedy $\operatorname{br} (K_1) + \operatorname{br}(K_2) = \operatorname{br}(K_1 \# K_2) + 1$.
\end{proposition}

\begin{proof}[Nieedowód]
    Schubert pokazał to blisko pół wieku temu w~\cite{schubert54}.
    Nowszy dowód pochodzi od Schultensa, w~artykule \cite{schultens03} skorzystał z~foliacji na brzegu węzła towarzyszącego sateltarnemu.
    Dokładniejszy opis powyższych prac wykraczałby poza zakres tego opracowania, zostanie więc pominięty.
\end{proof}

Tylko jeden węzeł jest jednomostowy, to niewęzeł.
Kolejne w~hierarchii skomplikowania, czyli dwumostowe,
to domknięcia wymiernych supłów, patrz fakt \ref{prp:two_bridge_tangle}.
Węzły trójmostowe pozostają nie do końca zbadane.

\begin{conjecture}
    Jeśli $K$ jest węzłem, to $c(K) \ge 3 br(K) - 3$, przy czym równość zachodzi dokładnie dla niewęzła, trójlistnika i~sumy spójnej trójlistników (rozdział 4.3 z~podręcznika \cite{murasugi96}).
\end{conjecture}

Nie istnieje związek między liczbą mostową oraz gordyjską.
Po pierwsze, węzły torusowe $T_{2,n}$ są dwumostowe, a~ich liczba gordyjska jest duża.
Po drugie, podwojenie węzła (poza specjalnymi przypadkami, jak pokazał Schubert) zwiększa liczbę mostową dwukrotnie; liczba gordyjska takiego podwojenia wynosi $1$.
Podobnie nie ma zależności między liczbą mostową oraz genusem.

% Koniec podsekcji Liczba mostowa

\subsection{Liczba warkoczowa} % (fold)
\label{sub:braid_number}
\index{liczba!warkoczowa}
Z angielskiego \emph{braid number}.
Do jej określenia potrzebna jest definicja \ref{braid_def}.

\begin{definition}
    Liczba warkoczowa to minimalna liczba pasm, na których można zbudować warkocz, którego domknięciem jest wyjściowy węzeł.
\end{definition}

\begin{proposition}
    Węzeł o~$n$ skrzyżowaniach można zapleść na $n - 1$ pasmach.
\end{proposition}

Indeks warkoczowy zależy od orientacji ogniw i~trudno się go wyznacza w~ogólnym przypadku.
Poniższy dowód zakłada znajomość warkoczy (rozdział piąty).

\begin{proposition}
    Indeksem warkoczowym węzła torusowego $K(q, r)$, $rq \neq 0$, jest $\min\{|q|, |r|\}$.
\end{proposition}

\begin{proof}
    Niech $K$ będzie węzłem torusowym typu $(q,r)$ z~minimalnym przedstawieniem jako warkocz $\beta$.
    Z konstrukcji domknięcia (czyli dołączenia rozłącznych półokręgów) wynika,
    że diagram $K$ ma dokładnie $b(K)$ lokalnych maksimów.
    Definicja indeksu mostowego orzeka, iż $br(K) \le b(K)$.
    Bez straty ogólności niech $q > r > 0$.
    Skoro węzeł $K$ powstaje z~$r$-warkocza $(\sigma_{r-1} \ldots \sigma_2\sigma_1)^q$,
    indeks $b(K)$ nie przekracza $r = br(K)$.

    Skorzystaliśmy z~faktu \ref{torus_bridge}.
\end{proof}

% Koniec podsekcji Liczba warkoczowa

\subsection{Indeks zaczepienia} % (fold)
\label{sub:linking_number}
\begin{definition} \label{sign_def}
    Na diagramie zorientowanego splotu, każdemu skrzyżowaniu przypisujemy \textbf{znak} równy $\pm 1$.
    Skrzyżowania dodatnie nazywamy praworęcznymi, ujemne zaś: leworęcznymi.
    \[
        \sign \Bigl(\,\,\begin{tikzpicture}[baseline=-0.65ex,scale=0.07]
        \begin{knot}[clip width=5]
        \strand[semithick,-Latex] (-5,-5) -- (5,5);
        \strand[semithick,Latex-] (-5,5) -- (5,-5);
        \end{knot}\end{tikzpicture}\,\,\Bigr) = +1 \quad
        \sign \Bigl(\,\,\begin{tikzpicture}[baseline=-0.65ex,scale=0.07]
        \begin{knot}[clip width=5, flip crossing/.list={1}]
        \strand[semithick,-Latex] (-5,-5) -- (5,5);
        \strand[semithick,Latex-] (-5,5) -- (5,-5);
        \end{knot}\end{tikzpicture}\,\,\Bigr) = -1
    \]
\end{definition}

\begin{definition} \label{sign_def}
    \index{indeks!zaczepienia}
    Niech $L = K_1 \sqcup K_2$ będzie splotem o dwóch ogniwach.
    Wielkość
    \[
        \operatorname{lk}(K_1, K_2) = \frac 12 \sum_i \sign c_i,
    \]
    gdzie sumowanie rozciąga się na wszystkie skrzyżowania, gdzie spotykają się łuki z różnych ogniw, nazywamy \textbf{indeksem zaczepienia} węzłów $K_1, K_2$.
    Ogólniej, jeśli $L = K_1 \sqcup \ldots \sqcup K_n$ jest splotem o $n$ ogniwach, to jego indeks zaczepienia wyznacza wzór $\operatorname{lk}(L) = \sum_{i < j} \operatorname{lk}(K_i, K_j)$.
\end{definition}

Zauważmy, że indeks zaczepienia splotu Hopfa wynosi $1$, natomiast splotu Whiteheada $0$.
Są zatem istotnie różne.
W obydwu przypadkach indeks zaczepienia jest liczbą całkowitą, nie stanowi to przypadku.
Na mocy twierdzenia Jordana $\operatorname{lk}$ jest funkcją o całkowitych wartościach.

\begin{proposition}
    Indeks zaczepienia jest dobrze określonym niezmiennikiem zorientowanych splotów.
\end{proposition}

\begin{proof}
    Sprawdźmy wpływ ruchów Reidemeistera na wartość
    $\operatorname{lk}(L)$:

    \[
        \fbox{
        \begin{tikzpicture}[baseline=-0.65ex,scale=0.07]
        \begin{knot}[clip width=5]
        \strand[semithick] (-10,10) .. controls (-10,2) and (-10,2) .. (-6,-2);
        \strand[semithick] (-10,-10) .. controls (-10,-2) and (-10,-1) .. (-9,0);
        \strand[semithick] (-7,1) -- (-6,2);
        \strand[semithick] (-6,2) .. controls (2,9) and (2,-9) .. (-6,-2);
        \end{knot}
        \end{tikzpicture}
        $\stackrel{R_1}{\cong} \,\,$
        \begin{tikzpicture}[baseline=-0.65ex,scale=0.07]
        \begin{knot}[clip width=5]
        \strand[semithick] (0,10) -- (0,-10);
        \end{knot}
        \end{tikzpicture}}
        %%%
        \quad \fbox{
        \begin{tikzpicture}[baseline=-0.65ex,scale=0.07]
        \begin{knot}[clip width=5]
        \strand[semithick] (4,-10) .. controls (4,-4) and (-4,-4) .. (-4,0);
        \strand[semithick] (4,10) .. controls (4, 4) and (-4, 4) .. (-4,0);
        \strand[semithick] (-4,-10) .. controls (-4,-4) and (4,-4) .. (4,0);
        \strand[semithick] (-4,10) .. controls (-4, 4) and (4,4) .. (4,0);
        \node[blue] at (-4,4)[left] {$a$};
        \node[blue] at (-4,-4)[left] {$-a$};
        \end{knot}
        \end{tikzpicture}
        $\stackrel{R_2}{\cong} \,\,$
        \begin{tikzpicture}[baseline=-0.65ex,scale=0.07]
        \begin{knot}[clip width=5]
        \strand[semithick] (4,-10) .. controls (4,-4) and (1,-4) .. (1,0);
        \strand[semithick] (4,10) .. controls (4, 4) and (1, 4) .. (1,0);
        \strand[semithick] (-4,-10) .. controls (-4,-4) and (-1,-4) .. (-1,0);
        \strand[semithick] (-4,10) .. controls (-4, 4) and (-1,4) .. (-1,0);
        \end{knot}
        \end{tikzpicture}}
        %%%
        \quad \fbox{
        \begin{tikzpicture}[baseline=-0.65ex,scale=0.07]
        \begin{knot}[clip width=5, flip crossing/.list={1,2,3}]
        \strand[semithick] (-10,-10) -- (10,10);
        \strand[semithick] (-10,10) -- (10,-10);
        \strand[semithick] (-10,-2) .. controls (-4, -2) and (-4,8) .. (0,8);
        \strand[semithick] (10,-2) .. controls (4, -2) and (4,8) .. (0,8);
        \node[blue] at (-6,4)[left] {$a$};
        \node[blue] at (6,4)[right] {$b$};
        \node[blue] at (0,-2)[below] {$c$};
        \end{knot}
        \end{tikzpicture}
        $\stackrel{R_3}{\cong} \,\,$
        \begin{tikzpicture}[baseline=-0.65ex,scale=0.07]
        \begin{knot}[clip width=5, flip crossing/.list={1,2,3}]
        \strand[semithick] (-10,-10) -- (10,10);
        \strand[semithick] (-10,10) -- (10,-10);
        \strand[semithick] (-10,2) .. controls (-4, 2) and (-4,-8) .. (0,-8);
        \strand[semithick] (10,2) .. controls (4, 2) and (4,-8) .. (0,-8);
        \node[blue] at (-6,-4)[left] {$a$};
        \node[blue] at (6,-4)[right] {$b$};
        \node[blue] at (0,2)[above] {$c$};
        \end{knot}
        \end{tikzpicture}}
    \]
    Na mocy twierdzenia Reidemeistera dowód został zakończony.
\end{proof}

Kawauchi definiuje jeszcze \emph{twisting number}: sumę spinów po wszystkich składowych.

% Koniec podsekcji Indeks zaczepienia

\subsection{Sygnatura} % (fold)
\label{sub:signature}
\index{sygnatura}
Sygnatura pojawia się w~fakcie \ref{slice_signature}.

\begin{definition}
    Sygnatura to niezmiennik topologiczny zadany (kłębiastą) relacją rekurencyjną:
    \begin{itemize}[leftmargin=*]
    \itemsep0em
        \item $\sigma (\LittleUnknot) = 0$,
        \item $\sigma (K_+) - \sigma (K_-) \in \{0, 2\}$,
        \item $4 \mid \sigma (K)$ wtedy i~tylko wtedy, gdy $\nabla(2i) > 0$ (wielomian Conwaya).
    \end{itemize}
\end{definition}

\begin{proposition} \label{prop_sigma_inverse}
    Mamy $\sigma(K^*) = -\sigma(K)$ oraz $\sigma(-K) = \sigma(K)$.
\end{proposition}

\begin{proof}
    To jest twierdzenie 6.4.5 z podręcznika \cite{murasugi96}.
\end{proof}

\begin{proposition} \label{prop_sigma_additive}
    Sygnatura jest addytywna: $\sigma(K_1 \shrap \ldots \shrap K_n) = \sum_{k=1}^n \sigma(K_k)$.
\end{proposition}

Węzły achiralne mają zerową sygnaturę, zatem trójlistnik nie jest achiralny.
Z faktów \ref{prop_sigma_inverse} oraz \ref{prop_sigma_additive} wynika, że suma tak samo skręconych trójlistników nie jest achiralna ($\sigma = \pm 4$), natomiast węzeł prosty (suma różnie skręconych) ma zerową sygnaturę i jak można przekonać się ze standardowego diagramu, jest achiralny.

Sygnatura pozwala uzyskać proste oszacowanie liczby gordyjskiej od dołu:

\begin{proposition}
    Mamy $2 u(K) \ge |\sigma(K)|$.
\end{proposition}

\begin{proof}
    To jest twierdzenie 6.4.8 z podręcznika \cite{murasugi96}.
\end{proof}

Nie istnieje bezpośredni związek między sygnaturą i~liczbą mostową.
Węzeł torusowy $T_{2,n}$ jest dwumostowy, jego sygnatura wynosi $n - 1$.
Suma spójna węzłów prostych ma zerową sygnaturę i~nieograniczoną liczbę mostową.
Wynika to z~ogólniejszego faktu:


Czy istnieje węzeł o~sygnaturze $4$ i~wyznaczniku postaci $n = 4k + 1$ dla $k$ całkowitego dodatniego?
Stojmenow twierdzi, że jeśli tak jest, to wszystkie pierwsze dzielniki $n$ dają resztę $1$ z~dzielenia przez $24$ i~są większe od $2857$.

% Koniec podsekcji Sygnatura

\subsection{Liczba patykowa} % (fold)
\label{sub:stick_index}
\index{liczba!patykowa}
Z angielskiego \emph{stick number}.

\begin{definition}
	Minimalną liczbę odcinków w~łamanej, która przedstawia węzeł $K$, nazywamy jego liczbą patykową i~oznaczamy $\operatorname{s}(K)$.
\end{definition}

Wielkość tę wprowadził do matematyki Randell w 1988 i~znalazł dokładną jej wartość dla niewęzła (3), trójlistnika (6) oraz ósemki (7).
Negami trzy lata później w~\cite{negami91} pokazał przy użyciu teorii grafów, że dla nietrywialnych węzłów prawdziwe są nierówności
\begin{equation}
    \frac{5+\sqrt{9 + 8 \operatorname{cr} K}}{2} \le \operatorname{s} K \le 2 \operatorname{cr} K.
\end{equation}

Trójlistnik to jedyny węzeł realizujący górne ograniczenie.
Z~pracy Elrifaia wynika, że dolne ograniczenie nie jest osiągane przez żaden węzeł o co najwyżej 26 skrzyżowaniach (\cite{elifrai06}).

Jin oraz Kim w 1993 ograniczyli liczby patykowe dla węzłów torusowych korzystając z~liczby supermostowej.
Wkrótce wynik został poprawiony przez samego Jina, w pracy \cite{jin97} znalazł dokładne wartości dla niektórych węzłów.
I~tak, jeśli $2 \le p < q \le 2p$, to $\operatorname{s} T_{p,q} = 2q$ oraz $\operatorname{s} T_{p, p-1} = 2$.
Ten sam wynik, choć dla węższego zakresu parametrów, odkryto w~\cite{greilsheimer97}.
Autorzy niezależnie od siebie znaleźli proste oszacowanie z góry dla liczby patykowej sumy spójnej:
\begin{equation}
	\operatorname{s}(K_1 \shrap K_2) \le \operatorname{s}(K_1) + \operatorname{s}(K_2) - 3.
\end{equation}

Koniec dekady przyniósł jeszcze jedną pracę McCabe'a z nierównością $\operatorname{s}(K) \le 3 + \operatorname{cr} (K)$ dla węzłów dwumostowych (\cite{mccabe98}) oraz odkrycie Calvo: jeśli ograniczymy się do łamanych o co najwyżej siedmiu odcinkach, ósemka przestaje być odwracalna.

Na początku XX wieku nierówności Negamiego poprawiono, z dołu dokonał tego Calvo w~\cite{calvo01}, z góry natomiast Huh, Oh w \cite{huh11}.
Górne ograniczenie można poprawić o $3/2$, jeżeli $K$ jest niealternującym węzłem pierwszym.
\begin{equation}
    \frac{7+\sqrt{1 + 8 \operatorname{cr} K}}{2} \le \operatorname{s} K \le \frac{3}{2} (1 + \operatorname{cr} K).
\end{equation}

% Koniec podsekcji Liczba patykowa

\subsection{Długość sznurowa} % (fold)
\label{sub:ropelength}
Długość sznurowa, z~angielskiego \emph{ropelength}, pochodzi z~fizycznej teorii węzłów, która bierze pod uwagę obiekty wykonane z~nieelastycznych materiałów

Długość sznurowa $L$ to stosunek długości węzła do jego grubości $\tau$ (mówimy, że węzeł $K$ jest grubości $\tau$, jeśli ma otoczenie rurowe bez samoprzecięć z~przekrojem poprzecznym o~promieniu $\tau$).
Przez wiele lat zastanawiano się, czy można zawiązać węzeł ze sznura o~długości jednej stopy i~promieniu jednego cala.
Nie jest to możliwe: rozumowanie oparte o~czterosieczne pokazuje, że długość sznurowa nietrywialnego węzła wynosi co najmniej $15.66$ (dla trójlistnika jest to co najmniej $16.372$).

Węzeł realizujący długość sznurową jest klasy $C^1$.

Prawdziwe są oszacowania asymptotyczne:
\[
    L = \Omega (\operatorname{cr}^{3/4}),  \quad
    L = O(\operatorname{cr} \log^5 \operatorname{cr})
\]

% Koniec podsekcji Ropelength

% Koniec sekcji Niezmienniki liczbowe
