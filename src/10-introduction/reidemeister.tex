\section{Ruchy Reidemeistera}
\label{sec:reidemeister_moves}
W kombinatorycznej teorii węzłów diagramy są dużo ważniejsze od gładkich włożeń okręgu w~przestrzeń $\R^3$,
dlatego przytoczymy proste kryterium decydujące o~tym,
kiedy dwa diagramy przedstawiają jeden węzeł.
Najpierw zdefiniujmy trzy lokalne operacje na diagramach.

\begin{definition}
    \index{ruchy Reidemeistera}
    Trzy ruchy Reidemeistera, $R_1$, $R_2$, oraz $R_3$, to następujące deformacje diagramu:
    \[
        \underbrace{\begin{tikzpicture}[baseline=-0.65ex,scale=0.1]
        \begin{knot}[clip width=5]
            \strand[thick] (-5, 10) to [in=left, out=down] (2, -5);
            \strand[thick] (5, 0) to [in=right, out=down] (2, -5);
            \strand[thick] (5, 0) to [in=right, out=up] (2, 5);
            \strand[thick] (-5, -10) to [in=left, out=up] (2, 5);
        \end{knot}
        \end{tikzpicture}
        \, \cong \,
        \begin{tikzpicture}[baseline=-0.65ex,scale=0.1]
        \begin{knot}[clip width=5]
            \strand[thick] (0,10) to (0,-10);
        \end{knot}
        \end{tikzpicture}}_{R_1}
        %%%
        \quad \quad \quad
        \underbrace{\begin{tikzpicture}[baseline=-0.65ex,scale=0.1]
        \begin{knot}[clip width=5]
            \strand[thick] (-5, 10) to [in=up, out=down] (5, 0);
            \strand[thick] (-5, -10) to [in=down, out=up] (5, 0);
            \strand[thick] (5, 10) to [in=up, out=down] (-5, 0);
            \strand[thick] (5, -10) to [in=down, out=up] (-5, 0);
        \end{knot}
        \end{tikzpicture}
        \, \cong \,
        \begin{tikzpicture}[baseline=-0.65ex,scale=0.1]
        \begin{knot}[clip width=5]
            \strand[thick] (-5, 10) to [in=up, out=down] (-2, 0);
            \strand[thick] (-5, -10) to [in=down, out=up] (-2, 0);
            \strand[thick] (5, 10) to [in=up, out=down] (2, 0);
            \strand[thick] (5, -10) to [in=down, out=up] (2, 0);
        \end{knot}
        \end{tikzpicture}}_{R_2}
        %%%
        \quad \quad \quad
        \underbrace{\begin{tikzpicture}[baseline=-0.65ex,scale=0.1]
        \begin{knot}[clip width=5, flip crossing/.list={1,2,3}]
            \strand[thick] (-10, -10) -- (10, 10);
            \strand[thick] (-10, 10) -- (10, -10);
            \strand[thick] (-10, 0) to [in=left, out=right] (0, 10);
            \strand[thick] (10, 0) to [in=right, out=left] (0, 10);
        \end{knot}
        \end{tikzpicture}
        \, \cong \,
        \begin{tikzpicture}[baseline=-0.65ex,scale=0.1]
        \begin{knot}[clip width=5, flip crossing/.list={1,2,3}]
            \strand[thick] (-10, -10) -- (10, 10);
            \strand[thick] (-10, 10) -- (10, -10);
            \strand[thick] (-10, 0) to [in=left, out=right] (0, -10);
            \strand[thick] (10, 0) to [in=right, out=left] (0, -10);
        \end{knot}
        \end{tikzpicture}}_{R_3}
    \]
\end{definition}

Ruch $R_i$ operuje więc na $i$ łukach diagramu.
Reidemeister w~swojej pierwszej pracy przyjął inną kolejność,
jego drugi ruch jest naszym pierwszym.

\begin{theorem}[Reidemeister, 1927]
    \label{thm:reidemeister}
    Każdy splot posiada diagram.
    Dwa diagramy przedstawiają równoważne sploty,
    wtedy i~tylko wtedy gdy pierwszy można otrzymać z~drugiego
    wykonując skończenie wiele ruchów Reidemeistera
    oraz gładko deformując łuki bez zmiany biegu skrzyżowań.
\end{theorem}

\begin{proof}
    Szkielet dowodu można znaleźć w~książce Burdego i~Zieschanga.
    Kluczowe pomysły zawiera ,,Knots, links, braids and $3$-manifolds''
    Prasołowa i~Sosińskiego.
    Innym przystępnym źródłem jest podręcznik \cite{murasugi96} Murasugiego ,,Knot theory and its applications''.
\end{proof}

Twierdzenie Reidemeistera jest użytecznym narzędziem,
z którego będziemy korzystać podczas definiowania większości niezmienników,
obiektów, które pozwalają odróżniać od siebie węzły.
Rzadko stosuje się je do przechodzenia między dwoma diagramami.
Istotnie, Coward i~Lackenby udowodnili w~\cite{coward11},
że jeśli dwa diagramy o~$n$ skrzyżowaniach przedstawiają jeden węzeł, wystarcza
\[
    R(n) = 2^{2^{\ldots^{2^n}}}
\]
(gdzie piętrowa potęga ma $10^{1000000n}$ warstw) ruchów Reidemeistera, by przejść między nimi.
Jeśli jeden z~diagramów jest pozbawiony skrzyżowań, wystarcza $(236n)^{11}$ ruchów.
Zapewne lepsze ograniczenia istnieją, ale ich nie znamy.
Ważne jest to, że wielkość $R(n)$ jest skończona.

\begin{definition}
    \index{węzeł!zorientowany}
    Węzeł zorientowany to taki, w~którym wybrano kierunek, w~którym należy się po nim poruszać.
\end{definition}

Twierdzenie Reidemeistera pozostaje prawdziwe dla węzłów zorientowanych.

% koniec sekcji Ruchy Reidemeistera
