\subsection{Liczba gordyjska} % (fold)
\label{sub:unknotting_number}
\index{liczba!gordyjska}
Z angielskiego \emph{unknotting number}.

\begin{definition}
    Liczba gordyjska $\operatorname{u}(K)$ węzła $K$ to minimalna liczba skrzyżowań,
    które trzeba odwrócić na pewnym jego diagramie, by dostać niewęzeł.
    Liczba $u$ nie musi być osiągana przez diagram o~minimalnej liczbie skrzyżowań.
\end{definition}

Dotychczas wyznaczono liczbę gordyjską dla prawie wszystkich węzłów pierwszych o~co najwyżej dziesięciu skrzyżowaniach,
Cha i~Livingston podają następującą listę wyjątków:
$10_{11}$, $10_{47}$, $10_{51}$, $10_{54}$, $10_{61}$, $10_{76}$, $10_{77}$, $10_{79}$, $10_{100}$ (stan na rok 2008).
S. Bleiler odkrył w~\cite{bleiler84} fascynujący przykład wymiernego węzła $2$-gordyjskiego,
czego świadkiem nie może być diagram o~minimalnej liczbie skrzyżowań
(ponieważ, co jeszcze bardziej fascynujące, węzeł ten posiada tylko jeden diagram o~dziesięciu skrzyżowaniach z~trzema do odwrócenia).
Koduje go liczba $29/6$, w~tabeli węzłów figuruje jako $10_8$.
Stojmenow w~\cite{stoimenow01} pokazuje, że jeden węzeł może mieć kilka diagramów minimalnych,
z których tylko niektóre są świadkiem $1$-gordyjskości (są to między innymi $14_{36750}$ oraz $14_{36760}$.)

Coward, Lackenby dowiedli w~\cite{coward11}, że jeśli $K$ jest 1-gordyjski i~o genusie 1, to z~dokładnością do pewnej relacji równoważności, tylko jedna zmiana skrzyżowania rozwiązuje go; chyba że $K$ jest ósemką -- wtedy takie zmiany są dwie.
Kanenobu, Murakami oraz Kohn dwadzieścia lat wcześniej wiedzieli, że wymierne sploty 1-gordyjskie posiadają rozwiązujące skrzyżowanie na minimalnym diagramie (\cite{kanenobu86}, \cite{kohn91}).

Jeśli odwrócenie pewnych skrzyżowań daje niewęzeł, to odwrócenie pozostałych także.
To daje proste, choć niezbyt pomocne oszacowanie liczby gordyjskiej: $2 \operatorname{u} (K) \le \operatorname{cr} (K)$.
Borodzik oraz Friedl podali niedawno całkiem mocne ograniczenia na liczbę gordyjską w~pracach \cite{borodzik14} i~\cite{borodzik15} opierając się o~parowanie Blanchfielda
(poprawiając ograniczenia z~sygnatury Levine'a-Tristrama, indeksu Nakanishiego oraz przeszkodą Lickorisha).
Dwadzieścia pięć węzłów o~co najwyżej dwunastu skrzyżowaniach ma liczbę gordyjską równą co najmniej trzy, co trudno pokazać innymi metodami.

Dla każdego nietrywialnego splotu istnieje diagram wymagający odwrócenia dowolnie wielu skrzyżowań (dowód zawiera praca \cite{taniyama09} Taniyamy).
Pokazany jest tam jeszcze jeden godny uwagi fakt.
Jeśli liczba gordyjska diagramu $D$ wynosi $\frac 12 (c(D) - 1)$,
co jest maksymalną możliwą wartością zgodnie z~naszym prostym ograniczeniem,
to węzeł jest $(2,p)$-torusowy albo wygląda jak diagram niewęzła po pierwszym ruchu Reidemeistera.

% Liczba gordyjska nietrywialnego węzła skręconego (definicja \ref{twist_knots}) to jeden, wystarczy bowiem rozwiązać jego splecione końce.
Patrz także stwierdzenie \ref{torus_unknotting} o liczbie gordyjskiej węzłów torusowych.

\begin{proposition}
    \label{unknotting_one_prime}
    Węzły $1$-gordyjskie są pierwsze.
\end{proposition}

Podejrzewał to H. Wendt w~1937 roku, kiedy policzył liczbę gordyjską węzła babskiego używając homologii rozgałęzionego nakrycia cyklicznego.

\begin{proof}[Niedowód]
    W pracy \cite{scharleman85} z~1985 roku M. Scharleman podał dość zawiłe uzasadnienie, w~które zamieszane były grafy planarne.
    Obecnie znamy prostsze dowody, patrz \cite{lackenby97} albo \cite{zhang91}.
\end{proof}

Scharlemann pokazał w \cite[wniosek 1.6]{scharlemann98}, że liczba gordyjska jest podaddytywna, to znaczy zachodzi $u(K_1 \shrap K_2) \le u(K_1) + u(K_2)$.
Stąd oraz z faktu \ref{unknotting_one_prime} wynika, że suma dwóch $1$-gordyjskich węzłów jest $2$-gordyjska, ale od początku teorii węzłów podejrzewano dużo więcej:

\begin{conjecture}
    Niech $K, J$ będą węzłami.
    Wtedy $u(K \shrap J) = u(K) + u(J)$, czyli liczba gordyjska jest addytywna.
\end{conjecture}

W pracy \cite{shimizu14} Ayaka Shimizu rozpatruje różne operacje, które rozwiązują węzły lub sploty.
Na przykład zamiana pod- i nadskrzyżowań wokół obszaru na diagramie rozwiązuje węzły, ale nie sploty.
Kontrprzykładem jest splot Hopfa.

Liczbę gordyjską można uogólnić w naturalny sposób do metryki.
Mianowicie mając dane dwa węzły $K_0, K_1$, rozpatrzmy wszystkie homotopie $f : [0,1] \times S^1 \to \R^3$ takie, że wszystkie funkcje $f_t$ są zanurzeniami z co najwyżej jednym punktem podwójnym.
Zażądajmy dodatkowo, by styczne do krótkich łuków, które przecinają się w tym punkcie, były od siebie różne.
Odległością gordyjską między węzłami $K_0, K_1$ jest minimalna liczba podwójnych punktów, jakie posiada homotopia $f$.

Przestrzeń węzłów z~tą metryką bywa sprzeczna z~intuicją: twierdzenie C~z~pracy \cite{gambaudo05} głosi, że zawiera ona prawie idealną kopię przestrzeni euklidesowej dowolnego wymiaru.
Dokładniej:

\begin{proposition}
    Dla każdej liczby całkowitej $n \ge 1$ istnieje funkcja $\xi: \Z^n \to \mathcal{K}$, dodatnie stałe $A, B, C$ oraz norma $\|\cdot\|$ na przestrzeni $\R^n$ takie, że spełniona jest podwójna nierówność
    \[
        A\|x-y\|  - B \le d(\xi(x), \xi(y)) \le C\|x-y\|.
    \]
\end{proposition}

Dowód korzysta z grup warkoczowych, które poznamy w sekcji \ref{sec:braid}.

\begin{conjecture}[Bernhard 1994, Jablan 1998] \label{bernhard_jablan}
    Każdy węzeł $K$ posiada diagram $D$ realizujący liczbę gordyjską oraz skrzyżowanie, którego odwrócenie daje nowy węzeł $K'$ z diagramem $D'$ o mniejszej liczbie gordyjskiej: $u(D') < u(D)$.
\end{conjecture}

Gdyby hipoteza była prawdziwa, mielibyśmy dość prosty algorytm do wyznaczania liczby gordyjskiej $u(K)$: wystarczy skonstruować skończenie wiele diagramów minimalnych dla węzła $K$, odwracać kolejne skrzyżowania i szukać rekursywnie liczb gordyjskich nowych węzłów.
Hipoteza jest prawdziwa dla węzłów do jedenastu skrzyżowań (patrz baza danych KnotInfo C. Livingstona) i splotów o dwóch składowych do dziewięciu skrzyżowań, pokazał to Kohn w pracy \cite{kohn93} z 1993 (!) roku.

Brittenham, Hermiller twierdzą, że hipoteza jest fałszywa.

% Koniec podsekcji Liczba gordyjska