\section{Grupa} % (fold)
\label{sec:section_name}
\index{grupa!kolorująca}
\begin{definition} \label{colgrp_def}
    Niech $L$ będzie splotem.
    Grupa kolorująca $L$, oznaczana $\operatorname{Col}(L)$, to abelowa grupa generowana przez łuki diagramu przedstawiającego $L$ z~równaniami skrzyżowań tegoż diagramu jako relacjami.
    Oprócz tego mamy jeszcze jedną relację, $a = 0$, gdzie $a$ jest ustalonym łukiem.
\end{definition}

\begin{proposition} \label{morphism_colour}
    Splot $L$ koloruje się modulo $n$, wtedy i~tylko wtedy gdy istnieje nietrywialny homomorfizm $\operatorname{Col}(L) \to \Z/n\Z$.
\end{proposition}

\begin{proof}
    Niech $a$ będzie ustalonym łukiem na diagramie splotu $L$ z definicji \ref{colgrp_def}.
    Funkcja $\varphi \colon \operatorname{Col}(L) \to \Z/n\Z$ jest nietrywialnym morfizmem, jeśli przyjmuje choć raz wartość różną od zera, spełnia warunek $\varphi(a) = 0$ i~dla każdego skrzyżowania prawdziwe jest równanie
    \begin{equation}
        \varphi(x_j) + \varphi(x_k) - 2\varphi(x_i) = 0
    \end{equation}
    przy oznaczeniach z definicji \ref{def:colour_equation}.
    To pokazuje, że niezerowe morfizmy $\operatorname{Col}(L) \to \Z/n\Z$ to dokładnie kolorowania modulo $n$, z kolorem $\varphi(l)$ na łuku $l$.
\end{proof}

\begin{corollary} \label{odd_determinant}
    Niech $K$ będzie węzłem.
    Wtedy $\det K$ jest liczbą nieparzystą.
\end{corollary}

\begin{proof}
    Bezpośredni wniosek z faktów \ref{no_knots_colours_mod_two} oraz \ref{morphism_colour}.
\end{proof}

\begin{proposition}
    Grupa kolorująca jest z dokładnością do izomorfizmu niezmiennikiem węzłów.
\end{proposition}

\begin{proof}[Szkic dowodu]
    Do wyznaczenia grupy kolorującej potrzebujemy diagramu $D$ z wybranym łukiem $a$.
    Musimy zatem sprawdzić, że ustalenie innego diagramu lub łuku prowadzi do grupy izomorficznej z wyjściową.
    Dla diagramów wystarczy sprawdzić, co dzieje się podczas ruchów Reidemeistera jak w dowodzie faktu \ref{color_invariant}.
    W przypadku łuku rozumowanie przebiega analogicznie do dowodu lematu \ref{lem:colouring_arc}.
\end{proof}

Oto metoda pozwalająca na znalezienie grupy kolorującej splotu $L$.
Wybierzmy diagram dla $L$ bez zamkniętych krzywych, etykietowanie $x_0, \ldots, x_m$ dla łuków oraz $0, \ldots, m$ dla skrzyżowań.
Utwórzmy macierz kolorującą $A$.
Grupy abelowe $\operatorname{Col}(L)$ oraz $\Z^m / A^t \Z^m$ są izomorficzne, wynika to bezpośrednio z~definicji macierzy $A$.
Następnie znajdźmy macierz diagonalną $D = \operatorname{diag}(d_1, \ldots, d_m)$, taką że $D = RAC$, gdzie całkowite macierze $R, C$ mają wyznacznik $1$.
Z~algebry liniowej wiemy, że to zawsze się uda: macierz $D$ nazywamy postacią normalną Smitha.
Wtedy funkcja
\begin{equation}
    f(x) \colon \Z^m/A^t \Z^m \to \Z^m / D\Z^m, \quad f(x) = C^t x
\end{equation}
stanowi izomorfizm, a~skoro $D$ jest macierzą diagonalną, to 
\begin{equation}
    \Z^m/D\Z^m \cong \bigoplus_{k=1}^m \Z/{|d_k|} \Z
\end{equation}
Macierz $A$ można zastąpić przez macierz Goeritza $G$.
Podsumujmy.

\begin{proposition}
    Niech $L$ będzie splotem z~macierzą kolorującą $A$, zaś $R, C$ macierzami o wyznaczniku $1$ tak, że macierz $D=RAC$ jest diagonalna, postaci $\operatorname{diag}(d_1, \ldots, d_n)$.
    Wtedy grupy $\operatorname{Col}(L)$ oraz $\Z/|d_1| \Z \times \ldots \times \Z/|d_n| \Z$ są izomorficzne.
\end{proposition}

\begin{corollary}
    Grupa kolorująca splotu jest skończona wtedy i tylko wtedy, gdy wyznacznik tego splotu jest niezerowy.
\end{corollary}

\begin{proof}
    Przy oznaczeniach z powyższego faktu, $\det(L) = |d_1| \cdot \ldots \cdot |d_n|$.
\end{proof}

\begin{corollary} \label{det_invariant}
    Wyznacznik jest niezmiennikiem splotów.
\end{corollary}

Grupa jest istotnie mocniejszym niezmiennikiem niż wyznacznik.
Na przykład węzły $6_1$ oraz $3_1 \shrap 3_1$ mają ten sam wyznacznik, $9$, ale różne grupy kolorujące: odpowiednio $\Z/9$ i~$(\Z/3)^2$.

Podamy teraz za S. Cyganem przykład nieskończonej rodziny węzłów,
której elementy rozróżnia właśnie wyznacznik.
Rozpatrzmy węzeł $K_k$ dla $k \ge 4$:
\[
\begin{tikzpicture}[baseline=-0.65ex, scale=0.1]
%\useasboundingbox (-5, -5) rectangle (5,5);
\begin{knot}[clip width=5, end tolerance=1pt, flip crossing/.list={1,2,3,6}]
    \strand[semithick] (-10, +3) .. controls (-4, +3) and (-4, -3) .. (0, -3);
    \strand[semithick] (-10, -3) .. controls (-4, -3) and (-4, +3) .. (0, +3);
    \node at (5, 0) {$\ldots$};
    \strand[semithick] (10+10, +3) .. controls (10+ 4, +3) and (10+ 4, -3) .. (10+0, -3);
    \strand[semithick] (10+ 10, -3) .. controls (10+ 4, -3) and (10+4, +3) .. (10+0, +3);
    \strand[semithick] (20+10, +3) .. controls (20+ 4, +3) and (20+ 4, -3) .. (20+0, -3);
\strand[semithick] (20+ 10, -3) .. controls (20+ 4, -3) and (20+4, +3) .. (20+0, +3);
    \strand[semithick] (30, 3) [in=up, out=right] to (35, -3);
    \strand[semithick,  ] (30, -3) [in=down, out=right] to (35, 3);
    \strand[semithick] (35, 3) [in=right, out=up] to (0, 10);
    \strand[semithick] (35, -3) [in=right, out=down] to (0, -10);
    \strand[semithick] (-10, -3) [in=down, out=left] to (-20, 0) to [in=left, out=up] (-10, 3);
    \strand[semithick] (-15, 5) [in=left, out=up] to (0, 10);
    \strand[semithick] (-15, 5) to (-15, -5) [in=left, out=down] to (0, -10);
    \node at (-5, -5) {$c_3$};
    \node at (15, -5) {$c_{k-2}$};
    \node at (25, -5) {$c_{k-1}$};
\end{knot}
\end{tikzpicture}\]

Niech $W_n$ będzie macierzą $n \times n$, na przekątnej której znajdują się $2$, zaś bezpośrednio nad i~pod nią -- wyrazy $-1$.
W macierzy kolorującej (ze skreślonym wierszem ,,$c_k$'' oraz kolumną ,,$a_k$'') zamieńmy miejscami dwie pierwsze kolumny, dodajmy do drugiej dwa razy pierwszą, zaś do drugiego wiersza -- dwa razy pierwszy wiersz.
Otrzymamy macierz
\[\begin{pmatrix}
    -1 & 0 & 0 \\
    0 & 3 & -1 \\
    0 & -1 & W_{k-3}
\end{pmatrix}\]

Powtarzając operacje: zamiana miejscami skrajnie lewych kolumn, dodanie do drugiej $2m+1$ razy pierwszej, odjęcie pierwszego wiersza od trzeciego, czyli sprowadzając naszą macierz do postaci normalnej Smitha przekonamy się, że na jej przekątnej znajdują się wyrazy $-1, -1, \ldots, -1, 2k-3$.
To oznacza, że $\det K_k = 2k-3$.

\begin{proposition}
    Niech $k \ge 9$ będzie takie, że wyznacznik węzła $K_k$ jest pewną potęgą $3$.
    Wtedy $K_k$ nie jest splotem trójlistników.
\end{proposition}

\begin{proof}
    Macierz diagonalna otrzymana po wniosku \ref{det_invariant} także jest niezmiennikiem węzłów.

    Macierzą dla splotu trójlistników $(3_1)^{\# n}$ trójlistników jest $\operatorname{diag}(1, \ldots, 1, 3, \ldots, 3)$, zaś dla węzła $K_k$: $\operatorname{diag} (1, \ldots, 1, 3^?)$.
\end{proof}
% koniec sekcji grupa
