\section{Macierz i wyznacznik} % (fold)
\label{sec:colour_matrix}
Zajmiemy się teraz wyznacznikiem, pierwszym nieoczywistym niezmiennikiem splotów, który przypisuje każdemu pewną liczbę całkowitą.
Jest on blisko związany z kolorowaniem.
Zauważmy, że pierwszy ruch Reidemeistera usuwa zamknięte krzywe, czyli pojedyncze łuki bez skrzyżowań.
Diagram bez takich krzywych ma tyle samo skrzyżowań, co łuków.

\begin{definition}[macierz kolorująca]
    Ustalmy diagram bez zamkniętych krzywych dla splotu $L$ z łukami $x_0, \ldots, x_m$ oraz skrzyżowaniami $0, \ldots, m$.
    Definiujemy macierz $A$, której wyraz $a_{lj}$ jest współczynnikiem przy $x_j$ w $l$-tym równaniu kolorującym: $x_j+x_k - 2x_i \equiv 0 \mod n$.
    Macierz kolorująca $A_+$ powstaje z macierzy $A$ przez skreślenie dowolnego wiersza i kolumny.
    \[\begin{tikzpicture}[baseline=-0.65ex, scale=0.12]
    \useasboundingbox (-5, -5) rectangle (5,5);
    \begin{knot}[clip width=5, end tolerance=1pt, flip crossing/.list={1}]
        \strand[semithick] (-5,5) to (5,-5);
        \strand[semithick] (-5,-5) to (5,5);
        \node[darkblue] at (5, 5)[below right] {$x_i$};
        \node[darkblue] at (5, -5)[above right] {$x_j$};
        \node[darkblue] at (-5, 5)[below left] {$x_k$};
    \end{knot}
    \end{tikzpicture}\]
\end{definition}

Taka macierz jest kwadratowa, ponieważ z każdego skrzyżowania wychodzą (tunelem) dwa włókna mające dwa końce.
Wykreślenie wiersza i kolumny jest konieczne.
Gdybyśmy tego zaniechali, otrzymana macierz nie byłaby odwracalna, bowiem wiersze sumują się do zera.
Dla alternujących diagramów możemy żądać, by górą $i$-tego skrzyżowania biegło $i$-te włókno, wtedy na diagonali macierzy $A$ znajdą się same minus dwójki.

\begin{definition}[wyznacznik]
    Wyznacznik splotu różnego od niewęzła to wyznacznik jego macierzy kolorującej bez znaku.
    Wyznacznikiem niewęzła jest $1$.
\end{definition}

\begin{definition}[defekt]
    Defekt (wymiar jądra) macierzy kolorującej modulo $p$ nazywamy defektem węzła.
\end{definition}

Defekty modulo różne liczby pierwsze są niezależne od siebie.
Na przykład suma spójna $k$ trójlistników i $j$ węzłów$T_{2,5}$ posiada defekt $k$ modulo $3$ oraz $j$ modulo $5$.
Podobne przykłady istnieją dla innych zbiorów liczb pierwszych.

Pokażemy później (po poznaniu grupy kolorującej, czyli we wniosku \ref{det_invariant}) lub jeszcze później (po wprowadzeniu wielomianu Alexandera, w dowodzie faktu \ref{alexander_invariance}), że wyznacznik splotu jest dobrze określony: nie zależy on od wyboru etykietowania, minora macierzy oraz diagramu i że jest niezmiennikiem.
Teraz ograniczymy się do jego kilku własności.

Defekt także jest niezmiennikiem, choć rzadziej używanym.
Węzeł o defekcie $n$ modulo $p$ posiada $p(p^n-1)$ kolorowań $p$ kolorami.
Węzły $8_{18}$ oraz $9_{24}$ mają ten sam wyznacznik, $45$.
Ich defekty modulo $3$ to $1$ i $2$, zatem są różne.

% \begin{proof}
%     \emph{Krok pierwszy}.
%     Pokażemy, że żaden ruch Reidemeistera nie zmienia wyznacznika.
%     \begin{enumerate}
%         \item \emph{Ruch $R_1$}. Diagram przed lub po ruchu zawiera co najmniej jedno włókno, które łączy tunel z mostem pewnego skrzyżowania.
%         \item \emph{Ruch $R_2$}.
%         \item \emph{Ruch $R_3$}.
%     \end{enumerate}

%     \emph{Krok drugi}.
%     Niech $A_{i,j}$ oznacza minor powstały przez skreślenie $i$-tego wiersza oraz $j$-tej kolumny.
%     Pokażemy, że wartość wyznacznika nie zależy od wyboru $i$ oraz $j$.

%     Niech $X$ będzie macierzą $k \times k$ złożoną z samych jedynek.
%     Suma elementów w każdej kolumnie oraz każdym wierszu macierzy $A + X$ wynosi $k$, ponieważ znaki równań zostały dobrze wybrane.
%     Wykonujemy kolejno operacje:
%     \begin{enumerate}
%         \item Dodajemy do $i$-tego wiersza sumę pozostałych.
%         \item Dodajemy do $j$-tej kolumny sumę pozostałych.
%         Teraz $i$-ty wiersz oraz $j$-ta kolumna zawierają wyrazy $k$ z wyjątkiem $a_{ij}$, który wynosi $k^2$.
%         \item Wyciągamy $k$ z $i$-tego wiersza przed wyznacznik.
%         \item Odejmujemy $i$-ty wiersz od pozostałych.
%     \end{enumerate}
%     Rozwinięcie Laplace'a względem $j$-tej kolumny mówi, że $|\det (A+X)| = k^2 |(-1)^{i+j} \det A_{i,j}|$, co kończy dowód drugiego kroku.

%     \emph{Krok trzeci}.
%     Pokażemy, że zmiana etykietowania nie zmienia wyznacznika.
% \end{proof}

\begin{proposition}
    Splot $L$ koloruje się modulo $n$ wtedy i tylko wtedy, gdy liczby $\det L$ oraz $n$ nie są względnie pierwsze.
\end{proposition}

\begin{proof}
    Bez straty ogólności ograniczmy się do tych kolorowań, gdzie $x_0 = 0$.
    Kolorowanie modulo $n$ istnieje dokładnie wtedy, gdy istnieje niezerowy wektor $(x_1, x_2, \ldots, x_m)$ taki, że $Ax \equiv 0 \mod n$.
    Ale z algebry liniowej wiemy, że dla pewnych całkowitoliczbowych macierzy $C, R$ macierz $RAC = diag(y_1, \ldots, y_m)$ jest diagonalna.
    Warunek z faktu tłumaczy się wtedy na istnienie rozwiązania dla przynajmniej jednego z równań $x_iy_i \equiv 0 \mod n$.
\end{proof}

Poniższy problem pochodzi od Stojmenowa.

\begin{conjecture}
    Niech $n$ będzie nieparzystą sumą dwóch kwadratów różną od $1, 9, 49$.
    Czy istnieje pierwszy, alternujący, achiralny węzeł o wyznaczniku $n$?
\end{conjecture}

Oto, co już wiemy.
Wyznacznik węzła achiralnego jest nieparzystą sumą dwóch kwadratów.
Implikacja odwrotna także jest prawdziwa, od węzła można dodatkowo żądać bycia pierwszym albo achiralnym (ale nie jednocześnie).
Jeśli węzeł z hipotezy istnieje, to $n > 2000$ nie jest kwadratem.

Istnieje jeszcze jedna kombinatoryczna metoda badania węzłów, która prowadzi między innymi do pojęcia wyznacznika.
W latach 30. ubiegłego wieku L. Goeritz pokazał, jak diagram węzła wyznacza specjalną formę kwadratowej.
Nieco później H. F. Trotter zmodyfikował jego pomysł, by sygnatura formy stanowiła niezmiennik splotów.
Gordon, Litherland ujednolicili dwa wyżej wymienione podejścia w pracy \cite{litherland81}.
My opiszemy krótko macierz Goeritza, gdyż jej wyznaczenie wymaga mniejszej ilości rachunków (niż macierz kolorująca).
Ustalmy diagram uszachowiony $D$ dla splotu $L$.
Oznaczmy białe regiony $0,1,\ldots,m$, przy czym $0$ jest regionem nieograniczonym.
Przydzielmy skrzyżowaniom znaki:
    \[\begin{tikzpicture}[baseline=-0.65ex, scale=0.12]
    \useasboundingbox (-5, -5) rectangle (5,5);
    \begin{knot}[clip width=5, end tolerance=1pt, flip crossing/.list={1}]
        \strand[semithick] (-5,5) to (5,-5);
        \strand[semithick] (-5,-5) to (5,5);
        \fill[blue!20!white] (-4, 5) to (0, 1) to (4, 5);
        \fill[blue!20!white] (-4, -5) to (0, -1) to (4, -5);
        \node[darkblue] at (-5, 0) {$+1$};
    \end{knot}
    \end{tikzpicture}
    \quad\quad\quad
    \quad\quad\quad
    \begin{tikzpicture}[baseline=-0.65ex, scale=0.12]
    \useasboundingbox (-5, -5) rectangle (5,5);
    \begin{knot}[clip width=5, end tolerance=1pt]
        \strand[semithick] (-5,5) to (5,-5);
        \strand[semithick] (-5,-5) to (5,5);
        \fill[blue!20!white] (-4, 5) to (0, 1) to (4, 5);
        \fill[blue!20!white] (-4, -5) to (0, -1) to (4, -5);
        \node[darkblue] at (-5, 0) {$-1$};
    \end{knot}
    \end{tikzpicture}\]

\begin{definition}
    Macierz Goeritza powstaje przez skreślenie z macierzy $G_+$ jednego wiersza oraz jednej kolumny:
    \[
        G_+=\begin{pmatrix}
        G_{00} & \cdots & G_{0m} \\
        \vdots & \ddots & \vdots \\
        G_{m0} & \cdots & G_{mm}
        \end{pmatrix},
    \]
    gdzie jeśli $i\neq j$, to $G_{ij}$ jest sumą znaków skrzyżowań przyległych do $i$ oraz $j$.
    Dla $i = j$, $G_{ii}$ jest minus sumą znaków skrzyżowań wokół $j$-tego obszaru.
\end{definition}

Macierz $G_+$ posiada dwie własności pozwalające wykryć proste błędy rachunkowe: jest symetryczna, a jej kolumny i wiersze sumują się do zera.

\begin{proposition}
    Macierz Goeritza oraz kolorująca mają ten sam wyznacznik (bez znaku).
\end{proposition}

Nie możemy niestety podać dowodu tego faktu, wymaga bowiem znajomości topologii algebraicznej, której wolelibyśmy nie zakładać.
Macierz Goeritza nie jest niezmiennikiem splotów.
Jeśli jednak równoważnym diagramom $D_1, D_2$ odpowiadają macierze $G_1, G_2$, to można między nimi przejść skończoną liczbą dwóch ruchów:
\begin{enumerate}[leftmargin=*]
\itemsep0em
    \item zamiany macierzy $G$ na $PGP^{-1}$, gdzie $P$ i $P^{-1}$ mają całkowite wyrazy
    \item dopisania lub skreślenia $\pm 1$ na przekątnej (dla węzłów) albo $-1, 0, 1$ (dla splotów).
\end{enumerate}

% \begin{proposition}
%     Wyznacznik jest niezmiennikiem splotów.
%     Jeśli diagramy $D_1, D_2$ przedstawiają ten sam splot, to od macierzy Goeritza $G_1$ pierwszego do macierzy $G_2$ drugiego można dojść, wykonując trzy rodzaje ruchów:
%     \begin{enumerate}
%         \item zamieniając $G$ z $P^t G P$, gdzie macierz $P$ jest całkowitoliczbowa i $\det P = \pm 1$.
%         \item zamieniając $G$ z
%         \[
%             \begin{pmatrix}
%             G & 0 \\
%             0 & k
%             \end{pmatrix},
%         \]
%         gdzie $k \in \{-1, 0, 1\}$.
%         Dla węzłów można ograniczyć się do $k = \pm 1$.
%     \end{enumerate}
% \end{proposition}

% Dowód opiera się na prostych rachunkach i ruchach Reidemeistera.

% Z macierzy Goeritza można otrzymać sygnaturę: $\sigma(G) - \mu$, gdzie $\mu$ to to suma znaków skrzyżowań o odpowiednio dobranej orientacji.

% Koniec sekcji Macierz i wyznacznik
