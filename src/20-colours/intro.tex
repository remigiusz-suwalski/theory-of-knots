Zamknęliśmy już pierwszy rozdział, ale wciąż nie pokazaliśmy, że istnieje przynajmniej jeden węzeł różny od niewęzła.
Temu między innymi poświęcony jest rozdział drugi.
Poznamy prosty niezmiennik, trójkolorowalność, do zdefiniowania którego nie trzeba odwoływać się nawet do pojęcia liczby.
Później zostanie on oczywiście odpowiednio wzmocniony.

Kolorowanie nie jest idealnym narzędziem klasyfikującym, istnieją węzły, których nie odróżnia.
Problem ten dotyka wielu późniejszych niezmienników, dość ważna hipoteza Jonesa \ref{jones_conjecture} pyta, czy wielomian Jonesa jest zupełny (tak nazywamy niezmienniki, które wykrywają każdą różnicę).
Pierwszy niezmiennik zupełny poznamy w rozdziale czwartym.
Nie stanowi to wielkiego powodu do radości ze względu na wysoką złożoność obliczeniową.

\section{Kolorowanie splotów} % (fold)
\label{sec:colour_links}
Nasz nowy niezmiennik, kolorowalność, opisuje, czy włókna diagramu dają się pomalować ograniczoną paletą kolorów.
Przed wyłożeniem właściwego materiału potrzebne nam będą dwie definicje.

\begin{definition}[włókno] 
	Fragment diagramu, który biegnie między dwoma kolejnymi tunelami (podskrzyżowaniami) nazywamy włóknem.
\end{definition}

\begin{definition}[rozszczepialność]
	Splot, który jest niespójną sumą niepustych splotów, nazywamy rozszczepialnym.
\end{definition}

Oto mniej mętny opis trójkolorowalności, czyli jak nietrudno się domyślić, kolorowalności trzema kolorami.
Diagram $D$ splotu $K$ jest trójkolorowalny, jeśli każdemu włóknu można przypisać jeden z trzech kolorów tak, by co najmniej dwa zostały użyte.
Wymagamy przy tym, by przy żadnym skrzyżowaniu nie spotykały się tylko dwa kolory.
Metoda ta została odkryta razem z uogólnieniem do $n$ kolorów przez Ralpha Foxa w 1956, kiedy próbował uczynić teorię węzłów bardziej przystępną dla studentów.

\begin{definition}[kolorowanie] \label{coleqn-definition}
	Niech $L$ będzie splotem, zaś $n$ liczbą naturalną.
	Mówimy, że splot $L$ jest kolorowalny modulo $n$, jeśli posiada diagram, którego włóknom można przypisać liczby całkowite $0, \ldots, n - 1$ tak, by równanie $a + b \equiv 2c$ modulo $n$ było spełnione przy każdym skrzyżowaniu oraz istniały dwa włókna różnych kolorów.
	Takie przyporządkowanie nazywamy (nietrywialnym) kolorowaniem.
	\[\begin{tikzpicture}[baseline=-0.65ex, scale=0.12]
	\useasboundingbox (-5, -5) rectangle (5,5);
	\begin{knot}[clip width=5, end tolerance=1pt, flip crossing/.list={1}] 
		\strand[semithick] (-5,5) to (5,-5);
		\strand[semithick] (-5,-5) to (5,5);
		\node[darkblue] at (5, 5)[below right] {$c$};
		\node[darkblue] at (5, -5)[above right] {$b$};
		\node[darkblue] at (-5, 5)[below left] {$a$};
	\end{knot}
	\end{tikzpicture}\]
\end{definition}

Opierając się jedynie na definicji kolorowania oraz ruchach Reidemeistera jesteśmy w stanie wykazać pierwsze własności niezmiennika.

\begin{proposition}
	Żaden węzeł nie koloruje się modulo dwa.
\end{proposition}

\begin{proof}
	Załóżmy nie wprost, że istnieje nietrywialne kolorowanie.
	Analiza czterech możliwych skrzyżowań pokazuje, 
	że włókna wychodzące z tunelu muszą mieć ten sam kolor.
	Przechodząc wzdłuż węzła widzimy jeden kolor, wbrew założeniu nie wprost.
\end{proof}

Oczywiście wszystkie sploty o wielu składowych kolorują się modulo dwa: wystarczy pomalować jedną składową zerem, a pozostałe jedynkami.
Powyższe rozumowanie dowodzi, że nie można pomalować żadnej składowej dwoma kolorami.
Co więcej, rozszczepialne sploty są kolorowalne modulo $n$ dla każdego $n \ge 2$: można skorzystać z tego samego schematu kolorowania.

\begin{proposition} \label{color_invariant}
	Kolorowalność modulo $n$ jest niezmiennikiem węzłów.
\end{proposition}

\begin{proof}
	Wystarczy sprawdzić, jak ruchy Reidemeistera zmieniają kolory.
	Pierwszy i drugi:
	\[
		\fbox{
		\begin{tikzpicture}[baseline=-0.65ex,scale=0.07]
		\begin{knot}[clip width=5] 
			\strand[semithick] (-10,10) .. controls (-10,2) and (-10,2) .. (-6,-2);
			\strand[semithick] (-10,-10) .. controls (-10,-2) and (-10,-1) .. (-9,0);

			\strand[semithick] (-7,1) -- (-6,2);
			\strand[semithick] (-6,2) .. controls (2,9) and (2,-9) .. (-6,-2);
			\node[darkblue] at (-10, 10)[below left] {$a$};
			\node[darkblue] at (-10, -10)[above left] {$b \equiv a$};
		\end{knot}
		\end{tikzpicture}
		$\stackrel{R_1}{\cong} \,\,$
		\begin{tikzpicture}[baseline=-0.65ex,scale=0.07]
		\begin{knot}[clip width=5] 
			\strand[semithick] (0,10) -- (0,-10);
			\node[darkblue] at (0, 0)[left] {$a$};
		\end{knot}
		\end{tikzpicture}}
		%%%
		\quad \fbox{
		\begin{tikzpicture}[baseline=-0.65ex,scale=0.07]
		\begin{knot}[clip width=5] 
			\strand[semithick] (4,-10) .. controls (4,-4) and (-4,-4) .. (-4,0);
			\node[darkblue] at (-4, -10)[above left] {$d \equiv b$};
			\strand[semithick] (4,10) .. controls (4, 4) and (-4, 4) .. (-4,0);
			\node[darkblue] at (4, 10)[below right] {$a$};
			\strand[semithick] (-4,-10) .. controls (-4,-4) and (4,-4) .. (4,0);
			\node[darkblue] at (4, 0) [right] {$c \equiv 2a-b$};
			\strand[semithick] (-4, 10) .. controls (-4, 4) and (4,4) .. (4,0);
			\node[darkblue] at (-4, 10) [below left] {$b$};
		\end{knot}
		\end{tikzpicture}
		$\stackrel{R_2}{\cong} \,\,$
		\begin{tikzpicture}[baseline=-0.65ex,scale=0.07]
		\begin{knot}[clip width=5] 
			\strand[semithick] (4,-10) .. controls (4,-4) and (1,-4) .. (1,0);
			\strand[semithick] (4,10) .. controls (4, 4) and (1, 4) .. (1,0);
			\strand[semithick] (-4,-10) .. controls (-4,-4) and (-1,-4) .. (-1,0);
			\strand[semithick] (-4,10) .. controls (-4, 4) and (-1,4) .. (-1,0);
		\end{knot}
		\end{tikzpicture}}
	\]
	Trzeci ruch także nie wymaga skomplikowanych rachunków.
	Najkrótszy łuk na diagramach ma kolor $2a-c$ po lewej oraz $2b-c$ po prawej stronie.
	\[
	 \fbox{
		\begin{tikzpicture}[baseline=-0.65ex,scale=0.07]
		\begin{knot}[clip width=5, flip crossing/.list={1,2,3}] 
			\node[darkblue] at (-10, 10) [above] {$b$};
			\node[darkblue] at (10, 10) [above] {$c$};
			\node[darkblue] at (-10, -10) [below] {$2a-2b+c$};
			\node[darkblue] at (10, -10) [below] {$2a-b$};
			\node[darkblue] at (-10, -2) [left] {$a$};
			\strand[semithick] (-10,-10) -- (10,10);
			\strand[semithick] (-10,10) -- (10,-10);
			\strand[semithick] (-10,-2) .. controls (-4, -2) and (-4,8) .. (0,8);
			\strand[semithick] (10,-2) .. controls (4, -2) and (4,8) .. (0,8);
		\end{knot}
		\end{tikzpicture}
		$\stackrel{R_3}{\cong} \,\,$
		\begin{tikzpicture}[baseline=-0.65ex,scale=0.07]
		\begin{knot}[clip width=5, flip crossing/.list={1,2,3}] 
			\node[darkblue] at (-10, 10) [above] {$b$};
			\node[darkblue] at (10, 10) [above] {$c$};
			\node[darkblue] at (-10, -10) [below] {$2a-2b+c$};
			\node[darkblue] at (10, -10) [below] {$2a-b$};
			\node[darkblue] at (10, 2) [right] {$a$};
			\strand[semithick] (-10,-10) -- (10,10);
			\strand[semithick] (-10,10) -- (10,-10);
			\strand[semithick] (-10,2) .. controls (-4, 2) and (-4,-8) .. (0,-8);
			\strand[semithick] (10,2) .. controls (4, 2) and (4,-8) .. (0,-8);
		\end{knot}
		\end{tikzpicture}}
	\]
\end{proof}

Trójlistnik koloruje się dokładnie modulo krotności trójk, ósemka zaś -- piątki.
Pierścienie Boromeuszy nie kolorują się modulo trzy, nie są zatem rozszczepialne.
Sama kolorowalność nie mówi wiele, splot jest kolorowalny lub nie.
Dowód faktu \ref{color_invariant} pokazuje coś więcej: liczba kolorowań, być może trywialnych, także jest niezmiennikiem.

Poniższy akapit można spokojnie pominąć przy pierwszym czytaniu.
Niech $L$ będzie splotem, zaś $\pi$ grupą podstawową jego dopełnienia.
Reprezentację $\rho$ dla $\pi$ na $D_{2n}$, grupę diedralną, nazywamy $n$-kolorowaniem Foxa.
Grupa splotu generowana jest przez ścieżki z punktu bazowego w $S^3$ do brzegu rurowego otoczenia splotu, wokół południka i znowu do bazowego punktu.
Obrazami tych generatorów są odbicia $n$-kąta foremne, które zapisujemy jako $ts^i$ ($t$ to odbicie, zaś $s$ jest najmniejszym obrotem). 
Każdy generator odpowiada pewnemu łukowi, przypiszmy mu liczbę $i \in \Z/n\Z$.
Dostaliśmy zwykłe kolorowanie.

Później (fakt \ref{morphism_colour}) pokażemy, że kolorowania to dokładnie homomorfizmy z pewnej grupy związanej z węzłem w $\Z/n$.

% Koniec sekcji Kolorowanie splotów
