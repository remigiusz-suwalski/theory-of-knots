\documentclass{createspace}
\newcommand{\N}{\mathbb N}
\newcommand{\Z}{\mathbb Z}
\newcommand{\Q}{\mathbb Q}
\newcommand{\R}{\mathbb R}
\newcommand{\C}{\mathbb C}

\newcommand{\shrap}{\mathbin{\#}}
\DeclareMathOperator*{\bigshrap}{\#}

\newcommand{\bracket}[1]{\left\langle{#1}\right\rangle}

\newcommand{\alexander}{\Delta}
\newcommand{\conway}{\nabla}
\newcommand{\jones}{V}

\newcommand{\bridge}{\operatorname{br}}
\newcommand{\crossing}{\operatorname{cr}}
\newcommand{\linking}{\operatorname{lk}}
\newcommand{\sign}{\operatorname{sgn}}
\newcommand{\volume}{\operatorname{vol}}
\newcommand{\writhe}{\operatorname{wr}}


\usepackage{enumitem}
\usepackage{booktabs}
\usepackage{longtable}
\usepackage[table]{xcolor}
\usepackage[colorinlistoftodos,prependcaption]{todonotes}
\usepackage{tikz}
\usetikzlibrary{arrows.meta}
\usetikzlibrary{decorations.markings}
\usetikzlibrary{decorations.pathreplacing}
\usetikzlibrary{knots}
\colorlet{darkblue}{blue!80!black}
% clean diagrams

\tikzset{
	->-/.style={decoration={markings, mark=at position .5 with {\arrow{>}}},postaction={decorate}},
	-<-/.style={decoration={markings, mark=at position .5 with {\arrow{<}}},postaction={decorate}},
	TIKZ_ARCH/.style ={
		draw=black,
		line join=miter,
		line cap=butt,
		miter limit=4.00,
		line width=0.2 mm
	},
}

\newcommand{\LittleUnknot} {\begin{tikzpicture}[baseline=-0.65ex, scale=0.02]
	\begin{knot}[clip width=5, end tolerance=1pt]
		\strand[semithick] (0,0) circle (5);
	\end{knot}
\end{tikzpicture}}

\newcommand{\Unknot} {\begin{tikzpicture}[baseline=-0.65ex, scale=0.04]
	\begin{knot}[clip width=5, end tolerance=1pt]
		\strand[semithick] (0,0) circle (5);
	\end{knot}
\end{tikzpicture}}

\newcommand{\LeftCrossing} {\begin{tikzpicture}[scale=0.03, baseline=-3]
	\begin{knot}[clip width=5, end tolerance=1pt]
		\strand[semithick] (-5,5) to (5,-5);
		\strand[semithick] (-5,-5) to (5,5);
	\end{knot}
\end{tikzpicture}}

\newcommand{\RightCrossing} {\begin{tikzpicture}[baseline=-0.65ex, scale=0.04]
	\useasboundingbox (-5, -5) rectangle (5,5);
	\begin{knot}[clip width=5, end tolerance=1pt, flip crossing/.list={1}]
		\strand[semithick] (-5,5) to (5,-5);
		\strand[semithick] (-5,-5) to (5,5);
	\end{knot}
\end{tikzpicture}}

\newcommand{\LittleLeftCrossing} {
	\begin{tikzpicture}[baseline=-0.65ex,scale=0.03]
	\begin{knot}[clip width=5, end tolerance=1pt]
		\strand[semithick] (-5, -5) to [out=up, in=up] (5, -5);
		\strand[semithick] (-5, 5) to [out=down, in=down] (5, 5);
	\end{knot}
	\end{tikzpicture}
}

\newcommand{\LittleRightCrossing} {\begin{tikzpicture}[baseline=-0.65ex, scale=0.03]
	\useasboundingbox (-5, -5) rectangle (5,5);
	\begin{knot}[clip width=5, end tolerance=1pt, flip crossing/.list={1}]
		\strand[semithick] (-5,5) to (5,-5);
		\strand[semithick] (-5,-5) to (5,5);
	\end{knot}
\end{tikzpicture}}

\newcommand{\LittleLeftSmoothing} {
	\begin{tikzpicture}[baseline=-0.65ex,scale=0.03]
	\begin{knot}[clip width=5, end tolerance=1pt]
		\strand[semithick] (-5, 5) [in=-135, out=-45] to (5,5);
		\strand[semithick] (-5, -5) [in=135, out=45] to (5,-5);
	\end{knot}
	\end{tikzpicture}
}

\newcommand{\LittleRightSmoothing} {
	\begin{tikzpicture}[baseline=-0.65ex,scale=0.03]
	\begin{knot}[clip width=5, end tolerance=1pt]
		\strand[semithick] (-4, -5) to [out=45, in=-45] (-4, 5);
		\strand[semithick] (4, -5) to [out=135, in=-135] (4, 5);
	\end{knot}
	\end{tikzpicture}
}

\newcommand{\LeftCrossSmoothing} {
	\begin{tikzpicture}[baseline=-0.65ex,yscale=0.07, xscale=0.1]
	\useasboundingbox (-5, -6) rectangle (5, 6);
	\begin{knot}[clip width=5, end tolerance=1pt]
		\strand[semithick] (-5, 5) [in=-135, out=-45] to (5,5);
		\strand[semithick] (-5, -5) [in=135, out=45] to (5,-5);
		\strand[semithick] (-5, 0) to (5, 0);
	\end{knot}
	\end{tikzpicture}
}

\newcommand{\RightCrossSmoothing} {
	\begin{tikzpicture}[baseline=-0.65ex,yscale=0.07, xscale=0.1]
	\useasboundingbox (-5, -6) rectangle (5, 6);
	\begin{knot}[clip width=5, end tolerance=1pt]
		\strand[semithick] (-4, -5) to [out=45, in=-45] (-4, 5);
		\strand[semithick] (4, -5) to [out=135, in=-135] (4, 5);
		\strand[semithick] (-5, 0) to (5, 0);
	\end{knot}
	\end{tikzpicture}
}

% dirty diagrams

\newcommand{\reidemeisterIa} {
\begin{tikzpicture}[baseline=-0.65ex, scale=0.07]
\useasboundingbox (-4, -5) rectangle (3, 5);
\begin{knot}[clip width=5, end tolerance=1pt]
	\strand[semithick] (-3,  5) [in=left, out=down] to (1, -2) [in=down, out=right] to (3, 0);
	\strand[semithick] (-3, -5) [in=left, out=up]   to (1,  2) [in=up,   out=right] to (3, 0);
\end{knot}
\end{tikzpicture}
}

\newcommand{\MalyreidemeisterIa} {\begin{tikzpicture}[baseline=-0.65ex, scale=0.03]
\useasboundingbox (-4, -5) rectangle (3, 5);
\begin{knot}[clip width=5, end tolerance=1pt]
	\strand[semithick] (-3,  5) [in=left, out=down] to (1, -2) [in=down, out=right] to (3, 0);
	\strand[semithick] (-3, -5) [in=left, out=up]   to (1,  2) [in=up,   out=right] to (3, 0);
	\end{knot}
	\end{tikzpicture}}

\newcommand{\MalyreidemeisterIb} {\begin{tikzpicture}[baseline=-0.65ex, scale=0.03]
	\begin{knot}[clip width=5, end tolerance=1pt]
		\strand[semithick] (0,-5) to (0, 5);
	\end{knot}
\end{tikzpicture}}

\newcommand{\reidemeisterIb} {
\begin{tikzpicture}[baseline=-0.65ex, scale=0.07]
\begin{knot}[clip width=5, end tolerance=1pt]
	\strand[semithick] (0,-5) to (0, 5);
\end{knot}
\end{tikzpicture}
}

% potrzebne do klamry Kauffmana
\newcommand{\reidemeisterIab} {
\begin{tikzpicture}[baseline=-0.65ex,scale=0.07]
\useasboundingbox (-5, -6) rectangle (5, 6);
\begin{knot}[clip width=5, end tolerance=1pt]
	\strand[semithick] (4,-5) .. controls (4,-3) and (-4,-3) .. (-4,-1);
	\strand[semithick] (-4,-5) .. controls (-4,-3) and (4,-3) .. (4,-1);
	\strand[semithick] (-4,-1) [in=left, out=up] to (0, 1) to [in=up, out=right] (4,-1);
	\strand[semithick] (-4, 5) [in=left, out=down] to (0, 3) to [in=down, out=right] (4, 5);
\end{knot}
\end{tikzpicture}
}

\newcommand{\reidemeisterIIa} {
\begin{tikzpicture}[baseline=-0.65ex,scale=0.07]
\useasboundingbox (-5, -6) rectangle (5, 6);
\begin{knot}[clip width=5, end tolerance=1pt]
	\strand[semithick] (4,-5) .. controls (4,-2) and (-4,-2) .. (-4,0);
	\strand[semithick] (4,5) .. controls (4, 2) and (-4, 2) .. (-4,0);
	\strand[semithick] (-4,-5) .. controls (-4,-2) and (4,-2) .. (4,0);
	\strand[semithick] (-4,5) .. controls (-4, 2) and (4,2) .. (4,0);
\end{knot}
\end{tikzpicture}
}

% reidemeister II a poziomo
\newcommand{\reidemeisterIIaa} {
\begin{tikzpicture}[baseline=-0.65ex,scale=0.05]
\begin{knot}[clip width=5, end tolerance=1pt]
	\strand[semithick] (-10, -5) to [out=right, in=left] ( 0,  5)
	                             to [out=right, in=left] (10, -5);
	\strand[semithick] (-10,  5) to [out=right, in=left] ( 0, -5)
	                             to [out=right, in=left] (10,  5);
\end{knot}
\end{tikzpicture}
}

\newcommand{\reidemeisterIIb} {
\begin{tikzpicture}[baseline=-0.65ex,scale=0.07]
\begin{knot}[clip width=5, end tolerance=1pt]
	\strand[semithick] (4,-5) .. controls (4,-2) and (1,-2) .. (1,0);
	\strand[semithick] (4,5) .. controls (4, 2) and (1, 2) .. (1,0);
	\strand[semithick] (-4,-5) .. controls (-4,-2) and (-1,-2) .. (-1,0);
	\strand[semithick] (-4,5) .. controls (-4, 2) and (-1,2) .. (-1,0);
\end{knot}
\end{tikzpicture}
}

\newcommand{\reidemeisterIIIa} {
\begin{tikzpicture}[baseline=-0.65ex,yscale=0.07, xscale=0.1]
\useasboundingbox (-5, -6) rectangle (5, 6);
\begin{knot}[clip width=5, flip crossing/.list={1,2,3}, end tolerance=1pt]
	\strand[semithick] (-5,-5) -- (5,5);
	\strand[semithick] (-5,5) -- (5,-5);
	\strand[semithick] (-5,-1) .. controls (-2, -1) and (-2,4) .. (0,4) .. controls (2, 4) and (2, -1) .. (5, -1);
\end{knot}
\end{tikzpicture}
}

\newcommand{\reidemeisterIIIb} {
\begin{tikzpicture}[baseline=-0.65ex,yscale=0.07, xscale=0.1]
\useasboundingbox (-5, -6) rectangle (5, 6);
\begin{knot}[clip width=5, flip crossing/.list={1,2,3}, end tolerance=1pt]
	\strand[semithick] (-5,-5) -- (5,5);
	\strand[semithick] (-5,5) -- (5,-5);
	\strand[semithick](-5,+1) .. controls (-2,+1) and (-2,-4) .. (0,-4) .. controls (2,-4) and (2,+1) .. (5,+1);
\end{knot}
\end{tikzpicture}
}

\newcommand{\skeinplus} {
\begin{tikzpicture}[baseline=-0.65ex,scale=0.1]
\useasboundingbox (-5, -6) rectangle (5, 6);
\begin{knot}[clip width=5, end tolerance=1pt]
	\strand[semithick,-latex] (-5, -5) to (5,  5);
	\strand[semithick,latex-] (-5,  5) to (5, -5);
	\node[darkblue] at (0,-4)[below] {$L_+$};
\end{knot}
\end{tikzpicture}
}

\newcommand{\skeinminus} {
\begin{tikzpicture}[baseline=-0.65ex,scale=0.1]
\useasboundingbox (-5, -6) rectangle (5, 6);
\begin{knot}[clip width=5, end tolerance=1pt, flip crossing/.list={1}]
	\strand[semithick,-latex] (-5, -5) to (5,  5);
	\strand[semithick,latex-] (-5,  5) to (5, -5);
	\node[darkblue] at (0,-4)[below] {$L_-$};
\end{knot}
\end{tikzpicture}
}

\newcommand{\skeinzero} {
\begin{tikzpicture}[baseline=-0.65ex,scale=0.1]
\useasboundingbox (-5, -6) rectangle (5, 6);
\begin{knot}[clip width=5, end tolerance=1pt, flip crossing/.list={1}]
	\strand[semithick,-latex] (-5, -5) to [out=45, in=-45] (-5, 5);
	\strand[semithick,-Latex] (5, -5) to [out=135, in=-135] (5, 5);
	\node[darkblue] at (0,-4)[below] {$L_0$};
\end{knot}
\end{tikzpicture}
}

\author{Leon Suwalski}
\title{Krótkie wprowadzenie do teorii węzłów (wersja robocza)}
\date{2018}

\begin{document}
\maketitle
\tableofcontents
\chapter{Preludium}
\label{cha:preludium}
Teoria węzłów to gałąź topologii,
która powstała z inspiracji węzłami,
jakie pojawiają się w~codziennym życiu: przy wiązaniu butów albo cumowaniu statków.
Zajmuje się ona badaniem przede wszystkim węzłów,
czyli pewnych włożeń okręgu $S^1$ w trójwymiarową przestrzeń euklidesową $\R^3$ lub sferę $S^3$,
ale także splotów (zaplątanych w sobie węzłów), warkoczy, supłów oraz podobnych obiektów.
Matematyczne węzły różnią się tym od zwykłych, że ich końce są ze sobą połączone.

Oto kilka przykładów.
Węzeł (a) nazywamy niewęzłem (jest to kalka angielskiego \emph{unknot}).
Następne w kolejce widoczne są trójlistnik (b,~\emph{trefoil}), ósemka (c,~\emph{figure-eight}), pięciolistnik (d,~\emph{cinquefoil}) oraz słynna para Perko (e,~f~wg oryginalnej numeracji Rolfsena).

\begin{figure}[H]
	\centering
	\begin{minipage}[b]{.14\linewidth}
		\centering
		$\begin{tikzpicture}[baseline=-0.65ex, scale=0.5] \begin{knot}[clip width=5, end tolerance=1pt] \strand[semithick] (0,0) circle (\linewidth); \end{knot}
\end{tikzpicture}$
		\subcaption{}
	\end{minipage}
	\begin{minipage}[b]{.14\linewidth}
		\centering
		\includegraphics[width=\linewidth]{../data/3_1.png}
		\subcaption{$3_1$}
	\end{minipage}
	\begin{minipage}[b]{.14\linewidth}
		\centering
		\includegraphics[width=\linewidth]{../data/4_1.png}
		\subcaption{$4_1$}
	\end{minipage}
	\begin{minipage}[b]{.14\linewidth}
		\centering
		\includegraphics[width=\linewidth]{../data/5_1.png}
		\subcaption{$5_1$}
	\end{minipage}
	\begin{minipage}[b]{.14\linewidth}
		\centering
		\includegraphics[width=\linewidth]{../data/perko1.png}
		\subcaption{$10_{161}$}
	\end{minipage}
	\begin{minipage}[b]{.14\linewidth}
		\centering
		\includegraphics[width=\linewidth]{../data/perko2.png}
		\subcaption{$10_{162}$}
	\end{minipage}
\end{figure}

Początkowo celem teorii węzłów była klasyfikacja wszystkich węzłów.
Od XIX wieku, kiedy teoria węzłów wyodrębniła się jako osobny dział matematyki,
zdążyliśmy skatalogować ponad sześć miliardów tych obiektów.
Pozornie tak samo wyglądające węzły mogą się od siebie różnić.
Do wykrywania tych subtelnych różnic używa się przede wszystkim niezmienników topologicznych takich jak grupy, wielomiany bądź liczby.
Poznamy je w dalszych rozdziałach.

Matematycy uogólnili pojęcie węzła:
można rozpatrywać je w wyższych wymiarach albo zastąpić okrąg inną przestrzenią topologiczną.
Będziemy starać się unikać tych uogólnień.

\section{Węzły i sploty}
Największą różnicą między węzłami matematycznymi oraz tymi z prawdziwego jest życia jest to, że te pierwsze nie mają luźnych końców.
Można przyjąć nieidealną, naiwną definicję:

\begin{definition}
	Ciągłe oraz różnowartościowe odwzorowanie $S^1 \to \R^3$ to \textbf{węzeł}.
\end{definition}

Zastanówmy się, jakim formalizmem opisać manipulowanie fizycznym sznurkiem.
Nie można użyć izotopii
(dwa węzły są izotopijne, jeśli istnieje ciągła funkcja $F \colon S^1 \times [0, 1] \to \R^3$ taka, że $F(-, 0)$ jest pierwszym, zaś $F(-,1)$ drugim węzłem).
Zauważmy, że każde splątanie ściaga się do punktu.
Dowolny węzeł jest równoważny z trywialnym, zatem istnieje tylko jedna klasa abstrakcji.
\begin{figure}[H]
	\begin{minipage}[b]{.23\linewidth}
		\centering
		\includegraphics[width=\linewidth]{../data/missing.jpg}
		\subcaption{węzeł}
	\end{minipage}
	\begin{minipage}[b]{.23\linewidth}
		\centering
		\includegraphics[width=\linewidth]{../data/missing.jpg}
		\subcaption{prostszy węzeł}
	\end{minipage}
	\begin{minipage}[b]{.23\linewidth}
		\centering
		\includegraphics[width=\linewidth]{../data/missing.jpg}
		\subcaption{niewęzeł}
	\end{minipage}
\end{figure}

Trzeba uwzględnić to, jak węzeł leży w przestrzeni.
Właściwym narzędziem jest więc izotopia otaczająca.
Intuicyjnie: dwa węzły uznajemy za równoważne,
jeśli można przejść od jednego do drugiego przy użyciu deformacji całej przestrzeni $\R^3$.

\begin{definition}[izotopia otaczająca] \label{def_ambient_isotopy}
	Ciągłe odwzorowanie $F \colon \R^3 \times [0,1] \to \R^3$,
	które staje się homeomorfizmem po ustaleniu drugiego argumentu i takie,
	że $F(-, 0)$ jest funkcją tożsamościową,
	zaś $F(-, 1)$ złożona z pierwszym węzłem daje drugi węzeł,
	nazywamy izotopią otaczającą.
\end{definition}

\begin{definition}[węzeł] \label{def:knot}
	Gładkie włożenie $S^1 \to \R^3$ izotopijne otaczająco z zamkniętą łamaną bez samoprzecięć nazywamy węzłem poskromionym.
\end{definition}

Wykluczamy w ten sposób patologiczne z kombinatorycznego punktu widzenia węzły dzikie.
Przez prawie całą książkę interesować nas będą jedynie węzły poskromione,
dlatego jeśli nie zaznaczono inaczej, przez węzeł rozumiemy węzeł poskromiony.
\todo{Wstawić przykład węzła dzikiego.}
Dość często będziemy utożsamiać węzeł z jego obrazem.

Jednocześnie homeomorfizmy $F(-,t)$ zastępujemy przez dyfeomorfizmy zachowujące orientację.
Chwila namysłu wystarcza do przekonania się, że definicja \ref{def_ambient_isotopy} obejmuje zwykłą izotopię,
a przy tym nie pozwala na rozwiązanie nietrywialnych węzłów przez ściągnięcie zaplątania do punktu.
Istnieje jeszcze jedna, konkurencyjna definicja węzłów równoważnych:

\begin{definition}
	Dwa węzły są równoważne, gdy jeden z nich jest obrazem drugiego przez zachowujący orientację homeomorfizm $\R^3 \to \R^3$.
\end{definition}

Stwierdzenie to przestaje być prawdziwe po zastąpieniu przestrzeni $\R^m$ przez $S^m$.

\begin{proof}
	Podany niżej dowód pochodzi z książki ,,Topology from the differentiable viewpoint'' Johna Millnora.
	Musimy pokazać, że dyfeomorfizm $f \colon \R^m \to \R^m$ jest gładko izotopijny z identycznością.
	Translacje są izotopiami, więc bez straty ogólności zakładamy, że $f(0) = 0$.
	Pochodna $f$ w zerze jest dana wzorem $\mathrm{d}f_0(x) = \lim_{t \to 0} f(tx) /t$,
	naturalną definicją	izotopii $F \colon \R^m \times [0, 1] \to \R^m$ jest więc
	\[
		F(x, t) = \begin{cases}
			f(tx) / t & 0 < t \le 1 \\
			\mathrm{d}f_0(x) & t = 0
		\end{cases} .
	\]

	Funkcja $F$ jest gładka,
	gdyż na mocy lematu Hadamarda funkcja $f$ zapisuje się jako suma $x_1 g_1(x) + \ldots + x_mg_m(x)$,
	gdzie funkcje $g_i$ są gładkie, co jakoś kończy dowód.
\end{proof}

\begin{definition}[splot] \label{def_link}
	Sumę rozłączną skończenie wielu węzłów nazywamy splotem.
\end{definition}

\begin{example}
	Whitehead w 1934 odkrył kontrprzykład do nieudanego dowodu hipotezy Poincarego.
	Był nim splot o dwóch składowych przedstawiony na poniższym rysunku.
	Splot Hopfa to najprostszy splot nietrywialny, którym w 1931 r. zajmował się Heinz Hopf,
	topolog niemiecki, w ramach badań nad tzw. rozwłóknieniem (Hopf fibration).

	\begin{figure}[H]
		\begin{minipage}[b]{.32\linewidth}
			\centering
			\includegraphics[width=\linewidth]{../data/missing.jpg}
			\subcaption{splot Whiteheada}
		\end{minipage}
		\begin{minipage}[b]{.32\linewidth}
			\centering
			\includegraphics[width=\linewidth]{../data/missing.jpg}
			\subcaption{splot Heada}
		\end{minipage}
		\begin{minipage}[b]{.32\linewidth}
			\centering
			\includegraphics[width=\linewidth]{../data/missing.jpg}
			\subcaption{splot jakiś inny}
		\end{minipage}
	\end{figure}
\end{example}

\begin{definition}[rozszczepialność]
	Splot, który jest niespójną sumą niepustych splotów, nazywamy rozszczepialnym.
\end{definition}

Jeśli dwa węzły są równoważne, to ich dopełnienia są oczywiście homeomorficzne.
Pytanie o prawdziwość implikacji odwrotnej jako pierwszy zadał najprawdopodobniej Tietze\footnote{Praca \emph{Über die topologischen Invarianten mehrdimensionaler Mannigfaltigkeiten}.} w 1908 roku.
Odpowiedź jest pozytywna: każdy węzeł jest wyznaczony jednoznacznie przez swoje dopełnienie.

\begin{theorem}[Gordon, Luecke, 1989] \label{thm_gordon_luecke}
	Poskromione węzły o homeomorficznych (z zachowaniem orientacji) dopełnieniach są wzajemnie izotopijne.
\end{theorem}

Wcześniej wiedziano tylko, że istnieją co najwyżej dwa węzły o zadanym dopełnieniu.

\begin{proof}[Niedowód]
	Wynika to z ogólniejszego stwierdzenia:
	nietrywialna chirurgia Dehna na węźle w~3-sferze nigdy nie daje 3-sfery.
	Pełny dowód zawiera praca \cite{gordon89}.
\end{proof}

Istnieje nieskończenie wiele (jakich?) kontrprzykładów do odpowiednika twierdzenia Gordona-Lueckego dla splotów,
których nie da się, patrząc na samo dopełnienie, odróżnić od splotu Whiteheada.

\section{Diagramy}
Chociaż w świetle definicji \ref{def:knot} węzły są pewnymi regularnymi podzbiorami przestrzeni $\R^3$,
z kombinatorycznego punktu widzenia wygodniej jest rysować je na  płaszczyźnie.

\begin{definition} [diagram] \label{def_diagrams}
	Cień to rzut węzła $K \subseteq \R^3$ na płaszczyznę.
	Diagram to cień bez katastrof
	(potrójnych przecięć, stycznych i dziobów)
	razem z informacją o tym, jak przebiegają skrzyżowania.
\end{definition}

\begin{definition} [włókno]
	Fragment diagramu, który biegnie między dwoma kolejnymi tunelami (podskrzyżowaniami) nazywamy włóknem.
\end{definition}

\begin{proposition}
	Każdy splot posiada diagram -- zbiór diagramów jest otwarty i gęsty w zbiorze wszystkich.
\end{proposition}

\begin{proof}
	Rzut splotu na równoległe płaszyczyzny jest taki sam,
	a te można sparametryzować prostymi przechodzącymi przez początek układu współrzędnych,
	które tworzą przestrzeń rzutową $\R \mathbb P^2$.
	Niech $S$ będzie zbiorem prostych, które dają złe rzuty.
	Wystarczy pokazać jego nigdziegęstość.
	Okazuje się, że $S$ jest też jednowymiarowy.
	(Dowód za \cite{crowell63}).
\end{proof}

\begin{definition}
	Diagram jest alternujący,
	gdy podczas poruszania się wzdłuż splotu
	mijamy jego skrzyżowania na zmianę z góry oraz z dołu.
	Splot jest alternujący, gdy posiada taki diagram.
\end{definition}

%Niestety pomimo upływu czasus, nikt nie napisał komputerowego programu realizującego ten algorytm (stan na 1994).
%Może podejmie się tego Czytelnik?
%Inne algorytmy istnieją, jednak wszystkie działają w wykładniczym czasie.

W 1961 roku W. Haken \cite{haken61} podał niezawodny przepis na wykrycie diagramu niewęzła,
częściowo rozwiązując jeden z ważniejszych problemów teorii węzłów.
Przez wiele lat nikt nie podjął się implementacji tego algorytmu,
udało się to na przykład B. Burtonowi, R. Budneyowi oraz W. Petterssonowi w komputerowym programie Regina
\footnote{Dostępny pod adresem \url{https://regina-normal.github.io/}.} na przełomie tysiącleci.
Burton, Rubinstein i Tillman pokazali w pracy \cite{burton12}, jak sprawdzać,
czy powierzchnia normalna na striangulowanej 3-rozmaitości jest (nie)ściśliwa w czasie wykładniczym.
To okazało się być wystarczającym do udzielenia negatywnej odpowiedzi na pytanie Thurstona:
,,czy przestrzeń Seiferta-Webera jest rozmaitością Hakena?'',
a zatem wykraczającego poza poziom tej pracy.
Patrz także {\url{http://geometrygames.org/SnapPea/index.html}.

Przykładami trudnych w rozpoznaniu niewęzłów są: niewęzeł Goritza, Freedmana.

\begin{figure}[H]
	\begin{minipage}[b]{.32\linewidth}
		\centering
		\includegraphics[width=\linewidth]{../data/missing.jpg}
		\subcaption{normalny}
	\end{minipage}
	\begin{minipage}[b]{.32\linewidth}
		\centering
		\includegraphics[width=\linewidth]{../data/missing.jpg}
		\subcaption{Goritza}
	\end{minipage}
	\begin{minipage}[b]{.32\linewidth}
		\centering
		\includegraphics[width=\linewidth]{../data/missing.jpg}
		\subcaption{Freedmana}
	\end{minipage}
\end{figure}

Więcej trudnych niewęzłów zawiera świeża praca
,,Algorithmic simplification of knot diagrams: new moves and experiments'
\footnote{\url{https://arxiv.org/pdf/1508.03226.pdf}}
C. Petronio, A. Zanellatiego.

Innym narzędziem wykrywającym niewęzły jest homologia Chowanowa (opisana później),
jak pokazał Kronheimer z Mrówką \cite{kronheimer11}.
Bar-Natan, topolog izraelski, napisał program liczący te homologie szybko \cite{barnatan07},
zapewne w czasie $O(\exp(c \sqrt n))$, dla diagramu o $n$ skrzyżowaniach.
Nie możemy liczyć na istotne przyspieszenie:
znalezienie przybliżenia wielomianu Jonesa jest problemem \#P-trudnym (\cite{kuperberg15}, \cite{vertigan05}),
a przy znanych homologiach -- wręcz trywialnym.
Patrz też \ref{jones_sharp_p_hard}.

\subsection{Metody kodowania węzłów i splotów}
Tait, Little wyprodukowali prawie bezbłędną tablicę węzłów o co najwyżej 11 skrzyżowaniach przy użyciu grafów.

\subsubsection{Notacja Conwaya}
Wprowadzona przez Conwaya w pracy ,,An Enumeration of Knots and Links, and Some of Their Algebraic Properties''.

\subsubsection{Notacja Dowkera--Thistlethwaite'a}
Poprawia notację Taita.

\subsubsection{Notacja Alexandera-Briggsa}
Najbardziej tradycyjna, wprowadzona w 1927 roku, rozszerzona później przez Rolfsena.
\section{Ruchy Reidemeistera}
\label{sec:reidemeister_moves}
W kombinatorycznej teorii węzłów diagramy są dużo ważniejsze od gładkich włożeń okręgu w przestrzeń $\R^3$,
dlatego przytoczymy proste kryterium decydujące o tym,
kiedy dwa diagramy przedstawiają jeden węzeł.
Najpierw zdefiniujmy trzy lokalne operacje na diagramach.

\begin{definition}
    Trzy ruchy Reidemeistera, $R_1$, $R_2$, oraz $R_3$, to następujące deformacje diagramu:
    \[
        \underbrace{\begin{tikzpicture}[baseline=-0.65ex,scale=0.1]
        \begin{knot}[clip width=5]
            \strand[thick] (-5, 10) to [in=left, out=down] (2, -5);
            \strand[thick] (5, 0) to [in=right, out=down] (2, -5);
            \strand[thick] (5, 0) to [in=right, out=up] (2, 5);
            \strand[thick] (-5, -10) to [in=left, out=up] (2, 5);
        \end{knot}
        \end{tikzpicture}
        \, \cong \,
        \begin{tikzpicture}[baseline=-0.65ex,scale=0.1]
        \begin{knot}[clip width=5]
            \strand[thick] (0,10) to (0,-10);
        \end{knot}
        \end{tikzpicture}}_{R_1}
        %%%
        \quad \quad \quad
        \underbrace{\begin{tikzpicture}[baseline=-0.65ex,scale=0.1]
        \begin{knot}[clip width=5]
            \strand[thick] (-5, 10) to [in=up, out=down] (5, 0);
            \strand[thick] (-5, -10) to [in=down, out=up] (5, 0);
            \strand[thick] (5, 10) to [in=up, out=down] (-5, 0);
            \strand[thick] (5, -10) to [in=down, out=up] (-5, 0);
        \end{knot}
        \end{tikzpicture}
        \, \cong \,
        \begin{tikzpicture}[baseline=-0.65ex,scale=0.1]
        \begin{knot}[clip width=5]
            \strand[thick] (-5, 10) to [in=up, out=down] (-2, 0);
            \strand[thick] (-5, -10) to [in=down, out=up] (-2, 0);
            \strand[thick] (5, 10) to [in=up, out=down] (2, 0);
            \strand[thick] (5, -10) to [in=down, out=up] (2, 0);
        \end{knot}
        \end{tikzpicture}}_{R_2}
        %%%
        \quad \quad \quad
        \underbrace{\begin{tikzpicture}[baseline=-0.65ex,scale=0.1]
        \begin{knot}[clip width=5, flip crossing/.list={1,2,3}]
            \strand[thick] (-10, -10) -- (10, 10);
            \strand[thick] (-10, 10) -- (10, -10);
            \strand[thick] (-10, 0) to [in=left, out=right] (0, 10);
            \strand[thick] (10, 0) to [in=right, out=left] (0, 10);
        \end{knot}
        \end{tikzpicture}
        \, \cong \,
        \begin{tikzpicture}[baseline=-0.65ex,scale=0.1]
        \begin{knot}[clip width=5, flip crossing/.list={1,2,3}]
            \strand[thick] (-10, -10) -- (10, 10);
            \strand[thick] (-10, 10) -- (10, -10);
            \strand[thick] (-10, 0) to [in=left, out=right] (0, -10);
            \strand[thick] (10, 0) to [in=right, out=left] (0, -10);
        \end{knot}
        \end{tikzpicture}}_{R_3}
    \]
\end{definition}

Ruch $R_i$ operuje więc na $i$ łukach diagramu.
Reidemeister w swojej pierwszej pracy przyjął inną kolejność,
jego drugi ruch jest naszym pierwszym.

\begin{theorem}[Reidemeister, 1927]
    Każdy splot posiada diagram.
    Dwa diagramy przedstawiają równoważne sploty,
    wtedy i tylko wtedy gdy pierwszy można otrzymać z drugiego
    wykonując skończenie wiele ruchów Reidemeistera
    oraz gładko deformując łuki bez zmiany biegu skrzyżowań.
\end{theorem}

\begin{proof}
    Szkielet dowodu można znaleźć w książce Burdego i Zieschanga.
    Kluczowe pomysły zawiera ,,Knots, links, braids and $3$-manifolds''
    Prasołowa i Sosińskiego.
    Innym przystępnym źródłem jest podręcznik \cite{murasugi96} Murasugiego ,,Knot theory and its applications''.
\end{proof}

Twierdzenie Reidemeistera jest użytecznym narzędziem,
z którego będziemy korzystać podczas definiowania większości niezmienników,
obiektów, które pozwalają odróżniać od siebie węzły.
Rzadko stosuje się je do przechodzenia między dwoma diagramami.
Istotnie, Coward i Lackenby udowodnili w \cite{coward11},
że jeśli dwa diagramy o $n$ skrzyżowaniach przedstawiają jeden węzeł, wystarcza
\[
    R(n) = 2^{2^{\ldots^{2^n}}}
\]
(gdzie piętrowa potęga ma $10^{1000000n}$ warstw) ruchów Reidemeistera, by przejść między nimi.
Jeśli jeden z diagramów jest pozbawiony skrzyżowań, wystarcza $(236n)^{11}$ ruchów.
Zapewne lepsze ograniczenia istnieją, ale ich nie znamy.
Ważne jest to, że wielkość $R(n)$ jest skończona.

\begin{definition}
    Węzeł zorientowany to taki, w którym wybrano kierunek, w którym należy się po nim poruszać.
\end{definition}

Twierdzenie Reidemeistera pozostaje prawdziwe dla węzłów zorientowanych.

% koniec sekcji Ruchy Reidemeistera

\section{Operacje na węzłach} % (fold)

W tej sekcji poznamy sposoby otrzymywania nowych obiektów z~już istniejących (rewers i~lustro splotu).
Rodzina węzłów wyposażona w~sumę spójną tworzy przemienny monoid z~jednoznacznością rozkładu.
Znacznie później (w sekcji \ref{sec:tangle}) określimy jeszcze sumę oraz iloczyn supłów.

\subsection{Lustro i~rewers} % (fold)
\begin{definition}[lustro]
    \index{lustro}
    Niech $L$ będzie zorientowanym splotem.
    Splot $mL$ powstały przez odbicie splotu $L$ względem dowolnej płaszczyzny nazywamy lustrem.
\end{definition}

\begin{definition}[rewers]
    \index{rewers}
    Niech $L$ będzie zorientowanym splotem.
    Splot $rL$ powstały przez odwrócenie orientacji wszystkich ogniw splotu $L$ nazywamy rewersem.
\end{definition}

\begin{comment}
\begin{figure}[H]
    \begin{minipage}[b]{.32\linewidth}
        \centering
        \includegraphics[width=\linewidth]{../data/link_mirror.png}
        \subcaption{lustro $mL$}
    \end{minipage}
    \begin{minipage}[b]{.32\linewidth}
        \centering
        \includegraphics[width=\linewidth]{../data/link.png}
        \subcaption{przykładowy splot $L$}
    \end{minipage}
    \begin{minipage}[b]{.32\linewidth}
        \centering
        \includegraphics[width=\linewidth]{../data/link_reverse.png}
        \subcaption{rewers $rL$}
    \end{minipage}
\end{figure}
\end{comment}

Na lewym obrazku odbiliśmy diagram względem poziomej prostej, innym sposobem na otrzymanie lustra jest odwrócenie wszystkich skrzyżowań, co odpowiada odbijaniu względem płaszczyzny papieru.
Zauważmy, że wykonując powyższe operacje na węźle możemy otrzymać mniej niż czterech różne obiekty ($L$, $mL$, $rL$, $mrL$) -- na przykład trójlistnik jest własnym rewersem, ale nie lustrem.

Wyróżniamy pięć typów symetrii węzłów:

\begin{definition}[całkowicie chiralny albo skrętny]
    \index{węzeł!chiralny (skrętny)}
    Węzły $K$, $rK$, $mK$ są parami nierównoważne. % chiral 9_32
\end{definition}

\begin{definition}[odwracalny]
    \index{węzeł!odwracalny}
    Węzły $K \cong rK$ są równoważne. % reversible 3_1
\end{definition}

\begin{definition}[zwierciadlany ujemnie]
    \index{węzeł!zwierciadlany}
    Węzły $K \cong mrK$ są równoważne. % negative amphicheiral 8_17
\end{definition}

\begin{definition}[zwierciadlany dodatnio]
    Węzły $K \cong mK$ są równoważne. % positive amphicheiral 12a_427
\end{definition}

\begin{definition}[całkowicie zwierciadlany]
    Węzły $K, rK, mK$ są parami równoważne. % fully amphicheiral 4_1
\end{definition}

\begin{example}
    Węzeł $9_{32}$ jest całkowicie skrętny.
\end{example}

Całkowicie skrętne są też między innymi wszystkie węzły torusowe.

\begin{example}
    \label{exm:trefoil_is_chiral}
    Trójlistnik jest odwracalny, ale nie zwierciadlany.
\end{example}

Po raz pierwszy odkrył to M. Dehn w 1914 roku \cite{dehn14}.
Oto, jak tego dokonał.
Iloraz grafu Cayleya dla grupy podstawowej trójlistnika, $G = \pi_1(S^3 - K)$, zanurza się w~produkt $\mathbb H^2 \times \R$, co pozwala wyznaczyć grupę zewnętrznych automorfizmów grupy $G$, $\Z/2\Z$.
Korzystając z południków i równoleżników pokazał następnie, że nietrywialny automorfizm odwraca orientację przestrzeni otaczającej.
My przekonamy się o~tym przez wyznaczenie wielomianu Jonesa trójlistnika, patrz wniosek \ref{cor:joines_of_amphicheiral}.

\begin{example}
    Węzeł $8_{17}$ jest zwierciadlany ujemnie, ale nie odwracalny.
\end{example}

Sześćdziesiąt lat temu nie było pewne, czy węzły nieodwracalne w~ogóle istnieją; obecnie wiadomo, że prawie wszystkie węzły są nieodwracalne (\cite[s.~46]{murasugi96}).
W~roku 1962 Ralph Fox wskazał kilku kandydatów do tego tytułu.
Hale Trotter odkrył rok później nieskończoną rodzinę nieodwracalnych precli, patrz \ref{prp:pretzel_not_invertible}.

\begin{example}
    Węzeł $12a427$ jest zwierciadlany dodatnio, ale nie odwracalny.
\end{example}

Żaden inny węzeł pierwszy o mniej niż 13 skrzyżowaniach nie ma tej cechy.

\begin{example}
    Ósemka $4_1$ jest całkowicie zwierciadlana.
\end{example}

To najprostszy typ symetrii, wystarczy jawnie wskazać przekształcenie między diagramem węzła, jego lustra oraz odwrotności.

Tait odnosił wrażenie, że zwierciadlane węzły mają parzysty indeks skrzyżowań,
ale Hoste (Thistlethwaite?) znalazł w~1998 kontrprzykład o~piętnastu skrzyżowaniach.
Jest on jedynym znanym nam dzisiaj.
Hipoteza Taita jest prawdziwa dla węzłów pierwszych, alternujących.

Poniższa tabela oparta jest (kolejno) o~ciągi
\href{https://oeis.org/A051766}{51766},
\href{https://oeis.org/A051769}{51769},
\href{https://oeis.org/A051768}{51768},
\href{https://oeis.org/A051767}{51767},
\href{https://oeis.org/A052400}{52400},
z bazy danych ``The On-Line Encyclopedia of Integer Sequences'' (OEIS).

\begin{table}[h]
    \centering
    \begin{tabular}{@{}*{20}l@{}} \toprule
        skrzyżowania & 3 & 4 & 5 & 6 & 7 & 8 & 9 & 10 & 11 & 12 & 13 & 14 \\ \midrule
        całkowicie skrętne & 0 & 0 & 0 & 0 & 0 & 0 & 2 & 27 & 187 & 1103 & 6919 & 37885 \\
        odwracalne & 1 & 0 & 2 & 2 & 7 & 16 & 47 & 125 & 365 & 1015 & 3069 & 8813 \\
        $-$ zwierciadlane & 0 & 0 & 0 & 0 & 0 & 1 & 0 & 6 & 0 & 40 & 0 & 227 \\
        $+$ zwierciadlane & 0 & 0 & 0 & 0 & 0 & 0 & 0 & 0 & 0 & 1 & 0 & 6 \\
        zwierciadlane & 0 & 1 & 0 & 1 & 0 & 4 & 0 & 7 & 0 & 17 & 0 & 41 \\
        \bottomrule
        \hline
    \end{tabular}
    \caption{Liczba węzłów o~poszczególnych typach symetrii}

\end{table}

Można wyróżnić jeszcze jeden rodzaj symetrii.

\begin{definition}
    \label{def:period}
    \index{węzeł!okresowy}
    Węzeł $K$ nazywamy $n$-okresowym, jeśli istnieje obrót $f \colon \R^3 \to \R^3$ o~kąt $2\pi/n$ wokół pewnej prostej $l$, rozłącznej z~węzłem, taki że $f(K) = K$.
\end{definition}

Trójlistnik jest 3-okresowy, węzeł $5_1$ 5-okresowy (widać to na standardowym diagramie) i~2-okresowy.
Tę drugą symetrię można dostrzec na diagramie realizującym indeks mostowy.

\begin{proposition}
    Zbiór wszystkich okresów jest niezmiennikiem węzłów.
\end{proposition}

Z~każdym węzłem okresowym związany jest inny, prostszy węzeł.
Niech $f$ będzie obrotem z definicji \ref{def:period}, zaś $p \colon \R^3 \to \R^3/f \simeq \R^3$ rzutem na przestrzeń ilorazową.
\index{węzeł!ilorazowy}
Wtedy $p(K)$ nazywamy \emph{węzłem ilorazowym}, zaś $K$ to jego $n$-krotne nakrycie.

Murasugi podał dwa warunki, które musi spełniać węzeł o~okresie $n = p^r$, gdzie $r$ jest liczbą pierwszą.
Do ich zrozumienia potrzebujemy prostej definicji.
Ustalmy półprostą, która nie jest styczna do węzła $K$, po czym zorientujmy ją oraz węzeł.
Indeksem zaczepienia $\lambda$ węzła $p(K)$ jest różnica między liczbą skrzyżowań dodatnich oraz ujemnych wzdłuż półprostej (bez znaku).

\begin{proposition}[warunek Murasugiego]
    \index{warunek Murasugiego}
    \label{prp:murasugi_periodic}
    Niech $K$ będzie węzłem o~okresie $n = p^r$, gdzie $p$ jest liczbą pierwszą.
    Niech $J$ będzie jego węzłem ilorazowym, z~indeksem zaczepienia $\lambda$.
    Wtedy wielomian $\alexander_J$ jest dzielnikiem wielomianu $\alexander_K$ oraz istnieje pewna całkowita liczba $k$, taka że
    \begin{equation}
        \alexander_K = \pm \alexander_J^n \left(1 + t + t^2 + \ldots + t^{\lambda - 1}\right)^{n-1} \mod p.
    \end{equation}
\end{proposition}

\begin{proof}
    Mozolne operacje na macierzach, których wyznacznikiem jest wielomian Alexandera.
    Patrz \cite{murasugi71}.
\end{proof}

% Koniec podsekcji Lustro i~rewers

\subsection{Suma niespójna i~suma spójna} % (fold)

Suma spójna węzłów to szczególny przypadek sklejenia dwóch rozmaitości wzdłuż brzegu.

\begin{definition}[suma niespójna]
    Niech $L_1$ oraz $L_2$ będą splotami, które leżą po różnych stronach ustalonej płaszczyzny w przestrzeni $\R^3$.
    Teoriomnogościową sumę $L_1 \sqcup L_2$ nazywamy sumą niespójną splotów.
\end{definition}

\begin{definition}[suma spójna]
    \index{suma!spójna, niespójna}
    Niech $K_1, K_2$ będą zorientowanymi węzłami.
    Natnijmy każdy z nich w dwóch punktach tego samego krótkiego łuku, a następnie zszyjmy dwoma łukami, które nie przecinają już istniejących, jak na obrazku.
    Otrzymany węzeł nazywamy sumą spójną węzłów $K_1$ oraz $K_2$.
\begin{comment}
    \[
        \begin{tikzpicture}[baseline=-0.65ex,scale=0.1]
        \begin{knot}[clip width=5, flip crossing/.list={5}, ignore endpoint intersections=false,]
            \strand[thick] (-3.5, -3.5) [in=down, out=up] to (3.5, 3.5);
            \strand[thick] (3.5, 3.5) [in=right, out=up] to (-4.5, 10);
            \strand[thick] (-4.5, 10) [in=up, out=left] to (-10, 3.5);
            \strand[thick] (-10, 3.5) to (-10, -3.5);
            \strand[thick] (-10, -3.5) [in=left, out=down] to (-4.5, -10);
            \strand[thick] (-4.5, -10) [in=down, out=right] to (3.5, -3.5);
            \strand[thick] (3.5, -3.5) [in=down, out=up] to (-3.5, 3.5);
            \strand[thick] (-3.5, 3.5) [in=left, out=up] to (4.5, 10);
            \strand[thick] (4.5, 10) [in=up, out=right] to (10, 3.5);
            \strand[thick, -Latex] (10, 3.5) to (10, -3.5);
            \strand[thick] (10, -3.5) [in=right, out=down] to (4.5, -10);
            \strand[thick] (4.5, -10) [in=down, out=left] to (-3.5, -3.5);
            \node at (0, -15) {$K_1$};
        \end{knot}
        \end{tikzpicture}
        \shrap
        \begin{tikzpicture}[baseline=-0.65ex,scale=0.1]
        \begin{knot}[clip width=5, flip crossing/.list={6}, ignore endpoint intersections=false,]
            \strand[thick] (-3.5, -3.5) [in=down, out=up] to (3.5, 3.5);
            %\strand[thick] (3.5, 3.5) [in=right, out=up] to (-4.5, 10);
            %\strand[thick] (-4.5, 10) [in=up, out=left] to (-10, 3.5);
            \strand[thick] (-10, -3.5) [in=left, out=up] to (0, 6.5);
            \strand[thick, Latex-] (-10, -3.5) [in=left, out=down] to (-4.5, -10);
            \strand[thick] (-4.5, -10) [in=down, out=right] to (3.5, -3.5);
            \strand[thick] (3.5, -3.5) [in=down, out=up] to (-3.5, 3.5);
            %\strand[thick] (-3.5, 3.5) [in=left, out=up] to (4.5, 10);
            %\strand[thick] (4.5, 10) [in=up, out=right] to (10, 3.5);
            \strand[thick] (10, -3.5) [in=right, out=up] to (0, 6.5);
            \strand[thick] (10, -3.5) [in=right, out=down] to (4.5, -10);
            \strand[thick] (4.5, -10) [in=down, out=left] to (-3.5, -3.5);
            %
            \strand[thick] (-3.5, 3.5) [in=left, out=up] to (0, 10);
            \strand[thick] (3.5, 3.5) [in=right, out=up] to (0, 10);
            \node at (0, -15) {$K_2$};
        \end{knot}
        \end{tikzpicture}
        =
        \begin{tikzpicture}[baseline=-0.65ex,scale=0.1]
        \begin{knot}[clip width=5, flip crossing/.list={5, 22, 23}, ignore endpoint intersections=false,]
            \strand[thick] (-18.5, -3.5) [in=down, out=up] to (-11.5, 3.5);
            \strand[thick] (-11.5, 3.5) [in=right, out=up] to (-19.5, 10);
            \strand[thick] (-19.5, 10) [in=up, out=left] to (-25, 3.5);
            \strand[thick] (-25, 3.5) to (-25, -3.5);
            \strand[thick] (-25, -3.5) [in=left, out=down] to (-19.5, -10);
            \strand[thick] (-19.5, -10) [in=down, out=right] to (-11.5, -3.5);
            \strand[thick] (-11.5, -3.5) [in=down, out=up] to (-18.5, 3.5);
            \strand[thick] (-18.5, 3.5) [in=left, out=up] to (-10.5, 10);
            \strand[thick] (-10.5, 10) [in=left, out=right] to (-5, 2);
            \strand[thick, -Latex] (-5, 2) to (-5+6, 2);
            \strand[thick] (5, 2) to (-5+6, 2);
            \strand[thick] (3, -2) to [in=left, out=right] (10.5, -10);
            \strand[thick, -Latex] (3, -2) to (0, -2);
            \strand[thick] (-5, -2) to (0, -2);
            \strand[thick] (-5, -2) [in=right, out=left] to (-10.5, -10);
            \strand[thick] (-10.5, -10) [in=down, out=left] to (-18.5, -3.5);
            %%%
            \strand[thick] (11.5, -3.5) [in=down, out=up] to (18.5, 3.5);
            \strand[thick] (-10 +15, 2) [in=left, out=right] to (15, 6.5);
            \strand[thick] (10.5, -10) [in=down, out=right] to (18.5, -3.5);
            \strand[thick] (18.5, -3.5) [in=down, out=up] to (11.5, 3.5);
            \strand[thick] (25, -3.5) [in=right, out=up] to (15, 6.5);
            \strand[thick] (25, -3.5) [in=right, out=down] to (19.5, -10);
            \strand[thick] (19.5, -10) [in=down, out=left] to (11.5, -3.5);
            \strand[thick] (11.5, 3.5) [in=left, out=up] to (15, 10);
            \strand[thick] (18.5, 3.5) [in=right, out=up] to (15, 10);
            %%%
            \node at (0, -15) {$K_1 \shrap K_2$};
        \end{knot}
        \end{tikzpicture}
    \]
\end{comment}
\end{definition}

\begin{tobedone}
The band sum operation is a special case of a hyperbolic transformation of a link (in 12.3) and also of a fusion of a link (in 13.1).
% Kawauchi
\end{tobedone}

Ważna jest orientacja składników: suma dwóch trójlistników może być węzłem babskim lub prostym.
Uzasadnienie, że te węzły są różne, nie jest łatwym zadaniem.
Fox pokazał w~1952 roku, że ich dopełnienia nie są homeomorficzne.
Suma przeciwnie skręconych trójlistników jest plastrowa, natomiast tak samo skręconych nie jest.
(To jedno z niewielu miejsc, gdzie nomenklatura pochodzi od żeglarzy: z~angielskiego \emph{granny knot, square knot}.)

Warunku, by zszywające łuki nie przecinały diagramów, nie można pominąć: Cromwell w~\cite[s.90]{cromwell04} pokazuje przykład dwóch niewęzłów, z~których otrzymano niepoprawnie dwie różne sumy, $6_1$ oraz $8_{20}$.

Uogólnieniem sumy spójnej oraz (nieopisanej w~naszej pracy) operacji \emph{plumbing} jest suma Murasugiego, dobrze wyjaśniona w~czwartym rozdziale książki \cite{kawauchi96}.

\begin{tobedone}
The Murasugi sum of Seifert surfaces was introduced originally in [Murasugi 1958, 1958',1958"] in order to estimate the degree of the Alexander polynomial of alter- nating links. After that, J. Stallings showed in [Stallings 1978] that a Seifert surface obtained by a Murasugi sum of fiber surfaces is a fiber surface.
\end{tobedone}

% TODO: \textbf{W topologii rozważa się podobną operację dla dwóch $n$-wymiarowych rozmaitości: z~każdej z nich usuwa się kulę, po czym skleja wzdłuż brzegu kuli. Kiedy zajmujemy się węzłami, nie interesuje nas jednak struktura rozmaitości (gdyż każdy węzeł jest równoważny z okręgiem), ale zanurzenie w~otaczającą przestrzeń.}

\begin{proposition}
    Suma spójna jest dobrze określonym działaniem.
\end{proposition}

Suma spójna nie jest dobrze określona dla splotów:
nie istnieje kanoniczny wybór, które ogniwa łączyć ze sobą.

\begin{proof}
    Niech dane będą węzły $K_1$ oraz $K_2$
    oraz dwa różne łuki $\gamma_1$, $\gamma_2$,
    których można użyć do konstrukcji sumy spójnej.
    Skurczmy $K_1$, przeciągnijmy najpierw przez łuk $\gamma_1$, a~następnie wzdłuż węzła $K_2$.
    Teraz wystarczy odwrócić ten proces z~$\gamma_2$ w~miejscu $\gamma_1$.
\end{proof}

\begin{proposition}
    Suma spójna jest działaniem łącznym oraz przemiennym.
    Niewęzeł stanowi jej element neutralny.
\end{proposition}

Prosty dowód tego faktu pozostawiamy Czytelnikowi.
W języku algebry mówimy, że węzły z~sumą spójną tworzą półgrupę (tak jak liczby naturalne z działaniem dodawania).
Dużo później pokażemy, że działaniu $\shrap$ brakuje elementów przeciwnych, więc ta struktura algebraiczna nie jest grupą.

\begin{proposition}
    Jeśli $K \shrap L = \Unknot$, to $K = L = \Unknot$.
\end{proposition}

\begin{proof}[Niedowód]
    Technika ta zwana jest szwindlem Mazura.
    Załóżmy, że $K \shrap L = \Unknot$ i~dopuśćmy wyjątkowo węzły dzikie.
    Skonstruujmy sumę $K \shrap L \shrap K \shrap \ldots$,
    przy czym kolejne składniki powinny zmniejszać się,
    aby ich suma nadal była węzłem.
    Wtedy
    \begin{align*}
        K & \simeq K \shrap [(L \shrap K) \shrap (L \shrap K) \ldots] \\
         & \simeq (K \shrap L) \shrap (K \shrap L) \shrap \ldots
         \simeq \Unknot \shrap \Unknot \shrap \ldots
         \simeq \Unknot.
    \end{align*}
    Analogicznie pokazujemy, że $L \simeq \Unknot$.
\end{proof}

\begin{tobedone}
    % Kawauchi
    Theorem 3.2.1 (Non-cancellation theorem) A connected sum L1~L2 of any two links L1 and L2 is not a trivial link unless both links L1 and L2 are trivial links.
The proof (whose details are left to the reader) is essentially obtained from the following two facts:
(1) If L1 and L2 are non-split links, then L1~L2 is a non-split link.
(2) If L1~L2 is a trivial knot, then L1 and L2 are trivial knots.
(1) is directly proved by a cut-and-paste argument of combinatorial topology. (2) is usually obtained from Schubert's result on the additivity of the knot genus (cf. 4.1.5) under the connected sum, i.e., g(Ll~L2) = geLd +g(L2) (which is also proved by a cut-and-paste argument).
\end{tobedone}

Prawdziwy dowód oparty jest na topologii algebraicznej, stanowi bezpośredni wniosek z~faktów \ref{prp:genus_detects_unknot} oraz \ref{prp:genus_of_sum}.

Półgrupę węzłów z~operacją sumy spójnej można ulepszyć do grupy na dwa sposoby:
albo poprzez zmianę działania, w~jakie jest wyposażona,
albo osłabiając definicję węzłów równoważnych.
Drugi pomysł jest dużo lepszy niż pierwszy.
Na początku lat pięćdziesiątych J. Milnor wprowadził do matematyki pojęcie zgodności
(z angielskiego \emph{concordance}), które zastąpiło zwykłą równoważność.
Element neutralny nowej grupy to węzły plastrowe, ich opis leży w~sekcji \ref{sec:slice}.
Zagadnienia te zakorzenione są w~czterowymiarowej topologii.

% Koniec podsekcji Suma niespójna i~suma spójna

% Koniec sekcji Operacje na węzłach

\section{Węzły pierwsze}
\label{sec:prime_knots}
Istnieje węzłowy odpowiednik liczb pierwszych.
Jest on ściśle związany z podaną wyżej operacją sumy spójnej.
Do jego dostatecznie dobrego zrozumienia wymagana jest znajomość powierzchni Seiferta (opisanych w sekcji \ref{sec:genus}).

\begin{definition} \label{primeknot}
    Nietrywialny węzeł nazywamy \textbf{pierwszym},
    kiedy nie można przedstawić go jako sumy spójnej $K_1 \shrap K_2$
    dwóch nietrywialnych węzłów $K_1, K_2$ (nie jest złożony).
\end{definition}

Okazuje się, że jeśli alternujący splot nie jest pierwszy,
to każdy jego alternujący diagram jest złożony.
Jako pierwszy fakt ten został wykazany przez Menasco w \cite{menasco84}.
Dowód opiera się na multiplikatywności wielomianu BLM/Ho (opisuje go definicja \ref{def:blm_ho}).

Czy węzłów pierwszych jest nieskończenie wiele?
Tak (patrz fakt \ref{infty_primes}), potrafimy nawet oszacować ich liczbę.
W roku 1987 C. Ernst, D. Sumners w oparciu o wyniki Thistlethwaite'a, Kauffmana, oraz Murasugiego dotyczące węzłów alternujących pokazali w \cite{ernst87},
że różnych węzłów pierwszych o $n$ skrzyżowaniach jest co najmniej $\frac 1 3 (2^{n- 2} - 1)$ dla $n \ge 4$,
przy czym węzły lustrzane traktowane są jako różne.
Welsh znalazł ograniczenie
\[
    \limsup_{n \to \infty} \sqrt[n]{N(n)} \le 13.5,
\]
gdzie $N(n)$ jest ilością pierwszych węzłów o $n$ skrzyżowaniach.

Czy niewęzeł nie daje się zapisać jako suma dwóch innych węzłów?
Byłoby to skrajnie niepożądane, gdyż każdy węzeł jest naturalnie spójną sumą siebie oraz niewęzła.
Na szczęście przy pomocy powierzchni Seiferta można pokazać, że tak się nie dzieje (jest to wniosek \ref{no_inverses}).
Prawdziwe jest dużo mocniejsze stwierdzenie,
którego nie udowodnimy ze względu na niedostatecznie rozwinięty aparat matematyczny.
Należy o nim myśleć jak o analogonie zasadniczego twierdzenia arytmetyki.

\begin{theorem}[Schubert, 1949]
    Każdy różny od niewęzła węzeł rozkłada się jednoznacznie na węzły pierwsze
    (jeśli tylko pominąć kolejność składników).
\end{theorem}

Pomimo to uda się nam pokazać samo istnienie rozkładu, jest to fakt \ref{genus_sum}.
% koniec sekcji Węzły pierwsze

\section{Niezmienniki liczbowe}
Jak wspomnieliśmy na początku rozdziału, sprawdzenie,
czy dwa diagramy przedstawiają sploty równoważne,
jest uciążliwym i~czasochłonnym zadaniem.
Aby je uprościć, podamy opis kilku prostych niezmienników o~całkowitych wartościach.
Zachodzą implikacje:
sploty równoważne $\Rightarrow$ ta sama wartość niezmiennika
oraz różne wartości niezmiennika $\Rightarrow$ różne sploty.

\subsection{Indeks skrzyżowaniowy} % (fold)
\label{sub:crossing_number}
\index{indeks!skrzyżowaniowy}
Z angielskiego \emph{crossing number}.

\begin{definition}
    Indeks skrzyżowaniowy $\operatorname{cr}(L)$ splotu $L$ to
    minimalna liczba skrzyżowań widocznych na diagramie,
    który przedstawia splot $L$.
\end{definition}

Jeśli $K_1, K_2$ są alternującymi węzłami o~$c_1, c_2$ skrzyżowaniach, to istnieje alternujący diagram ich sumy $K_1 \shrap K_2$ o~$c_1 + c_2$ skrzyżowaniach.
Lackenby w~pracy \cite{lackenby09} pokazał, że dla pewnej stałej $N \le 152$ zachodzi
\[
    \frac 1 N \sum_{i=1}^n \operatorname{cr} K_i \le \operatorname{cr} \left(\bigshrap_{i=1}^n K_i\right) \le \sum_{i=1}^n \operatorname{cr} K_i.
\]
(Tylko pierwsza nierówność jest ciekawa).
Jego argumentu (wykorzystującego powierzchnie normalne) nie można poprawić tak, by otrzymać stałą $N = 1$.
Wiadomo jednak, że indeks skrzyżowaniowy jest addytywny dla niektórych klas węzłów: alternujących (\cite{kauffman88}), adekwatnych\footnote{Węzły adekwatne nie pojawiają się na żadnej innej stronie tej książki.} czy torusowych (\cite{gruber03}).

\begin{proposition} \label{prp:bankwitz}
    Wyznacznik splotu alternującego $L$ jest nie mniejszy od jego indeksu skrzyżowaniowego: $\det L \ge \operatorname{cr} L$.
\end{proposition}

W ten sposób pokazano, że niealternujące węzły istnieją: $\det 8_{19} = 3 < 8 = \operatorname{cr} 8_{19}$.
Bankwitz podał w 1930 niepoprawny dowód dla specjalnego przypadku, kiedy $L$ jest węzłem.
Pierwszy pełny dowód, oparty o teorię grafów, przedstawił blisko trzy dekady później Crowell w \cite{crowell57}.
Znalazł też mocniejszą nierówność $\det L + 3 \ge \operatorname{cr} L$ dla pierwszych, alternujących splotów, które nie są $(2, n)$-torusowe.
To prawie wystarczyło do rozstrzygnięcia, które z węzłów o mniej niż 10 skrzyżowaniach są alternujące.
Otwartym problemem pozostały $9_{45}$, $9_{47}$, $9_{48}$ oraz $9_{49}$.

% Koniec podsekcji Indeks skrzyżowaniowy

\subsection{Liczba gordyjska} % (fold)
\label{sub:unknotting_number}
\index{liczba!gordyjska}
Z angielskiego \emph{unknotting number}.

\begin{definition}
    Liczba gordyjska $\operatorname{u}(K)$ węzła $K$ to minimalna liczba skrzyżowań,
    które trzeba odwrócić na pewnym jego diagramie, by dostać niewęzeł.
    Liczba $u$ nie musi być osiągana przez diagram o~minimalnej liczbie skrzyżowań.
\end{definition}

Dotychczas wyznaczono liczbę gordyjską dla prawie wszystkich węzłów pierwszych o~co najwyżej dziesięciu skrzyżowaniach,
Cha i~Livingston podają następującą listę wyjątków:
$10_{11}$, $10_{47}$, $10_{51}$, $10_{54}$, $10_{61}$, $10_{76}$, $10_{77}$, $10_{79}$, $10_{100}$ (stan na rok 2008).
S. Bleiler odkrył w~\cite{bleiler84} fascynujący przykład wymiernego węzła $2$-gordyjskiego,
czego świadkiem nie może być diagram o~minimalnej liczbie skrzyżowań
(ponieważ, co jeszcze bardziej fascynujące, węzeł ten posiada tylko jeden diagram o~dziesięciu skrzyżowaniach z~trzema do odwrócenia).
Koduje go liczba $29/6$, w~tabeli węzłów figuruje jako $10_8$.
Stojmenow w~\cite{stoimenow01} pokazuje, że jeden węzeł może mieć kilka diagramów minimalnych,
z których tylko niektóre są świadkiem $1$-gordyjskości (są to między innymi $14_{36750}$ oraz $14_{36760}$.)

Coward, Lackenby dowiedli w~\cite{coward11}, że jeśli $K$ jest 1-gordyjski i~o genusie 1, to z~dokładnością do pewnej relacji równoważności, tylko jedna zmiana skrzyżowania rozwiązuje go; chyba że $K$ jest ósemką -- wtedy takie zmiany są dwie.
Kanenobu, Murakami oraz Kohn dwadzieścia lat wcześniej wiedzieli, że wymierne sploty 1-gordyjskie posiadają rozwiązujące skrzyżowanie na minimalnym diagramie (\cite{kanenobu86}, \cite{kohn91}).

Jeśli odwrócenie pewnych skrzyżowań daje niewęzeł, to odwrócenie pozostałych także.
To daje proste, choć niezbyt pomocne oszacowanie liczby gordyjskiej: $2 \operatorname{u} (K) \le \operatorname{cr} (K)$.
Borodzik oraz Friedl podali niedawno całkiem mocne ograniczenia na liczbę gordyjską w~pracach \cite{borodzik14} i~\cite{borodzik15} opierając się o~parowanie Blanchfielda
(poprawiając ograniczenia z~sygnatury Levine'a-Tristrama, indeksu Nakanishiego oraz przeszkodą Lickorisha).
Dwadzieścia pięć węzłów o~co najwyżej dwunastu skrzyżowaniach ma liczbę gordyjską równą co najmniej trzy, co trudno pokazać innymi metodami.

Dla każdego nietrywialnego splotu istnieje diagram wymagający odwrócenia dowolnie wielu skrzyżowań (dowód zawiera praca \cite{taniyama09} Taniyamy).
Pokazany jest tam jeszcze jeden godny uwagi fakt.
Jeśli liczba gordyjska diagramu $D$ wynosi $\frac 12 (c(D) - 1)$,
co jest maksymalną możliwą wartością zgodnie z~naszym prostym ograniczeniem,
to węzeł jest $(2,p)$-torusowy albo wygląda jak diagram niewęzła po pierwszym ruchu Reidemeistera.

% Liczba gordyjska nietrywialnego węzła skręconego (definicja \ref{twist_knots}) to jeden, wystarczy bowiem rozwiązać jego splecione końce.
Patrz także stwierdzenie \ref{torus_unknotting}.

Suma dwóch węzłów $1$-gordyjskch jest $2$-gordyjska, pokazał to Scharleman.
Od początku teorii węzłów podejrzewano dużo więcej:

\begin{conjecture}
    Niech $K, J$ będą węzłami.
    Wtedy $u(K \shrap J) = u(K) + u(J)$, czyli liczba gordyjska jest addytywna.
\end{conjecture}

Z tej nieudowodnionej do dzisiaj (stan na 2018) hipotezy można wyciągnąć wniosek,
że jeśli do rozwiązania węzła wystarcza odwrócenie skrzyżowania, to jest pierwszy.
Podejrzewał to H. Wendt w~1937 roku,
kiedy policzył liczbę gordyjską węzła babskiego używając homologii rozgałęzionego nakrycia cyklicznego.

\begin{proposition}
    Węzły $1$-gordyjskie są pierwsze.
\end{proposition}

\begin{proof}[Niedowód]
    W pracy \cite{scharleman85} z~1985 roku M. Scharleman podał dość zawiłe uzasadnienie, w~które zamieszane były grafy planarne.
\end{proof}

W pracy \cite{shimizu14} Ayaka Shimizu pokazuje przykład innej operacji rozwiązującej węzły, ale nie wszystkie sploty:
zmianę skrzyżowań wokół obszaru na diagramie.

Liczbę gordyjską można uogólnić w naturalny sposób do metryki.
Mianowicie mając dane dwa węzły $K_0, K_1$, rozpatrzmy wszystkie homotopie $f : [0,1] \times S^1 \to \R^3$ takie, że wszystkie funkcje $f_t$ są zanurzeniami z co najwyżej jednym punktem podwójnym.
Zażądajmy dodatkowo, by styczne do krótkich łuków, które przecinają się w tym punkcie, były od siebie różne.
Odległością gordyjską między węzłami $K_0, K_1$ jest minimalna liczba podwójnych punktów, jakie posiada homotopia $f$.

Przestrzeń węzłów z~tą metryką bywa sprzeczna z~intuicją: twierdzenie C~z~pracy \cite{gambaudo05} głosi, że zawiera ona prawie idealną kopię przestrzeni euklidesowej dowolnego wymiaru.
Dokładniej:

\begin{proposition}
    Dla każdej liczby całkowitej $d \ge 1$ istnieje funkcja $\xi: \Z^d \to \mathcal{K}$, dodatnie stałe $A, B, C$ oraz norma $\|\cdot\|$ na przestrzeni $\R^d$ takie, że spełniona jest podwójna nierówność
    \[
        A\|x-y\|  - B \le d(\xi(x), \xi(y)) \le C\|x-y\|.
    \]
\end{proposition}

Dowód korzysta z grup warkoczowych, które poznamy w sekcji \ref{sec:braid}.

\begin{conjecture}[Bernhard 1994, Jablan 1998] \label{bernhard_jablan}
    Każdy węzeł $K$ posiada diagram $D$ realizujący liczbę gordyjską oraz skrzyżowanie, którego odwrócenie daje nowy węzeł $K'$ z diagramem $D'$ o mniejszej liczbie gordyjskiej: $u(D') < u(D)$.
\end{conjecture}

Gdyby hipoteza była prawdziwa, mielibyśmy dość prosty algorytm do wyznaczania liczby gordyjskiej $u(K)$: wystarczy skonstruować skończenie wiele diagramów minimalnych dla węzła $K$, odwracać kolejne skrzyżowania i szukać rekursywnie liczb gordyjskich nowych węzłów.
Hipoteza jest prawdziwa dla węzłów do jedenastu skrzyżowań (patrz baza danych KnotInfo C. Livingstona) i splotów o dwóch składowych do dziewięciu skrzyżowań, pokazał to Kohn w pracy \cite{kohn93} z 1993 (!) roku.
Brittenham, Hermiller twierdzą, że hipoteza jest fałszywa.

% Koniec podsekcji Liczba gordyjska

\subsection{Liczba mostowa} % (fold)
\label{sub:bridge_index}
\index{liczba!mostowa}
Z angielskiego \emph{bridge number}.
Wprowadzona w~1954 przez Schuberta.
\begin{definition}
    Liczba mostowa $\operatorname{br}(K)$ to minimalna liczba mostów:
    długich łuków, które biegną przez nadskrzyżowania.
\end{definition}

Można pokazać, że $n$-mostowe węzły rozkładają się na sumę dwóch trywialnych $n$-supłów.

\begin{proposition}
    Niech $K_1, K_2$ będą węzłami.
    Wtedy $\operatorname{br} (K_1) + \operatorname{br}(K_2) = \operatorname{br}(K_1 \# K_2) + 1$.
\end{proposition}

\begin{proof}[Nieedowód]
    Schubert pokazał to blisko pół wieku temu w~\cite{schubert54}.
    Nowszy dowód pochodzi od Schultensa, w~artykule \cite{schultens03} skorzystał z~foliacji na brzegu węzła towarzyszącego sateltarnemu.
    Dokładniejszy opis powyższych prac wykraczałby poza zakres tego opracowania, zostanie więc pominięty.
\end{proof}

Tylko jeden węzeł jest jednomostowy, to niewęzeł.
Kolejne w~hierarchii skomplikowania, czyli dwumostowe,
to domknięcia wymiernych supłów, patrz fakt \ref{prp:two_bridge_tangle}.
Węzły trójmostowe pozostają nie do końca zbadane.

\begin{conjecture}
    Jeśli $K$ jest węzłem, to $c(K) \ge 3 br(K) - 3$, przy czym równość zachodzi dokładnie dla niewęzła, trójlistnika i~sumy spójnej trójlistników (rozdział 4.3 z~podręcznika \cite{murasugi96}).
\end{conjecture}

Nie istnieje związek między liczbą mostową oraz gordyjską.
Po pierwsze, węzły torusowe $T_{2,n}$ są dwumostowe, a~ich liczba gordyjska jest duża.
Po drugie, podwojenie węzła (poza specjalnymi przypadkami, jak pokazał Schubert) zwiększa liczbę mostową dwukrotnie; liczba gordyjska takiego podwojenia wynosi $1$.
Podobnie nie ma zależności między liczbą mostową oraz genusem.

% Koniec podsekcji Liczba mostowa

\subsection{Liczba warkoczowa} % (fold)
\label{sub:braid_number}
\index{liczba!warkoczowa}
Z angielskiego \emph{braid number}.
Do jej określenia potrzebna jest definicja \ref{braid_def}.

\begin{definition}
    Liczba warkoczowa to minimalna liczba pasm, na których można zbudować warkocz, którego domknięciem jest wyjściowy węzeł.
\end{definition}

\begin{proposition}
    Węzeł o~$n$ skrzyżowaniach można zapleść na $n - 1$ pasmach.
\end{proposition}

Indeks warkoczowy zależy od orientacji ogniw i~trudno się go wyznacza w~ogólnym przypadku.
Poniższy dowód zakłada znajomość warkoczy (rozdział piąty).

\begin{proposition}
    Indeksem warkoczowym węzła torusowego $K(q, r)$, $rq \neq 0$, jest $\min\{|q|, |r|\}$.
\end{proposition}

\begin{proof}
    Niech $K$ będzie węzłem torusowym typu $(q,r)$ z~minimalnym przedstawieniem jako warkocz $\beta$.
    Z konstrukcji domknięcia (czyli dołączenia rozłącznych półokręgów) wynika,
    że diagram $K$ ma dokładnie $b(K)$ lokalnych maksimów.
    Definicja indeksu mostowego orzeka, iż $br(K) \le b(K)$.
    Bez straty ogólności niech $q > r > 0$.
    Skoro węzeł $K$ powstaje z~$r$-warkocza $(\sigma_{r-1} \ldots \sigma_2\sigma_1)^q$,
    indeks $b(K)$ nie przekracza $r = br(K)$.

    Skorzystaliśmy z~faktu \ref{torus_bridge}.
\end{proof}

% Koniec podsekcji Liczba warkoczowa

\subsection{Indeks zaczepienia} % (fold)
\label{sub:linking_number}
\begin{definition} \label{sign_def}
    Na diagramie zorientowanego splotu, każdemu skrzyżowaniu przypisujemy \textbf{znak} równy $\pm 1$.
    Skrzyżowania dodatnie nazywamy praworęcznymi, ujemne zaś: leworęcznymi.
    \[
        \sign \Bigl(\,\,\begin{tikzpicture}[baseline=-0.65ex,scale=0.07]
        \begin{knot}[clip width=5]
        \strand[semithick,-Latex] (-5,-5) -- (5,5);
        \strand[semithick,Latex-] (-5,5) -- (5,-5);
        \end{knot}\end{tikzpicture}\,\,\Bigr) = +1 \quad
        \sign \Bigl(\,\,\begin{tikzpicture}[baseline=-0.65ex,scale=0.07]
        \begin{knot}[clip width=5, flip crossing/.list={1}]
        \strand[semithick,-Latex] (-5,-5) -- (5,5);
        \strand[semithick,Latex-] (-5,5) -- (5,-5);
        \end{knot}\end{tikzpicture}\,\,\Bigr) = -1
    \]
\end{definition}

\begin{definition} \label{sign_def}
    \index{indeks!zaczepienia}
    Niech $L = K_1 \sqcup K_2$ będzie splotem o dwóch ogniwach.
    Wielkość
    \[
        \operatorname{lk}(K_1, K_2) = \frac 12 \sum_i \sign c_i,
    \]
    gdzie sumowanie rozciąga się na wszystkie skrzyżowania, gdzie spotykają się łuki z różnych ogniw, nazywamy \textbf{indeksem zaczepienia} węzłów $K_1, K_2$.
    Ogólniej, jeśli $L = K_1 \sqcup \ldots \sqcup K_n$ jest splotem o $n$ ogniwach, to jego indeks zaczepienia wyznacza wzór $\operatorname{lk}(L) = \sum_{i < j} \operatorname{lk}(K_i, K_j)$.
\end{definition}

Zauważmy, że indeks zaczepienia splotu Hopfa wynosi $1$, natomiast splotu Whiteheada $0$.
Są zatem istotnie różne.
W obydwu przypadkach indeks zaczepienia jest liczbą całkowitą, nie stanowi to przypadku.
Na mocy twierdzenia Jordana $\operatorname{lk}$ jest funkcją o całkowitych wartościach.

\begin{proposition}
    Indeks zaczepienia jest dobrze określonym niezmiennikiem zorientowanych splotów.
\end{proposition}

\begin{proof}
    Sprawdźmy wpływ ruchów Reidemeistera na wartość
    $\operatorname{lk}(L)$:

    \[
        \fbox{
        \begin{tikzpicture}[baseline=-0.65ex,scale=0.07]
        \begin{knot}[clip width=5]
        \strand[semithick] (-10,10) .. controls (-10,2) and (-10,2) .. (-6,-2);
        \strand[semithick] (-10,-10) .. controls (-10,-2) and (-10,-1) .. (-9,0);
        \strand[semithick] (-7,1) -- (-6,2);
        \strand[semithick] (-6,2) .. controls (2,9) and (2,-9) .. (-6,-2);
        \end{knot}
        \end{tikzpicture}
        $\stackrel{R_1}{\cong} \,\,$
        \begin{tikzpicture}[baseline=-0.65ex,scale=0.07]
        \begin{knot}[clip width=5]
        \strand[semithick] (0,10) -- (0,-10);
        \end{knot}
        \end{tikzpicture}}
        %%%
        \quad \fbox{
        \begin{tikzpicture}[baseline=-0.65ex,scale=0.07]
        \begin{knot}[clip width=5]
        \strand[semithick] (4,-10) .. controls (4,-4) and (-4,-4) .. (-4,0);
        \strand[semithick] (4,10) .. controls (4, 4) and (-4, 4) .. (-4,0);
        \strand[semithick] (-4,-10) .. controls (-4,-4) and (4,-4) .. (4,0);
        \strand[semithick] (-4,10) .. controls (-4, 4) and (4,4) .. (4,0);
        \node[blue] at (-4,4)[left] {$a$};
        \node[blue] at (-4,-4)[left] {$-a$};
        \end{knot}
        \end{tikzpicture}
        $\stackrel{R_2}{\cong} \,\,$
        \begin{tikzpicture}[baseline=-0.65ex,scale=0.07]
        \begin{knot}[clip width=5]
        \strand[semithick] (4,-10) .. controls (4,-4) and (1,-4) .. (1,0);
        \strand[semithick] (4,10) .. controls (4, 4) and (1, 4) .. (1,0);
        \strand[semithick] (-4,-10) .. controls (-4,-4) and (-1,-4) .. (-1,0);
        \strand[semithick] (-4,10) .. controls (-4, 4) and (-1,4) .. (-1,0);
        \end{knot}
        \end{tikzpicture}}
        %%%
        \quad \fbox{
        \begin{tikzpicture}[baseline=-0.65ex,scale=0.07]
        \begin{knot}[clip width=5, flip crossing/.list={1,2,3}]
        \strand[semithick] (-10,-10) -- (10,10);
        \strand[semithick] (-10,10) -- (10,-10);
        \strand[semithick] (-10,-2) .. controls (-4, -2) and (-4,8) .. (0,8);
        \strand[semithick] (10,-2) .. controls (4, -2) and (4,8) .. (0,8);
        \node[blue] at (-6,4)[left] {$a$};
        \node[blue] at (6,4)[right] {$b$};
        \node[blue] at (0,-2)[below] {$c$};
        \end{knot}
        \end{tikzpicture}
        $\stackrel{R_3}{\cong} \,\,$
        \begin{tikzpicture}[baseline=-0.65ex,scale=0.07]
        \begin{knot}[clip width=5, flip crossing/.list={1,2,3}]
        \strand[semithick] (-10,-10) -- (10,10);
        \strand[semithick] (-10,10) -- (10,-10);
        \strand[semithick] (-10,2) .. controls (-4, 2) and (-4,-8) .. (0,-8);
        \strand[semithick] (10,2) .. controls (4, 2) and (4,-8) .. (0,-8);
        \node[blue] at (-6,-4)[left] {$a$};
        \node[blue] at (6,-4)[right] {$b$};
        \node[blue] at (0,2)[above] {$c$};
        \end{knot}
        \end{tikzpicture}}
    \]
    Na mocy twierdzenia Reidemeistera dowód został zakończony.
\end{proof}

Kawauchi definiuje jeszcze \emph{twisting number}: sumę spinów po wszystkich składowych.

% Koniec podsekcji Indeks zaczepienia

\subsection{Sygnatura} % (fold)
\label{sub:signature}
\index{sygnatura}
Sygnatura pojawia się w~fakcie \ref{slice_signature}.

\begin{definition}
    Sygnatura to niezmiennik topologiczny zadany (kłębiastą) relacją rekurencyjną:
    \begin{itemize}[leftmargin=*]
    \itemsep0em
        \item $\sigma (\LittleUnknot) = 0$,
        \item $\sigma (K_+) - \sigma (K_-) \in \{0, 2\}$,
        \item $4 \mid \sigma (K)$ wtedy i~tylko wtedy, gdy $\nabla(2i) > 0$ (wielomian Conwaya).
    \end{itemize}
\end{definition}

\begin{proposition} \label{prop_sigma_inverse}
    Mamy $\sigma(K^*) = -\sigma(K)$ oraz $\sigma(-K) = \sigma(K)$.
\end{proposition}

\begin{proof}
    To jest twierdzenie 6.4.5 z podręcznika \cite{murasugi96}.
\end{proof}

\begin{proposition} \label{prop_sigma_additive}
    Sygnatura jest addytywna: $\sigma(K_1 \shrap \ldots \shrap K_n) = \sum_{k=1}^n \sigma(K_k)$.
\end{proposition}

Węzły achiralne mają zerową sygnaturę, zatem trójlistnik nie jest achiralny.
Z faktów \ref{prop_sigma_inverse} oraz \ref{prop_sigma_additive} wynika, że suma tak samo skręconych trójlistników nie jest achiralna ($\sigma = \pm 4$), natomiast węzeł prosty (suma różnie skręconych) ma zerową sygnaturę i jak można przekonać się ze standardowego diagramu, jest achiralny.

Sygnatura pozwala uzyskać proste oszacowanie liczby gordyjskiej od dołu:

\begin{proposition}
    Mamy $2 u(K) \ge |\sigma(K)|$.
\end{proposition}

\begin{proof}
    To jest twierdzenie 6.4.8 z podręcznika \cite{murasugi96}.
\end{proof}

Nie istnieje bezpośredni związek między sygnaturą i~liczbą mostową.
Węzeł torusowy $T_{2,n}$ jest dwumostowy, jego sygnatura wynosi $n - 1$.
Suma spójna węzłów prostych ma zerową sygnaturę i~nieograniczoną liczbę mostową.
Wynika to z~ogólniejszego faktu:


Czy istnieje węzeł o~sygnaturze $4$ i~wyznaczniku postaci $n = 4k + 1$ dla $k$ całkowitego dodatniego?
Stojmenow twierdzi, że jeśli tak jest, to wszystkie pierwsze dzielniki $n$ dają resztę $1$ z~dzielenia przez $24$ i~są większe od $2857$.

% Koniec podsekcji Sygnatura

\subsection{Liczba patykowa} % (fold)
\label{sub:stick_index}
\index{liczba!patykowa}
Z angielskiego \emph{stick number}.

\begin{definition}
	Minimalną liczbę odcinków w~łamanej, która przedstawia węzeł $K$, nazywamy jego liczbą patykową i~oznaczamy $\operatorname{s}(K)$.
\end{definition}

Wielkość tę wprowadził do matematyki Randell w 1988 i~znalazł dokładną jej wartość dla niewęzła (3), trójlistnika (6) oraz ósemki (7).
Negami trzy lata później w~\cite{negami91} pokazał przy użyciu teorii grafów, że dla nietrywialnych węzłów prawdziwe są nierówności
\begin{equation}
    \frac{5+\sqrt{9 + 8 \operatorname{cr} K}}{2} \le \operatorname{s} K \le 2 \operatorname{cr} K.
\end{equation}

Trójlistnik to jedyny węzeł realizujący górne ograniczenie.
Z~pracy Elrifaia wynika, że dolne ograniczenie nie jest osiągane przez żaden węzeł o co najwyżej 26 skrzyżowaniach (\cite{elrifai06}).

Jin oraz Kim w 1993 ograniczyli liczby patykowe dla węzłów torusowych korzystając z~liczby supermostowej.
Wkrótce wynik został poprawiony przez samego Jina, w pracy \cite{jin97} znalazł dokładne wartości dla niektórych węzłów.
I~tak, jeśli $2 \le p < q \le 2p$, to $\operatorname{s} T_{p,q} = 2q$ oraz $\operatorname{s} T_{p, p-1} = 2$.
Ten sam wynik, choć dla węższego zakresu parametrów, odkryto w~\cite{greilsheimer97}.
Autorzy niezależnie od siebie znaleźli proste oszacowanie z góry dla liczby patykowej sumy spójnej:
\begin{equation}
	\operatorname{s}(K_1 \shrap K_2) \le \operatorname{s}(K_1) + \operatorname{s}(K_2) - 3.
\end{equation}

Koniec dekady przyniósł jeszcze jedną pracę McCabe'a z nierównością $\operatorname{s}(K) \le 3 + \operatorname{cr} (K)$ dla węzłów dwumostowych (\cite{mccabe98}) oraz odkrycie Calvo: jeśli ograniczymy się do łamanych o co najwyżej siedmiu odcinkach, ósemka przestaje być odwracalna.

Na początku XX wieku nierówności Negamiego poprawiono, z dołu dokonał tego Calvo w~\cite{calvo01}, z góry natomiast Huh, Oh w \cite{huh11}.
Górne ograniczenie można poprawić o $3/2$, jeżeli $K$ jest niealternującym węzłem pierwszym.
\begin{equation}
    \frac{7+\sqrt{1 + 8 \operatorname{cr} K}}{2} \le \operatorname{s} K \le \frac{3}{2} (1 + \operatorname{cr} K).
\end{equation}

% Koniec podsekcji Liczba patykowa

\subsection{Długość sznurowa} % (fold)
\label{sub:ropelength}
Długość sznurowa, z~angielskiego \emph{ropelength}, pochodzi z~fizycznej teorii węzłów, która bierze pod uwagę obiekty wykonane z~nieelastycznych materiałów

Długość sznurowa $L$ to stosunek długości węzła do jego grubości $\tau$ (mówimy, że węzeł $K$ jest grubości $\tau$, jeśli ma otoczenie rurowe bez samoprzecięć z~przekrojem poprzecznym o~promieniu $\tau$).
Przez wiele lat zastanawiano się, czy można zawiązać węzeł ze sznura o~długości jednej stopy i~promieniu jednego cala.
Nie jest to możliwe: rozumowanie oparte o~czterosieczne pokazuje, że długość sznurowa nietrywialnego węzła wynosi co najmniej $15.66$ (dla trójlistnika jest to co najmniej $16.372$).

Węzeł realizujący długość sznurową jest klasy $C^1$.

Prawdziwe są oszacowania asymptotyczne:
\[
    L = \Omega (\operatorname{cr}^{3/4}),  \quad
    L = O(\operatorname{cr} \log^5 \operatorname{cr})
\]

% Koniec podsekcji Ropelength

% Koniec sekcji Niezmienniki liczbowe


\chapter{Niezmienniki kolorowe}
Zamknęliśmy już pierwszy rozdział,
Wciąż nie pokazaliśmy jednak pełnego dowodu, że dwa konkretne węzły (na przykład niewęzeł i~trójlistnik) są od siebie różne.
Dlatego teraz podamy proste narzędzie odróżniające węzły: \emph{trójkolorowalność}, która przypisuje włóknom diagramu różne kolory.
Następnie rozszerzymy paletę do dowolnie wielu kolorów i~zastąpimy ją grupą skończoną.
Nawet wzmocniony wariant nie jest idealnym narzędziem klasyfikującym.
Istnieją węzły, których nie odróżnia.
Problem ten dotyka wielu późniejszych niezmienników, pierwszy niezmiennik zupełny poznamy dopiero w~rozdziale czwartym.
Nie stanowi to wielkiego powodu do radości ze względu na trudności w~jego wyznaczaniu.

\section{Kolorowanie splotów} % (fold)
\label{sec:colour_links}
Oto mniej mętny opis trójkolorowalności, czyli jak nietrudno się domyślić, kolorowalności trzema kolorami.
Diagram $D$ splotu $K$ jest trójkolorowalny, jeśli każdemu włóknu można przypisać jeden z~trzech kolorów tak, by co najmniej dwa zostały użyte.
Wymagamy przy tym, by przy żadnym skrzyżowaniu nie spotykały się dokładnie dwa kolory.

Dla własnej wygody jako kolorów używać będziemy kolejnych liczb naturalnych $0, 1, 2, \ldots$.
Pozwala to zapisać warunek kolorowalności równaniem algebraicznym.

\begin{definition}[kolorowanie] \label{def:colour_equation}
    Niech $L$ będzie splotem, zaś $n$ liczbą naturalną.
    Mówimy, że splot $L$ jest kolorowalny modulo $n$, jeśli posiada diagram, którego włóknom można przypisać liczby całkowite $0, \ldots, n - 1$ tak, by
    \begin{enumerate}[leftmargin=*]
        \item istniały dwa włókna różnych kolorów,
        \item równanie $a + b \equiv 2c$ modulo $n$ było spełnione przy każdym skrzyżowaniu:
    \end{enumerate}
    \[
        \begin{tikzpicture}[baseline=-0.65ex, scale=0.12]
            \useasboundingbox (-5, -5) rectangle (5,5);
            \begin{knot}[clip width=5, end tolerance=1pt, flip crossing/.list={1}]
                \strand[semithick] (-5,5) to (5,-5);
                \strand[semithick] (-5,-5) to (5,5);
                \node[darkblue] at (5, 5)[below right] {$c$};
                \node[darkblue] at (5, -5)[above right] {$b$};
                \node[darkblue] at (-5, 5)[below left] {$a$};
            \end{knot}
        \end{tikzpicture}.
    \]
    Takie przyporządkowanie nazywamy (nietrywialnym) kolorowaniem.
\end{definition}

Metoda ta została odkryta razem z~uogólnieniem do $n$ kolorów przez Ralpha Foxa w~1956, kiedy próbował uczynić teorię węzłów bardziej przystępną dla studentów.
Opierając się tylko na definicji kolorowania oraz ruchach Reidemeistera możemy wykazać pierwsze własności kolorowań.

\begin{proposition} \label{color_invariant}
    ,,Bycie $n$-kolorowalnym'' jest niezmiennikiem węzłów.
\end{proposition}

\begin{proof}
    Wystarczy sprawdzić, jak ruchy Reidemeistera zmieniają kolory.
    Pierwszy i~drugi:
    \[
        \fbox{
        \begin{tikzpicture}[baseline=-0.65ex,scale=0.07]
        \begin{knot}[clip width=5]
            \strand[semithick] (-10,10) .. controls (-10,2) and (-10,2) .. (-6,-2);
            \strand[semithick] (-10,-10) .. controls (-10,-2) and (-10,-1) .. (-9,0);

            \strand[semithick] (-7,1) -- (-6,2);
            \strand[semithick] (-6,2) .. controls (2,9) and (2,-9) .. (-6,-2);
            \node[darkblue] at (-10, 10)[below left] {$a$};
            \node[darkblue] at (-10, -10)[above left] {$b \equiv a$};
        \end{knot}
        \end{tikzpicture}
        $\stackrel{R_1}{\cong} \,\,$
        \begin{tikzpicture}[baseline=-0.65ex,scale=0.07]
        \begin{knot}[clip width=5]
            \strand[semithick] (0,10) -- (0,-10);
            \node[darkblue] at (0, 0)[left] {$a$};
        \end{knot}
        \end{tikzpicture}}
        %%%
        \quad \fbox{
        \begin{tikzpicture}[baseline=-0.65ex,scale=0.07]
        \begin{knot}[clip width=5]
            \strand[semithick] (4,-10) .. controls (4,-4) and (-4,-4) .. (-4,0);
            \node[darkblue] at (-4, -10)[above left] {$d \equiv b$};
            \strand[semithick] (4,10) .. controls (4, 4) and (-4, 4) .. (-4,0);
            \node[darkblue] at (4, 10)[below right] {$a$};
            \strand[semithick] (-4,-10) .. controls (-4,-4) and (4,-4) .. (4,0);
            \node[darkblue] at (4, 0) [right] {$c \equiv 2a-b$};
            \strand[semithick] (-4, 10) .. controls (-4, 4) and (4,4) .. (4,0);
            \node[darkblue] at (-4, 10) [below left] {$b$};
        \end{knot}
        \end{tikzpicture}
        $\stackrel{R_2}{\cong} \,\,$
        \begin{tikzpicture}[baseline=-0.65ex,scale=0.07]
        \begin{knot}[clip width=5]
            \strand[semithick] (4,-10) .. controls (4,-4) and (1,-4) .. (1,0);
            \strand[semithick] (4,10) .. controls (4, 4) and (1, 4) .. (1,0);
            \strand[semithick] (-4,-10) .. controls (-4,-4) and (-1,-4) .. (-1,0);
            \strand[semithick] (-4,10) .. controls (-4, 4) and (-1,4) .. (-1,0);
        \end{knot}
        \end{tikzpicture}}
    \]
    Trzeci ruch także nie wymaga skomplikowanych rachunków.
    Najkrótszy łuk na diagramach ma kolor $2a-c$ po lewej oraz $2b-c$ po prawej stronie.
    \[
     \fbox{
        \begin{tikzpicture}[baseline=-0.65ex,scale=0.07]
        \begin{knot}[clip width=5, flip crossing/.list={1,2,3}]
            \node[darkblue] at (-10, 10) [above] {$b$};
            \node[darkblue] at (10, 10) [above] {$c$};
            \node[darkblue] at (-10, -10) [below] {$2a-2b+c$};
            \node[darkblue] at (10, -10) [below] {$2a-b$};
            \node[darkblue] at (-10, -2) [left] {$a$};
            \strand[semithick] (-10,-10) -- (10,10);
            \strand[semithick] (-10,10) -- (10,-10);
            \strand[semithick] (-10,-2) .. controls (-4, -2) and (-4,8) .. (0,8);
            \strand[semithick] (10,-2) .. controls (4, -2) and (4,8) .. (0,8);
        \end{knot}
        \end{tikzpicture}
        $\stackrel{R_3}{\cong} \,\,$
        \begin{tikzpicture}[baseline=-0.65ex,scale=0.07]
        \begin{knot}[clip width=5, flip crossing/.list={1,2,3}]
            \node[darkblue] at (-10, 10) [above] {$b$};
            \node[darkblue] at (10, 10) [above] {$c$};
            \node[darkblue] at (-10, -10) [below] {$2a-2b+c$};
            \node[darkblue] at (10, -10) [below] {$2a-b$};
            \node[darkblue] at (10, 2) [right] {$a$};
            \strand[semithick] (-10,-10) -- (10,10);
            \strand[semithick] (-10,10) -- (10,-10);
            \strand[semithick] (-10,2) .. controls (-4, 2) and (-4,-8) .. (0,-8);
            \strand[semithick] (10,2) .. controls (4, 2) and (4,-8) .. (0,-8);
        \end{knot}
        \end{tikzpicture}} \qedhere
    \]
\end{proof}

Trójlistnik koloruje się dokładnie modulo krotności trójki, ósemka zaś -- piątki.
Sama kolorowalność nie mówi wiele, splot jest kolorowalny lub nie.
Dowód faktu \ref{color_invariant} pokazuje coś więcej: liczba kolorowań, być może trywialnych, jest mocniejszym niezmiennikiem.
Liczbę kolorowań splotu $L$ modulo $n$, trywialnych lub nie, oznaczamy przez $\tau_n(L)$.

\begin{lemma}
    \label{lem:colouring_arc}
    Niech $D$ będzie diagramem z definicji \ref{def:colour_equation} z wybranym łukiem, zaś $k \in \{0, \ldots, n - 1\}$ pewnym kolorem.
    Bez straty ogólności możemy założyć, że krótki łuk jest koloru $k$.
\end{lemma}

Kolorem tym zazwyczaj jest $0$.

\begin{proof}
    Dodanie tej samej wartości do wszystkich łuków na dobrze pokolorowanym diagramie daje nowy, także dobrze pokolorowany diagram.
\end{proof}

\begin{proposition}
    \label{no_knots_colours_mod_two}
    Żaden węzeł nie koloruje się modulo dwa.
\end{proposition}

\begin{proof}
    Załóżmy nie wprost, że istnieje nietrywialne kolorowanie.
    Analiza czterech możliwych skrzyżowań pokazuje, że włókna wychodzące z~tunelu muszą mieć ten sam kolor.
    Przechodząc wzdłuż węzła widzimy jeden kolor, wbrew założeniu nie wprost.
\end{proof}

\begin{proposition}
    Każdy splot o co najmniej dwóch ogniwach koloruje się modula dwa.
\end{proposition}

\begin{proof}
    Wystarczy pomalować jedną składową zerem, a~pozostałe jedynkami.
\end{proof}

Sploty rozszczepialne są $n$-kolorowalne dla każdego $n \ge 2$, można skorzystać z~tego samego schematu kolorowania.
Pierścienie Boromeuszy nie kolorują się modulo trzy, nie są zatem rozszczepialne.
Sploty, które nie są kolorowalne modulo $n$ dla każdej liczby $n \in \N$ nazywa się czasem niewidzialnymi, dwa węzły do dziesięciu skrzyżowań mają tę własność: $10_{124}$ oraz $10_{153}$.

Pokażemy teraz, że suma równań kolorujących z dobrze wybranymi znakami jest postaci $0 \equiv 0 \mod n$.
Będziemy potrzebować kilku pomocniczych definicji.
Każdy diagram węzła rozcina płaszczyznę na obszary, z~czego jeden jest nieograniczony.

\begin{definition}[uszachowienie]
    Przyporządkowanie każdemu z~obszarów, na jakie diagram rozcina płaszczyznę, jednego z~dwóch kolorów tak, by sąsiadujące ze sobą obszary były różnych kolorów, nazywamy uszachowieniem.
\end{definition}

Ustalmy węzeł $K$ oraz dowolne uszachowienie dla jego diagramu.
Skojarzmy z~każdym skrzyżowaniem równanie kolorujące, zgodnie z~poniższym schematem:
\[\begin{tikzpicture}[baseline=-0.65ex, scale=0.12]
    \useasboundingbox (-5, -10) rectangle (5,5);
    \begin{knot}[clip width=5, end tolerance=1pt, flip crossing/.list={1}]
        \strand[semithick] (-5,5) to (5,-5);
        \strand[semithick] (-5,-5) to (5,5);
        \fill[blue!20!white] (-4, 5) to (0, 1) to (4, 5);
        \fill[blue!20!white] (-4, -5) to (0, -1) to (4, -5);
        \node[darkblue] at (-5, -5)[left] {$a$};
        \node[darkblue] at (-5, +5)[left] {$b$};
        \node[darkblue] at (+5, -5)[right] {$c$};
        \node[darkblue] at (+5, +5)[right] {$a$};
        \node[darkblue] at (0, -10) {$+a-b+a-c=0 \mod n$};
    \end{knot}
    \end{tikzpicture}
    \quad\quad\quad\quad\quad\quad\quad\quad\quad\quad\quad\quad
    \begin{tikzpicture}[baseline=-0.65ex, scale=0.12]
    \useasboundingbox (-5, -10) rectangle (5,5);
    \begin{knot}[clip width=5, end tolerance=1pt, flip crossing/.list={1}]
        \strand[semithick] (-5,5) to (5,-5);
        \strand[semithick] (-5,-5) to (5,5);
        \fill[blue!20!white] (5, -4) to (1, 0) to (5, 4);
        \fill[blue!20!white] (-5, -4) to (-1, 0) to (-5, 4);
        \node[darkblue] at (-5, -5)[left] {$a$};
        \node[darkblue] at (-5, +5)[left] {$b$};
        \node[darkblue] at (+5, -5)[right] {$c$};
        \node[darkblue] at (+5, +5)[right] {$a$};
        \node[darkblue] at (0, -10) {$-a+b-a+c=0 \mod n$};
    \end{knot}
    \end{tikzpicture}
\]

\begin{proposition}
    \label{prp:colouring_sum_zero}
    Sumą równań kolorujących o dobrze wybranych znakach jest $0 \equiv 0 \mod n$.
\end{proposition}

Będziemy potrzebować tego do pokazania, że wyznacznik determinuje kolorowalność splotu.

\begin{proof}
    Każde równanie kolorujące składa się z~czterech wyrazów, po jednym od każdej nici, która spotyka się w~danym skrzyżowaniu.
    Nić biegnie między dwoma skrzyżowaniami, więc suma wszystkich równań kolorujących składa się z~par składników, po jednej parze na nić.
    Składniki te są przeciwnych znaków, zatem wzajemnie się znoszą.
    Suma równań kolorujących jest sumą zer, a~to należało udowodnić.
\end{proof}

\begin{proposition}
    Jeśli $K, L$ są węzłami, to $3\tau_3(K \shrap L) = \tau_3(K)\tau_3(L)$.
\end{proposition}

\begin{corollary}
    Istnieje nieskończenie wiele węzłów.
\end{corollary}

\begin{proof}
    Suma spójna $n$ trójlistników ma $3^n$ (być może trywialnych) $3$-kolorowań.
\end{proof}

Jako kolorów użyjemy teraz elementów $g_1, \ldots, g_n$ pewnej skończonej grupy $G$.

\begin{definition}[etykietowanie]
    Mówimy, że zorientowany węzeł $K$ jest etykietowalny grupą $G$ generowaną przez elementy $g_1, \ldots, g_n$, jeśli posiada diagram, którego włóknom przypisano elementy $g_1, \ldots, g_n$ tak, by równanie $gk=hg$ było spełnione przy każdym skrzyżowaniu ($g$: włókno biegnące górą, $k$: bo jego lewej stronie, $h$: po prawej).
    \[
        \begin{tikzpicture}[baseline=-0.65ex, scale=0.12]
            \useasboundingbox (-5, -5) rectangle (5,5);
            \begin{knot}[clip width=5, end tolerance=1pt, flip crossing/.list={1}]
                \strand[semithick] (-5,5) to (5,-5);
                \strand[semithick, -Latex] (-5,-5) to (5,5);
                \node[darkblue] at (5, 5)[below right] {$g$};
                \node[darkblue] at (5, -5)[above right] {$h$};
                \node[darkblue] at (-5, 5)[below left] {$k$};
            \end{knot}
        \end{tikzpicture}
    \]
\end{definition}

Równanie $gkg^{-1}=h$ mówi, że etykiety włókien wchodzących oraz wychodzących są sprzężone.
Wynika stąd, że wszystkie etykiety pochodzą z~jednej klasy sprzężoności.
Muszą jednocześnie generować całą grupę, dlatego $G$ musi być grupą nieprzemienną lub trywialną.
Etykietowalność jest niezmiennikiem węzłów i~nie zależy od orientacji węzła:
jeżeli elementy $g_1, \ldots, g_n$ generują grupę, to ich odwrotności także.

Rozpatrzmy węzły $6_1$ oraz $9_{46}$ i~spróbujmy etykietować je transpozycjami z~grupy $S_4$.
Wybranie dwóch etykiet przy jednym skrzyżowaniu $6_1$ wymusza etykiety dla wszystkich włókien.
Dwie transpozycje nie mogą generować grupy $S_4$, natomiast włókna węzła $9_{46}$ dają się etykietować samymi transpozycjami.
Węzły te są więc różne, choć mają te same własności kolorujące.

Etykietowanie jest mocnym narzędziem odróżniającym węzły.
Thistlethwaite w 1985 roku korzystając z niego klasyfikował węzły o~co najwyżej 13 skrzyżowaniach (jest ich, jak ostatecznie się okazało, 12965).
Mają one tylko 5639 różnych wielomianów Alexandera, ale etykietowania trzynastoma różnymi grupami pozwoliły zmniejszyć liczbę nierozpoznanych węzłów do około tysiąca.
Wśród nich 30 posiada wielomian Conwaya $1 + 2z^2 + 2z^4$, ale pary rozróżniane wielomianem HOMFLY mają też różne wielomiany Jonesa.
Wielomiany opisujemy w~rozdziale trzecim.

Niech $p \ge 3$ będzie liczbą pierwszą, natomiast $D_p = \langle r, s \mid r^p = s^2 = e, rsr = s \rangle$ grupą diedralną rzędu $2p$.
Elementy tej grupy to $1, r, r^2, \ldots, r^{p-1}, s, sr, \ldots, sr^{p-1}$.
,,Obrót'' $r^k$ jest sprzężony tylko ze swoją odwrotnością, ale ,,symetrie osiowe'' $sr^k$ tworzą jedną klasę sprzężoności.
Łatwo widać, że dowolne dwie z~nich generują całą grupę $D_p$.

\begin{proposition}
    Węzeł $K$ jest $p$-kolorowalny wtedy i~tylko wtedy, gdy jest $D_p$-etykietowalny.
\end{proposition}

\begin{proof}
    Załóżmy, że $K$ ma $n$ włókien.
    Wiemy już, że każde $D_p$-etykietowanie wykorzystuje tylko elementy $sr^{a_1}, \ldots, sr^{a_n}$ dla $1 \le a_i \le p$.
    Jest ono prawidłowe dokładnie wtedy, gdy analogiczne kolorowanie liczbami $a_1, \ldots, a_n$ jest prawidłowe.
\end{proof}

Kolorowania definiowano kiedyś jako surjekcje $\rho \colon \pi \to D_{2n}$ z~grupy podstawowej.
Jak mówi prezentacja Wirtingera, grupa splotu generowana jest przez ścieżki z~punktu bazowego w~$S^3$ do brzegu rurowego otoczenia splotu, wokół południka i~znowu do bazowego punktu.
Fox zauważył, że z~surjektywności $\rho$ wynika, iż generatory mapują się na symetrie osiowe $sr^k$.
Ponieważ istnieje wzajemnie jednoznaczna odpowiedniość między generatorami grupy splotu oraz łukami diagramu, każdemu możemy przypisać liczbę całkowitą $k$.
Etykietowania są więc uogólnieniem kolorowań.
Rozumowanie, które przedstawiliśmy, prowadzi do prostej klasyfikacji grup, których można użyć do etykietowania.

\begin{proposition}
    Niech $K$ będzie węzłem, $\pi$ grupą podstawową jego dopełnienia, zaś $G$ dowolną grupą.
    Następujące warunki są równoważne: $K$ jest $G$-etykietowalny; istnieje surjekcja $\pi_1 \to G$.
\end{proposition}

Historycznie, prezentacja Wirtingera była pierwsza, zaś etykietowania odkryto później.

\begin{proposition}[Perko]
    Niech $K$ będzie węzłem etykietowalnym grupą $S_3$.
    Wtedy $K$ jest też etykietowalny grupą $S_4$.
\end{proposition}

Nie znam innych nietrywialnych faktów dotyczących etykietowań.

% Koniec sekcji Kolorowanie splotów

\section{Macierz i wyznacznik} % (fold)
\label{sec:colour_matrix}
Zajmiemy się teraz wyznacznikiem, pierwszym nieoczywistym niezmiennikiem splotów, który przypisuje każdemu pewną liczbę całkowitą.
Jest on blisko związany z kolorowaniem.
Zauważmy, że pierwszy ruch Reidemeistera usuwa zamknięte krzywe, czyli pojedyncze łuki bez skrzyżowań.
Diagram bez takich krzywych ma tyle samo skrzyżowań, co łuków.

\begin{definition}[macierz kolorująca]
    Ustalmy diagram bez zamkniętych krzywych dla splotu $L$ z łukami $x_0, \ldots, x_m$ oraz skrzyżowaniami $0, \ldots, m$.
    Definiujemy macierz $A$, której wyraz $a_{lj}$ jest współczynnikiem przy $x_j$ w $l$-tym równaniu kolorującym: $x_j+x_k - 2x_i \equiv 0 \mod n$.
    Macierz kolorująca $A_+$ powstaje z macierzy $A$ przez skreślenie dowolnego wiersza i kolumny.
    \[\begin{tikzpicture}[baseline=-0.65ex, scale=0.12]
    \useasboundingbox (-5, -5) rectangle (5,5);
    \begin{knot}[clip width=5, end tolerance=1pt, flip crossing/.list={1}]
        \strand[semithick] (-5,5) to (5,-5);
        \strand[semithick] (-5,-5) to (5,5);
        \node[darkblue] at (5, 5)[below right] {$x_i$};
        \node[darkblue] at (5, -5)[above right] {$x_j$};
        \node[darkblue] at (-5, 5)[below left] {$x_k$};
    \end{knot}
    \end{tikzpicture}\]
\end{definition}

Taka macierz jest kwadratowa, ponieważ z każdego skrzyżowania wychodzą (tunelem) dwa włókna mające dwa końce.
Wykreślenie wiersza i kolumny jest konieczne.
Gdybyśmy tego zaniechali, otrzymana macierz nie byłaby odwracalna, bowiem wiersze sumują się do zera.
Dla alternujących diagramów możemy żądać, by górą $i$-tego skrzyżowania biegło $i$-te włókno, wtedy na diagonali macierzy $A$ znajdą się same minus dwójki.

\begin{definition}[wyznacznik]
    Wyznacznik splotu różnego od niewęzła to wyznacznik jego macierzy kolorującej bez znaku.
    Wyznacznikiem niewęzła jest $1$.
\end{definition}

\begin{definition}[defekt]
    Defekt (wymiar jądra) macierzy kolorującej modulo $p$ nazywamy defektem węzła.
\end{definition}

Defekty modulo różne liczby pierwsze są niezależne od siebie.
Na przykład suma spójna $k$ trójlistników i $j$ węzłów$T_{2,5}$ posiada defekt $k$ modulo $3$ oraz $j$ modulo $5$.
Podobne przykłady istnieją dla innych zbiorów liczb pierwszych.

Pokażemy później (po poznaniu grupy kolorującej, czyli we wniosku \ref{det_invariant}) lub jeszcze później (po wprowadzeniu wielomianu Alexandera, w dowodzie faktu \ref{alexander_invariance}), że wyznacznik splotu jest dobrze określony: nie zależy on od wyboru etykietowania, minora macierzy oraz diagramu i że jest niezmiennikiem.
Teraz ograniczymy się do jego kilku własności.

Defekt także jest niezmiennikiem, choć rzadziej używanym.
Węzeł o defekcie $n$ modulo $p$ posiada $p(p^n-1)$ kolorowań $p$ kolorami.
Węzły $8_{18}$ oraz $9_{24}$ mają ten sam wyznacznik, $45$.
Ich defekty modulo $3$ to $1$ i $2$, zatem są różne.

% \begin{proof}
%     \emph{Krok pierwszy}.
%     Pokażemy, że żaden ruch Reidemeistera nie zmienia wyznacznika.
%     \begin{enumerate}
%         \item \emph{Ruch $R_1$}. Diagram przed lub po ruchu zawiera co najmniej jedno włókno, które łączy tunel z mostem pewnego skrzyżowania.
%         \item \emph{Ruch $R_2$}.
%         \item \emph{Ruch $R_3$}.
%     \end{enumerate}

%     \emph{Krok drugi}.
%     Niech $A_{i,j}$ oznacza minor powstały przez skreślenie $i$-tego wiersza oraz $j$-tej kolumny.
%     Pokażemy, że wartość wyznacznika nie zależy od wyboru $i$ oraz $j$.

%     Niech $X$ będzie macierzą $k \times k$ złożoną z samych jedynek.
%     Suma elementów w każdej kolumnie oraz każdym wierszu macierzy $A + X$ wynosi $k$, ponieważ znaki równań zostały dobrze wybrane.
%     Wykonujemy kolejno operacje:
%     \begin{enumerate}
%         \item Dodajemy do $i$-tego wiersza sumę pozostałych.
%         \item Dodajemy do $j$-tej kolumny sumę pozostałych.
%         Teraz $i$-ty wiersz oraz $j$-ta kolumna zawierają wyrazy $k$ z wyjątkiem $a_{ij}$, który wynosi $k^2$.
%         \item Wyciągamy $k$ z $i$-tego wiersza przed wyznacznik.
%         \item Odejmujemy $i$-ty wiersz od pozostałych.
%     \end{enumerate}
%     Rozwinięcie Laplace'a względem $j$-tej kolumny mówi, że $|\det (A+X)| = k^2 |(-1)^{i+j} \det A_{i,j}|$, co kończy dowód drugiego kroku.

%     \emph{Krok trzeci}.
%     Pokażemy, że zmiana etykietowania nie zmienia wyznacznika.
% \end{proof}

\begin{proposition}
    Splot $L$ koloruje się modulo $n$ wtedy i tylko wtedy, gdy liczby $\det L$ oraz $n$ nie są względnie pierwsze.
\end{proposition}

\begin{proof}
    Bez straty ogólności ograniczmy się do tych kolorowań, gdzie $x_0 = 0$.
    Kolorowanie modulo $n$ istnieje dokładnie wtedy, gdy istnieje niezerowy wektor $(x_1, x_2, \ldots, x_m)$ taki, że $Ax \equiv 0 \mod n$.
    Ale z algebry liniowej wiemy, że dla pewnych całkowitoliczbowych macierzy $C, R$ macierz $RAC = diag(y_1, \ldots, y_m)$ jest diagonalna.
    Warunek z faktu tłumaczy się wtedy na istnienie rozwiązania dla przynajmniej jednego z równań $x_iy_i \equiv 0 \mod n$.
\end{proof}

Poniższy problem pochodzi od Stojmenowa.

\begin{conjecture}
    Niech $n$ będzie nieparzystą sumą dwóch kwadratów różną od $1, 9, 49$.
    Czy istnieje pierwszy, alternujący, achiralny węzeł o wyznaczniku $n$?
\end{conjecture}

Oto, co już wiemy.
Wyznacznik węzła achiralnego jest nieparzystą sumą dwóch kwadratów.
Implikacja odwrotna także jest prawdziwa, od węzła można dodatkowo żądać bycia pierwszym albo achiralnym (ale nie jednocześnie).
Jeśli węzeł z hipotezy istnieje, to $n > 2000$ nie jest kwadratem.

Istnieje jeszcze jedna kombinatoryczna metoda badania węzłów, która prowadzi między innymi do pojęcia wyznacznika.
W latach 30. ubiegłego wieku L. Goeritz pokazał, jak diagram węzła wyznacza specjalną formę kwadratowej.
Nieco później H. F. Trotter zmodyfikował jego pomysł, by sygnatura formy stanowiła niezmiennik splotów.
Gordon, Litherland ujednolicili dwa wyżej wymienione podejścia w pracy \cite{litherland81}.
My opiszemy krótko macierz Goeritza, gdyż jej wyznaczenie wymaga mniejszej ilości rachunków (niż macierz kolorująca).
Ustalmy diagram uszachowiony $D$ dla splotu $L$.
Oznaczmy białe regiony $0,1,\ldots,m$, przy czym $0$ jest regionem nieograniczonym.
Przydzielmy skrzyżowaniom znaki:
    \[\begin{tikzpicture}[baseline=-0.65ex, scale=0.12]
    \useasboundingbox (-5, -5) rectangle (5,5);
    \begin{knot}[clip width=5, end tolerance=1pt, flip crossing/.list={1}]
        \strand[semithick] (-5,5) to (5,-5);
        \strand[semithick] (-5,-5) to (5,5);
        \fill[blue!20!white] (-4, 5) to (0, 1) to (4, 5);
        \fill[blue!20!white] (-4, -5) to (0, -1) to (4, -5);
        \node[darkblue] at (-5, 0) {$+1$};
    \end{knot}
    \end{tikzpicture}
    \quad\quad\quad
    \quad\quad\quad
    \begin{tikzpicture}[baseline=-0.65ex, scale=0.12]
    \useasboundingbox (-5, -5) rectangle (5,5);
    \begin{knot}[clip width=5, end tolerance=1pt]
        \strand[semithick] (-5,5) to (5,-5);
        \strand[semithick] (-5,-5) to (5,5);
        \fill[blue!20!white] (-4, 5) to (0, 1) to (4, 5);
        \fill[blue!20!white] (-4, -5) to (0, -1) to (4, -5);
        \node[darkblue] at (-5, 0) {$-1$};
    \end{knot}
    \end{tikzpicture}\]

\begin{definition}
    Macierz Goeritza powstaje przez skreślenie z macierzy $G_+$ jednego wiersza oraz jednej kolumny:
    \[
        G_+=\begin{pmatrix}
        G_{00} & \cdots & G_{0m} \\
        \vdots & \ddots & \vdots \\
        G_{m0} & \cdots & G_{mm}
        \end{pmatrix},
    \]
    gdzie jeśli $i\neq j$, to $G_{ij}$ jest sumą znaków skrzyżowań przyległych do $i$ oraz $j$.
    Dla $i = j$, $G_{ii}$ jest minus sumą znaków skrzyżowań wokół $j$-tego obszaru.
\end{definition}

Macierz $G_+$ posiada dwie własności pozwalające wykryć proste błędy rachunkowe: jest symetryczna, a jej kolumny i wiersze sumują się do zera.

\begin{proposition}
    Macierz Goeritza oraz kolorująca mają ten sam wyznacznik (bez znaku).
\end{proposition}

Nie możemy niestety podać dowodu tego faktu, wymaga bowiem znajomości topologii algebraicznej, której wolelibyśmy nie zakładać.
Macierz Goeritza nie jest niezmiennikiem splotów.
Jeśli jednak równoważnym diagramom $D_1, D_2$ odpowiadają macierze $G_1, G_2$, to można między nimi przejść skończoną liczbą dwóch ruchów:
\begin{enumerate}[leftmargin=*]
\itemsep0em
    \item zamiany macierzy $G$ na $PGP^{-1}$, gdzie $P$ i $P^{-1}$ mają całkowite wyrazy
    \item dopisania lub skreślenia $\pm 1$ na przekątnej (dla węzłów) albo $-1, 0, 1$ (dla splotów).
\end{enumerate}

% \begin{proposition}
%     Wyznacznik jest niezmiennikiem splotów.
%     Jeśli diagramy $D_1, D_2$ przedstawiają ten sam splot, to od macierzy Goeritza $G_1$ pierwszego do macierzy $G_2$ drugiego można dojść, wykonując trzy rodzaje ruchów:
%     \begin{enumerate}
%         \item zamieniając $G$ z $P^t G P$, gdzie macierz $P$ jest całkowitoliczbowa i $\det P = \pm 1$.
%         \item zamieniając $G$ z
%         \[
%             \begin{pmatrix}
%             G & 0 \\
%             0 & k
%             \end{pmatrix},
%         \]
%         gdzie $k \in \{-1, 0, 1\}$.
%         Dla węzłów można ograniczyć się do $k = \pm 1$.
%     \end{enumerate}
% \end{proposition}

% Dowód opiera się na prostych rachunkach i ruchach Reidemeistera.

% Z macierzy Goeritza można otrzymać sygnaturę: $\sigma(G) - \mu$, gdzie $\mu$ to to suma znaków skrzyżowań o odpowiednio dobranej orientacji.

% Koniec sekcji Macierz i wyznacznik

\section{Grupa splotu. Prezentacja Wirtingera} % (fold)
\label{sec:group_wirtinger}

Ponieważ dopełnienie dowolnego splotu, zarówno w przestrzeni $\R^3$ jak i $S^3$, jest łukowo spójne, jego grupa podstawowa nie zależy od wyboru punktu bazowego.
Dzięki temu poniższa definicja ma sens:

\begin{definition}
    \label{def:knot_group}
    \index{grupa!węzła}
    Niech $L$ będzie splotem.
    Grupę podstawową jego dopełnienia, $\pi_1(\R^3 \setminus L)$, nazywamy grupą splotu.
\end{definition}

Kiedy mówimy o~grupie węzła, zazwyczaj mamy na myśli obiekt opisany powyżej, a nie grupę kolorującą z~definicji \ref{colgrp_def}.
Nie należy ich mylić, grupa węzła ma bowiem dużo większe znaczenie.

Podamy teraz kilka przykładów węzłów oraz ich grup.

\begin{example}
    Niewęzeł: $\Z$.
\end{example}

\begin{example}
    Trójlistnik: grupa warkoczowa $B_3 \cong \langle x, y \mid x^2 = y^3\rangle$.
\end{example}

\begin{proof}
    Wynika to z równości
    % https://en.wikipedia.org/wiki/Tietze_transformations
    \begin{align}
        \pi_1(S^3 \setminus 3_1) & = \langle x, y, z \mid xz = yx, zy = xz, yx = zy \rangle \\
                                 & = \langle x, y \mid xyx = yxy \rangle \\
                                 & = \langle x, y, a, b \mid xyx = yxy, a = yx, b = xyx \rangle \\
                                 & = \langle x, a, b \mid xa = a^2x^{-1}, b = xa \rangle \\
                                 & = \langle a, b \mid b = a^2(ba^{-1})^{-1} \rangle \\
                                 & = \langle a, b \mid a^3 = b^2 \rangle,
    \end{align}
    prawdziwych na mocy transformacji Tietzego.
\end{proof}

\begin{example}
    Węzeł $(p,q)$-torusowy: $\langle x, y \mid x^p = y^q \rangle$.
\end{example}

\begin{example}
    Węzeł ósemkowy: $\langle x, y \mid yxy^{{-1}}xy=xyx^{{-1}}yx \rangle$.
\end{example}

\begin{proposition}
    \label{prop:knot_group_invariant}
    Grupa węzła jest niezmiennikiem węzłów.
\end{proposition}

\begin{proof}
    Gdy dwa węzły są równoważne, istnieje izotopijny z~identycznością homeomorfizm $\R^3 \to \R^3$, który posyła pierwszy węzeł na drugi.
    Obcięty do dopełnień węzłów indukuje izomorfizm grup podstawowych.
\end{proof}

\begin{proposition}
    Grupa węzła jest niezmiennikiem mocniejszym od genusu, a~w~przypadku węzłów złożonych, także od indeksu mostowego.
\end{proposition}

\begin{proof}[Niedowód]
    Jest to wniosek 3 z~pracy \cite{feustel78}.
\end{proof}

\begin{proposition}
    Niech $K_1, K_2$ będą węzłami pierwszymi.
    Jeżeli ich grupy są izomorficzne, to same węzły są równoważne.
\end{proposition}

\begin{proof}
    Jak piszą Gordon, Luecke w \cite{gordon89}, jest to bezpośredni wniosek z ich twierdzenia 2: nietrywialna chirurgia Dehna na nietrywialnym węźle nigdy nie daje $S^3$.
\end{proof}

Wcześniej Whitten wiedział tylko, że jeśli węzły pierwsze mają izomorficzne grupy, to dopełnienia tych węzłów są homeomorficzne.
Jak sam wspomina w \cite{whitten87}: ,,\emph{The group of a prime knot does not, however, necessarily determine the topological type of the exterior. Dehn hips on certain “essential” solid tori in the exteriors of torus knots and of cable knots produce Haken manifolds that are homotopically equivalent but not homeomorphic to the original exteriors and that, in fact, cannot be imbedded in $S^3$.}''.

Na przykładzie grupy $\langle x,y,z \mid xyx=yxy,xzx=zxz\rangle$, która odpowiada zarówno sumie prostej różno-, jak i~jednoskrętnych trójlistników, widać że założenia o pierwszości nie można pominąć.
Prawdziwe jest ogólniejsze stwierdzenie:

\begin{proposition}
    \label{prop:knot_group_sum}
    Niech $K_1, K_2$ będą zorientowanymi węzłami.
    Wtedy węzłom $K_1 \shrap K_2$, $K_1 \shrap mr K_2$ odpowiadają izomorficzne grupy.
\end{proposition}

\index{prezentacja Wirtingera}
Wiemy więc już trochę o~nowym niezmienniku, ale nie umiemy go jeszcze wyznaczać.
Jak zauważył Wilhelm Wirtinger około roku 1905, a więc jeszcze przed narodzinami teorii węzłów, grupa węzła zawsze posiada pewną specjalną prezentację, nazwaną na jego cześć prezentacją Wirtingera.
Jest to skończona prezentacja, w~której wszystkie relacje są postaci $w g_i w^{-1} = g_j$, gdzie $w$ to pewne słowo na generatorach, $g_1, \ldots, g_k$.
Przedstawimy ją zaraz ze względu na użyteczność w~rachunkach, dowodząc jednocześnie jej istnienia.

\begin{proposition}
    \label{prop:wirtinger}
    Grupa każdego węzła posiada prezentację Wirtingera.
\end{proposition}

\begin{proof}
    Oto zarys konstruktywnego dowodu.
    Przedstawiony algorytm jest bardzo wygodnym sposobem na wyznaczenie grupy węzła.
    Niech $K$ będzie węzłem z~diagramem o~$n$ łukach i~$m$ skrzyżowaniach.
    Wtedy
    \begin{equation}
        \pi_1(K) \cong \langle a_1, \ldots, a_n \mid r_1, \ldots, r_m\rangle,
    \end{equation}
    gdzie $a_i$ to włókna diagramu, zaś $r_x$ to relacje Wirtingera: $a_ia_ja_i^{-1}a_k^{-1}=1$,
\begin{comment}
    \[
    \begin{tikzpicture}[baseline=-0.65ex,scale=0.15]
    \begin{knot}[clip width=15]
        \strand[semithick,-Latex] (-5, -5) to (5, 5);
        \strand[semithick,-Latex] (-5, 5) to (5, -5);
        \node[darkblue] at (5, 5)[below right] {$a_i$};
        \node[darkblue] at (5, -5)[above right] {$a_k$};
        \node[darkblue] at (-5, 5)[below left] {$a_j$};
    \end{knot}
    \end{tikzpicture}
    \quad\quad
    \begin{tikzpicture}[baseline=-0.65ex,scale=0.15]
    \begin{knot}[clip width=15, flip crossing/.list={1}]
        \strand[semithick,-Latex] (-5, -5) to (5, 5);
        \strand[semithick,-Latex] (-5, 5) to (5, -5);
        \node[darkblue] at (5, 5)[below right] {$a_j$};
        \node[darkblue] at (-5, -5)[above left] {$a_k$};
        \node[darkblue] at (-5, 5)[below left] {$a_i$};
    \end{knot}
    \end{tikzpicture}
    \]
\end{comment}
    w~których łuk $a_i$ biegnie górą, zaś $a_j$ leży po jego lewej stronie.
\end{proof}

\begin{figure}[H]
    \begin{minipage}[b]{.48\linewidth}
        \[
            \begin{tikzpicture}[baseline=-0.65ex, scale=0.2]
                \useasboundingbox (-5, -5) rectangle (5,5);
                \begin{knot}[clip width=3.5, end tolerance=1pt, flip crossing/.list={1}]
                    \strand[thick, Latex-] (-5,5) to (5,-5);
                    \strand[thick, -Latex] (-5,-5) to (5,5);
                    % top left
                    \strand[thick, Latex-, darkblue] (-5, 1) to (-1, 5);
                    % bottom left
                    \strand[thick, Latex-, darkblue] (-5, -1) to (-1, -5);
                    % bottom right
                    \strand[thick, -Latex, darkblue] (5, -1) to (1, -5);
                    % top right
                    \strand[thick, -Latex, darkblue] (5, 1) to (1, 5);
                    \node[darkblue] at (-7, -2) {$x_k$};
                    \node[darkblue] at (-7, 2) {$x_{j+1}$};
                    \node[darkblue] at (7, -2) {$x_j$};
                    \node[darkblue] at (7, 2) {$x_k$};
                \end{knot}
            \end{tikzpicture}
        \]
        \subcaption{skrzyżowanie dodatnie: $x_j = x_k x_{j+1} x_k^{-1}$}
    \end{minipage}
    \begin{minipage}[b]{.48\linewidth}
        \[
            \begin{tikzpicture}[baseline=-0.65ex, scale=0.2]
                \useasboundingbox (-5, -5) rectangle (5,5);
                \begin{knot}[clip width=3.5, end tolerance=1pt, flip crossing/.list={1}]
                    \strand[thick, Latex-] (-5,5) to (5,-5);
                    \strand[thick, Latex-] (-5,-5) to (5,5);
                    % top left
                    \strand[thick, Latex-, darkblue] (-5, 1) to (-1, 5);
                    % bottom left
                    \strand[thick, -Latex, darkblue] (-5, -1) to (-1, -5);
                    % bottom right
                    \strand[thick, -Latex, darkblue] (5, -1) to (1, -5);
                    % top right
                    \strand[thick, Latex-, darkblue] (5, 1) to (1, 5);
                    \node[darkblue] at (-7, -2) {$x_k$};
                    \node[darkblue] at (-7, 2) {$x_{j+1}$};
                    \node[darkblue] at (7, -2) {$x_j$};
                    \node[darkblue] at (7, 2) {$x_k$};
                \end{knot}
            \end{tikzpicture}
        \]
        \subcaption{skrzyżowanie ujemne: $x_j = x_k^{-1} x_{j+1} x_k$}
    \end{minipage}
\end{figure}

\begin{corollary}
    \label{prop:knot_group_abelianisation}
    Niech $G$ będzie grupą węzła.
    Wtedy jej abelianizacją jest $G^{\operatorname{ab}} = \Z$.
\end{corollary}

\begin{proof}
    Relacja $a_ia_ja_i^{-1}a_k^{-1}=1$ po przejściu do abelianizacji przyjmuje postać $a_j = a_k$.
    Oznacza to, że etykieta łuku nie zmienia się podczas przejścia pod każdym skrzyżowaniem, zatem wszystkie etykiety są takie same.

    Można też zauważyć, że abelianizacją grupy podstawowej węzła jest pierwsza grupa homologii okręgu, czyli $\Z$.
\end{proof}

Istnieje alternatywna prezentacja grupy węzła, która pochodzi od Dehna, gdzie zamiast etykietować łuki, przypisuje się różne litery czterem częściom płaszczyzny, które są rozcinane przez skrzyżowanie.
Pomijamy tę prezentację dla oszczędności miejsca.
Klasycznie, jak na przykład w~\cite{crowell63}, macierz, a~co za tym idzie, także wielomian Alexandera wprowadza się przy użyciu prezentacji Wirtingera i~pochodnej Foxa.
Oryginalna praca Alexandera była jednak bliższa duchem pomysłom Dehna.

\begin{definition}[pochodna Foxa]
    \index{pochodna Foxa}
    Niech $G$ będzie wolną grupą generowaną przez (niekoniecznie skończony) podzbiór $\{g_i\}_{i \in I}$.
    Odwzorowanie $\partial/\partial g_i \colon G \to \Z G$ spełniające trzy aksjomaty:
    \begin{align}
        \frac{\partial}{\partial g_i} (e) & = 0 \\
        \frac{\partial}{\partial g_i} (g_j) & = \delta_{ij} \\
        \forall u, v \in G : \frac{\partial}{\partial g_i} (uv) & = \frac{\partial}{\partial g_i}(u) + u \frac{\partial}{\partial g_i} (w),
    \end{align}
    gdzie $\delta_{ij}$ oznacza deltę Kroneckera, nazywamy pochodną cząstkową Foxa.
\end{definition}

Ustalmy prezentację grupy węzła z $n$ relacjami (słowami) $w_1, \ldots, w_n$ nad $n$-literowym alfabetem $x_1, \ldots, x_n$.
Zdefiniujmy następnie macierz Jacobiego wymiaru $n \times n$, elementami której są pochodne Foxa słów $w_i$ względem zmiennych $x_j$:
\begin{equation}
    J = \left(\frac{\partial w_i}{\partial x_j}\right).
\end{equation}

Wykreślmy z macierzy $J$ najpiew jedną kolumnę oraz jeden wiersz z tej macierzy, po czym podstawmy za wszystkie litery zmienną $t$ i policzmy wyznacznik.
Otrzymaliśmy znowu wielomian Alexandera.
Fox napisał cykl pięciu artykułów \cite{fox53}, \cite{fox54}, \cite{fox56}, \cite{fox58}, \cite{fox60} poświęcony wolnemu rachunkowi różniczkowemu, powyższa definicja jest tylko małym wycinkiem tego cyklu opublikowanego w Annals of Mathematics.

Dwa następne stwierdzenia są już trudniejsze w~dowodzie,
na przykład uzasadnienie pierwszego może wymagać:
twierdzenia o~sferze, o~pętli oraz hipotezy Knesera.

\begin{proposition}
    \label{prop:knot_group_split}
    Niech $L \subseteq S^3$ będzie splotem.
    Następujące warunki są równoważne:
    \begin{enumerate}
        \item grupa podstawowa splotu $L$ nie jest produktem wolnym,
        \item splot $L$ nie jest rozszczepialny,
        \item splot $L$ jest rozmaitością Hakena o~nieściśliwym brzegu.
    \end{enumerate}
\end{proposition}

\begin{proof}[Niedowód]
    Kawauchi w \cite{kawauchi96}, patrz twierdzenie 6.1.4.
\end{proof}

\begin{proposition}
    \label{prop:knot_group_free}
    Niech $L \subseteq S^3$ będzie splotem.
    Następujące warunki są równoważne:
    \begin{enumerate}
        \item grupa podstawowa splotu $L$ jest wolna, rangi $n$,
        \item splot $L$ jest trywialny, złożony z $n$ ogniw.
    \end{enumerate}
\end{proposition}

\begin{proof}[Niedowód]
    Kawauchi w \cite{kawauchi96}, patrz wniosek 6.1.5.
\end{proof}

Twierdzenie Dehna z~1915 mówi, że jedynym węzłem, którego grupą są liczby całkowite $\mathbb Z$, jest niewęzeł.
Wynik ten został później istotnie uogólniony.
Michael Kervaire pokazał w~1966 roku (w \cite{kervaire65}) jakie warunki musi spełniać grupa $G$, by istniał pewien węzeł, którego grupą jest właśnie $G$.
Patrz też twierdzenie 14.1.1 w \cite{kawauchi96}.

\begin{proposition}
    Niech $G$ będzie grupą węzła $S^n \subseteq S^{n+2}$.
    Wtedy:
    \begin{enumerate}[leftmargin=*]
        \itemsep0em
        \item grupa $G$ jest skończenie prezentowana,
        \item abelianizacja $G/G'$ jest nieskończoną grupą cykliczną,
        \item druga grupa homologii $H_2(G) = 0$ jest trywialna,
        \item istnieje element $x \in G$ zwany południkiem taki, że $G$ jest najmniejszą podgrupą normalną $G$, która zawiera $x$.
    \end{enumerate}
\end{proposition}

Wyżej wymienione warunki konieczne są także wystarczające, jeżeli $n \ge 3$, jednakże problem pełnej charakteryzacji w~czwartym wymiarze jest otwarty.
Warunki 2. i 3. wynikają z~dualności Alexandera, zaś 1. i 4. stanowią przeformułowanie prezentacji Wirtingera.

% Koniec sekcji Grupa węzła. Prezentacja Wirtingera


\chapter{Niezmienniki wielomianowe}
Zamknęliśmy już pierwszy rozdział,
Wciąż nie pokazaliśmy jednak pełnego dowodu, że dwa konkretne węzły (na przykład niewęzeł i~trójlistnik) są od siebie różne.
Dlatego teraz podamy proste narzędzie odróżniające węzły: \emph{trójkolorowalność}, która przypisuje włóknom diagramu różne kolory.
Następnie rozszerzymy paletę do dowolnie wielu kolorów i~zastąpimy ją grupą skończoną.
Nawet wzmocniony wariant nie jest idealnym narzędziem klasyfikującym.
Istnieją węzły, których nie odróżnia.
Problem ten dotyka wielu późniejszych niezmienników, pierwszy niezmiennik zupełny poznamy dopiero w~rozdziale czwartym.
Nie stanowi to wielkiego powodu do radości ze względu na trudności w~jego wyznaczaniu.

\section{Kolorowanie splotów} % (fold)
\label{sec:colour_links}
Oto mniej mętny opis trójkolorowalności, czyli jak nietrudno się domyślić, kolorowalności trzema kolorami.
Diagram $D$ splotu $K$ jest trójkolorowalny, jeśli każdemu włóknu można przypisać jeden z~trzech kolorów tak, by co najmniej dwa zostały użyte.
Wymagamy przy tym, by przy żadnym skrzyżowaniu nie spotykały się dokładnie dwa kolory.

Dla własnej wygody jako kolorów używać będziemy kolejnych liczb naturalnych $0, 1, 2, \ldots$.
Pozwala to zapisać warunek kolorowalności równaniem algebraicznym.

\begin{definition}[kolorowanie] \label{def:colour_equation}
    Niech $L$ będzie splotem, zaś $n$ liczbą naturalną.
    Mówimy, że splot $L$ jest kolorowalny modulo $n$, jeśli posiada diagram, którego włóknom można przypisać liczby całkowite $0, \ldots, n - 1$ tak, by
    \begin{enumerate}[leftmargin=*]
        \item istniały dwa włókna różnych kolorów,
        \item równanie $a + b \equiv 2c$ modulo $n$ było spełnione przy każdym skrzyżowaniu:
    \end{enumerate}
    \[
        \begin{tikzpicture}[baseline=-0.65ex, scale=0.12]
            \useasboundingbox (-5, -5) rectangle (5,5);
            \begin{knot}[clip width=5, end tolerance=1pt, flip crossing/.list={1}]
                \strand[semithick] (-5,5) to (5,-5);
                \strand[semithick] (-5,-5) to (5,5);
                \node[darkblue] at (5, 5)[below right] {$c$};
                \node[darkblue] at (5, -5)[above right] {$b$};
                \node[darkblue] at (-5, 5)[below left] {$a$};
            \end{knot}
        \end{tikzpicture}.
    \]
    Takie przyporządkowanie nazywamy (nietrywialnym) kolorowaniem.
\end{definition}

Metoda ta została odkryta razem z~uogólnieniem do $n$ kolorów przez Ralpha Foxa w~1956, kiedy próbował uczynić teorię węzłów bardziej przystępną dla studentów.
Opierając się tylko na definicji kolorowania oraz ruchach Reidemeistera możemy wykazać pierwsze własności kolorowań.

\begin{proposition} \label{color_invariant}
    ,,Bycie $n$-kolorowalnym'' jest niezmiennikiem węzłów.
\end{proposition}

\begin{proof}
    Wystarczy sprawdzić, jak ruchy Reidemeistera zmieniają kolory.
    Pierwszy i~drugi:
    \[
        \fbox{
        \begin{tikzpicture}[baseline=-0.65ex,scale=0.07]
        \begin{knot}[clip width=5]
            \strand[semithick] (-10,10) .. controls (-10,2) and (-10,2) .. (-6,-2);
            \strand[semithick] (-10,-10) .. controls (-10,-2) and (-10,-1) .. (-9,0);

            \strand[semithick] (-7,1) -- (-6,2);
            \strand[semithick] (-6,2) .. controls (2,9) and (2,-9) .. (-6,-2);
            \node[darkblue] at (-10, 10)[below left] {$a$};
            \node[darkblue] at (-10, -10)[above left] {$b \equiv a$};
        \end{knot}
        \end{tikzpicture}
        $\stackrel{R_1}{\cong} \,\,$
        \begin{tikzpicture}[baseline=-0.65ex,scale=0.07]
        \begin{knot}[clip width=5]
            \strand[semithick] (0,10) -- (0,-10);
            \node[darkblue] at (0, 0)[left] {$a$};
        \end{knot}
        \end{tikzpicture}}
        %%%
        \quad \fbox{
        \begin{tikzpicture}[baseline=-0.65ex,scale=0.07]
        \begin{knot}[clip width=5]
            \strand[semithick] (4,-10) .. controls (4,-4) and (-4,-4) .. (-4,0);
            \node[darkblue] at (-4, -10)[above left] {$d \equiv b$};
            \strand[semithick] (4,10) .. controls (4, 4) and (-4, 4) .. (-4,0);
            \node[darkblue] at (4, 10)[below right] {$a$};
            \strand[semithick] (-4,-10) .. controls (-4,-4) and (4,-4) .. (4,0);
            \node[darkblue] at (4, 0) [right] {$c \equiv 2a-b$};
            \strand[semithick] (-4, 10) .. controls (-4, 4) and (4,4) .. (4,0);
            \node[darkblue] at (-4, 10) [below left] {$b$};
        \end{knot}
        \end{tikzpicture}
        $\stackrel{R_2}{\cong} \,\,$
        \begin{tikzpicture}[baseline=-0.65ex,scale=0.07]
        \begin{knot}[clip width=5]
            \strand[semithick] (4,-10) .. controls (4,-4) and (1,-4) .. (1,0);
            \strand[semithick] (4,10) .. controls (4, 4) and (1, 4) .. (1,0);
            \strand[semithick] (-4,-10) .. controls (-4,-4) and (-1,-4) .. (-1,0);
            \strand[semithick] (-4,10) .. controls (-4, 4) and (-1,4) .. (-1,0);
        \end{knot}
        \end{tikzpicture}}
    \]
    Trzeci ruch także nie wymaga skomplikowanych rachunków.
    Najkrótszy łuk na diagramach ma kolor $2a-c$ po lewej oraz $2b-c$ po prawej stronie.
    \[
     \fbox{
        \begin{tikzpicture}[baseline=-0.65ex,scale=0.07]
        \begin{knot}[clip width=5, flip crossing/.list={1,2,3}]
            \node[darkblue] at (-10, 10) [above] {$b$};
            \node[darkblue] at (10, 10) [above] {$c$};
            \node[darkblue] at (-10, -10) [below] {$2a-2b+c$};
            \node[darkblue] at (10, -10) [below] {$2a-b$};
            \node[darkblue] at (-10, -2) [left] {$a$};
            \strand[semithick] (-10,-10) -- (10,10);
            \strand[semithick] (-10,10) -- (10,-10);
            \strand[semithick] (-10,-2) .. controls (-4, -2) and (-4,8) .. (0,8);
            \strand[semithick] (10,-2) .. controls (4, -2) and (4,8) .. (0,8);
        \end{knot}
        \end{tikzpicture}
        $\stackrel{R_3}{\cong} \,\,$
        \begin{tikzpicture}[baseline=-0.65ex,scale=0.07]
        \begin{knot}[clip width=5, flip crossing/.list={1,2,3}]
            \node[darkblue] at (-10, 10) [above] {$b$};
            \node[darkblue] at (10, 10) [above] {$c$};
            \node[darkblue] at (-10, -10) [below] {$2a-2b+c$};
            \node[darkblue] at (10, -10) [below] {$2a-b$};
            \node[darkblue] at (10, 2) [right] {$a$};
            \strand[semithick] (-10,-10) -- (10,10);
            \strand[semithick] (-10,10) -- (10,-10);
            \strand[semithick] (-10,2) .. controls (-4, 2) and (-4,-8) .. (0,-8);
            \strand[semithick] (10,2) .. controls (4, 2) and (4,-8) .. (0,-8);
        \end{knot}
        \end{tikzpicture}} \qedhere
    \]
\end{proof}

Trójlistnik koloruje się dokładnie modulo krotności trójki, ósemka zaś -- piątki.
Sama kolorowalność nie mówi wiele, splot jest kolorowalny lub nie.
Dowód faktu \ref{color_invariant} pokazuje coś więcej: liczba kolorowań, być może trywialnych, jest mocniejszym niezmiennikiem.
Liczbę kolorowań splotu $L$ modulo $n$, trywialnych lub nie, oznaczamy przez $\tau_n(L)$.

\begin{lemma}
    \label{lem:colouring_arc}
    Niech $D$ będzie diagramem z definicji \ref{def:colour_equation} z wybranym łukiem, zaś $k \in \{0, \ldots, n - 1\}$ pewnym kolorem.
    Bez straty ogólności możemy założyć, że krótki łuk jest koloru $k$.
\end{lemma}

Kolorem tym zazwyczaj jest $0$.

\begin{proof}
    Dodanie tej samej wartości do wszystkich łuków na dobrze pokolorowanym diagramie daje nowy, także dobrze pokolorowany diagram.
\end{proof}

\begin{proposition}
    \label{no_knots_colours_mod_two}
    Żaden węzeł nie koloruje się modulo dwa.
\end{proposition}

\begin{proof}
    Załóżmy nie wprost, że istnieje nietrywialne kolorowanie.
    Analiza czterech możliwych skrzyżowań pokazuje, że włókna wychodzące z~tunelu muszą mieć ten sam kolor.
    Przechodząc wzdłuż węzła widzimy jeden kolor, wbrew założeniu nie wprost.
\end{proof}

\begin{proposition}
    Każdy splot o co najmniej dwóch ogniwach koloruje się modula dwa.
\end{proposition}

\begin{proof}
    Wystarczy pomalować jedną składową zerem, a~pozostałe jedynkami.
\end{proof}

Sploty rozszczepialne są $n$-kolorowalne dla każdego $n \ge 2$, można skorzystać z~tego samego schematu kolorowania.
Pierścienie Boromeuszy nie kolorują się modulo trzy, nie są zatem rozszczepialne.
Sploty, które nie są kolorowalne modulo $n$ dla każdej liczby $n \in \N$ nazywa się czasem niewidzialnymi, dwa węzły do dziesięciu skrzyżowań mają tę własność: $10_{124}$ oraz $10_{153}$.

Pokażemy teraz, że suma równań kolorujących z dobrze wybranymi znakami jest postaci $0 \equiv 0 \mod n$.
Będziemy potrzebować kilku pomocniczych definicji.
Każdy diagram węzła rozcina płaszczyznę na obszary, z~czego jeden jest nieograniczony.

\begin{definition}[uszachowienie]
    Przyporządkowanie każdemu z~obszarów, na jakie diagram rozcina płaszczyznę, jednego z~dwóch kolorów tak, by sąsiadujące ze sobą obszary były różnych kolorów, nazywamy uszachowieniem.
\end{definition}

Ustalmy węzeł $K$ oraz dowolne uszachowienie dla jego diagramu.
Skojarzmy z~każdym skrzyżowaniem równanie kolorujące, zgodnie z~poniższym schematem:
\[\begin{tikzpicture}[baseline=-0.65ex, scale=0.12]
    \useasboundingbox (-5, -10) rectangle (5,5);
    \begin{knot}[clip width=5, end tolerance=1pt, flip crossing/.list={1}]
        \strand[semithick] (-5,5) to (5,-5);
        \strand[semithick] (-5,-5) to (5,5);
        \fill[blue!20!white] (-4, 5) to (0, 1) to (4, 5);
        \fill[blue!20!white] (-4, -5) to (0, -1) to (4, -5);
        \node[darkblue] at (-5, -5)[left] {$a$};
        \node[darkblue] at (-5, +5)[left] {$b$};
        \node[darkblue] at (+5, -5)[right] {$c$};
        \node[darkblue] at (+5, +5)[right] {$a$};
        \node[darkblue] at (0, -10) {$+a-b+a-c=0 \mod n$};
    \end{knot}
    \end{tikzpicture}
    \quad\quad\quad\quad\quad\quad\quad\quad\quad\quad\quad\quad
    \begin{tikzpicture}[baseline=-0.65ex, scale=0.12]
    \useasboundingbox (-5, -10) rectangle (5,5);
    \begin{knot}[clip width=5, end tolerance=1pt, flip crossing/.list={1}]
        \strand[semithick] (-5,5) to (5,-5);
        \strand[semithick] (-5,-5) to (5,5);
        \fill[blue!20!white] (5, -4) to (1, 0) to (5, 4);
        \fill[blue!20!white] (-5, -4) to (-1, 0) to (-5, 4);
        \node[darkblue] at (-5, -5)[left] {$a$};
        \node[darkblue] at (-5, +5)[left] {$b$};
        \node[darkblue] at (+5, -5)[right] {$c$};
        \node[darkblue] at (+5, +5)[right] {$a$};
        \node[darkblue] at (0, -10) {$-a+b-a+c=0 \mod n$};
    \end{knot}
    \end{tikzpicture}
\]

\begin{proposition}
    \label{prp:colouring_sum_zero}
    Sumą równań kolorujących o dobrze wybranych znakach jest $0 \equiv 0 \mod n$.
\end{proposition}

Będziemy potrzebować tego do pokazania, że wyznacznik determinuje kolorowalność splotu.

\begin{proof}
    Każde równanie kolorujące składa się z~czterech wyrazów, po jednym od każdej nici, która spotyka się w~danym skrzyżowaniu.
    Nić biegnie między dwoma skrzyżowaniami, więc suma wszystkich równań kolorujących składa się z~par składników, po jednej parze na nić.
    Składniki te są przeciwnych znaków, zatem wzajemnie się znoszą.
    Suma równań kolorujących jest sumą zer, a~to należało udowodnić.
\end{proof}

\begin{proposition}
    Jeśli $K, L$ są węzłami, to $3\tau_3(K \shrap L) = \tau_3(K)\tau_3(L)$.
\end{proposition}

\begin{corollary}
    Istnieje nieskończenie wiele węzłów.
\end{corollary}

\begin{proof}
    Suma spójna $n$ trójlistników ma $3^n$ (być może trywialnych) $3$-kolorowań.
\end{proof}

Jako kolorów użyjemy teraz elementów $g_1, \ldots, g_n$ pewnej skończonej grupy $G$.

\begin{definition}[etykietowanie]
    Mówimy, że zorientowany węzeł $K$ jest etykietowalny grupą $G$ generowaną przez elementy $g_1, \ldots, g_n$, jeśli posiada diagram, którego włóknom przypisano elementy $g_1, \ldots, g_n$ tak, by równanie $gk=hg$ było spełnione przy każdym skrzyżowaniu ($g$: włókno biegnące górą, $k$: bo jego lewej stronie, $h$: po prawej).
    \[
        \begin{tikzpicture}[baseline=-0.65ex, scale=0.12]
            \useasboundingbox (-5, -5) rectangle (5,5);
            \begin{knot}[clip width=5, end tolerance=1pt, flip crossing/.list={1}]
                \strand[semithick] (-5,5) to (5,-5);
                \strand[semithick, -Latex] (-5,-5) to (5,5);
                \node[darkblue] at (5, 5)[below right] {$g$};
                \node[darkblue] at (5, -5)[above right] {$h$};
                \node[darkblue] at (-5, 5)[below left] {$k$};
            \end{knot}
        \end{tikzpicture}
    \]
\end{definition}

Równanie $gkg^{-1}=h$ mówi, że etykiety włókien wchodzących oraz wychodzących są sprzężone.
Wynika stąd, że wszystkie etykiety pochodzą z~jednej klasy sprzężoności.
Muszą jednocześnie generować całą grupę, dlatego $G$ musi być grupą nieprzemienną lub trywialną.
Etykietowalność jest niezmiennikiem węzłów i~nie zależy od orientacji węzła:
jeżeli elementy $g_1, \ldots, g_n$ generują grupę, to ich odwrotności także.

Rozpatrzmy węzły $6_1$ oraz $9_{46}$ i~spróbujmy etykietować je transpozycjami z~grupy $S_4$.
Wybranie dwóch etykiet przy jednym skrzyżowaniu $6_1$ wymusza etykiety dla wszystkich włókien.
Dwie transpozycje nie mogą generować grupy $S_4$, natomiast włókna węzła $9_{46}$ dają się etykietować samymi transpozycjami.
Węzły te są więc różne, choć mają te same własności kolorujące.

Etykietowanie jest mocnym narzędziem odróżniającym węzły.
Thistlethwaite w 1985 roku korzystając z niego klasyfikował węzły o~co najwyżej 13 skrzyżowaniach (jest ich, jak ostatecznie się okazało, 12965).
Mają one tylko 5639 różnych wielomianów Alexandera, ale etykietowania trzynastoma różnymi grupami pozwoliły zmniejszyć liczbę nierozpoznanych węzłów do około tysiąca.
Wśród nich 30 posiada wielomian Conwaya $1 + 2z^2 + 2z^4$, ale pary rozróżniane wielomianem HOMFLY mają też różne wielomiany Jonesa.
Wielomiany opisujemy w~rozdziale trzecim.

Niech $p \ge 3$ będzie liczbą pierwszą, natomiast $D_p = \langle r, s \mid r^p = s^2 = e, rsr = s \rangle$ grupą diedralną rzędu $2p$.
Elementy tej grupy to $1, r, r^2, \ldots, r^{p-1}, s, sr, \ldots, sr^{p-1}$.
,,Obrót'' $r^k$ jest sprzężony tylko ze swoją odwrotnością, ale ,,symetrie osiowe'' $sr^k$ tworzą jedną klasę sprzężoności.
Łatwo widać, że dowolne dwie z~nich generują całą grupę $D_p$.

\begin{proposition}
    Węzeł $K$ jest $p$-kolorowalny wtedy i~tylko wtedy, gdy jest $D_p$-etykietowalny.
\end{proposition}

\begin{proof}
    Załóżmy, że $K$ ma $n$ włókien.
    Wiemy już, że każde $D_p$-etykietowanie wykorzystuje tylko elementy $sr^{a_1}, \ldots, sr^{a_n}$ dla $1 \le a_i \le p$.
    Jest ono prawidłowe dokładnie wtedy, gdy analogiczne kolorowanie liczbami $a_1, \ldots, a_n$ jest prawidłowe.
\end{proof}

Kolorowania definiowano kiedyś jako surjekcje $\rho \colon \pi \to D_{2n}$ z~grupy podstawowej.
Jak mówi prezentacja Wirtingera, grupa splotu generowana jest przez ścieżki z~punktu bazowego w~$S^3$ do brzegu rurowego otoczenia splotu, wokół południka i~znowu do bazowego punktu.
Fox zauważył, że z~surjektywności $\rho$ wynika, iż generatory mapują się na symetrie osiowe $sr^k$.
Ponieważ istnieje wzajemnie jednoznaczna odpowiedniość między generatorami grupy splotu oraz łukami diagramu, każdemu możemy przypisać liczbę całkowitą $k$.
Etykietowania są więc uogólnieniem kolorowań.
Rozumowanie, które przedstawiliśmy, prowadzi do prostej klasyfikacji grup, których można użyć do etykietowania.

\begin{proposition}
    Niech $K$ będzie węzłem, $\pi$ grupą podstawową jego dopełnienia, zaś $G$ dowolną grupą.
    Następujące warunki są równoważne: $K$ jest $G$-etykietowalny; istnieje surjekcja $\pi_1 \to G$.
\end{proposition}

Historycznie, prezentacja Wirtingera była pierwsza, zaś etykietowania odkryto później.

\begin{proposition}[Perko]
    Niech $K$ będzie węzłem etykietowalnym grupą $S_3$.
    Wtedy $K$ jest też etykietowalny grupą $S_4$.
\end{proposition}

Nie znam innych nietrywialnych faktów dotyczących etykietowań.

% Koniec sekcji Kolorowanie splotów

\section{Wielomian Alexandera}
Większość niezmienników, jakie dotąd widzieliśmy, przypisuje każdemu węzłowi lub splotowi pewną liczbę całkowitą: niezmienniki z pierwszego rozdziału, liczba kolorowań, wyznacznik czy też defekt są właśnie takie.
Poznamy teraz wielomian, opisany po raz pierwszy przez Jamesa Waddella Alexandera w~1923 roku \cite{alexander23}.
Przez blisko sześć dekad pozostał on jedynym wielomianowym niezmiennikiem węzłów.
Ze względu na elementarność takiego podejścia, pominiemy teorię homologii, zamiast tego skupiając naszą uwagę na równaniach kolorujących.
Najpierw trzeba je jednak uogólnić.

\begin{definition}
    Wielomianowe równanie kolorujące związane ze skrzyżowaniem
\begin{comment}
    \[\begin{tikzpicture}[baseline=-0.65ex, scale=0.12]
    \useasboundingbox (-5, -5) rectangle (5,5);
    \begin{knot}[clip width=5, end tolerance=1pt, flip crossing/.list={1}]
        \strand[semithick] (-5,5) to (5,-5);
        \strand[semithick,-Latex] (-5,-5) to (5,5);
        \node[darkblue] at (5, 5)[below right] {$a$};
        \node[darkblue] at (5, -5)[above right] {$b$};
        \node[darkblue] at (-5, 5)[below left] {$c$};
    \end{knot}
    \end{tikzpicture}\]
\end{comment}
    splotu zorientowanego to $a + tc - ta - b = 0$.
    Tylko orientacja górnej wiązki ma znaczenie.
\end{definition}

Istotnie, wystarczy podstawić tutaj $t = -1$, by otrzymać mniej ogólną definicję \ref{def:colouring_equation}.

\begin{definition}[wielomian Alexandera]
    \label{def:alexander_polynomial}
    \index{wielomian!Alexandera}
    Niech $L$ będzie zorientowanym splotem z~diagramem bez krzywych zamkniętych.
    Przypiszmy etykiety $x_0, \ldots, x_m$ do włókien oraz $0, \ldots, m$ do skrzyżowań.
    Niech $P_{ij}$ będzie współczynnikiem przy $x_j$ w~wielomianowym równaniu kolorującym nad wierzchołkiem $i$.
    Z macierzy $P=(P_{ij})$ wykreślmy jedną kolumnę i~jeden wiersz.
    Wyznacznik tak otrzymanej macierzy nazywamy wielomianem Alexandera i~oznaczamy $\alexander_L(t)$.
\end{definition}

Nasz nowy niezmiennik nie jest zwykłym wielomianem, tylko wielomianem Laurenta jednej zmiennej, czyli elementem pierścienia $\Z[t, t^{-1}]$.

\begin{proposition}
    \label{alexander_invariance}
    Wielomian Alexandera z~dokładnością do mnożenia przez jedności:
    \begin{equation}
        f(t) \equiv g(t) \iff \exists m \in \Z: f(t) = \pm t^m g(t)
    \end{equation}
    jest niezmiennikiem zorientowanych splotów.
\end{proposition}

W dowodzie niezmienniczości wyznacznika węzła skorzystaliśmy z~relacji między nim a~grupą kolorującą.
Poprzednie wydania książki zawierały sugestię, że elementarny (czyli taki, który nie korzysta z~teorii modułów) dowód niezmienniczości wielomianu Alexandera nie istnieje.
Sugestia ta była błędna, wystarczy użyć alternatywnej definicji.

\begin{proof}
    Ustalmy diagram o~$k$ skrzyżowaniach, który rozcina płaszczyznę na $k+2$ obszarów i~utwórzmy macierz o~wymiarach $k \times k$, której kolumny odpowiadają obszarom, wiersze zaś skrzyżowaniom -- pomijając przy tym dwa sąsiadujące ze sobą obszary -- o~wyrazach ze współczynników równań kolorujących.
    Jej wyznacznik jest wielomianem Alexandera.

    Sąsiadującym ze sobą obszarom przypiszmy kolejne liczby całkowite tak, by obszar leżący po prawej stronie włókna miał niższy indeks.
    Pokażemy najpierw, że skasowanie kolumny indeksu $n$ oraz $n+1$ sprawia, że wyznacznik zmienia się co najwyżej o~czynnik $\pm t^m$ dla pewnego $m$.
    Niech $S_n$ oznacza sumę kolumn indeksu $n$.
    Każdy wiersz macierzy zawiera cztery niezerowe wyrazy: $\pm 1, \pm t$, zatem $\sum_n S_n = 0$.
    Równość ta zachodzi nawet po przemnożeniu kolumny indeksu $n$ przez $t^{-n}$: $\sum_n t^{-n}S_n = 0$, co prowadzi do relacji $\sum_n (t^{-n}-1) S_n = 0$.
    Jeśli więc indeks kolumny $v_j$ wynosi $n$, to $(t^{-n}-1)v_j$ jest kombinacją liniową innych kolumn niezerowego indeksu (ponieważ $t^0 - 1 = 0$).

    Rozpatrzmy macierze $M_{0,j}, M_{0,k}$, gdzie indeksy $j$-tej i~$k$-tej kolumny to odpowiednio $p$ i~$q$.
    Z powyższych rozważań wynika, że $(t^{-q}-1) \alexander_{0,j} = \pm (t^{-p}-1)\alexander_{0,k}$, ale indeksy obszarów są wyznaczone z~dokładnością do stałej addytywnej.
    Biorąc $i$-tą oraz $l$-tą kolumnę, indeksów $r$ oraz $s$, dostaniemy zależności
    \begin{align}
        (t^{r-q}-1) \alexander_{l,j} & = \pm (t^{r-p} - 1)\alexander_{l,i} \\
        (t^{q-s}-1) \alexander_{k,l} & = \pm (t^{q-r} - 1)\alexander_{k,i}
    \end{align}
    co prowadzi do
    \begin{equation}
        \alexander_{l,j} = \pm \frac{t^{q-r}(t^{r-p}-1)}{t^{q-s}-1} \alexander_{k,i}
    \end{equation}
    Położenie $p = r +1$, $s =q+1$ pokazuje, że różny wybór kolumn do skreślenia zmienia wyznacznik macierzy co najwyżej o~czynnik $\pm t^m$.

    Wprowadźmy jeszcze jedną techniczną definicję.
    Dwie kwadratowe macierze będą dla nas równoważne, jeśli można przejść od jednej do drugiej przy użyciu pięciu operacji:
    \begin{enumerate}[leftmargin=*]
    \itemsep0em
        \item przemnożenie wiersza lub kolumny przez $-1$;
        \item zamiana dwóch wierszy lub kolumn miejscami;
        \item dodanie jednego wiersza do innego (lub kolumny do innej);
        \item przemnożenie lub podzielenie kolumny przez $t$;
        \item rozszerzenie lub zmniejszenie macierzy o~$1$ na przekątnej i~zera w~innych miejscach.
    \end{enumerate}

    Ruchy Reidemeistera prowadzą do macierzy równoważnych wyjściowym.
    Każda z~tych operacji zmienia wyznacznik macierzy o~czynnik $\pm t^{-m}$, co kończy dowód.
\end{proof}

Zwyczajowo wielomian normalizuje się: bierze reprezentanta, który jest symetryczny w~zmiennych $t$ i $t^{-1}$ oraz przyjmuje w~punkcie $1$ wartość $\alexander_L(1) = 1$.
Odwrotnie, dowolny wielomian Laurenta z~całkowitymi współczynnikami o~takich własnościach jest wielomianem Alexandera pewnego węzła:

\begin{proposition}
    \label{prp:alexander_hosokawa}
    Każdy wielomian Laurenta $p(t)$ o~całkowitych współczynnikach taki, że $p(1/t) = p(t)$ i~$p(1) = \pm 1$ jest wielomianem Alexandera pewnego węzła.
\end{proposition}

\begin{proof}[Niedowód]
    Hosokawa w \cite{hosokawa58} udowodnił to dla pomocniczego wielomianu splotów
    \begin{equation}
        \frac{\Delta(t, \ldots, t)}{(1-t)^{\max(0, \mu - 2)}},
    \end{equation}
    gdzie $\mu$ oznacza liczbę ogniw.
    Książka \cite{rolfsen76} Rolfsena na stronach 171-172 zawiera natomiast jawną konstrukcję węzła o~danym wielomianie Alexandera.
\end{proof}

Istnieje nieopisana przez nas odmiana wielomianu Alexandera, która liczy sobie tyle zmiennych, ile ogniw posiada splot.
Fakt \ref{prp:alexander_hosokawa} można częściowo uogólnić: Torres znalazł w~\cite{torres53} dwie geometryczne własności tych wielomianów, nazwane później warunkami Torresa.
\index{warunek!Torresa}
Nie są one warunkami wystarczającymi, jak odkrył ponad ćwierć wieku później Hillman \cite{hillman81}: wielomian
\begin{equation}
    D(x,y) = \frac{1}{1-xy} \cdot \left((1 - x^6y^6)(x - 1 + 1/x) - 2(1 - x^5y^5)(1 - x)(1 - y)\right)
\end{equation}
spełnia warunki Torresa, ale nie jest wielomianem Alexandera.

Wielomian Alexandera nie odróżnia luster i~rewersów od wyjściowych węzłów:

\begin{proposition}
    Niech $L$ będzie zorientowanym splotem.
    Wtedy $\alexander_{mL}(t) = \alexander_L(1/t) = \alexander_{rL}(t)$.
\end{proposition}

\begin{proof}
    Po odbiciu diagramu względem pionowej prostej skrzyżowanie z~definicji \ref{def:colouring_equation} również się odbija.
    Równanie związane z~nim zmienia się według schematu:
    \begin{equation}
        a + tc - ta - b = 0 \rightleftharpoons a + tb - ta - c = 0
    \end{equation}
    Pierwsze równanie z~$t$ zamienionym na $1/t$ staje się drugim równaniem przemnożonym przez $-1/t$.
    Dowód drugiej równości przebiega analogicznie.
\end{proof}

Nasz niezmiennik nie wykrywa niewęzła.
Na przykład $11_{471} = 11n_{34}$, $11_{473} = 11n_{42}$ albo $(-3, 5, 7)$-precel posiadają trywialny wielomian Alexandera, zjawisko to nie występuje wśród nietrywialnych węzłów o co najwyżej 10 skrzyżowaniach.

\begin{proposition}
    \label{prp:alexander_determinant}
    Niech $L$ będzie zorientowanym splotem.
    Wtedy $|\alexander_L(-1)| = \det L$.
\end{proposition}

\begin{proof}
    Wystarczy porównać definicję dla $\alexander_L$ (\ref{def:alexander_polynomial}) oraz $\det L$ (\ref{def:determinant}).
\end{proof}

\begin{proposition}
    \label{prp:alexander_multiplicative}
    Niech $K_1, K_2$ będą zorientowanymi węzłami.
    Wtedy
    \begin{equation}
        \alexander_{K_1 \shrap K_2}(t) \equiv \alexander_{K_1}(t) \alexander_{K_2}(t)
    \end{equation}
\end{proposition}

\begin{proof}
    Wybierzmy poniższe diagramy dla węzłów $K_1$ oraz $K$:
\begin{comment}
    \[\begin{tikzpicture}[baseline=-0.65ex, scale=0.07]
    %\useasboundingbox (-5, -5) rectangle (5,5);
    \begin{knot}[clip width=5, end tolerance=1pt]
        \strand[semithick] (-70, -10) rectangle (-30, 10);
        \strand[semithick] ( 30, -10) rectangle ( 70, 10);
        \strand[semithick,Latex-] (-30, 5) .. controls (-22, 5) and (-18, -5) .. (-10, -5);
        \strand[semithick] (-30,-5) .. controls (-22, -5) and (-18, 5) .. (-10,  5);
        \strand[semithick] (-10, 5) [in=up, out=right] to (-5, 0) [in=right, out=down] to (-10, -5);

        % prawe strzalki
        \strand[semithick,-Latex] (30, 5) .. controls (22, 5) and (18, -5) .. (10, -5);
        \strand[semithick] (30,-5) .. controls (22, -5) and (18, 5) .. (10,  5);
        \strand[semithick] (10, 5) [in=up, out=left] to (5, 0) [in=left, out=down] to (10, -5);

        \node[darkblue] at (-50,5) {$x_1,\ldots,x_{m-1}$};
        \node[red] at (-50,-5) {$1,\ldots,m$};

        \node[darkblue] at (50,5) {$y_1,\ldots,y_{n-1}$};
        \node[red] at (50,-5) {$1,\ldots,n$};

        \node[darkblue] at (-30,-5)[below right] {$x_m$};
        \node[darkblue] at (-15,-5)[below] {$x_0$};
        \node[darkblue] at (30,-5)[below left] {$y_n$};
        \node[darkblue] at (15,-5)[below] {$y_0$};
        \node[red] at ( 19.5,  1)[above]{$0$};
        \node[red] at (-19.5,  1)[above]{$0$};
    \end{knot}
    \end{tikzpicture}
\]
\end{comment}
    Niech $A$ oraz $B$ oznaczają macierze otrzymane z~wielomianowych równań kolorujących dla $K_1$ oraz $K_2$ przez skreślenie skrajnie lewej kolumny i~górnego wiersza.
    Wtedy $\alexander_{K_1}(t) = \det A$ oraz $\alexander_{K_2}(t) = \det B$.
    Poniższy diagram przedstawia sumę $K_1 \shrap K_2$:

\begin{comment}
\[\begin{tikzpicture}[baseline=-0.65ex, scale=0.07]
    %\useasboundingbox (-5, -5) rectangle (5,5);
    \begin{knot}[clip width=5, end tolerance=1pt]
        \strand[semithick] (-70, -10) rectangle (-30, 10);
        \strand[semithick] ( 30, -10) rectangle ( 70, 10);
        \strand[semithick,Latex-] (-30, 5) .. controls (-22, 5) and (-18, -5) .. (-10, -5);
        \strand[semithick] (-30,-5) .. controls (-22, -5) and (-18, 5) .. (-10,  5);

        % prawe strzalki
        \strand[semithick] (30, 5) .. controls (22, 5) and (18, -5) .. (10, -5);
        \strand[semithick] (30,-5) .. controls (22, -5) and (18, 5) .. (10,  5);
        \strand[semithick] (10, 5) to (-10, 5);
        \strand[semithick,-Latex] (10, -5) to (-10, -5);

        \node[darkblue] at (-50,5) {$x_1,\ldots,x_{m-1}$};
        \node[red] at (-50,-5) {$1,\ldots,m$};

        \node[darkblue] at (50,5) {$y_1,\ldots,y_{n-1}$};
        \node[red] at (50,-5) {$1,\ldots,n$};

        \node[darkblue] at (-30,-5)[below right] {$x_m$};
        \node[darkblue] at (0,-5)[below] {$x_0 = y_0$};
        \node[darkblue] at (0, 5)[above] {$z$};
        \node[darkblue] at (30,-5)[below left] {$y_n$};
        \node[red] at ( 19.5,  1)[above]{$\zeta$};
        \node[red] at (-19.5,  1)[above]{$0$};
    \end{knot}
    \end{tikzpicture}\]
\end{comment}

    Uporządkujmy łuki na diagramie jako $x_0 = y_0$, $x_1, \ldots, x_m$, $y_1, \ldots, y_n$, $z$; skrzyżowania: $0, 1, \ldots, m$ (z $K_1$), $1, \ldots, n$ (z $K_2$), $\zeta$.
    Wielomianowe równanie kolorujące dla $K_1 \shrap K_2$ nad skrzyżowaniami $1, \ldots, m$ ($1, \ldots, n$) są takie same, jak przed dodaniem do siebie węzłów.
    Nad skrzyżowaniem $\zeta$ równanie orzeka, że $(1-t)y_0+t z-y_n=0$.

    Wynika stąd, że $\alexander_{K_1 \shrap K_2}(t)$ jest wyznacznikiem macierzy
    \begin{align*}
        M &= \left(\begin{array}{cc|cc|c}
            & & & & \\
            \multicolumn{2}{c|}{\smash{\raisebox{.5\normalbaselineskip}{$A$}}} & & \\
            \hline \\[-\normalbaselineskip]
            & & & & \\
            & & \multicolumn{2}{c|}{\smash{\raisebox{.5\normalbaselineskip}{$B$}}}\\ \hline
            & & & -1 & t
    \end{array}\right)
    \end{align*}

    Skreśliliśmy lewą kolumnę oraz górny wiersz.
    Zatem $\alexander_{K_1 \shrap K_2}(t) = t^?\alexander_{K_1}(t) \alexander_{K_2}(t)$, jeśli nie pomyliliśmy się w~obliczeniach.
\end{proof}

\begin{proposition}
    Wielomian Alexandera zadaje ograniczenie na indeks skrzyżowaniowy $c$:
    \begin{equation}
        \deg \alexander_K(t) < c(K).
    \end{equation}
\end{proposition}

Być może istnieje bezpośredni dowód tej nierówności, ale jedyne uzasadnienie, jakie znam, opiera się na fakcie \ref{prp:alexander_genus} oraz wniosku \ref{cor:crossing_genus}.

\begin{proposition}
    Tylko skończenie wiele węzłów alternujących może mieć ten sam wielomian Alexandera.
\end{proposition}

\begin{proof}
    Załóżmy nie wprost, że istnieje nieskończony ciąg $K_n$ węzłów alternujących o~tym samym wielomianie Alexandera $\alexander_K(t)$.
    Wszystkie jego wyrazy mają ten sam wyznacznik, ponieważ $\det K_n = |\alexander_K(-1)|$.
    Z faktu \ref{prp:bankwitz} wynika, że indeks skrzyżowaniowy węzłów $K_n$ jest wspólnie ograniczony: $c_k \le \det K_n = \det K$.
    To prowadzi do sprzeczności: węzłów o~danym indeksie skrzyżowaniowym jest tylko skończenie wiele.
\end{proof}

Przedstawimy teraz kilka innych definicji, które prowadzą do tego samego niezmiennika.

\begin{definition}[relacja kłębiasta]
    Niech $L$ będzie zorientowanym splotem z ustalonym diagramem oraz skrzyżowaniem.
    Oznaczmy przez $L_+, L_-, L_0$ trzy diagramy splotów, które różnią się jedynie na małym obszarze wokół ustalonego skrzyżowania:
\begin{comment}
    \[
        \skeinplus \quad\quad\quad\quad
        \skeinminus \quad\quad\quad\quad
        \skeinzero
    \]
\end{comment}
    Mówimy, że niezmiennik zorientowanych splotów $f$ spełnia relację kłębiastą, jeżeli wartości $f(L_+)$, $f(L_-)$ i $f(L_0)$ są związane pewnym wielomianowym równaniem, niezależnie od wyboru splotu $L$.
\end{definition}

Termin ,,skein'' (kłąb) wprowadził Conway około roku 1970, kontynuując tradycję używania słów, które kojarzą się ze sznurkami.

\begin{definition}
    Niech $L$ będzie zorientowanym splotem.
    Wielomian Laurenta $\alexander_L(t) \in \Z[t^{\pm 1/2}]$, który spełnia relację kłębiastą
    \begin{equation}
        \alexander_{L_+}(t) - \alexander_{L_-}(t) - (t^{1/2} - t^{-1/2}) \alexander_{L_0}(t) = 0
    \end{equation}
    z warunkiem brzegowym $\alexander_{\LittleUnknot}(t) = 1$, nazywamy wielomianem Alexandera.
\end{definition}

Wzór ten, choć znany był Alexanderowi, nie zyskał przez wiele dekad uwagi matematyków.
Mogło tak być, gdyż w pracy \cite{alexander28} znalazł się on na samym końcu, pod nagłówkiem ,,twierdzenia różne''.
Na nowo odkrył go Conway: chcąc szybko liczyć wielomian Alexandera zaproponował, by reparametryzować go wzorem $\alexander(x^2) = \conway(x - 1/x)$.
Spełnia wtedy zależność
\begin{equation}
    \conway_{L_+}(x)- \conway_{L_-}(x) = x \conway_{L_0}(x).
\end{equation}

Relacja kłębiasta wystarcza do wyznaczenia $\alexander_L$ każdego splotu na mocy lematu \ref{lem:unknotting_well_defined}.
Dzięki niej wiemy też, że wielomian Alexandera nie odróżnia od siebie niesplotów.
Wady tej nie posiada wielomian Jonesa.

\begin{proposition}
    \label{prp:alexander_unlinks}
    Niech $L$ będzie splotem rozszczepialnym.
    Wtedy $\alexander_L(t) \equiv 0$.
\end{proposition}

\begin{proof}
    Skorzystamy z~relacji kłębiastej.
    Niech $L_0$ będzie splotem rozsczepialnym z~dwoma ogniwami.
    Wtedy węzły $L_+$ oraz $L_-$ powstałe przez dodanie skrzyżowania między ogniwami są tego samego typu, zatem
    \begin{equation}
        \alexander_{L_0} = \frac{\alexander_{L_+} - \alexander_{L_-}}{t^{1/2} - t^{-1/2}} = 0,
    \end{equation}
    a to chcieliśmy udowodnić.
\end{proof}

Implikacja w drugą stronę jest fałszywa.
Niech $\sigma_* = \sigma_{2} \sigma_{3}^{-2} \sigma_{2}$.
Domknięcie warkocza $\sigma_{1} \sigma_* \sigma_{1} \sigma_{3} \sigma_* \sigma_{1} \sigma_{3} \sigma_* \sigma_{3}$ nie jest rozszczepialne, ale jego wielomian Alexandera jest zerem.
\index{warkocz}
Warkocze poznamy w rozdziale piątym.

% Wielomian Jonesa (zdefiniowany w~kolejnej sekcji) spełnia podobną równość.
% Ich istnienie może nasunąć przypuszczenie, że dają się wspólnie uogólnić.
% Tak jest w~rzeczywistości -- mocniejszym niezmiennikiem okazał się wielomian HOMFLY-PT.

Murasugi podejrzewał, że w~przypadku węzłów alternujących ciąg współczynników jest unimodalny.
Dowód podano dla węzłów algebraicznych (Murasugi \cite{murasugi85}) oraz genusu dwa (Ozsvath i~Szabo w~\cite{ozsvath03}).
Hipoteza w~ogólnym przypadku pozostaje otwarta.
% Wiemy natomiast, że jeśli węzeł jest alternujący, to kolejne współczynniki wielomianu Conwaya $\alexander$ są przeciwnym znaków? (Murasugi, 242)

% Remark (M. Hutchings) There does exist a~categorification of the Alexander polynomial, or more precisely of ∆K(t)/(1 − t)2, where ∆K(t) denotes the (symmetrized) Alexander polynomial of the knot K. It is a~kind of Seiberg-Witten Floer homology of the three-manifold obtained by zero surgery on K.
% One can regard it as Z×Z/2Z graded, although in fact the column whose Euler characteristic gives the coefficient of tk is relatively Z/2kZ graded.

Kondo pokazał w \cite{kondo79}, że dla każdego węzła można znaleźć inny, 1-gordyjski węzeł o tym samym wielomianie Alexandera.
Wynika stąd, że nie ma związku między wielomianem $\Delta$ oraz liczbą gordyjską.

Na sam koniec pozostawiliśmy najstarszą definicję wielomianu Alexandera.
Niech $K$ będzie węzłem w~3-sferze, zaś $X$ nieskończonym nakryciem cyklicznym jego dopełnienia.
Można je otrzymać rozcinając dopełnienie wzdłuż powierzchni Seiferta.
Na $X$, a~przez to także na grupie homologii $H_1(X)$, działa automorfizm $t$, który czyni z~niej moduł nad pierścieniem $\Z[t, t^{-1}]$, i~to skończenie prezentowalny.
Jeśli posiada przedstawienie z~$r$ generatorami i~$s$ relacjami, gdzie $r \le s$, rozpatrzmy ideał generowany przez minory $r \times r$ macierzy prezentacji (jeśli nie, weźmy ideał zerowy).
Alexander pokazał, że ideał ten zawsze jest niezerowy i~główny.

\begin{tobedone}
    Definicja przy użyciu różniczek Foxa:
    https://math.berkeley.edu/~hutching/teach/215b-2004/yu.pdf
\end{tobedone}
\section{Wielomian Jonesa} % (fold)


Relacja kłębiasta okazała się być kluczem do sukcesu w poszukiwaniu nowych niezmienników wielomianowych.
Vaughan Jones, matematyk nowozelandzki, zaprezentował w~roku 1984 nowy niezmiennik splotów jako produkt uboczny podczas pracy nad algebrami operatorowymi, które nas nie interesują.
Był to przełomowy rezultat, a~już cztery miesiące później ogłoszono znalezienie jeszcze bardziej wyrafinowanego niezmiennika, któremu przyjrzymy się w~kolejnej sekcji.

Aby lepiej zrozumieć wielomian Jonesa, zbadamy najpierw nieco prostszy obiekt, nawias Kauffmana.
Później zajmiemy się węzłami alternującymi.

\subsection{Nawias Kauffmana} % (fold)
\label{sub:kauffman_bracket}
Zaczniemy od zdefiniowania nawiasu Kauffmana.
Przypomnijmy, wielomian Laurenta zmiennej $X$ to formalny symbol $f=a_r X^r + \ldots + a_s X^s$,
gdzie $r, s, a_r, \ldots, a_s$ są całkowite i $r \le s$.

Poszukujemy niezmiennika dla splotów o kilku prostych własnościach.
Przede wszystkim żądamy,
by niewęzłowi przypisany był wielomian $1$: $\bracket{\LittleUnknot} = 1$.
Po drugie chcemy wyznaczać nawiasy znając je dla prostszych splotów,
co zapiszemy symbolicznie $\bracket{\LittleRightCrossing} = A \bracket{\LittleRightSmoothing} + B \bracket{\LittleLeftSmoothing}$.
Zależy nam wreszcie na tym, by móc dodać do splotu trywialną składową:
$\langle L \cup \LittleUnknot \rangle = C \langle L \rangle$.

Prosty rachunek pokazuje wpływ drugiego ruchu Reidemeistera na nawias:
\[
    \bracket{\reidemeisterIIaa}
    = (A^2 + ABC + B^2) \bracket{\LittleLeftSmoothing} + BA \bracket{\LittleRightSmoothing}
    \stackrel{?}{=} \bracket{\LittleRightSmoothing}.
\]

Aby zachodziła ostatnia równość wystarczy (chociaż wcale nie trzeba) przyjąć
$B = A^{-1}$, co wymusza na nas $C = -A^2 - A^{-2}$.
W ten sposób odkryliśmy następującą definicję.

\begin{definition}[nawias Kauffmana]
    \index{nawias!Kauffmana}
    Wielomian Laurenta $\bracket{D}$ dla diagramu splotu $D$ zmiennej $A$,
    który jest niezmienniczy ze względu na gładkie deformacje diagramu,
    a przy tym spełnia trzy poniższe aksjomaty, to nawias Kauffmana.
    \begin{align}
        \bracket{\LittleUnknot} & = 1 \\
        \bracket{D \sqcup \LittleUnknot} & = (-A^{-2} - A^2) \bracket{D} \\
        \bracket{\LittleRightCrossing} & = A \bracket{\LittleRightSmoothing} + A^{-1} \bracket{\LittleLeftSmoothing}
    \end{align}
\end{definition}

Tutaj $\LittleUnknot$ oznacza standardowy diagram dla niewęzła,
zaś trzy symbole $\LittleRightCrossing$, $\LittleRightSmoothing$ oraz $\LittleLeftSmoothing$ odnoszą się do diagramów,
które są identyczne wszędzie poza małym obszarem (tak jak w relacji kłębiastej).

Diagramy $\LittleRightSmoothing$ oraz $\LittleLeftSmoothing$ nazywa się odpowiednio
dodatnim (prawym) i ujemnym (lewym) wygładzeniem $\LittleRightCrossing$.

\begin{lemma}
    Nawias Kauffmana każdego diagramu wyznacza się w skończenie wielu krokach.
\end{lemma}

\begin{proof}
    Jeżeli diagram $D$ ma $n$ skrzyżowań, to nieustanne stosowanie aksjomatu trzeciego pozwala na zapisanie $\bracket{D}$ jako sumy $2^n$ składników,
    z których każdy jest po prostu zamkniętą krzywą i ma trywialny nawias ($\bracket{\LittleUnknot} = 1$).
    Nawias sumy wyznacza się korzystając z drugiego aksjomatu.
\end{proof}

Przedstawimy teraz wpływ ruchów Reidemeistera na nawias Kauffmana.

\begin{lemma}
    Drugi i trzeci ruch Reidemeistera nie ma wpływu na klamrę Kauffmana,
    pierwszy ruch zmienia ją zgodnie z regułą:
    \[
        \bracket{\reidemeisterIa} = -A^{-3} \bracket{\,\reidemeisterIb\,}.
    \]
\end{lemma}

\begin{proof}
Pierwszy ruch Reidemeistera:
\[
    \bracket{\reidemeisterIa} \stackrel{K3}{=} A \bracket{
    \begin{tikzpicture}[baseline=-0.65ex,scale=0.07]
    \useasboundingbox (-4, -5) rectangle (3, 5);
    \begin{knot}[clip width=5, end tolerance=1pt]
        \strand[semithick]
            (-3, 5) [in=left, out=down] to (-1,1) [in=left, out=right]
                                        to (1,3)
                                        to [in=up, out=right] (3,0);
        \strand[semithick]
            (-3, -5) [in=left, out=up] to (-1,-1) [in=left, out=right]
                                       to (1, -3)
                                       to [in=down, out=right] (3,0);
    \end{knot}
    \end{tikzpicture}}
    + A^{-1} \bracket{\,
    \begin{tikzpicture}[baseline=-0.65ex,scale=0.07]
    \begin{knot}[clip width=5]
        \strand[semithick] (0,-5) [in=down, out=up] to (1, -2) to (1, 2) to (0, 5);
        \strand[semithick] (4,0) circle (1.5);
    \end{knot}
    \end{tikzpicture}}
    \stackrel{K2}{=} A \bracket{\,\reidemeisterIb\,} + A^{-1}(-A^{-2}-A^2) \bracket{\,\reidemeisterIb\,}
    = -A^{-3}\bracket{\,\reidemeisterIb\,}
\]

Pierwsza równość wynika z $K3$, druga z $K2$, trzecia jest oczywista.
Dla drugiego ruchu:
\begin{align*}
    \bracket{\reidemeisterIIa} &\stackrel{K3}{=} A
    \bracket{\reidemeisterIab}
    + A^{-1} \bracket{\begin{tikzpicture}[baseline=-0.65ex,scale=0.07]
    \useasboundingbox (-5, -6) rectangle (5, 6);
    \begin{knot}[clip width=5, end tolerance=1pt]
        \strand[semithick] (4,-5) .. controls (4,-2) and (-4,-2) .. (-4,0);
        \strand[semithick] (4,5) to (4,0);
        \strand[semithick] (-4,-5) .. controls (-4,-2) and (4,-2) .. (4,0);
        \strand[semithick] (-4,5) to (-4,0);
    \end{knot}
    \end{tikzpicture}}
    \stackrel{K1}{=} -A^{-2} \bracket{\LittleLeftSmoothing} + A^{-1}
    \bracket{\begin{tikzpicture}[baseline=-0.65ex,scale=0.07]
    \useasboundingbox (-5, -6) rectangle (5, 6);
    \begin{knot}[clip width=5, end tolerance=1pt]
        \strand[semithick] (4,-5) .. controls (4,-2) and (-4,-2) .. (-4,0);
        \strand[semithick] (4,5) to (4,0);
        \strand[semithick] (-4,-5) .. controls (-4,-2) and (4,-2) .. (4,0);
        \strand[semithick] (-4,5) to (-4,0);
    \end{knot}
    \end{tikzpicture}}
    \\ & \stackrel{K3}{=} -A^{-2} \bracket{\LittleLeftSmoothing}
    + A^{-1}A \bracket{\LittleRightSmoothing} + A^{-1}A^{-1} \bracket{\LittleLeftSmoothing}
    = \bracket{\LittleRightSmoothing}
\end{align*}

Dla trzeciego ruchu:
\begin{align*}
\bracket{\,\reidemeisterIIIa\,} &\stackrel{K3}{=} A
\bracket{\,\begin{tikzpicture}[baseline=-0.65ex,yscale=0.07, xscale=0.1]
    \useasboundingbox (-5, -6) rectangle (5, 6);
    \begin{knot}[clip width=5, flip crossing/.list={1,2,3}, end tolerance=1pt]
        \strand[semithick] (-5, 5) [in=-135, out=-45] to (5,5);
        \strand[semithick] (-5, -5) [in=135, out=45] to (5,-5);
        \strand[semithick] (-5, 0) .. controls (-2, 0) and (-2,5) .. (0,5) .. controls (2, 5) and (2, 0) .. (5, 0);
    \end{knot}
    \end{tikzpicture}\,}
+A^{-1} \bracket{\RightCrossSmoothing} \\
%\stackrel{R2}{=} A \bracket{\,\LeftCrossSmoothing\,} +A^{-1} \bracket{\RightCrossSmoothing} \\
& \stackrel{R2}{=} A \bracket{\,\LeftCrossSmoothing\,} +A^{-1} \bracket{\RightCrossSmoothing}
\stackrel{K3}{=} \bracket{\,\reidemeisterIIIb\,}
\end{align*}
korzystaliśmy tu z własności drugiego ruchu.
\end{proof}

Okazało się, że użycie najprostszego, I ruchu Reidemeistera, ,,psuje'' nawias!
W akcie desperacji moglibyśmy zmienić definicję,
zaniechamy tego i przejdziemy do kolejnego składnika w przepisie na wielomian Jonesa.

Gregor Schaumann w notatkach \cite{schaumann16} wprowadza rachunek schematyczny,
który pozwala spojrzeć na nawias Kauffmana z nowej perspektywy.
Definiuje kwantowe niezmienniki oraz plątaniny (równoważne wtedy,
kiedy związane są przez ciąg tożsamości wężowych, ruchów Turaewa oraz Reidemeistera).
Wspomniane są też dokonania Wasiljewa.
% Koniec podsekcji Nawias Kauffmana


\subsection{Definicja algebraiczna -- algebra Temperleya-Lieba} % (fold)
\label{sub:jones_paper}
Jones otrzymał swój wielomian jako efekt uboczny badań nad algebrami operatorowymi: wziął ślad pewnej reprezentacji warkoczy w~algebrę, która miała ważne znaczenie w~mechanice statystycznej.
Dalszy opis pochodzi z Wikipedii.
Zaletą tego podejścia jest możliwość wyboru algebry, która reprezentuje grupę warkoczy.

\begin{definition}[algebra Temperleya-Lieba]
    Niech $R$ będzie przemiennym pierścieniem, w~którym ustalono element $\delta \in R$.
    Wtedy $R$-algebrę $TL_n(\delta)$ generowaną przez elementy $e_1, \ldots, e_{n-1}$, które związane są relacjami
    \begin{align}
        e_i^2 & = \delta e_i, \\
        e_i e_{i \pm 1} e_i & = e_i, \\
        e_i e_j & = e_j e_i
    \end{align}
    dla $|i-j| \ge 2$, nazywamy algebrą Temperleya-Lieba.
    % Algebra Temperleya-Lieba $A_n$ to wolna addytywna algebra na multiplikatywnych generatorach $e_1, \ldots, e_{n-1}$ traktowana jako $\C[\tau, \tau^{-1}]$-moduł.
    % Zmienna $\tau$ komutuje ze wszystkimi generatorami, generatory zaś spełniają relacje ($j$ jest różne od $i - 1, i, i+1$):
\end{definition}

$TL_n(\delta)$ można przedstawić przy użyciu diagramów: prostokątów, których przeciwległe boki zawierają po $n$ punktów połączonych w~pary tak, by uniknąć samoprzecięć. Mnożenie elementów algebry odpowiada sklejaniu dwóch diagramów, przy czym każdą zamkniętą pętlę zamieniamy na dodatkowy czynnik $\delta$.
To w~gruncie rzeczy są warkocze.

\begin{definition}[ślad Markowa]
    Niech $K \in TL_n(\delta)$ będzie elementem algebry Temperleya-Lieba będącym iloczynem generatorów\footnote{Czyli element $K$ utożsamia się z~pewnym warkoczem o~$n$ pasmach.} $e_1, \ldots, e_{n-1}$, którego domknięcie rozpada się na $m$ składowych spójności.
    Śladem Markowa elementu $K$ nazywamy wielkość $\operatorname{tr} K = \delta^{m-n}$.
\end{definition}

Na mocy twierdzenia Alexandera, każdy splot $L$ jest domknięciem warkocza zaplecionego na pewnej liczbie pasm.
Zdefiniujmy reprezentację $\rho \colon B_n \to TL_n$ grupy warkoczy w~algebrę Temperleya-Lieba związaną z pierścieniem $R = \Z[A, 1/A]$ oraz elementem $\delta = -A^2 - A^{-2}$ wzorem
\begin{equation}
    \rho(\sigma_i) = A \cdot e_i + \frac{1}{A} \cdot 1.
\end{equation}
Wtedy $\langle K \rangle = \delta^{n-1} \operatorname{tr} \rho (\sigma)$ jest klamrą Kauffmana.
Pozostaje sprawdzić wpływ ruchów Markowa na złożenie $\operatorname{tr} \circ \rho$, ponieważ sploty nie przedstawiają się jako domknięcia warkoczy jednoznacznie.

% Koniec podsekcji Oryginalna praca Jonesa


\subsection{Rozpiętość i~wielomian Jonesa} % (fold)
\label{sub:span}
Przytoczmy raz jeszcze treść pierwszej hipotezy Taita (\ref{conj_tait_i}).

\begin{conjecture}[Tait]
    Jeśli zorientowany splot $L$ posiada zredukowany, spójny, alternujący diagram o~$n$ skrzyżowaniach, to każdy jego diagram składa się z co najmniej $n$ skrzyżowań.
\end{conjecture}

Zanim przejdziemy do dowodu, wyjaśnimy użyte w tym stwierdzeniu przymiotniki.

\begin{definition}
    \index{diagram!zredukowany}
    Diagram jest zredukowany, gdy nie zawiera usuwalnych skrzyżowań:
    \[
        \begin{tikzpicture}[baseline=-0.65ex,scale=0.07]
        \begin{knot}[clip width=5]
            \strand[semithick] (-5,-5) rectangle (5,5);
            \strand[semithick] (-5, -3) [in=right, out=left] to (-15, 3);
            \strand[semithick] (-5, 3) [in=right, out=left] to (-15, -3);

            \node at (-20, -3) {$\ldots$};
            \node at (-20,  3) {$\ldots$};
        \end{knot}
        \end{tikzpicture}
    \]
\end{definition}

\begin{definition}
    Diagram jest spójny, gdy nie można go podzielić na dwie niepuste części, które nie spotykają się na żadnym skrzyżowaniu.
\end{definition}

W dowodzie hipotezy Taita użyjemy rozpiętości wielomianu Jonesa.

\begin{definition}
    Rozpiętością wielomianu Laurenta $f$ jednej zmiennej $X$ nazywamy różnicę między najwyższym oraz najmniejszym wykładnikiem pojawiającym się w~$f$: $\operatorname{span} f = M_f - m_f$.
\end{definition}

Zajmiemy się teraz wzorem pozwalającym na wyznaczenie nawiasu Kauffmana dowolnego splotu o~$n$ skrzyżowaniach przez dodanie $2^n$ wyrazów (które odpowiadają digramom bez skrzyżowań).
Wzór ten okaże się użyteczny przy dowodzeniu późniejszych twierdzeń.

\begin{definition}[stan]
    Niech $D$ będzie diagramem splotu.
    Każdą funkcję $s$ ze zbioru skrzyżowań diagramu $D$ o wartościach $\pm 1$ nazywamy stanem.
    Sumę wszystkich wartości $s$ oznaczamy $|s|$.
\end{definition}

\begin{definition}
    Niech $D$ będzie diagramem splotu, zaś $s$ jego stanem.
    Diagram powstały przez wygładzenie wszystkich skrzyżowań zgodnie z~ich stanem oznaczamy $sD$.
    Przez $|sD|$ rozumiemy liczbę zamkniętych, nieprzecinających się krzywych, z których składa się nowy diagram.
\end{definition}

\begin{proposition}[o sumowaniu stanów]
    Niech $D$ będzie diagramem splotu.
    Wtedy
    \begin{equation}
        \langle D\rangle = \sum_s \underbrace{(-A^2-A^{-2})^{|sD|-1} A^{|s|}}_{\langle D \mid s \rangle},
    \end{equation}
    gdzie sumujemy po wszystkich stanach $s$ dla diagramu $D$.
\end{proposition}

\begin{proof}
    Oznaczmy prawą stronę dowodzonej równości przez $[D]$.
    Pokażemy, że spełnia ona $[\LittleUnknot]=1$, $[D\sqcup\LittleUnknot]=(-A^{-2}-A^2) [D]$ oraz $[\LittleRightCrossing] = A [\LittleRightSmoothing] + A^{-1}[\LittleLeftSmoothing]$.
    Stąd wynika już, że $[D] = \bracket{D}$.

    Niewęzeł $\LittleUnknot$ ma tylko jeden stan $s$ z~$|s| = 0$ i~$|s\,\LittleUnknot| = 1$.

    Zauważmy, że $D \sqcup \LittleUnknot$ i~$D$ mają te same skrzyżowania,
    więc możemy utożsamiać stany $s$ dla $D$ ze stanami $u$ dla $D \sqcup \LittleUnknot$.
    Wtedy $|u| = |s|$ oraz $|u(D \sqcup \LittleUnknot)| = |sD| + 1$.
    Zatem
    \begin{align*}
        \left[D \sqcup \LittleUnknot\right]
        & = \sum_u (-A^2-A^{-2})^{|u(D\sqcup\LittleUnknot)|-1} A^{|u|} \\
        & = \sum_s (-A^2-A^{-2})^{|sD|} A^{|s|} = (-A^2-A^{-2}) [D].
    \end{align*}

    Pozostała trzecia własność. Z definicji $A[\LittleRightSmoothing]=\sum_u(-A^2-A^{-2})^{|u\LittleRightSmoothing|-1}A^{|u|+1}$,
    gdzie $u$ przebiega wszystkie stany $\LittleRightSmoothing$.
    Ale $\LittleRightSmoothing$ to $\LittleRightCrossing$ ze skrzyżowaniem (powiedzmy, $c$) wygładzonym dodatnio,
    co daje bijekcję między stanami $u$ dla $\LittleRightSmoothing$ i~$s$ dla $\LittleRightCrossing$, dla których $s(c) = + 1$.
    Wtedy $|s\LittleRightCrossing|=|u\LittleRightSmoothing|$ i~$|s|=|u|+1$ oraz
    \begin{equation}
    A\left[\LittleRightSmoothing\right] =
    \sum_u (-A^2-A^{-2})^{|u\,\LittleRightSmoothing|-1}A^{|u|+1} =
    \sum_{s(c)=1}(-A^2-A^{-2})^{|s\,\LittleRightCrossing|-1}A^{|s|},
    \end{equation}
    podobne rozumowanie pokazuje, że $A^{-1}[\LittleLeftSmoothing] = \sum_{s(c)=-1}(-A^2-A^{-2})^{|s\,\LittleRightCrossing|-1}A^{|s|}$.
    Teraz wystarczy dodać do siebie dwa ostatnie równania. %: $A[\PrawyGladki]+A^{-1}[\LewyGladki] = \sum_s(-A^2-A^{-2})^{|s\,\LittleRightCrossing|-1}A^{|s|} = [\RightCrossing]$.
\end{proof}

Zbadamy teraz dwa najprostsze stany dowolnego diagramu.

\begin{definition}
    Stan przypisujący znak $+ 1$ każdemu skrzyżowaniu nazywamy $s_+$.
    Analogicznie definiujemy $s_-$.
\end{definition}

Niech $D$ będzie alternującym, zredukowanym diagramem spójnym.
Wtedy wszystkie jego skrzyżowania mają ten sam znak.
Wybierzmy dla niego uszachowienie.
\[
    \begin{tikzpicture}[baseline=-0.65ex,scale=0.15]
    \begin{knot}[clip width=5]
        \strand[semithick] (-25, 0) to (25, 0);
        \strand[semithick] (10*0-15, -5) to (10*0-15, 5);
        \strand[semithick] (10*1-15, -5) to (10*1-15, 5);
        \strand[semithick] (10*2-15, -5) to (10*2-15, 5);
        \strand[semithick] (10*3-15, -5) to (10*3-15, 5);
        \draw[fill=blue!10!white,draw=none] (-25, 0) rectangle (-15, -5);
        \draw[fill=blue!10!white,draw=none] (-15, 0) rectangle (-5, 5);
        \draw[fill=blue!10!white,draw=none] (-5, 0) rectangle (5, -5);
        \draw[fill=blue!10!white,draw=none] (5, 0) rectangle (15, 5);
        \draw[fill=blue!10!white,draw=none] (15, 0) rectangle (25, -5);
        \node[above left] at (-15, 0) {$+1$};
        \node[above left] at (5, 0) {$+1$};
        \node[below left] at (-5, 0) {$+1$};
        \node[below left] at (15, 0) {$+1$};
    \end{knot}
    \end{tikzpicture}
\]

Zamieniając biały i~czarny w~razie potrzeby możemy założyć, że wszystkie skrzyżowania są dodatnie ($+1$).
Takie uszachowienie nazywamy \emph{standardowym}.
Porównajmy wygładzenie $s_+D$ z~$s_-D$:
\[
    \begin{tikzpicture}[baseline=-0.65ex,scale=0.10]
        \node at (0, 8) {$s_+D$};
        \draw[fill=blue!10!white,draw=none] (-25, -5) rectangle (25, 5);
        \draw[fill=white, draw=none] (-15, -5) [in=left, out=up] to (-12, 0) -- (-8, 0) [in=up, out=right] to (-5, -5);
        \draw[fill=white, draw=none] (5, -5) [in=left, out=up] to (8, 0) -- (12, 0) [in=up, out=right] to (15, -5);
        \draw[fill=white, draw=none] (-5, 5) [in=left, out=down] to (-2, 0) -- (2, 0) [in=down, out=right] to (5, 5);
        \draw[fill=white, draw=none] (-25, 0) -- (-18, 0) [in=down, out=right] to (-15, 5) -- (-25, 5);
        \draw[fill=white, draw=none] ( 25, 0) -- ( 18, 0) [in=down, out=left] to ( 15, 5) -- ( 25, 5);
    \end{tikzpicture}
    \quad
    \begin{tikzpicture}[baseline=-0.65ex,scale=0.10]
        \node at (0, 8) {$s_-D$};
        \draw[fill=blue!10!white, draw=none] (-15, 5) [in=left, out=up] to (-12, 0) -- (-8, 0) [in=up, out=right] to (-5, 5);
        \draw[fill=blue!10!white, draw=none] (5, 5) [in=left, out=up] to (8, 0) -- (12, 0) [in=up, out=right] to (15, 5);
        \draw[fill=blue!10!white, draw=none] (-5, -5) [in=left, out=down] to (-2, 0) -- (2, 0) [in=down, out=right] to (5, -5);
        \draw[fill=blue!10!white, draw=none] (-25, 0) -- (-18, 0) [in=down, out=right] to (-15, -5) -- (-25, -5);
        \draw[fill=blue!10!white, draw=none] ( 25, 0) -- ( 18, 0) [in=down, out=left] to ( 15, -5) -- ( 25, -5);
    \end{tikzpicture}
\]

Zamknięte krzywe tworzące $s_+D$ są brzegami białych obszarów uszachowienia,
podczas gdy te tworzące $s_-D$ stanowią brzeg czarnych obszarów.
Zauważmy jeszcze, że na każdym skrzyżowaniu są cztery różne czarne i~białe obszary
(nie mogą się spotkać w~innych miejscach), gdyż diagram był zredukowany.

\begin{lemma}
    Niech $D$ będzie spójnym diagramem splotu o~$n$ skrzyżowaniach.
    Wtedy
    \begin{equation}
        |s_+D|+|s_-D|\le n+2,
    \end{equation}
    z~równością gdy diagram $D$ jest alternujący i~zredukowany.
\end{lemma}

\begin{proof}
    Skorzystamy z~indukcji względem $n$.
    Łatwo widać prawdziwość lematu dla $n = 0$.
    Załóżmy, że jest on prawdziwy dla wszystkich diagramów o~$n - 1$ skrzyżowaniach, następnie ustalmy diagram $D$ o~$n$ skrzyżowaniach.

    Wybierzmy skrzyżowanie z~$D$. Można je wygładzić na dwa sposoby, jeden z~nich daje spójny diagram $D'$.
    Bez straty ogólności przyjmijmy, że jest to dodatnie wygładzenie.
    Wtedy zachodzi $|s_+D'| = |s_+D|$, ale $|s_-D'|=|s_-D|\pm 1$, ponieważ $s_-D'$ powstaje z~$s_-D$ przez zastąpienie pewnej części
    $\LittleRightSmoothing$ z~$\LittleLeftSmoothing$.
    To rozrywa jedną krzywą na dwa kawałki lub scala dwie krzywe w~jedną.
    Teraz z założenia indukcyjnego wynika
    \begin{equation}
    |s_+D|+|s_-D|
    =   |s_+D'| + |s_-D'| \pm 1
    \le (n - 1) + 2 \pm 1
    \le n + 2.
    \end{equation}

    Załóżmy, że $D$ jest spójny, alternujący oraz zredukowany.
    Musimy pokazać, że ostatnie dwie nierówności tak naprawdę są równościami.
    Pierwsza wynika z~tego, że $D'$ jest spójny, alternujący i~zredukowany.
    Z drugiej strony $|s_-D'|=|s_-D|-1$, ponieważ przejście od $s_-D$ do $s_-D'$ skleja dwa czarne obszary.
    To pokazuje drugą równość i~kończy dowód.
    \[
        \begin{tikzpicture}[baseline=-0.65ex,scale=0.20]
        \begin{knot}[clip width=5]
            \strand[semithick] (-5, 0) to (5, 0);
            \strand[semithick] (0, -5) to (0, 5);
            \draw[fill=blue!10!white,draw=none] (-5, -5) rectangle (0, 0);
            \draw[fill=blue!10!white,draw=none] ( 5,  5) rectangle (0, 0);
            \node at (0, -8) {$D$};
        \end{knot}
        \end{tikzpicture}
        \quad
        \begin{tikzpicture}[baseline=-0.65ex,scale=0.20]
            \draw[fill=blue!10!white, draw=none] (-5, 0) -- (-2, 0) [in=up, out=right] to (0, -2) -- (0, -5) -- (-5, -5);
            \draw[fill=blue!10!white, draw=none] (5, 0) -- (2, 0) [in=down, out=left] to (0, 2) -- (0, 5) -- (5, 5);
            \draw[semithick] (-5, 0) -- (-2, 0) [in=up, out=right] to (0, -2) -- (0, -5);
            \draw[semithick] (5, 0) -- (2, 0) [in=down, out=left] to (0, 2) -- (0, 5);
            \node at (0, -8) {$s_-D$};
        \end{tikzpicture}
        \quad
        \begin{tikzpicture}[baseline=-0.65ex,scale=0.20]
            \draw[fill=blue!10!white, draw=none] (-5, -5) rectangle (5, 5);
            \draw[fill=white, draw=none] (5, 0) -- (2, 0) [in=up, out=left] to (0, -2) -- (0, -5) -- (5, -5);
            \draw[fill=white, draw=none] (-5, 0) -- (-2, 0) [in=down, out=right] to (0, 2) -- (0, 5) -- (-5, 5);
            \draw[semithick] (5, 0) -- (2, 0) [in=up, out=left] to (0, -2) -- (0, -5);
            \draw[semithick] (-5, 0) -- (-2, 0) [in=down, out=right] to (0, 2) -- (0, 5);
            \node at (0, -8) {$s_-D'$};
        \end{tikzpicture}
        \qedhere
    \]
    \end{proof}

 \begin{lemma}
    Niech $D$ będzie diagramem splotu o~$n$ skrzyżowaniach.
    Wtedy
    \begin{enumerate}
        \item $M \langle D \rangle \le n+2|s_+D|-2$
        \item $m \langle D \rangle \ge -n-2|s_-D|+2$
    \end{enumerate}
    z równością, jeżeli $D$ jest alternujący, zredukowany i~spójny.
\end{lemma}

\begin{proof}
    Udowodnimy tylko pierwszą część, druga jest do niej podobna.
    Oznaczymy przez $\langle D \mid s \rangle$ wielkość $(-A^{-2}-A^2)^{|sD|-1}A^{|s|}$.
    Zauważmy, że $M\langle D|s\rangle=2|sD|+|s|-2$,
    a więc w~szczególności $M\langle D|s_+\rangle=2|s_+D|+n-2$.
    Gdyby udało się nam pokazać, że $M\langle D|s\rangle \le M\langle D|s_+\rangle$
    dla wszystkich innych stanów $s$, dowód nierówności byłby zakończony.
    Ale możemy znaleźć ciąg $s_+ = s_0$, $s_1$, \ldots, $s_r=s$,
    w którym $s_{i+1}$ powstaje z~$s_i$ przez pojedynczą zmianę $+1$ na $-1$.

    Teraz $|s_{i+1}|=|s_i|-2$, podczas gdy $|s_{i+1}D|=|s_iD|\pm 1$,
    ponieważ $s_{i+1}D$ uzyskujemy z~$s_{i}D$ przez połączenie dwóch zamkniętych krzywych lub podział jednej zamkniętej krzywej na dwie części.
    Zatem
    \begin{align*}
        M \langle D \mid s_{i+1} \rangle & =
        2|s_{i+1}D|+|s_{i+1}|-2 \\ & =
        (2|s_iD| + |s_i| -2 ) + (\pm 2-2) \le
        M \langle D|s_i\rangle.
    \end{align*}

    Widać już, że $M\langle D \mid s\rangle =M\langle D \mid s_r\rangle \le\ldots\le M\langle D \mid s_0\rangle=M\langle D \mid s_+\rangle$.

    Pokażemy teraz, że jeśli $D$ jest zredukowany, alternujący i~spójny, to nierówność zamienia się w~równość.
    Będzie to wynikało z~ $M\langle D|s\rangle<M\langle D| s_+\rangle$
    dla $s\neq s_+$, jeżeli tylko powołamy się na powyższy argument.
    Wystarczy ograniczyć się do tych $s$, które powstają z~$s_+$ przez zmianę pojedynczego stanu $+1$ na $-1$.
    Ale to już jest oczywiste, gdyż $sD$ otrzymujemy przez sklejenie dwóch białych obszarów $s_+ D$.
\end{proof}

Możemy wreszcie zająć się rozpiętością wielomianu Jonesa.

\begin{proposition}
    Niech $L$ będzie zorientowanym splotem o~spójnym diagramie $D$ z~$n$ skrzyżowaniami.
    Wtedy $\operatorname{span}(V(L)) \le n$, z~równością dla zredukowanego i~alternującego $D$.
\end{proposition}

\begin{proof}
    Pokażemy prawdziwość innego, równoważnego stwierdzenia: $\operatorname{span} \langle D\rangle\le 4n$
    z~równością dla zredukowanego i~alternującego $D$.
    Dwa poprzednie lematy mówią, że
    \begin{align*}
        \operatorname{span}\langle D\rangle
        & = M\langle D\rangle - m\langle D\rangle \le (2|s_+D|+n-2)+(2|s_-D|+n-2) \\
        & = 2(|s_+D|+|s_- D|)+2n-4 \le 2(n+2)+2n-4 = 4n. \qedhere
    \end{align*}
\end{proof}

Jesteśmy już w~stanie podać dowód twierdzenia \ref{taitjones} wspomnianego na początku sekcji.

\begin{proof}
    Założenia mówią nam, że $\operatorname{span} (V(L)) = n$.
    Gdyby istniał diagram o~mniejszej liczbie skrzyżowań,
    mielibyśmy $\operatorname{span} (V(L)) < n$, co prowadzi do sprzeczności.
\end{proof}

Szukanie wielomianu Jonesa splotu bywa uciążliwe,
jednak czasami możemy oszacować jego rozpiętość korzystając z~następujących nierówności:

\begin{corollary}
    Niech $L$ będzie zorientowanym splotem ze spójnym diagramem $D$ o~$n$ skrzyżowaniach.
    Wtedy
    \begin{align*}
        3w(D)-2|s_+D|+2-n & \le 4 m(V(L) \\
        3w(D)+2|s_-D|+n-2 & \ge 4 M(V(L)),
    \end{align*}
    z~równością dla zredukowanego i~alternującego $D$.
\end{corollary}

\begin{proof}
    Proste ćwiczenie.
\end{proof}

% Koniec podsekcji Rozpiętość i~wielomian Jonesa


% Koniec sekcji Wielomian Jonesa

\section{Niezmiennik Cahita Arfa} % (fold)
\label{sub:arf}
Niezmiennik Arfa dla węzłów można zdefiniować na kilka sposobów, z~których żaden jest istotnie lepszy od pozostałych.
Dwa węzły nazwiemy równoważnymi przez przejścia, jeśli są związane skończenie wieloma ,,przejściami'':
\[
	\begin{tikzpicture}[baseline=-0.65ex,scale=0.35]
	\begin{knot}[clip width=7]
		\strand[-latex, thick] (-2.5,-1.0) to (2.5,-1.0);
		\strand[-latex, thick] (2.5,1.0) to (-2.5,1.0);
		\strand[-latex, thick] (-1.0,-2.5) to (-1.0,2.5);
		\strand[-latex, thick] (1.0,2.5) to (1.0,-2.5);
	\end{knot}
	\end{tikzpicture}
	\cong
	\begin{tikzpicture}[baseline=-0.65ex,scale=0.35]
	\begin{knot}[clip width=7, flip crossing/.list={1,2,3,4}]
		\strand[-latex, thick] (-2.5,-1.0) to (2.5,-1.0);
		\strand[-latex, thick] (2.5,1.0) to (-2.5,1.0);
		\strand[-latex, thick] (-1.0,-2.5) to (-1.0,2.5);
		\strand[-latex, thick] (1.0,2.5) to (1.0,-2.5);
	\end{knot}
	\end{tikzpicture}
	\quad\mbox{albo}\quad
	\begin{tikzpicture}[baseline=-0.65ex,scale=0.35]
	\begin{knot}[clip width=7]
		\strand[-latex, thick] (-2.5,-1.0) to (2.5,-1.0);
		\strand[-latex, thick] (2.5,1.0) to (-2.5,1.0);
		\strand[latex-, thick] (-1.0,-2.5) to (-1.0,2.5);
		\strand[latex-, thick] (1.0,2.5) to (1.0,-2.5);
	\end{knot}
	\end{tikzpicture}
	\cong
	\begin{tikzpicture}[baseline=-0.65ex,scale=0.35]
	\begin{knot}[clip width=7, flip crossing/.list={1,2,3,4}]
		\strand[-latex, thick] (-2.5,-1.0) to (2.5,-1.0);
		\strand[-latex, thick] (2.5,1.0) to (-2.5,1.0);
		\strand[latex-, thick] (-1.0,-2.5) to (-1.0,2.5);
		\strand[latex-, thick] (1.0,2.5) to (1.0,-2.5);
	\end{knot}
	\end{tikzpicture}
\]

\index{niezmiennik!Arfa}
Każdy węzeł jest równoważny przez przejścia z~niewęzłem (powiemy, że wyznacznik Arfa wynosi $0$) albo trójlistnikiem (że niezmiennik Arfa przyjmuje wartość $1$).

Takie podejście zaproponował Louis Kauffman.

\begin{proposition}[Murasugi]
	$\operatorname{Arf}(K) = 0$ wtedy i~tylko wtedy, gdy $\Delta_K(-1) \equiv \pm 1 \mod 8$.
\end{proposition}

\begin{proposition}[Jones, 1985]
	$\operatorname{Arf}(K) = V_K(i)$.
\end{proposition}

\begin{proposition}[Robertello]
	Niech
	\begin{equation}
		\Delta (t)=c_{0}+c_{1}t+\cdots +c_{n}t^{n}+\cdots +c_{0}t^{2n}
	\end{equation}
	będzie wielomianem Alexandera.
	Wtedy niezmiennik Arfa to $ c_{n-1}+c_{n-3}+\cdots +c_{r}\mod 2$, gdzie $r = 0$ dla nieparzystych $n$, $r = 1$ w~przeciwnym razie.
\end{proposition}

\begin{proposition}
	Niech $(v_{ij})$ będzie macierzą Seiferta powstałą z~krzywych genusu $g$, które reprezentują bazę pierwszej grupy homologii powierzchni.
	To oznacza, że macierz $V$ wymiaru $2g \times 2g$ ma następującą własność: różnica $V - V^t$ jest symplektyczna.
	Niezmiennik Arfa to
	\begin{equation}
		\sum^g_{i=1}v_{2i-1,2i-1}v_{2i,2i} \pmod 2.
	\end{equation}
\end{proposition}

\begin{proposition}
	Niezmiennik Arfa jest $\shrap$-addytywny.
\end{proposition}

\begin{proposition}
	Niezmiennik Arfa znika na węzłach plastrowych.
	%It is additive under connected sum, and vanishes on slice knots, so is atv invariant. https://en.wikipedia.org/wiki/Link_concordance
\end{proposition}

Węzłom plastrowym poświęcona jest sekcja \ref{sec:slice}.

% OEIS http://oeis.org/A131433, ...1434
% Koniec sekcji Niezmiennik Cahita Arfa

\section{Wielomian HOMFLY} % (fold)
\label{sec:homfly}
Po tym, jak Jones przedstawił światu swój wielomian w~1984 roku, matematycy
zaczęli szukać jego uogólnienia zależnego nie od jednej, lecz dwóch zmiennych.
Pierwszym takim węzłowym niezmiennikiem okazał się wielomian (Laurenta) HOMFLY.
Jego nazwa pochodzi od sześciu odkrywców -- Hoste, Ocneanu, Millett, Freyd, Lickorish, Yetter z~pracy \cite{homfly85}.

Dwa lata później i~niezależnie od nich, Przytycki z~Traczykiem otrzymał ten sam obiekt w~\cite{przytycki87}.
Ich nazwiska czasami są pomijane ze względu na wolno działającą pocztę (!).
\todo[inline]{Citation needed.}

\begin{definition}
    \label{homflydef}
    \index{wielomian!HOMFLY}
    Wielomian HOMFLY zorientowanego splotu $L$ to niezmienniczy na izotopie wielomian Laurenta dwóch zmiennych $m, l$ taki, że $P(\LittleUnknot) = 1$ oraz
    \[
        l P(L_+) +  \frac 1l P(L_-) + mP(L_0) = 0,
    \]
    przy oznaczeniach ze stwierdzenia \ref{tracheotomia}.
\end{definition}
\todo[inline]{Ta postać bierze się z~jednorodnej (każdy jednomian tego samego stopnia) i~podstawienia $z = 1$}


Relacja kłębiasta pozwala wywnioskować własności wielomianu sumy spójnej i~prostej.
Potrzebować będziemy lematu:

\begin{lemma}
    \label{links_homfly}
    Wielomianem HOMFLY dla niesplotu o~$n$ składowych jest
    \[
        P(U_n) = \left(-\frac{l+1/l}{m}\right)^{n-1}.
    \]
\end{lemma}

%\begin{proof}
    % Dla dowodu tej równości nie trzeba powoływać się na fakt \ref{homfly_sums}.
Można potraktować to jako proste ćwiczenie z~indukcji matematycznej, pozostawiam je uwadze Czytelnika.
%\end{proof}

\begin{proposition}
    \label{homfly_sums}
    Dla splotów $L_1, L_2$ zachodzą równości:
    \begin{align*}
        P(L_1 \sqcup L_2) & = - \frac{l + 1/l}{m} \cdot P(L_1) P(L_2) \\
        P(L_1 \shrap L_2) & = P(L_1) P(L_2)
    \end{align*}
\end{proposition}

\begin{proof}
    Dowód analogiczny do ,,multiplikatywności'' wielomianu Jonesa/Alexandera.

    Każdy splot $L$ jest kombinacją liniową (o współczynnikach będących wielomianami w~$m$ oraz $l$) niesplotów $U_k$ o~różnej liczbie składowych $k$.
    Mamy więc
    \[
        L_1 = \sum_{k=1}^n a_k U_k, \quad
        L_2 = \sum_{k=1}^n b_k U_k.
    \]

    Skorzystamy w~tym miejscu z~lematu \ref{links_homfly}.
    Wynika z~niego, że $P(U_i)P(U_j) = P(U_{i+j-1})$ i~bezpośrednio
    \[
        P(L_1)P(L_2) = \sum_{k=1}^{2n} \sum_{i=1}^{k-1} a_i b_{k-i}P(U_{k-1}).
    \]

    Pozostało spojrzeć na sumę spójną diagramów  $L_1 \shrap L_2$.
    Jeśli usuniemy teraz wszystkie skrzyżowania z~diagramu $L_1$ relacją kłębiastą, dostaniemy $L_1 \# L_2 = \sum_{k=1}^n a_k (U_{k-1} \cup L_2)$, gdyż jedna z~niezawęźlonych składowych zostanie wchłonięta do diagramu diagramu $L_2$.
    Rozwijamy dalej i~otrzymujemy
    \begin{align*}
        L_1 \# L_2 & = \sum_{k=1}^n a_k \left(U_{k-1} \cup \sum_{k=1}^n b_k U_k\right) \\
                   & = \sum_{k=1}^{2n} \sum_{i=1}^{k-1} a_i b_{k-i} U_{i-1} \cup U_{k-i} \\
                   & = \sum_{k=1}^{2n} \sum_{i=1}^{k-1} a_i b_{k-i} U_{k-1},
    \end{align*}
    co kończy dowód.
\end{proof}

Nie wiemy jeszcze, czy definicja \ref{homflydef} pozwala wyliczyć wielomian w~skończenie wielu krokach, ani czy wynik jest jednoznaczny.
Przejdę do pokazania, że tak istotnie jest.

\begin{lemma}
    W dowolnym rzucie splotu można odwrócić pewne skrzyżowania tak, by uzyskać diagram niesplotu.
\end{lemma}

Zauważmy, że wyznaczenie nawiasu Kauffmana było prostsze, ponieważ w~każdym kroku liczba skrzyżowań ulegała zmniejszeniu.

\begin{proof}
Bez straty ogólności założę, że diagram przedstawia węzeł.
Ustalmy zatem diagram węzła i~wybierzmy jakiś początkowy punkt na nim, różny od skrzyżowania wraz z~kierunkiem, wzdłuż którego będziemy przemierzać węzeł.
Za każdym razem, kiedy odwiedzamy nowe skrzyżowanie, zmieniamy je w~razie potrzeby na takie, przez które przemieszczamy się górnym łukiem.
Skrzyżowań już odwiedzonych nie zmieniamy wcale.
Po powrocie do punktu wybranego na początku uzyskamy diagram niewęzła.

Teraz wyobraźmy sobie nasz węzeł w~trójwymiarowej przestrzeni $\mathbb R^3$, przy czym oś $z$ skierowana jest z~płaszczyzny, w~której leży diagram, w~naszą stronę.
Umieśćmy początkowy punkt tak, by jego trzecią współrzędną była $z = 1$.

Przemierzając węzeł, zmniejszamy stopniowo tę współrzędną, aż osiągniemy wartość $0$ tuż przed punktem, z~którego wyruszyliśmy.
Połączmy obydwa punkty (początkowy oraz ten, w~którym osiągamy współrzędną $z = 0$) pionowym odcinkiem.
Zauważmy, że kiedy patrzymy na węzeł w~kierunku osi $z$, nie widzimy żadnych skrzyżowań.

Oznacza to, że dostaliśmy niewęzeł.
\end{proof}

\begin{proposition}
    Wielomian HOMFLY z~definicji \ref{homflydef} dla zorientowanych splotów można wyznaczyć w~skończenie wielu krokach.
\end{proposition}

\begin{proof}
    Niech $L$ będzie splotem, którego wielomian HOMFLY próbujemy wyznaczyć.
    Ustalmy jego dowolny diagram i~wybierzmy jedno ze skrzyżowań, które należy odwrócić, by uzyskać niesplot.

    Początkowy diagram odpowiada $L_+$ lub $L_-$, relacja kłębiasta pozwala na uzyskanie wielomianu wyjściowego splotu zależnego od wielomianu splotu z
    diagramem, na którym jest mniej skrzyżowań oraz splotu, który jest ,,jedno skrzyżowanie bliżej'' niesplotu.

    Powtarzając tę procedurę dojdziemy w~pewnym momencie do splotów trywialnych, gdzie powołujemy się na wniosek \ref{links_homfly}.
\end{proof}

Wielomian HOMFLY jest dużo mocniejszy od wielomianów Jonesa czy Alexandera -- są one jego szczególnymi przypadkami.
Warto pamiętać, że żaden z~nich nie jest mocniejszy od drugiego, gdyż istnieją pary węzłów, które są rozróżnialne przez dokładnie jeden z~nich.

\begin{proposition}
    Dla dowolnego zorientowanego węzła zachodzą równości:
        \begin{align*}
        V(t) & = P(l = i/t, m = i(t^{-1/2} - t^{1/2})) \\
        \Delta(t) & = P(l = i, m = i(t^{-1/2} - t^{1/2}))
    \end{align*}
\end{proposition}

Zaletą wielomianu HOMFLY jest to, że często wykrywa chiralność (węzeł chiralny nie jest równoważny swemu lustrzanemu odbiciu), ale nie odróżnia enancjomerów węzłów $9_{42}$, $10_{48}$, $10_{71}$, $10_{91}$, $10_{104}$, oraz $10_{125}$.
Wśród węzłów do dziesięciu skrzyżowań dokładnie dwa opierają się testom chiralności opartym na niezmiennikach Jonesa, HOMFLY oraz Kauffmana: $9_{42}$ oraz $10_{71}$.
Do jej wykrycia potrzebna jest na przykład $SU(2)$-teoria Cherna-Simonsa, wyjaśniona w~kolejnej wersji tego skryptu.

M. B. Thistlethwaite wyznaczył wielomiany HOMFLY dla wszystkich 12966 węzłów, które mają mniej niż 14 skrzyżowań.
Wśród nich 30 posiada wielomian Conwaya $1 + 2z^2 + 2z^4$, ale pary rozróżniane wielomianem HOMFLY mają także różne wielomiany Jonesa.
Wielomian HOMFLY odróżnia $11_{388}$ od swojego odbicia, wielomiany Jonesa i~Conwaya nie -- jest więc od nich istotnie mocniejszy.

\todo[inline]{Powtórzenie z~innymi mutantami?}
Pomimo swej mocy nie jest jednak niezmiennikiem zupełnym, nie odróżnia od siebie mutantów\footnote{z definicji \ref{def:mutant}}.
Konkretne przykłady kolizji to $5_1$ oraz $10_{132}$, $8_{8}$ oraz $10_{129}$, $8_{16}$ oraz $10_{156}$ albo $10_{25}$ oraz $10_{56}$.
Co bardziej dramatyczne, Kanenobu wskazał (przy użyciu elementarnych metod!) w~pracy \cite{kanenobu86} z~1986 roku przeliczalnie wiele węzłów, które są jednocześnie hiperboliczne, włókniste\footnote{fibered}, taśmowe\footnote{ribbon}, 3-mostowe, o~genusie 2.
Żadnych dwóch spośród nich nie można rozróżnić wielomianem HOMFLY, ale posiadają nieizomorficzne moduły Alexandera, więc są (parami) różne.

% A POLYNOMIAL INVARIANT OF ORIENTED LINKS W. B. R. LICKORISH and KENNETH C. MILLET
% Example 16 - książka

Dla każdego splotu $L$ o~$k$ składowych, $P_L(x,y) - 1$ jest krotnością $x+y-1$, zaś $P_L(x,y) + (-1)^K$ jest krotnością $x +y + 1$, zatem wielomian HOMFLY nigdy nie jest zerem.
Uwaga: to jest inna parametryzacja!

Podstawiając $x = a/z$ i~$y = -1/az$ mamy $a = l$, $z = -m$
% Koniec sekcji Wielomian HOMFLY

\section{Wielomian BLM/Ho} % (fold)
\label{sec:blm_ho}
Na przełomie grudnia i~stycznia 1984 i~1985 roku, K. Nowiński sugerował uwzględnienie czwartego diagramu w~relacji kłębiastej, poza $L_+$, $L_-$, $L_0$.
Nie udało się uzyskać niezmiennika splotów.
Sukces odnieśli, dla niezorientowanych splotów, mniej więcej rok później Brandt, Lickorish, Millett oraz Ho (w pracy \cite{brandt86}).

\begin{definition}
    \label{def:blm_ho}
    \index{wielomian!BLM/Ho}
    Wielomian BLM/Ho to niezmiennik zdefiniowany relacją kłębiastą,
    \begin{equation}
        Q_{L_+}(x) + Q_{L_-}(x) = x (Q_{L_0}(x) + Q_{L_\infty}(x)),
    \end{equation}
    z warunkiem początkowym $Q_{\LittleUnknot} = 1$.
\end{definition}

Jest multiplikatywny.
Nie odróżnia luster ani mutantów (patrz definicja \ref{def:mutant}) i~potrafi liczyć składowe:
jeśli jest ich $c$, to najmniejszą potęgą $x$ występującą w~wielomianie $Q_L$ jest $x^{1-c}$.
Jego stopień nie przekracza indeksu skrzyżowaniowego.

W tej samej pracy (\cite{brandt86}) podano jawne wzory na wartości wielomianu BLM/Ho w~czterech punktach:

\begin{proposition}
    Niech $L$ będzie splotem.
    Wtedy $Q_L(1) = 1$.
\end{proposition}

\begin{proposition}
    Niech $L$ będzie splotem.
    Wtedy $Q_L(2) = (\det L)^2$.
\end{proposition}

\begin{proposition}
    Niech $L$ będzie splotem o $c$ ogniwach.
    Wtedy $Q_L(-2) = (-1)^{c-1}$.
\end{proposition}

\begin{proposition}
    Niech $L$ będzie splotem z $d$-wymiarową homologią modulo $3$ dla jego dwukrotnego nakrycia.
    Wtedy $Q_L(-1) = 3^d$.
\end{proposition}

Kanenobu w~pracy \cite{kanenobu89} podał prosty test potrafiący czasem wykrywać, które węzły nie są dwumostowe.
Jak pisze Stojmenow w~\cite{stoimenow00}: \emph{,,The converse of this criterion turns out not to be true; (...) these examples have been suggested by empirical calculations (...), which nevertheless reveal \ref{kanenobu_rationality_test} to be a surprisingly powerful test''}.

\begin{proposition}
    \label{prop:blmho_twobridge}
    Jeśli $L$ jest węzłem dwumostowym, to
    \begin{equation}
        \label{kanenobu_rationality_test}
        z Q_L(z) = 2 \jones_L(t) \jones_L (1-2z^{-1}+t^{-1}),
    \end{equation}
    gdzie $z = -t - t^{-1}$.
\end{proposition}

% Koniec sekcji Wielomian BLM/Ho

\section{Wielomian Kauffmana} % (fold)
\label{sec:kauffman_polynomial}
Mniej więcej w~tym samym czasie, gdy odkryto wielomian BLM/Ho, Kauffman zaproponował sposób, dzięki któremu ten wielomian można uogólnić do odróżniającego lustra.
Dopuszczał stosowanie tylko drugiego i~trzeciego ruchu Reidemeistera.

\index{wielomian!Kauffmana}
Wielomianu Kauffmana nie należy mylić z~nawiasem Kauffmana!
Jest to półzorientowany wielomian dwóch zmiennych dany wzorem
\begin{equation}
	F_L(a, z) = a^{-w(L)} \langle |L| \rangle,
\end{equation}
\todo[inline]{Co oznacza $z$?}
gdzie $w$ to writhe zorientowanego diagramu splotu, zaś $|L|$ to $L$ bez orientacji.
Stanowi uogólnienie wielomianu BLM/Ho: $F(1, x) = Q(x)$ oraz wielomianu Jonesa:
\begin{equation}
	V(t)=F(-t^{-3/4},t^{-1/4}+t^{1/4}).
\end{equation}

Jego związki z~wielomianem HOMFLY pozostają nieznane.
% Koniec sekcji Wielomian Kauffmana


\chapter{Topologia algebraiczna}
\section{Grupa splotu. Prezentacja Wirtingera} % (fold)
\label{sec:group_wirtinger}

Ponieważ dopełnienie dowolnego splotu, zarówno w przestrzeni $\R^3$ jak i $S^3$, jest łukowo spójne, jego grupa podstawowa nie zależy od wyboru punktu bazowego.
Dzięki temu poniższa definicja ma sens:

\begin{definition}
    \label{def:knot_group}
    \index{grupa!węzła}
    Niech $L$ będzie splotem.
    Grupę podstawową jego dopełnienia, $\pi_1(\R^3 \setminus L)$, nazywamy grupą splotu.
\end{definition}

Kiedy mówimy o~grupie węzła, zazwyczaj mamy na myśli obiekt opisany powyżej, a nie grupę kolorującą z~definicji \ref{colgrp_def}.
Nie należy ich mylić, grupa węzła ma bowiem dużo większe znaczenie.

Podamy teraz kilka przykładów węzłów oraz ich grup.

\begin{example}
    Niewęzeł: $\Z$.
\end{example}

\begin{example}
    Trójlistnik: grupa warkoczowa $B_3 \cong \langle x, y \mid x^2 = y^3\rangle$.
\end{example}

\begin{proof}
    Wynika to z równości
    % https://en.wikipedia.org/wiki/Tietze_transformations
    \begin{align}
        \pi_1(S^3 \setminus 3_1) & = \langle x, y, z \mid xz = yx, zy = xz, yx = zy \rangle \\
                                 & = \langle x, y \mid xyx = yxy \rangle \\
                                 & = \langle x, y, a, b \mid xyx = yxy, a = yx, b = xyx \rangle \\
                                 & = \langle x, a, b \mid xa = a^2x^{-1}, b = xa \rangle \\
                                 & = \langle a, b \mid b = a^2(ba^{-1})^{-1} \rangle \\
                                 & = \langle a, b \mid a^3 = b^2 \rangle,
    \end{align}
    prawdziwych na mocy transformacji Tietzego.
\end{proof}

\begin{example}
    Węzeł $(p,q)$-torusowy: $\langle x, y \mid x^p = y^q \rangle$.
\end{example}

\begin{example}
    Węzeł ósemkowy: $\langle x, y \mid yxy^{{-1}}xy=xyx^{{-1}}yx \rangle$.
\end{example}

\begin{proposition}
    \label{prop:knot_group_invariant}
    Grupa węzła jest niezmiennikiem węzłów.
\end{proposition}

\begin{proof}
    Gdy dwa węzły są równoważne, istnieje izotopijny z~identycznością homeomorfizm $\R^3 \to \R^3$, który posyła pierwszy węzeł na drugi.
    Obcięty do dopełnień węzłów indukuje izomorfizm grup podstawowych.
\end{proof}

\begin{proposition}
    Grupa węzła jest niezmiennikiem mocniejszym od genusu, a~w~przypadku węzłów złożonych, także od indeksu mostowego.
\end{proposition}

\begin{proof}[Niedowód]
    Jest to wniosek 3 z~pracy \cite{feustel78}.
\end{proof}

\begin{proposition}
    Niech $K_1, K_2$ będą węzłami pierwszymi.
    Jeżeli ich grupy są izomorficzne, to same węzły są równoważne.
\end{proposition}

\begin{proof}
    Jak piszą Gordon, Luecke w \cite{gordon89}, jest to bezpośredni wniosek z ich twierdzenia 2: nietrywialna chirurgia Dehna na nietrywialnym węźle nigdy nie daje $S^3$.
\end{proof}

Wcześniej Whitten wiedział tylko, że jeśli węzły pierwsze mają izomorficzne grupy, to dopełnienia tych węzłów są homeomorficzne.
Jak sam wspomina w \cite{whitten87}: ,,\emph{The group of a prime knot does not, however, necessarily determine the topological type of the exterior. Dehn hips on certain “essential” solid tori in the exteriors of torus knots and of cable knots produce Haken manifolds that are homotopically equivalent but not homeomorphic to the original exteriors and that, in fact, cannot be imbedded in $S^3$.}''.

Na przykładzie grupy $\langle x,y,z \mid xyx=yxy,xzx=zxz\rangle$, która odpowiada zarówno sumie prostej różno-, jak i~jednoskrętnych trójlistników, widać że założenia o pierwszości nie można pominąć.
Prawdziwe jest ogólniejsze stwierdzenie:

\begin{proposition}
    \label{prop:knot_group_sum}
    Niech $K_1, K_2$ będą zorientowanymi węzłami.
    Wtedy węzłom $K_1 \shrap K_2$, $K_1 \shrap mr K_2$ odpowiadają izomorficzne grupy.
\end{proposition}

\index{prezentacja Wirtingera}
Wiemy więc już trochę o~nowym niezmienniku, ale nie umiemy go jeszcze wyznaczać.
Jak zauważył Wilhelm Wirtinger około roku 1905, a więc jeszcze przed narodzinami teorii węzłów, grupa węzła zawsze posiada pewną specjalną prezentację, nazwaną na jego cześć prezentacją Wirtingera.
Jest to skończona prezentacja, w~której wszystkie relacje są postaci $w g_i w^{-1} = g_j$, gdzie $w$ to pewne słowo na generatorach, $g_1, \ldots, g_k$.
Przedstawimy ją zaraz ze względu na użyteczność w~rachunkach, dowodząc jednocześnie jej istnienia.

\begin{proposition}
    \label{prop:wirtinger}
    Grupa każdego węzła posiada prezentację Wirtingera.
\end{proposition}

\begin{proof}
    Oto zarys konstruktywnego dowodu.
    Przedstawiony algorytm jest bardzo wygodnym sposobem na wyznaczenie grupy węzła.
    Niech $K$ będzie węzłem z~diagramem o~$n$ łukach i~$m$ skrzyżowaniach.
    Wtedy
    \begin{equation}
        \pi_1(K) \cong \langle a_1, \ldots, a_n \mid r_1, \ldots, r_m\rangle,
    \end{equation}
    gdzie $a_i$ to włókna diagramu, zaś $r_x$ to relacje Wirtingera: $a_ia_ja_i^{-1}a_k^{-1}=1$,
\begin{comment}
    \[
    \begin{tikzpicture}[baseline=-0.65ex,scale=0.15]
    \begin{knot}[clip width=15]
        \strand[semithick,-Latex] (-5, -5) to (5, 5);
        \strand[semithick,-Latex] (-5, 5) to (5, -5);
        \node[darkblue] at (5, 5)[below right] {$a_i$};
        \node[darkblue] at (5, -5)[above right] {$a_k$};
        \node[darkblue] at (-5, 5)[below left] {$a_j$};
    \end{knot}
    \end{tikzpicture}
    \quad\quad
    \begin{tikzpicture}[baseline=-0.65ex,scale=0.15]
    \begin{knot}[clip width=15, flip crossing/.list={1}]
        \strand[semithick,-Latex] (-5, -5) to (5, 5);
        \strand[semithick,-Latex] (-5, 5) to (5, -5);
        \node[darkblue] at (5, 5)[below right] {$a_j$};
        \node[darkblue] at (-5, -5)[above left] {$a_k$};
        \node[darkblue] at (-5, 5)[below left] {$a_i$};
    \end{knot}
    \end{tikzpicture}
    \]
\end{comment}
    w~których łuk $a_i$ biegnie górą, zaś $a_j$ leży po jego lewej stronie.
\end{proof}

\begin{figure}[H]
    \begin{minipage}[b]{.48\linewidth}
        \[
            \begin{tikzpicture}[baseline=-0.65ex, scale=0.2]
                \useasboundingbox (-5, -5) rectangle (5,5);
                \begin{knot}[clip width=3.5, end tolerance=1pt, flip crossing/.list={1}]
                    \strand[thick, Latex-] (-5,5) to (5,-5);
                    \strand[thick, -Latex] (-5,-5) to (5,5);
                    % top left
                    \strand[thick, Latex-, darkblue] (-5, 1) to (-1, 5);
                    % bottom left
                    \strand[thick, Latex-, darkblue] (-5, -1) to (-1, -5);
                    % bottom right
                    \strand[thick, -Latex, darkblue] (5, -1) to (1, -5);
                    % top right
                    \strand[thick, -Latex, darkblue] (5, 1) to (1, 5);
                    \node[darkblue] at (-7, -2) {$x_k$};
                    \node[darkblue] at (-7, 2) {$x_{j+1}$};
                    \node[darkblue] at (7, -2) {$x_j$};
                    \node[darkblue] at (7, 2) {$x_k$};
                \end{knot}
            \end{tikzpicture}
        \]
        \subcaption{skrzyżowanie dodatnie: $x_j = x_k x_{j+1} x_k^{-1}$}
    \end{minipage}
    \begin{minipage}[b]{.48\linewidth}
        \[
            \begin{tikzpicture}[baseline=-0.65ex, scale=0.2]
                \useasboundingbox (-5, -5) rectangle (5,5);
                \begin{knot}[clip width=3.5, end tolerance=1pt, flip crossing/.list={1}]
                    \strand[thick, Latex-] (-5,5) to (5,-5);
                    \strand[thick, Latex-] (-5,-5) to (5,5);
                    % top left
                    \strand[thick, Latex-, darkblue] (-5, 1) to (-1, 5);
                    % bottom left
                    \strand[thick, -Latex, darkblue] (-5, -1) to (-1, -5);
                    % bottom right
                    \strand[thick, -Latex, darkblue] (5, -1) to (1, -5);
                    % top right
                    \strand[thick, Latex-, darkblue] (5, 1) to (1, 5);
                    \node[darkblue] at (-7, -2) {$x_k$};
                    \node[darkblue] at (-7, 2) {$x_{j+1}$};
                    \node[darkblue] at (7, -2) {$x_j$};
                    \node[darkblue] at (7, 2) {$x_k$};
                \end{knot}
            \end{tikzpicture}
        \]
        \subcaption{skrzyżowanie ujemne: $x_j = x_k^{-1} x_{j+1} x_k$}
    \end{minipage}
\end{figure}

\begin{corollary}
    \label{prop:knot_group_abelianisation}
    Niech $G$ będzie grupą węzła.
    Wtedy jej abelianizacją jest $G^{\operatorname{ab}} = \Z$.
\end{corollary}

\begin{proof}
    Relacja $a_ia_ja_i^{-1}a_k^{-1}=1$ po przejściu do abelianizacji przyjmuje postać $a_j = a_k$.
    Oznacza to, że etykieta łuku nie zmienia się podczas przejścia pod każdym skrzyżowaniem, zatem wszystkie etykiety są takie same.

    Można też zauważyć, że abelianizacją grupy podstawowej węzła jest pierwsza grupa homologii okręgu, czyli $\Z$.
\end{proof}

Istnieje alternatywna prezentacja grupy węzła, która pochodzi od Dehna, gdzie zamiast etykietować łuki, przypisuje się różne litery czterem częściom płaszczyzny, które są rozcinane przez skrzyżowanie.
Pomijamy tę prezentację dla oszczędności miejsca.
Klasycznie, jak na przykład w~\cite{crowell63}, macierz, a~co za tym idzie, także wielomian Alexandera wprowadza się przy użyciu prezentacji Wirtingera i~pochodnej Foxa.
Oryginalna praca Alexandera była jednak bliższa duchem pomysłom Dehna.

\begin{definition}[pochodna Foxa]
    \index{pochodna Foxa}
    Niech $G$ będzie wolną grupą generowaną przez (niekoniecznie skończony) podzbiór $\{g_i\}_{i \in I}$.
    Odwzorowanie $\partial/\partial g_i \colon G \to \Z G$ spełniające trzy aksjomaty:
    \begin{align}
        \frac{\partial}{\partial g_i} (e) & = 0 \\
        \frac{\partial}{\partial g_i} (g_j) & = \delta_{ij} \\
        \forall u, v \in G : \frac{\partial}{\partial g_i} (uv) & = \frac{\partial}{\partial g_i}(u) + u \frac{\partial}{\partial g_i} (w),
    \end{align}
    gdzie $\delta_{ij}$ oznacza deltę Kroneckera, nazywamy pochodną cząstkową Foxa.
\end{definition}

Ustalmy prezentację grupy węzła z $n$ relacjami (słowami) $w_1, \ldots, w_n$ nad $n$-literowym alfabetem $x_1, \ldots, x_n$.
Zdefiniujmy następnie macierz Jacobiego wymiaru $n \times n$, elementami której są pochodne Foxa słów $w_i$ względem zmiennych $x_j$:
\begin{equation}
    J = \left(\frac{\partial w_i}{\partial x_j}\right).
\end{equation}

Wykreślmy z macierzy $J$ najpiew jedną kolumnę oraz jeden wiersz z tej macierzy, po czym podstawmy za wszystkie litery zmienną $t$ i policzmy wyznacznik.
Otrzymaliśmy znowu wielomian Alexandera.
Fox napisał cykl pięciu artykułów \cite{fox53}, \cite{fox54}, \cite{fox56}, \cite{fox58}, \cite{fox60} poświęcony wolnemu rachunkowi różniczkowemu, powyższa definicja jest tylko małym wycinkiem tego cyklu opublikowanego w Annals of Mathematics.

Dwa następne stwierdzenia są już trudniejsze w~dowodzie,
na przykład uzasadnienie pierwszego może wymagać:
twierdzenia o~sferze, o~pętli oraz hipotezy Knesera.

\begin{proposition}
    \label{prop:knot_group_split}
    Niech $L \subseteq S^3$ będzie splotem.
    Następujące warunki są równoważne:
    \begin{enumerate}
        \item grupa podstawowa splotu $L$ nie jest produktem wolnym,
        \item splot $L$ nie jest rozszczepialny,
        \item splot $L$ jest rozmaitością Hakena o~nieściśliwym brzegu.
    \end{enumerate}
\end{proposition}

\begin{proof}[Niedowód]
    Kawauchi w \cite{kawauchi96}, patrz twierdzenie 6.1.4.
\end{proof}

\begin{proposition}
    \label{prop:knot_group_free}
    Niech $L \subseteq S^3$ będzie splotem.
    Następujące warunki są równoważne:
    \begin{enumerate}
        \item grupa podstawowa splotu $L$ jest wolna, rangi $n$,
        \item splot $L$ jest trywialny, złożony z $n$ ogniw.
    \end{enumerate}
\end{proposition}

\begin{proof}[Niedowód]
    Kawauchi w \cite{kawauchi96}, patrz wniosek 6.1.5.
\end{proof}

Twierdzenie Dehna z~1915 mówi, że jedynym węzłem, którego grupą są liczby całkowite $\mathbb Z$, jest niewęzeł.
Wynik ten został później istotnie uogólniony.
Michael Kervaire pokazał w~1966 roku (w \cite{kervaire65}) jakie warunki musi spełniać grupa $G$, by istniał pewien węzeł, którego grupą jest właśnie $G$.
Patrz też twierdzenie 14.1.1 w \cite{kawauchi96}.

\begin{proposition}
    Niech $G$ będzie grupą węzła $S^n \subseteq S^{n+2}$.
    Wtedy:
    \begin{enumerate}[leftmargin=*]
        \itemsep0em
        \item grupa $G$ jest skończenie prezentowana,
        \item abelianizacja $G/G'$ jest nieskończoną grupą cykliczną,
        \item druga grupa homologii $H_2(G) = 0$ jest trywialna,
        \item istnieje element $x \in G$ zwany południkiem taki, że $G$ jest najmniejszą podgrupą normalną $G$, która zawiera $x$.
    \end{enumerate}
\end{proposition}

Wyżej wymienione warunki konieczne są także wystarczające, jeżeli $n \ge 3$, jednakże problem pełnej charakteryzacji w~czwartym wymiarze jest otwarty.
Warunki 2. i 3. wynikają z~dualności Alexandera, zaś 1. i 4. stanowią przeformułowanie prezentacji Wirtingera.

% Koniec sekcji Grupa węzła. Prezentacja Wirtingera

\section{Genus} % (fold)
\label{sec:genus}
W tej sekcji zbadamy bardzo geometryczny niezmiennik węzłów, genus.
Wyznaczenie genusu konkretnego węzła przysparza wiele trudności, jednak jest on potężnym narzędziem podczas dowodzenia twierdzeń.
Zaczniemy jednak od przyjrzenia się powierzchniom.

\begin{definition}
\index{powierzchnia}
    Powierzchnia to dwuwymiarowa podrozmaitość topologiczna $M \subseteq \R^3$, czyli obiekt wyglądający lokalnie jak płaszczyzna: każdy punkt $x \in M$ posiada otwarte otoczenie homeomorficzne z otwartym dyskiem $\{(x,y) : x^2 + y^2 < 1\}$.
\end{definition}

Przykładami powierzchni są sfera oraz brzeg torusa, ale też koło otwarte albo pierścień $\{(x, y) \in \R^2 : 0 < a < x^2 + y^2 < b\}$.
Dwa ostatnie przykłady mają brzeg, dwa pierwsze są brzegu pozbawione (nazywamy je powierzchniami zamkniętymi).

Rozważane przez nas powierzchnie będą posiadać dodatkowo dwie cechy: ograniczoność oraz orientowalność.
Ta druga własność sprowadza się na mocy klasyfikacji powierzchni (spójna powierzchnia zamknięta jest sferą, sumą spójną torusów lub sumą spójną rzutowych (rzeczywistych) płaszczyzn) do niezawierania homeomorficznej kopii wstęgi Möbiusa.
Podane wcześniej powierzchnie są orientowalne.

\begin{definition}
\index{charakterystyka Eulera}
    Charakterystyka Eulera $\chi(M) \in \Z$ to niezmiennik
    topologiczny powierzchni $M$, którego definicji nie podamy.
    Zamiast tego wymienimy dość reguł,
    by wyznaczyć go w interesujących nas przypadkach:
    \begin{itemize}
        \item Charakterystyka dysku to $1$.
        \item $\chi(M_1 \sqcup M_2)=\chi(M_1) + \chi(M_2)$ dla każdych powierzchni $M_1, M_2$.
        \item Dołączenie paska zmniejsza charakterystykę powierzchni o $1$.
        \item Dołączenie dysku do całej składowej spójności brzegu powierzchni zwiększa charakterystykę o $1$.
    \end{itemize}
\end{definition}

\begin{definition}
\index{genus}
    Genus powierzchni $M$ posiadającej $c$ składowych spójności brzegu to
    \[
        g(M) = 1 - \frac{\chi(M) + c}{2}.
    \]
\end{definition}

\begin{proposition}
    Dwie spójne, zorientowane powierzchnie są homeomorficzne wtedy i tylko
    wtedy, gdy mają ten sam genus i tyle samo składowych spójności brzegu.
\end{proposition}

Najważniejsze dla nas są powierzchnie Seiferta:

\begin{definition}
\index{powierzchnia!Seiferta}
    Powierzchnia Seiferta związana z węzłem $K$ to spójna,
    orientowanlna powierzchnia zanurzona w $\R^3$, której brzegiem jest $K$.
\end{definition}

% \begin{example}
% Powierzchnia Seiferta dla trójlistnika:
% \begin{center}
% \begin{tikzpicture}
% [scale=0.1]
%   \clip (-17,-15) rectangle (17,15);
%   \foreach \d in {0,180} {
%       \path[OBSZAR    ,rotate=\d] (-1.25,11.5)
%       .. controls (2,14) and (6,13.5) ..  (10,12)
%       .. controls (23,7) and (15,-20)  .. (3,-13)
%       -- (1.25, -11.5)
%       .. controls (4.5,-8) and (4.5,-4) .. (0,0)
%       .. controls (4,4) and (4.5,5.5) .. (-1.25,11.5);}
%   \path[TIKZ_ARCH] (0,10) .. controls (10,0) and (-10,0) .. (0,-10);
%   \foreach \d in {0,180} {
%   \path[TIKZ_ARCH, rotate=\d] (-1.5,1.5) .. controls (-6,6) and (-3,17) .. (10,12)
%   .. controls (23,7) and (15,-20)  .. (3,-13);}
% \end{tikzpicture}
% \end{center}
% \end{example}

Nie każde uszachowienie diagramu węzła prowadzi do powierzchni Seiferta:
widać to po standardowym diagramie trójlistnika.
Pomimo to prawdziwe jest następujące stwierdzenie.

\begin{proposition}[Pontriagin, Frankl 1930]
    Każdy węzeł posiada powierzchnię Seiferta.
\end{proposition}

Dowód tego faktu opiera się na bezpośredniej konstrukcji i pochodzi od Seiferta.

\begin{proof}
    Wybierzmy diagram $D$ dla węzła oraz orientację,
    a następnie wyprostujmy wszystkie skrzyżowania zgodnie z ich orientacją:

    \[
    \begin{tikzpicture}[scale=0.12, baseline=-3]
        \begin{knot}[clip width=15, end tolerance=1pt,flip crossing/.list={1}]
            \strand[semithick,Latex-] (-5,5) to (5,-5);
            \strand[semithick,-Latex] (-5,-5) to (5,5);
        \end{knot}
    \end{tikzpicture}
    \quad\longrightarrow\quad
        \begin{tikzpicture}[baseline=-0.65ex, scale=0.12]
        \useasboundingbox (-5, -6) rectangle (5, 6);
        \draw[semithick,-Latex] (-4, -5) to [out=45, in=-45] (-4, 5);
        \draw[semithick,-Latex] (4, -5) to [out=135, in=-135] (4, 5);
        \end{tikzpicture}
        \quad\longleftarrow\quad
    \begin{tikzpicture}[scale=0.12, baseline=-3]
        \begin{knot}[clip width=15, end tolerance=1pt]
            \strand[semithick,Latex-] (-5,5) to (5,-5);
            \strand[semithick,-Latex] (-5,-5) to (5,5);
        \end{knot}
    \end{tikzpicture}
    \]

    Otrzymany diagram składa się teraz z pewnej liczby zamkniętych krzywych,
    zwanych okręgami Seiferta, które wypełniamy do dysków.
    Tam, gdzie jeden okrąg leżał wewnątrz drugiego, podnosimy wewnętrzny nad zewnętrzny.
    Przy każdym skrzyżowaniu pierwotnego diagramu doklejamy skręcony pasek do obydwu dysków.

    Dyski są dwustronne, więc ich górnej stronie przypisujemy znak $+$,
    jeśli tylko brzeg jest zorientowany dodatnio i $-$ w przeciwnym razie.

    \begin{center}
        %\includegraphics[width=0.75\textwidth]{seifert.png}
        (brakująca grafika)
    \end{center}

\end{proof}

Otrzymana powierzchnia zorientowana, $M_D$,
nie daje się łatwo wyobrazić, ale można policzyć jej genus.
Posiada bowiem dokładnie jedną składową spójności brzegu.

%\todo[inline]{Fałsz, jest to możliwe, ale muszę nauczyć się rysować kręcone paski.}

\begin{proposition}
    Niech $K$ będzie węzłem z diagramem $D$. Wtedy $\chi(M_D) = s - n$, gdzie $n$ jest liczbą skrzyżowań $D$, zaś $s$ jest liczbą okręgów Seiferta.
\end{proposition}

\begin{proof}
Powierzchnia $M_d$ powstaje z sumy rozłącznej dysków (po jednym dla każdego okręgu Seiferta) przez dołączanie pasków do ich brzegów (po jednym dla każdego skrzyżowania).
\end{proof}

Skupimy się wreszcie na genusie węzła.
Do matematyki wprowadził go Seifert w 1934 roku.

\begin{definition}
    Genusem węzła $K$ nazywamy minimalny genus jego powierzchni Seiferta.
\end{definition}

To daje namiastkę tego, dlaczego wyznaczenie genusu jest trudne.
W rzeczywistości jest jeszcze gorzej: w 1986 Morton pokazał, że genus pewnych węzłów nie jest realizowany przez żaden diagram (do którego stosuje się algorytm Seiferta), choćby $10_{165}$.
Następujący przypadek stanowi wyjątek: algorytm Seiferta zastosowany do alternującego diagramu zawsze daje powierzchnię o minimalnej powierzchni.
Najprostszy dowód pochodzi od Davida Gabaia.

\begin{example}
    Dubel Whiteheada dowolnego nietrywialnego węzła posiada genus $1$.
\end{example}

Z pracy Mortona wynika, że algorytm Seiferta zastosowany do dubla trójlistnika produkuje powierzchnie o genusie co najmniej $3$ (to dolne ograniczenie oparte jest o wielomian HOMFLY).
Patrz \cite{morton86}.

\begin{proposition}
\label{genus_one}
    Dokładnie jeden węzeł posiada zerowy genus: niewęzeł.
\end{proposition}

\begin{proof}
    Oto szkic dowodu.
    Jeżeli genus wynosi zero, to $K$ ma powierzchnię Seiferta o jednej składowej brzegowej i genusie zero.
Dysk też ma te własności, więc korzystamy z klasyfikacji powierzchni (dysk jest powierzchnią).
\end{proof}

\begin{proposition}
\label{genus_sum}
    Jeśli $J, K$ są węzłami, to $g (J \shrap K) = g(J) + g(K)$.
\end{proposition}

\begin{proof}
    Pokażemy najpierw, że $g(J \# K) \le g(J) + g(K)$.
    Wybierzmy powierzchnie Seiferta $M_J$ oraz $M_K$ dla $J$ oraz $K$ o minimalnym genusie.
    Suma $J \shrap K$ powstaje z $J$ oraz $K$, podobnie jest z powierzchniami Seiferta:
    \[
        \begin{tikzpicture}[baseline=-0.65ex,scale=0.12]
        \draw[semithick,-Latex] (-7, -5) to (-5, -5) [in=right, out=right] to (-5, 5) to (-7, 5);
        \draw[semithick,Latex-] ( 7, -5) to ( 5, -5) [in=left, out=left] to ( 5, 5) to ( 7, 5);
        \node at (-5, 0) {$J$};
        \node at (5, 0) {$K$};
        \end{tikzpicture}
        \longrightarrow
        \begin{tikzpicture}[baseline=-0.65ex,scale=0.12]
        \draw[semithick,-Latex] (-7, -5) to (-5, -5) to [out=right, in=left] (-2, -2) -- (2, -2) to [out=right, in=left] (5, -5) to (7, -5);
        \draw[semithick,Latex-] (-7, 5) to (-5,  5) to [out=right, in=left] (-2,  2) -- (2,  2) to [out=right, in=left] (5,  5) to (7, 5);
        \node at (0, -5) {$J \# K$};
        \end{tikzpicture}
        \quad\quad
        \begin{tikzpicture}[baseline=-0.65ex,scale=0.12]
        \draw[semithick,fill=blue!10!white] (-10, -5) to (-5, -5) [in=right, out=right] to (-5, 5) to (-10, 5);
        \draw[semithick,fill=blue!10!white] ( 10, -5) to ( 5, -5) [in=left, out=left] to ( 5, 5) to (10, 5);
        \node at (-6.5, 0) {$M_J$};
        \node at (6.5, 0) {$M_K$};
        \end{tikzpicture}
        \longrightarrow
        \begin{tikzpicture}[baseline=-0.65ex,scale=0.12]
        \fill[blue!10!white] (-7, -5) rectangle (7, 5);
        \draw[semithick,fill=white] (-7, -5) to (-5, -5) to [out=right, in=left] (-2, -2) -- (2, -2) to [out=right, in=left] (5, -5) to (7, -5);
        \draw[semithick,fill=white] (-7, 5) to (-5,  5) to [out=right, in=left] (-2,  2) -- (2,  2) to [out=right, in=left] (5,  5) to (7, 5);
        \node at (0, 0) {$M_{J \# K}$};
        \end{tikzpicture}
    \]

    Skoro $M_{J\#K}$ powstaje z $M_J \sqcup M_K$ przez dołączenie paska do brzegu, mamy
    \[
        \chi(M_{J\#K}) = \chi(M_J \sqcup M_K) - 1 = \chi(M_J) + \chi(M_K)-1,
    \]
    a przez to
    \[
        g(M_{J\#K}) = \frac{1-\chi(M_{J\#K})}{2} =
        \frac{1-\chi(M_{J})}{2} + \frac{1-\chi(M_{K})}{2}
        % = %g(M_J)+g(M_K)
        = g(J) + g(K).
    \]
    To kończy dowód pierwszej nierówności.
    Pokażemy jeszcze, że $g(J \# K) \ge g(J)+g(K)$.
    Zaczynamy od powierzchni Seiferta $M_{J\#K}$ dla $J\#K$ o minimalnym genusie $g(M_{J\#K})$ równym $g(J\#K)$.
    Poprzez wykonanie chirurgii na powierzchni, możemy przyjąć specjalną postać jak w poprzednim dowodzie:
    \[
        \begin{tikzpicture}[baseline=-0.65ex,scale=0.16]
            \fill[blue!10!white] (-5, -5) rectangle(5, 5);
        \draw[semithick,fill=white] (-5, -5) to [out=right, in=left] (-2, -2) -- (2, -2) to [out=right, in=left] (5, -5);
        \draw[semithick,fill=white] (-5,  5) to [out=right, in=left] (-2,  2) -- (2,  2) to [out=right, in=left] (5,  5);
            \node at (0, 0) {$M_{J \# K}$};
        \end{tikzpicture}
    \]

    Usunięcie paska daje powierzchnie Seiferta dla $M_J$ oraz $M_K$ takie, że
    \[
        g(M_J)+g(M_K)=g(M_{J\#K})=g(J\#K).
    \]
    Oznacza to, że $g(J)+g(K)\leqslant g(M_J)+g(M_K)=g(J\#K)$ i tak naprawdę mamy równość.
\end{proof}

\begin{corollary}
\label{no_inverses}
    Jeśli suma spójna dwóch węzłów jest niewęzłem, to oba składniki także nim są.
\end{corollary}

Powrócimy teraz do węzłów pierwszych (definicja \ref{primeknot}).

\begin{proposition}
    Jeśli genus węzła wynosi $g(K) = 1$, to sam węzeł jest pierwszy.
\end{proposition}

\begin{proof}
    Załóżmy, że $K=K_1\#K_2$.
    Wtedy $g(K_1)+g(K_2)=g(K)=1$ na mocy faktu  \ref{genus_sum} o genusie sumy.
    Jedyna możliwość: $K_1$ lub $K_1$ ma genus zero, drugi z nich ma genus $1$. To oznacza, że jeden z nich jest trywialny, więc $K$ jest pierwszy.
\end{proof}

\begin{proposition}
    Każdy węzeł można zapisać jako suma spójna pewnej liczby węzłów pierwszych (niewęzeł jest sumą pustej rodziny węzłów).
\end{proposition}

\begin{proof}
    Dowodzimy przez indukcję względem genusu $g(K)$.
    Przypadek bazowy $g(K) = 0$ jest oczywisty, gdyż wtedy $K$ to niewęzeł.
    Załóżmy więc, że fakt zachodzi dla węzłów $J$ genusu co najwyżej $n$.
    Niech $K$ będzie genusu $n + 1$.

    Jeśli $K$ jest pierwszy, nie ma czego dowodzić.
    W przeciwnym razie jest równoważny z $J_1 \shrap J_2$, gdzie $J_1$ i $J_2$ są nietrywialne.
    Mamy $g(J_1)+g(J_2)=g(K)$ oraz $g(J_1),g(J_2)\geqslant 1$.
    Zatem $g(J_1),g(J_2)\leqslant n$.
    Na mocy hipotezy indukcyjnej, $J_1$ oraz $J_2$ są równoważne sumom
    \[
        J_1 \cong K_1\#\cdots\# K_s,\qquad
        J_2 \cong K_{s+1}\#\cdots\# K_r,
    \]
    gdzie $K_i$ są pierwsze.
    Zatem $K$ jest równoważny z $K_1\#\cdots\# K_r$, co kończy dowód.
\end{proof}

Rozkład jest jednoznaczny, ale nie pokażemy tego.

\begin{proposition}
\label{infty_primes}
    Istnieje nieskończenie wiele węzłów pierwszych.
\end{proposition}

\begin{proof}
    Pokażemy, że wszystkie węzły $(2n+2)_1$ są pierwsze, gdzie $n \ge 1$.
    Istotnie, algorytm Seiferta zastosowany do diagramu tego węzła wyprodukuje $2n+1$ okręgów.
    \[
        \begin{tikzpicture}[baseline=-0.65ex,scale=0.055]
        \begin{knot}[clip width=10, flip crossing/.list={1,4,5},end tolerance=1pt]
            \node at (0,10) {$\cdots$};
            \strand[semithick] (-30, -5) -- (-5, -5);
            \strand[semithick,-Latex]  (5, -5) -- (30, -5);
            \strand[semithick,Latex-]  (-30,-15) -- (-5,-15);
            \strand[semithick,Latex-]  (5,-15) -- (30,-15);

            \strand[semithick,domain=-90:90] plot ({7.5*cos(\x)-5}, {5*sin(\x)-10});
            \strand[semithick,domain=90:270] plot ({7.5*cos(\x)+5}, {5*sin(\x)-10});

            % zewnętrzne obręcze -- lewa strona
            \strand[semithick] (-30, 15) to [out=left, in=up]   (-45, 0);
            \strand[semithick] (-30,-15) to [out=left, in=down] (-45, 0);
            \strand[semithick] (-30,  5) to [out=left, in=up]   (-35, 0);
            \strand[semithick] (-30, -5) to [out=left, in=down] (-35, 0);

            % zewnętrzne obręcze -- prawastrona
            \strand[semithick] (30, 15) to [out=right, in=up]   (45,0);
            \strand[semithick] (30,-15) to [out=right, in=down] (45,0);
            \strand[semithick] (30,  5) to [out=right, in=up]   (35,0);
            \strand[semithick] (30, -5) to [out=right, in=down] (35,0);

            % jak w drugim ruchu Reidemeistera - lewe
            \strand[semithick] (-30, 15) .. controls (-24, 15) and (-24,  5) .. (-20,  5);
            \strand[semithick] (-30,  5) .. controls (-24,  5) and (-24, 15) .. (-20, 15);
            \strand[semithick] (-10, 15) .. controls (-16, 15) and (-16,  5) .. (-20,  5);
            \strand[semithick] (-10,  5) .. controls (-16,  5) and (-16, 15) .. (-20, 15);

            % jak w drugim ruchu Reidemeistera - prawe
            \strand[semithick] (30, 15) .. controls (24, 15) and (24,  5) .. (20,  5);
            \strand[semithick] (10, 15) .. controls (16, 15) and (16,  5) .. (20,  5);
            \strand[semithick] (30,  5) .. controls (24,  5) and (24, 15) .. (20, 15);
            \strand[semithick] (10,  5) .. controls (16,  5) and (16, 15) .. (20, 15);
        \end{knot}
        \end{tikzpicture}
        \longrightarrow
        \begin{tikzpicture}[baseline=-0.65ex,scale=0.055]
            \node at (0,10) {$\cdots$};
            \draw[semithick] (-30,  -5) -- (30, -5);
            \draw[semithick] (-30, -15) -- (30,-15);

            \draw[semithick] (0,-10) circle (3);

                % zewnętrzne obręcze -- lewa strona
            \draw[semithick] (-30, 15) to [out=left, in=up]   (-45, 0);
            \draw[semithick] (-30,-15) to [out=left, in=down] (-45, 0);
            \draw[semithick] (-30,  5) to [out=left, in=up]   (-35, 0);
            \draw[semithick] (-30, -5) to [out=left, in=down] (-35, 0);

                % zewnętrzne obręcze -- prawastrona
            \draw[semithick] (30, 15) to [out=right, in=up]   (45,0);
            \draw[semithick] (30,-15) to [out=right, in=down] (45,0);
            \draw[semithick] (30,  5) to [out=right, in=up]   (35,0);
            \draw[semithick] (30, -5) to [out=right, in=down] (35,0);

            \draw[semithick] (-30, 15) to [out=right, in=up] (-20,10);
            \draw[semithick] (-30,  5) to [out=right, in=down] (-20,10);

            \draw[semithick] (30, 15) to [out=left, in=up] (20,10);
            \draw[semithick] (30,  5) to [out=left, in=down] (20,10);

            \draw[semithick] (-10, 10) circle (5);
            \draw[semithick] (10,  10) circle (5);
        \end{tikzpicture}
    \]
    Wynika stąd, że genus wynosi $\frac 12 (1 - (1+2n) + (2+2n)) = 1$, ponieważ wyznacznik ma wartość $4n+1$,
    węzły $2n+2)_1$ nie są trywialne i są parami różne.
\end{proof}

\begin{proposition}
    Genus węzła $K$ jest związany z wielomianem Alexandera oraz liczbą skrzyżowań:
    \[
        c(K) \ge 2 g(K) \ge \operatorname{Span}(\Delta_K(t)),
    \]
    przy czym (między innymi) dla węzłów o co najwyżej 10 skrzyżowaniach mamy nawet równość po prawej stronie.
\end{proposition}

\index{homologia!Floera}
Wielomian Alexandera uogólnia się do (skomplikowanej) homologii Floera, pozwala ona dokładniej szacować genus węzła.

\index{węzeł!rozwłókniony}
Wspomnijmy jeszcze krótko o specjalnym rodzaju węzłów.
Mówimy, że węzeł $K \subseteq S^3$ jest rozwłókniony\footnote{fibered}, jeśli istnieje rodzina $F_t$ powierzchni Seiferta dla $K$ sparametryzowana przez $t \in S^1$ taka, że $F_t \cap F_s = K$ dla $t \neq s$.
Dawniej nazywano je węzłami Neuwirtha.

Niewęzeł, trójlistnik i ósemka są rozwłóknione.
Pierwszy i ostatni współczynnik wielomianu Alexandera węzła rozwłóknionego to $\pm 1$.
Wielomianem węzła skręconego z $q$ półskrętami jest $\Delta_q = qt - (2q+1)+q/t$, więc nie jest on rozwłókniony (dla $q \neq 1$).

% Węzeł jest rozwłókniony dokładnie wtedy, gdy stanowi grzbiet pewnego 'open book decomposition' $S^3$.

% Koniec sekcji Genus

\section{Kwandle i wraki} % (fold)
% TODO: Problems  on invariants of knots and 3-manifolds, rozdział 3
\label{sec:quandle}
Sekcja ta powstała częściowo w~oparciu o~notatki autorstwa Andrew Bergera, Chrisa Geriga\footnote{dostępne pod adresem \url{https://math.berkeley.edu/~cgerig/notes}} oraz Andrew Bergera, Brandona Flannery'ego i~Chrisa Sumnichta\footnote{dostępne pod adresem \url{https://github.com/thyrgle/191_Final_Project/blob/master/paper.pdf}}.
Kwandle, z~angielskiego \emph{quandle}, są strukturami algebraicznymi przypominającymi grupy.
Aksjomaty grupy stanowią uogólnienie symetrii -- symetrie są odwracalne, można je składać, identyczność jest symetrią.
Aksjomaty kwandli będą odzwierciedlać ruchy Reidemeistera.

David Joyce zapytany o znaczenie słowa \emph{quandle} odpowiedział: ,,I needed a usable word. “Distributive algebra” had too many syllables. Piffle was already taken. I tried trindle and quagle, but they didn’t seem right, so I went with quandle.''.

% DICTIONARY;quandle;kwandel
\begin{definition}[kwandl]
    \index{quandle}
    Zbiór $X$ wyposażony w dwuargumentowe działanie $\triangleright$ taki, że dla wszystkich elementów $x, y, z \in X$ zachodzi:
    \begin{enumerate}
        \item $x \triangleright x = x$,
        \item odwzorowanie $\beta_y \colon X \to X$ dane wzorem $\beta_y(x) = x \triangleright y$ jest odwracalne,
        \item $(x \triangleright y) \triangleright z = (x \triangleright z) \triangleright (y \triangleright z)$
    \end{enumerate}
    nazywamy kwandlem.
\end{definition}

Kwandle można rozpatrywać jako samodzielne konstrukcje algebraiczne.
My pokażemy, że są naturalnym niezmiennikiem węzłów.

Niech $X$ będzie skończonym kwandlem, zaś $K$ węzłem.
Elementy $x \in X$ będą dla nas kolorami, którymi oznaczymy długie łuki na diagramie węzła $K$.
Gdy trzy kolory spotykają się przy jednym skrzyżowaniu, definiujemy funkcję $\triangleright \colon X \times X \to X$, jak na rysunku.
To znaczy: kiedy łuk o kolorze $x$ przechodzi pod łukiem koloru $y$, staje się łukiem w kolorze $x \triangleright y$.

\begin{comment}
\[
    \begin{tikzpicture}[scale=0.18, baseline=0]
        \path[TIKZ_ARCH,Latex-] (-4,0) -- (4,8);
        \path[TIKZ_ARCH] (4,0) -- (1,3);
        \path[TIKZ_ARCH] (-1,5) -- (-4,8);
        \node[darkblue] at (-4, 0)[above left] {$y$};
        \node[darkblue] at (4, 0)[above right] {$x \triangleright y$};
        \node[darkblue] at (-4, 8)[below left] {$x$};
    \end{tikzpicture}
\]
\end{comment}

Ta definicja pochodzi z~nieopublikowanej korespondencji między Johnem Conwayem i~Gavenem Wraithem, którzy w 1959 byli studentami I stopnia na uniwersytecie w Cambridge.
Ponownie odkryto ją w latach 80. XX wieku: Joyce w 1982 po raz pierwszy nazwał te obiekty kwandlami, Matwiejow w tym samym roku jako grupoidy rozdzielne, Brieskorn w 1986 jako zbiory automorficzne.

Drugi aksjomat nazywa się czasem odwracalnością z prawej strony: znając $x \triangleright y$ oraz $y$ możemy odtworzyć element $x$, jednak znając $x$ być może nie jesteśmy w stanie odtworzyć elementu $y$.
Jedyny element $x$ taki, że $x \triangleright y = z$ nazwijmy $y \triangleleft z$.
To pozwala podać trochę inną definicję kwandli, my nie będziemy jej używać.

\begin{definition}
    Zbiór $X$ z dwuargumentowymi działaniami $\triangleright, \triangleleft$ taki, że dla wszystkich $x, y, z \in X$ zachodzi:
    \begin{align*}
    x \triangleright x = x \triangleleft x & = x \\
    (x \triangleleft y) \triangleright x & = y \\
    x \triangleleft (y \triangleright x) & = y \\
     (x \triangleright z) \triangleright (y \triangleright z) & = (x \triangleright y) \triangleright z \\
    (x \triangleleft y) \triangleleft (x \triangleleft z) & = x \triangleleft (y \triangleleft z)
    \end{align*}
    nazywamy kwandlem.
\end{definition}

Teraz możemy przetłumaczyć ruchy Reidemeistera w aksjomaty kwandli.

\begin{proposition}
    Niech $X$ będzie skończonym kwandlem.
    Liczba etykietowań diagramu elementami kwandla $X$ jest niezmiennikiem węzłów, zwanym niezmiennikiem zliczającym.
\end{proposition}

\begin{proof}
    Musimy pokazać, że etykiety na diagramiem przed każdym ruchem Reidemeistera wyznaczają jednoznacznie układ etykiet po tym ruchu.
    Pierwszy ruch:
\begin{comment}
    \[
        \begin{tikzpicture}[baseline=-0.65ex,scale=0.07]
        \begin{knot}[clip width=5,flip crossing/.list={1}]
            \strand[semithick,Latex-] (15, 0) [in=up,out=left] to (-5, -7);
            \strand[semithick] (-5, -7) [in=down,out=down] to (5, -7);
            \strand[semithick] (5, -7) [in=right,out=up] to (-15, 0);
            \node[darkblue] at (-15, 0)[left] {$x$};
            \node[darkblue] at (15, 0)[right] {$x \triangleright x$};
        \end{knot}
        \end{tikzpicture}
        \stackrel{R_1}{\cong} \,\,
        \begin{tikzpicture}[baseline=-0.65ex,scale=0.07]
        \begin{knot}[clip width=5]
            \strand[semithick,-Latex] (-15, 0) to (15, 0);
            \node[darkblue] at (-15, 0)[left] {$x$};
        \end{knot}
        \end{tikzpicture}
    \]
\end{comment}
    Drugi ruch:
\begin{comment}
    \[
        \begin{tikzpicture}[baseline=-0.65ex,scale=0.07]
        \begin{knot}[clip width=4, flip crossing/.list={1,2}]
            \strand[semithick,-Latex] (-10, -4) to (-7, -4) [in=left, out=right] to (0, 4) [in=left, out=right] to (7, -4) to (10, -4) to (15, -4);
            \strand[semithick] (-10, 4) to (-7, 4) [in=left, out=right] to (0, -4) [in=left, out=right] to (7, 4) to (10, 4) to (15, 4);
            \node[darkblue] at (-10, -4)[left] {$y$};
            \node[darkblue] at (-10, 4)[left] {$x$};
            \node[darkblue] at (0, 4) [above] {$y \triangleright x$};
            \node[darkblue] at (15, -4) [right] {$x \triangleleft (y \triangleright x)$};
        \end{knot}
        \end{tikzpicture}
        \stackrel{R_2}{\cong} \,\,
        \begin{tikzpicture}[baseline=-0.65ex,scale=0.07]
        \begin{knot}[clip width=4]
            \strand[semithick,-Latex] (-10, -4) to (-7, -4) [in=left, out=right] to (0, -1) [in=left, out=right] to (7, -4) to (10, -4) to (15, -4);
            \strand[semithick] (-10, 4) to (-7, 4) [in=left, out=right] to (0, 1) [in=left, out=right] to (7, 4) to (10, 4) to (15, 4);
            \node[darkblue] at (-10, -4)[left] {$y$};
            \node[darkblue] at (-10, 4)[left] {$x$};
        \end{knot}
        \end{tikzpicture}
    \]
\end{comment}
    Trzeci ruch:
\begin{comment}
    \[
        \begin{tikzpicture}[baseline=-0.65ex,scale=0.07]
        \begin{knot}[clip width=5, flip crossing/.list={3}]
            \node[darkblue] at (-10, 10) [left] {$z$};
            \strand[semithick,-Latex] (-10, 10) [in=left, out=right] to (10,-10) to (15,-10);
            \node[darkblue] at (-10, 0) [left] {$y$};
            \strand[semithick,-Latex] (-10, 0) [in=left, out=right] to (0, 10) [in=left, out=right] to (10, 0) to (15, 0);
            \node[darkblue] at (-10, -10) [left] {$x$};
            \strand[semithick,-Latex] (-10, -10) [in=left, out=right] to (10,10) to (15,10);
            \node[darkblue] at (0, 10) [above] {$y \triangleright z$};
            \node[darkblue] at (15, 10) [right] {$(x \triangleright z) \triangleright (y \triangleright z)$};
        \end{knot}
        \end{tikzpicture}
        \stackrel{R_3}{\cong} \,\,
        \begin{tikzpicture}[baseline=-0.65ex,scale=0.07]
        \begin{knot}[clip width=5, flip crossing/.list={3}]
            \node[darkblue] at (-10, 10) [left] {$z$};
            \strand[semithick,-Latex] (-10, 10) [in=left, out=right] to (10,-10)  to (15, -10);
            \node[darkblue] at (-10, 0) [left] {$y$};
            \strand[semithick,-Latex] (-10, 0) [in=left, out=right] to (0, -10) [in=left, out=right] to (10, 0) to (15, 0);
            \node[darkblue] at (-10, -10) [left] {$x$};
            \strand[semithick,-Latex] (-10, -10) [in=left, out=right] to (10,10) to (15, 10);
            \node[darkblue] at (15, 10) [right] {$(x \triangleright y) \triangleright z$};
        \end{knot}
        \end{tikzpicture}
        \qedhere
    \]
\end{comment}
\end{proof}

Homomorfizmy definiujemy standardowo, przez analogię do grup:

\begin{definition}
    Niech $Q_1, Q_2$ będą kwandlami.
    Odwzorowanie $f \colon Q_1 \to Q_2$, które dla waszystkich $x,y \in Q_1$ spełnia warunek $f(x \triangleright y) = f(x) \triangleright f(y)$, nazywamy homomorfizmem.
\end{definition}

Wiele znanych struktur algebraicznych okazuje się być źródłem kwandli.

\begin{example}[kwandl cykliczny/diedralny]
    Grupa abelowa z działaniem $x \triangleright y = 2y - x$.
\end{example}

\begin{example}[kwandle sprzężone]
    Grupa z działaniem $x \triangleright y = y^{-n} x y^n$.
\end{example}

\begin{example}[kwandl Alexandera]
    Moduł nad pierścieniem $\Z[t, 1/t]$ wielomianów Laurenta z~działaniem $x \triangleright y =tx + (1-t) y$.
\end{example}

\begin{example}[kwandl symplektyczny]
    Przestrzeń liniowa i antysymetryczna forma dwuliniowa $\langle \cdot | \cdot \rangle$ z działaniem $x \triangleright y = x + \langle x | y \rangle y$.
\end{example}

D. Joyce w swojej rozprawie doktorskiej przypisał każdemu węzłowi $K$ pewien szczególny kwandl $Q(K)$, kwandl podstawowy.
Definicja tego obiektu jest dość zawiła: łuki diagramu są generatorami, zaś skrzyżowania odpowiadają za relacje.
Joyce pokazał, że kwandl $Q(K)$ wyznacza węzeł $K$ jednoznacznie z dokładnością do orientacji.
Nie czyni to jednak nowego niezmiennika użytecznym, gdyż wyznaczenie go nawet w najprostszych przypadkach stanowi trudność.
Niebrzydowski, Przytycki pokazali w 2008 roku, że kwandl podstawowy trójlistnika jest izomorficzny z~rzutowym pierwotnym podkwandlem pewnych odwzorowań liniowych przestrzeni symplektycznej $\Z \oplus \Z$, cokolwiek to znaczy.

Aksjomaty grupy można wzmacniać (grupy abelowe) lub osłabiać (monoidy).
Podobnie czyni się z aksjomatami kwandli.
Kwandle inwolutywne odpowiadają węzłom bez orientacji, wraki dobrze opisują węzły obramowane (\emph{framed}), i tak dalej.

\begin{definition}[kwandl inwolutywny]
    Kwandl $Q$, w którym dla wszystkich $x, y \in Q$ zachodzi $x \triangleleft (x \triangleleft y) = y$, nazywamy inwolutywnym (albo kei).
\end{definition}

Kwandle inwolutywne badał jako pierwszy Mituhisa Takasaki (1943).
Szukał niełącznej struktury, która dobrze opisywałaby odbicia w skończonej geometrii.

\begin{definition}[półka]
    \index{shelf}
    Zbiór $X$ wyposażony w dwuargumentowe działanie $\triangleright$ taki, że dla wszystkich elementów $x, y, z \in X$ zachodzi $(x \triangleright y) \triangleright z = (x \triangleright z) \triangleright (y \triangleright z)$, nazywamy półką.
\end{definition}

\begin{example}
    Niech $B_\infty$ oznacza grupę wszystkich warkoczy, zaś $\phi$ będzie jej endomorfizmem posyłającym generator $\sigma_k$ na $\sigma_{k+1}$.
    Zbiór $B_\infty$ z działaniem $a \triangleleft b = a\phi(b)\sigma_1 \phi{a} ^{-1}$ jest półką.
\end{example}

Półki zdefiniowała Alissa Crans w pracy doktorskiej.
To nieprzetłumaczalna gra słów: dwie półki (\emph{shelves}), lewa i prawa, które dobrze do siebie pasują, dają stojak (\emph{rack}).
Półka stanowi uogólnienie dwóch obiektów -- wraków i wrzecion.

\begin{definition}[wrak]
    \index{wrack}
    Zbiór $X$ z dwuargumentowym działaniem $\triangleright$ taki, że dla wszystkich elementów $x, y, z \in X$ zachodzi:
    \begin{enumerate}
        \item odwzorowanie $\beta_y \colon X \to X$ dane wzorem $\beta_y(x) = x \triangleright y$ jest odwracalne,
        \item $(x \triangleright y) \triangleright z = (x \triangleright z) \triangleright (y \triangleright z)$
    \end{enumerate}
    nazywamy wrakiem (z angielskiego \emph{wrack}).
\end{definition}

\begin{example}
    Zbiór $X = \{1, 2, \ldots, n\}$ i permutacja $\sigma \in S_n$ z działaniem $x \triangleright y = \sigma(x)$.
\end{example}

\begin{example}
    Moduł nad pierścieniem $\Z[t^{\pm 1}, s]/(s^2 - (1-t)s)$ z działaniem $x \triangleright y = tx+sy$.
\end{example}

Wraki dobrze współgrają z podwójnym pierwszym ruchem Reidemeistera, który to nie zmienia spinu diagramu:
\begin{comment}
\[
    \begin{tikzpicture}[baseline=-0.65ex,scale=0.07]
    \begin{knot}[clip width=5,flip crossing/.list={1}]
        \strand[semithick] (15, 0) [in=up,out=left] to (-5, -7);
        \strand[semithick] (-5, -7) [in=down,out=down] to (5, -7);
        \strand[semithick] (5, -7) [in=right,out=up] to (-15, 0);
        \strand[semithick,Latex-] (45, 0) [in=up,out=left] to (25, -7);
        \strand[semithick] (25, -7) [in=down,out=down] to (35, -7);
        \strand[semithick] (35, -7) [in=right,out=up] to (15, 0);
        \node[darkblue] at (-15, 0)[left] {$x$};
        \node[darkblue] at (15, 0)[above] {$x \triangleright x$};
        \node[darkblue] at (45, 0)[right] {$x$};
    \end{knot}
    \end{tikzpicture}
    \cong
    \begin{tikzpicture}[baseline=-0.65ex,scale=0.07]
    \begin{knot}[clip width=5]
        \strand[semithick,-Latex] (-15, 0) to (15, 0);
        \node[darkblue] at (-15, 0)[left] {$x$};
    \end{knot}
    \end{tikzpicture}
\]
\end{comment}

\begin{definition}[wrzeciono]
    \index{spindle}
    Zbiór $X$ z dwuargumentowym działaniem $\triangleright$ taki, że dla wszystkich elementów $x, y, z \in X$ zachodzi:
    \begin{enumerate}
        \item $x \triangleright x = x$,
        \item $(x \triangleright y) \triangleright z = (x \triangleright z) \triangleright (y \triangleright z)$
    \end{enumerate}
    nazywamy wrzecionem (z angielskiego \emph{spindle}).
\end{definition}

Zatem kwandle to wraki, które są też wrzecionami.
Muszę w~tym miejscu wtrącić uwagę językową.
Conway nazwał wraki wrakami (\emph{wracks}), by częściowo zażartować z~nazwiska jego kolegi Gavina Wraitha, a częściowo by zaznaczyć, że są one tym, co zostaje z~grupy, w~której zapomniano o~mnożeniu, ale nie sprzęganiu (w~języku angielskim co najmniej od XVI wieku funkcjonuje zwrot ,,wrack and ruin'' oznaczający zniszczenie).
Obecnie dominuje określenie \emph{racks}.

%Pokażemy, że kwandle uogólniają kolorowania.
%Niech $X$ będzie zbiorem kolorów z~operacją $\triangleright$, które spełnia aksjomaty z~definicji kwandli.
%Wtedy przy każdym skrzyżowaniu występują trzy kolory: $x$, $y$ oraz $x \triangleright y$.

Clark, Elhamdadi, Saito oraz Yeatman pokazali niedawno (\cite{clark13}) zbiór 26 kwandli, które razem odróżniają od siebie wszystkie 2977 zorientowanych węzłów pierwszych o~co najwyżej 12 skrzyżowaniach.
Największy z~nich jest rzędu 182.
Wcześniej (w 2003 roku) Dionisio, Lopes znaleźli 10 kwandli Alexandera, które odróżniają 249 węzłów pierwszych do 10 skrzyżowań.
Vendramin znalazł w \cite{vendramin12} wszystkie 431 kwandli spójnych rzędu 35 lub mniejszego.

%Przypomnijmy, że 3-kolorowanie diagramu polegało na przypisaniu każdemu włóknu pewnego koloru (z trzech) tak, by każdy został użyty, a~żadne skrzyżowanie nie stało się dwubarwne.
%Ogólniej, jeśli kolorami były liczby $0, \ldots, n - 1$, żądaliśmy od skrzyżowań, by kolor $y$ po przejściu pod kolorem $x$ stawał się $z$, gdzie $z \equiv 2x - y$ modulo $n$.
%Można to uogólnić jeszcze bardziej, właśnie do quandli: $\Z/n$-kolorowanie węzła to quandle związany z~pierścieniem $\Z/n$ operacją $x \triangleright y =  2y - x$}

% Koniec sekcji Wraki i~kwandle

\section{Homologie} % (fold)
\label{sec:homology}
Z powodu naszego ograniczonego rozumienia topologii algebraicznej oraz teorii kategorii
wyłożony poniżej materiał jest tak naprawdę tylko przytoczeniem podstawowych definicji i~faktów.
Prezentowane podejście do homologii Chowanowa pochodzi od Oleg Viro (praca \cite{viro04}.

Kompleks łańcuchowy to ciąg grup abelowych $C_n$ indeksowanych liczbami całkowitymi
wraz z~różniczkami, morfizmami $\partial_n \colon C_n \to C_{n-1}$ takimi,
że złożenie $\partial_{n-1} \circ \partial_n = 0$ jest trywialne.
Iloraz $\ker \partial_n / \operatorname{im} \partial_{n+1}$ nazywamy $n$-tą grupą homologii, $H_n$.

Innym narzędziem wykrywającym niewęzły jest homologia Chowanowa (opisana później),
jak pokazał Kronheimer z~Mrówką \cite{kronheimer11}.
Bar-Natan, topolog izraelski, napisał program liczący te homologie szybko \cite{barnatan07},
zapewne w~czasie $O(\exp(c \sqrt n))$, dla diagramu o~$n$ skrzyżowaniach.
Nie możemy liczyć na istotne przyspieszenie:
znalezienie przybliżenia wielomianu Jonesa jest problemem \#P-trudnym (\cite{kuperberg15}, \cite{vertigan05}),
a przy znanych homologiach -- wręcz trywialnym.
Patrz też \ref{jones_sharp_p_hard}.

% Przykładem takiego obiektu jest kompleks symplicjalny
% Kompleks symplicjalny: para $K = (V, P)$, gdzie $P \subseteq \mathfrak P(V)$ jest
% taką rodziną skończonych podzbiorów zamkniętą na branie podzbiorów, że $v \in V \implies \{v\} \in P$.
% $V$ zbiór wierzchołków, $P$ sympleksów.

\begin{definition}
    Niech $D$ będzie diagram splotu.
    Niezredukowanym nawiasem Kauffmana nazywamy wielomian
    \[
        [D] = (-A^2 - A^{-2}) \langle D \rangle = \sum_s A^{\sigma(s)} (-A^2 - A^{-2})^{|D_s|}.
    \]
\end{definition}

\begin{definition}
    Rozszerzonym stanem Kauffmana nazywamy parę uporządkowaną $S = (s, r)$,
    gdzie $s$ to stan Kauffmana,
    zaś $r$ to odwzorowanie $D_s \to \{\pm 1\}$ ze zbioru składowych diagramu.
\end{definition}

\begin{definition}
    Zbiór rozszerzonych stanów Kauffmana $\mathcal S(D)$:
    rozbija się na podzbiory indeksowane przez pary liczb całkowitych:
    $\mathcal S_{i, j} = \{S : \sigma(s) = i, \sigma(s) + 2 \tau(s) = j\}$.
\end{definition}

Tutaj lokalnie $\sigma(s) = |s|$, oraz $\tau(s) = |r^{-1}[1]| - |r^{-1}[-1]|$.

%%% Zauważmy, że $|D_s| \equiv r(s) \mod 2$. DLACZEGO?

\begin{definition}
    \index{homologia!Chowanowa}
    Niech $C_{i, j}$ będzie wolną grupą abelową generowaną przez zbiór $\mathcal S_{i, j}$.
    Ponumerujmy skrzyżowania diagramu $D$ liczbami $1, 2, \ldots, n$.
    Homologie Chowanowa splotu o~diagramie $D$ to homologie kompleksu
    \[
        C(D) = \bigoplus_{i, j \in \Z} C_{i, j}(D),
    \]
    z różniczkami $\partial_{i, j} \colon C_{i,j} \to C_{i-2, j}$ danymi wzorem
    \[
        \partial_{i, j}(S) = \sum_{S'} (-1)^{t(S, S')}  S'.
    \]
    Sumowanie odbywa się po tych $S' \in \mathcal S_{i-2, j}$,
    które różnią się od $S$ na dokładnie jednym skrzyżowaniu $v$:
    $S(v) = 1$, $S'(v) = -1$ oraz $\tau(S') = 1 + \tau (S)$.

    Liczba $t(S, S')$ to liczba skrzyżowań $D$ mniejszych od $v$, dla których $S$ przyjmuje wartość $-1$.
\end{definition}

Chowanow w~pracy \cite{khovanov00} przypisał każdemu rozszerzonemu stanowi Kauffmana
element w~bialgebrze $A^{\otimes |D_s|}$, gdzie $A = \Z[X]/(X^2)$.
Różniczka wyraża się w~terminach mnożenia i~komnożenia,
gdy okręgi dodatnie (z $r^{-1}[1]$) zamienimy na $X$, zaś pozostałe na $1$.

% Patrz ,,system Frobeniusa''.

% Koniec sekcji Homologie


\chapter{Wybrane rodziny węzłów}
Zamknęliśmy już pierwszy rozdział,
Wciąż nie pokazaliśmy jednak pełnego dowodu, że dwa konkretne węzły (na przykład niewęzeł i~trójlistnik) są od siebie różne.
Dlatego teraz podamy proste narzędzie odróżniające węzły: \emph{trójkolorowalność}, która przypisuje włóknom diagramu różne kolory.
Następnie rozszerzymy paletę do dowolnie wielu kolorów i~zastąpimy ją grupą skończoną.
Nawet wzmocniony wariant nie jest idealnym narzędziem klasyfikującym.
Istnieją węzły, których nie odróżnia.
Problem ten dotyka wielu późniejszych niezmienników, pierwszy niezmiennik zupełny poznamy dopiero w~rozdziale czwartym.
Nie stanowi to wielkiego powodu do radości ze względu na trudności w~jego wyznaczaniu.

\section{Kolorowanie splotów} % (fold)
\label{sec:colour_links}
Oto mniej mętny opis trójkolorowalności, czyli jak nietrudno się domyślić, kolorowalności trzema kolorami.
Diagram $D$ splotu $K$ jest trójkolorowalny, jeśli każdemu włóknu można przypisać jeden z~trzech kolorów tak, by co najmniej dwa zostały użyte.
Wymagamy przy tym, by przy żadnym skrzyżowaniu nie spotykały się dokładnie dwa kolory.

Dla własnej wygody jako kolorów używać będziemy kolejnych liczb naturalnych $0, 1, 2, \ldots$.
Pozwala to zapisać warunek kolorowalności równaniem algebraicznym.

\begin{definition}[kolorowanie] \label{def:colour_equation}
    Niech $L$ będzie splotem, zaś $n$ liczbą naturalną.
    Mówimy, że splot $L$ jest kolorowalny modulo $n$, jeśli posiada diagram, którego włóknom można przypisać liczby całkowite $0, \ldots, n - 1$ tak, by
    \begin{enumerate}[leftmargin=*]
        \item istniały dwa włókna różnych kolorów,
        \item równanie $a + b \equiv 2c$ modulo $n$ było spełnione przy każdym skrzyżowaniu:
    \end{enumerate}
    \[
        \begin{tikzpicture}[baseline=-0.65ex, scale=0.12]
            \useasboundingbox (-5, -5) rectangle (5,5);
            \begin{knot}[clip width=5, end tolerance=1pt, flip crossing/.list={1}]
                \strand[semithick] (-5,5) to (5,-5);
                \strand[semithick] (-5,-5) to (5,5);
                \node[darkblue] at (5, 5)[below right] {$c$};
                \node[darkblue] at (5, -5)[above right] {$b$};
                \node[darkblue] at (-5, 5)[below left] {$a$};
            \end{knot}
        \end{tikzpicture}.
    \]
    Takie przyporządkowanie nazywamy (nietrywialnym) kolorowaniem.
\end{definition}

Metoda ta została odkryta razem z~uogólnieniem do $n$ kolorów przez Ralpha Foxa w~1956, kiedy próbował uczynić teorię węzłów bardziej przystępną dla studentów.
Opierając się tylko na definicji kolorowania oraz ruchach Reidemeistera możemy wykazać pierwsze własności kolorowań.

\begin{proposition} \label{color_invariant}
    ,,Bycie $n$-kolorowalnym'' jest niezmiennikiem węzłów.
\end{proposition}

\begin{proof}
    Wystarczy sprawdzić, jak ruchy Reidemeistera zmieniają kolory.
    Pierwszy i~drugi:
    \[
        \fbox{
        \begin{tikzpicture}[baseline=-0.65ex,scale=0.07]
        \begin{knot}[clip width=5]
            \strand[semithick] (-10,10) .. controls (-10,2) and (-10,2) .. (-6,-2);
            \strand[semithick] (-10,-10) .. controls (-10,-2) and (-10,-1) .. (-9,0);

            \strand[semithick] (-7,1) -- (-6,2);
            \strand[semithick] (-6,2) .. controls (2,9) and (2,-9) .. (-6,-2);
            \node[darkblue] at (-10, 10)[below left] {$a$};
            \node[darkblue] at (-10, -10)[above left] {$b \equiv a$};
        \end{knot}
        \end{tikzpicture}
        $\stackrel{R_1}{\cong} \,\,$
        \begin{tikzpicture}[baseline=-0.65ex,scale=0.07]
        \begin{knot}[clip width=5]
            \strand[semithick] (0,10) -- (0,-10);
            \node[darkblue] at (0, 0)[left] {$a$};
        \end{knot}
        \end{tikzpicture}}
        %%%
        \quad \fbox{
        \begin{tikzpicture}[baseline=-0.65ex,scale=0.07]
        \begin{knot}[clip width=5]
            \strand[semithick] (4,-10) .. controls (4,-4) and (-4,-4) .. (-4,0);
            \node[darkblue] at (-4, -10)[above left] {$d \equiv b$};
            \strand[semithick] (4,10) .. controls (4, 4) and (-4, 4) .. (-4,0);
            \node[darkblue] at (4, 10)[below right] {$a$};
            \strand[semithick] (-4,-10) .. controls (-4,-4) and (4,-4) .. (4,0);
            \node[darkblue] at (4, 0) [right] {$c \equiv 2a-b$};
            \strand[semithick] (-4, 10) .. controls (-4, 4) and (4,4) .. (4,0);
            \node[darkblue] at (-4, 10) [below left] {$b$};
        \end{knot}
        \end{tikzpicture}
        $\stackrel{R_2}{\cong} \,\,$
        \begin{tikzpicture}[baseline=-0.65ex,scale=0.07]
        \begin{knot}[clip width=5]
            \strand[semithick] (4,-10) .. controls (4,-4) and (1,-4) .. (1,0);
            \strand[semithick] (4,10) .. controls (4, 4) and (1, 4) .. (1,0);
            \strand[semithick] (-4,-10) .. controls (-4,-4) and (-1,-4) .. (-1,0);
            \strand[semithick] (-4,10) .. controls (-4, 4) and (-1,4) .. (-1,0);
        \end{knot}
        \end{tikzpicture}}
    \]
    Trzeci ruch także nie wymaga skomplikowanych rachunków.
    Najkrótszy łuk na diagramach ma kolor $2a-c$ po lewej oraz $2b-c$ po prawej stronie.
    \[
     \fbox{
        \begin{tikzpicture}[baseline=-0.65ex,scale=0.07]
        \begin{knot}[clip width=5, flip crossing/.list={1,2,3}]
            \node[darkblue] at (-10, 10) [above] {$b$};
            \node[darkblue] at (10, 10) [above] {$c$};
            \node[darkblue] at (-10, -10) [below] {$2a-2b+c$};
            \node[darkblue] at (10, -10) [below] {$2a-b$};
            \node[darkblue] at (-10, -2) [left] {$a$};
            \strand[semithick] (-10,-10) -- (10,10);
            \strand[semithick] (-10,10) -- (10,-10);
            \strand[semithick] (-10,-2) .. controls (-4, -2) and (-4,8) .. (0,8);
            \strand[semithick] (10,-2) .. controls (4, -2) and (4,8) .. (0,8);
        \end{knot}
        \end{tikzpicture}
        $\stackrel{R_3}{\cong} \,\,$
        \begin{tikzpicture}[baseline=-0.65ex,scale=0.07]
        \begin{knot}[clip width=5, flip crossing/.list={1,2,3}]
            \node[darkblue] at (-10, 10) [above] {$b$};
            \node[darkblue] at (10, 10) [above] {$c$};
            \node[darkblue] at (-10, -10) [below] {$2a-2b+c$};
            \node[darkblue] at (10, -10) [below] {$2a-b$};
            \node[darkblue] at (10, 2) [right] {$a$};
            \strand[semithick] (-10,-10) -- (10,10);
            \strand[semithick] (-10,10) -- (10,-10);
            \strand[semithick] (-10,2) .. controls (-4, 2) and (-4,-8) .. (0,-8);
            \strand[semithick] (10,2) .. controls (4, 2) and (4,-8) .. (0,-8);
        \end{knot}
        \end{tikzpicture}} \qedhere
    \]
\end{proof}

Trójlistnik koloruje się dokładnie modulo krotności trójki, ósemka zaś -- piątki.
Sama kolorowalność nie mówi wiele, splot jest kolorowalny lub nie.
Dowód faktu \ref{color_invariant} pokazuje coś więcej: liczba kolorowań, być może trywialnych, jest mocniejszym niezmiennikiem.
Liczbę kolorowań splotu $L$ modulo $n$, trywialnych lub nie, oznaczamy przez $\tau_n(L)$.

\begin{lemma}
    \label{lem:colouring_arc}
    Niech $D$ będzie diagramem z definicji \ref{def:colour_equation} z wybranym łukiem, zaś $k \in \{0, \ldots, n - 1\}$ pewnym kolorem.
    Bez straty ogólności możemy założyć, że krótki łuk jest koloru $k$.
\end{lemma}

Kolorem tym zazwyczaj jest $0$.

\begin{proof}
    Dodanie tej samej wartości do wszystkich łuków na dobrze pokolorowanym diagramie daje nowy, także dobrze pokolorowany diagram.
\end{proof}

\begin{proposition}
    \label{no_knots_colours_mod_two}
    Żaden węzeł nie koloruje się modulo dwa.
\end{proposition}

\begin{proof}
    Załóżmy nie wprost, że istnieje nietrywialne kolorowanie.
    Analiza czterech możliwych skrzyżowań pokazuje, że włókna wychodzące z~tunelu muszą mieć ten sam kolor.
    Przechodząc wzdłuż węzła widzimy jeden kolor, wbrew założeniu nie wprost.
\end{proof}

\begin{proposition}
    Każdy splot o co najmniej dwóch ogniwach koloruje się modula dwa.
\end{proposition}

\begin{proof}
    Wystarczy pomalować jedną składową zerem, a~pozostałe jedynkami.
\end{proof}

Sploty rozszczepialne są $n$-kolorowalne dla każdego $n \ge 2$, można skorzystać z~tego samego schematu kolorowania.
Pierścienie Boromeuszy nie kolorują się modulo trzy, nie są zatem rozszczepialne.
Sploty, które nie są kolorowalne modulo $n$ dla każdej liczby $n \in \N$ nazywa się czasem niewidzialnymi, dwa węzły do dziesięciu skrzyżowań mają tę własność: $10_{124}$ oraz $10_{153}$.

Pokażemy teraz, że suma równań kolorujących z dobrze wybranymi znakami jest postaci $0 \equiv 0 \mod n$.
Będziemy potrzebować kilku pomocniczych definicji.
Każdy diagram węzła rozcina płaszczyznę na obszary, z~czego jeden jest nieograniczony.

\begin{definition}[uszachowienie]
    Przyporządkowanie każdemu z~obszarów, na jakie diagram rozcina płaszczyznę, jednego z~dwóch kolorów tak, by sąsiadujące ze sobą obszary były różnych kolorów, nazywamy uszachowieniem.
\end{definition}

Ustalmy węzeł $K$ oraz dowolne uszachowienie dla jego diagramu.
Skojarzmy z~każdym skrzyżowaniem równanie kolorujące, zgodnie z~poniższym schematem:
\[\begin{tikzpicture}[baseline=-0.65ex, scale=0.12]
    \useasboundingbox (-5, -10) rectangle (5,5);
    \begin{knot}[clip width=5, end tolerance=1pt, flip crossing/.list={1}]
        \strand[semithick] (-5,5) to (5,-5);
        \strand[semithick] (-5,-5) to (5,5);
        \fill[blue!20!white] (-4, 5) to (0, 1) to (4, 5);
        \fill[blue!20!white] (-4, -5) to (0, -1) to (4, -5);
        \node[darkblue] at (-5, -5)[left] {$a$};
        \node[darkblue] at (-5, +5)[left] {$b$};
        \node[darkblue] at (+5, -5)[right] {$c$};
        \node[darkblue] at (+5, +5)[right] {$a$};
        \node[darkblue] at (0, -10) {$+a-b+a-c=0 \mod n$};
    \end{knot}
    \end{tikzpicture}
    \quad\quad\quad\quad\quad\quad\quad\quad\quad\quad\quad\quad
    \begin{tikzpicture}[baseline=-0.65ex, scale=0.12]
    \useasboundingbox (-5, -10) rectangle (5,5);
    \begin{knot}[clip width=5, end tolerance=1pt, flip crossing/.list={1}]
        \strand[semithick] (-5,5) to (5,-5);
        \strand[semithick] (-5,-5) to (5,5);
        \fill[blue!20!white] (5, -4) to (1, 0) to (5, 4);
        \fill[blue!20!white] (-5, -4) to (-1, 0) to (-5, 4);
        \node[darkblue] at (-5, -5)[left] {$a$};
        \node[darkblue] at (-5, +5)[left] {$b$};
        \node[darkblue] at (+5, -5)[right] {$c$};
        \node[darkblue] at (+5, +5)[right] {$a$};
        \node[darkblue] at (0, -10) {$-a+b-a+c=0 \mod n$};
    \end{knot}
    \end{tikzpicture}
\]

\begin{proposition}
    \label{prp:colouring_sum_zero}
    Sumą równań kolorujących o dobrze wybranych znakach jest $0 \equiv 0 \mod n$.
\end{proposition}

Będziemy potrzebować tego do pokazania, że wyznacznik determinuje kolorowalność splotu.

\begin{proof}
    Każde równanie kolorujące składa się z~czterech wyrazów, po jednym od każdej nici, która spotyka się w~danym skrzyżowaniu.
    Nić biegnie między dwoma skrzyżowaniami, więc suma wszystkich równań kolorujących składa się z~par składników, po jednej parze na nić.
    Składniki te są przeciwnych znaków, zatem wzajemnie się znoszą.
    Suma równań kolorujących jest sumą zer, a~to należało udowodnić.
\end{proof}

\begin{proposition}
    Jeśli $K, L$ są węzłami, to $3\tau_3(K \shrap L) = \tau_3(K)\tau_3(L)$.
\end{proposition}

\begin{corollary}
    Istnieje nieskończenie wiele węzłów.
\end{corollary}

\begin{proof}
    Suma spójna $n$ trójlistników ma $3^n$ (być może trywialnych) $3$-kolorowań.
\end{proof}

Jako kolorów użyjemy teraz elementów $g_1, \ldots, g_n$ pewnej skończonej grupy $G$.

\begin{definition}[etykietowanie]
    Mówimy, że zorientowany węzeł $K$ jest etykietowalny grupą $G$ generowaną przez elementy $g_1, \ldots, g_n$, jeśli posiada diagram, którego włóknom przypisano elementy $g_1, \ldots, g_n$ tak, by równanie $gk=hg$ było spełnione przy każdym skrzyżowaniu ($g$: włókno biegnące górą, $k$: bo jego lewej stronie, $h$: po prawej).
    \[
        \begin{tikzpicture}[baseline=-0.65ex, scale=0.12]
            \useasboundingbox (-5, -5) rectangle (5,5);
            \begin{knot}[clip width=5, end tolerance=1pt, flip crossing/.list={1}]
                \strand[semithick] (-5,5) to (5,-5);
                \strand[semithick, -Latex] (-5,-5) to (5,5);
                \node[darkblue] at (5, 5)[below right] {$g$};
                \node[darkblue] at (5, -5)[above right] {$h$};
                \node[darkblue] at (-5, 5)[below left] {$k$};
            \end{knot}
        \end{tikzpicture}
    \]
\end{definition}

Równanie $gkg^{-1}=h$ mówi, że etykiety włókien wchodzących oraz wychodzących są sprzężone.
Wynika stąd, że wszystkie etykiety pochodzą z~jednej klasy sprzężoności.
Muszą jednocześnie generować całą grupę, dlatego $G$ musi być grupą nieprzemienną lub trywialną.
Etykietowalność jest niezmiennikiem węzłów i~nie zależy od orientacji węzła:
jeżeli elementy $g_1, \ldots, g_n$ generują grupę, to ich odwrotności także.

Rozpatrzmy węzły $6_1$ oraz $9_{46}$ i~spróbujmy etykietować je transpozycjami z~grupy $S_4$.
Wybranie dwóch etykiet przy jednym skrzyżowaniu $6_1$ wymusza etykiety dla wszystkich włókien.
Dwie transpozycje nie mogą generować grupy $S_4$, natomiast włókna węzła $9_{46}$ dają się etykietować samymi transpozycjami.
Węzły te są więc różne, choć mają te same własności kolorujące.

Etykietowanie jest mocnym narzędziem odróżniającym węzły.
Thistlethwaite w 1985 roku korzystając z niego klasyfikował węzły o~co najwyżej 13 skrzyżowaniach (jest ich, jak ostatecznie się okazało, 12965).
Mają one tylko 5639 różnych wielomianów Alexandera, ale etykietowania trzynastoma różnymi grupami pozwoliły zmniejszyć liczbę nierozpoznanych węzłów do około tysiąca.
Wśród nich 30 posiada wielomian Conwaya $1 + 2z^2 + 2z^4$, ale pary rozróżniane wielomianem HOMFLY mają też różne wielomiany Jonesa.
Wielomiany opisujemy w~rozdziale trzecim.

Niech $p \ge 3$ będzie liczbą pierwszą, natomiast $D_p = \langle r, s \mid r^p = s^2 = e, rsr = s \rangle$ grupą diedralną rzędu $2p$.
Elementy tej grupy to $1, r, r^2, \ldots, r^{p-1}, s, sr, \ldots, sr^{p-1}$.
,,Obrót'' $r^k$ jest sprzężony tylko ze swoją odwrotnością, ale ,,symetrie osiowe'' $sr^k$ tworzą jedną klasę sprzężoności.
Łatwo widać, że dowolne dwie z~nich generują całą grupę $D_p$.

\begin{proposition}
    Węzeł $K$ jest $p$-kolorowalny wtedy i~tylko wtedy, gdy jest $D_p$-etykietowalny.
\end{proposition}

\begin{proof}
    Załóżmy, że $K$ ma $n$ włókien.
    Wiemy już, że każde $D_p$-etykietowanie wykorzystuje tylko elementy $sr^{a_1}, \ldots, sr^{a_n}$ dla $1 \le a_i \le p$.
    Jest ono prawidłowe dokładnie wtedy, gdy analogiczne kolorowanie liczbami $a_1, \ldots, a_n$ jest prawidłowe.
\end{proof}

Kolorowania definiowano kiedyś jako surjekcje $\rho \colon \pi \to D_{2n}$ z~grupy podstawowej.
Jak mówi prezentacja Wirtingera, grupa splotu generowana jest przez ścieżki z~punktu bazowego w~$S^3$ do brzegu rurowego otoczenia splotu, wokół południka i~znowu do bazowego punktu.
Fox zauważył, że z~surjektywności $\rho$ wynika, iż generatory mapują się na symetrie osiowe $sr^k$.
Ponieważ istnieje wzajemnie jednoznaczna odpowiedniość między generatorami grupy splotu oraz łukami diagramu, każdemu możemy przypisać liczbę całkowitą $k$.
Etykietowania są więc uogólnieniem kolorowań.
Rozumowanie, które przedstawiliśmy, prowadzi do prostej klasyfikacji grup, których można użyć do etykietowania.

\begin{proposition}
    Niech $K$ będzie węzłem, $\pi$ grupą podstawową jego dopełnienia, zaś $G$ dowolną grupą.
    Następujące warunki są równoważne: $K$ jest $G$-etykietowalny; istnieje surjekcja $\pi_1 \to G$.
\end{proposition}

Historycznie, prezentacja Wirtingera była pierwsza, zaś etykietowania odkryto później.

\begin{proposition}[Perko]
    Niech $K$ będzie węzłem etykietowalnym grupą $S_3$.
    Wtedy $K$ jest też etykietowalny grupą $S_4$.
\end{proposition}

Nie znam innych nietrywialnych faktów dotyczących etykietowań.

% Koniec sekcji Kolorowanie splotów

\section{Węzły torusowe} % (fold)
\label{sec:torus}
W tej sekcji przyjrzymy się węzłom o~specjalnym ułożeniu w~przestrzeni $\R^3$.
Do ich określenia potrzebny jest torus trywialny,
powierzchnia otrzymana przez obrót okręgu $(x-2)^2 + y^2 = 1$ wokół osi $y$.
Można go także uzyskać przez sklejenie podstaw walca tak, by go przy tym nie zapętlić.
Oczywiście istnieją też nietrywialne torusy, na przykład rurowe otoczenie trójlistnika.

\begin{definition}
    \index{węzeł!torusowy}
    Węzeł (splot) torusowy to taki, który leży na powierzchni niezaplątanego torusa.
\end{definition}

Na walcu $S^2 \times [0,1]$, którego podstawa leży w~płaszczyźnie $xy$, rozpatrzmy $r$ skierowanych odcinkach (dla $k = 0, 1, \ldots, r - 1$) o~końcach w~punktach
\begin{align*}
    \left(\cos \frac{2k \pi}{r}, \sin \frac{2k\pi}{r}, 0 \right), \quad
    \left(\cos \frac{2k \pi}{r}, \sin \frac{2k\pi}{r}, 1 \right).
\end{align*}
Przekręćmy górną podstawę walca wokół osi $z$ o~skierowany kąt $2\pi q / r$ oraz utożsammy ze sobą pary punktów $(x, y, 0) \sim (x, y, 1)$,
Uzyskaliśmy splot torusowy $K_{q, r}$: okrąża on $q$ razy rdzeń torusa i~$p$ razy jego oś symetrii obrotowej.
Określimy jeszcze kilka splotów torusowych.
Węzeł $K_{0, 0}$ leży na powierzchni torusa i~jest ściągalny do punktu, zaś $K_{1, 0}$ to nawinięta toroidalnie pętla.
Węzeł $K_{p, q}$ posiada następującą parametryzację:
\[
    x = (2+\cos q \phi) \cos p \phi, \quad
    y = (2+\cos q \phi) \sin p \phi, \quad
    z = - \sin q \phi, \quad
    0 \le \phi \le 2\pi.
\]
Poniżej przedstawiamy trzy węzły torusowe.

\begin{figure}[H]
    \begin{minipage}[b]{.3\linewidth}
        \centering
        \includegraphics[width=\linewidth]{../data/torus-p2-q3.pdf}
        \subcaption{trójlistnik: $p = 2, q = 3$}
    \end{minipage}
    \begin{minipage}[b]{.3\linewidth}
        \centering
        \includegraphics[width=\linewidth]{../data/torus-p2-q11.pdf}
        \subcaption{$p = 2, q = 11$}
    \end{minipage}
    \begin{minipage}[b]{.3\linewidth}
        \centering
        \includegraphics[width=\linewidth]{../data/torus-p11-q2.pdf}
        \subcaption{$p = 11, q = 2$}
    \end{minipage}
\end{figure}

Okazuje się, że innych obiektów już nie ma.

\begin{proposition}
    Jeśli żadna ze składowych splotu torusowego nie jest postaci $K_{0, 0}$ lub $K_{1, 0}$, to splot ten jest równoważny ze splotem $K_{q, r}$ dla pewnych $q, r$.
    Największy wspólny dzielnik indeksów $q, r$ jest jednocześnie liczbą składowych splotu.
\end{proposition}

%Węzeł ten leży na torusie $(r - 2)^2 + z^2 = 1$.
% p = 5;
% q = 3;
% ParametricPlot3D[
% {
% Cos [2 Pi p t] (2 + Cos[2 Pi q t]),
% (2 + Cos[2 Pi q t]) Sin[2 Pi p t],
% -Sin[2 Pi q t]},
% {t, 0, 1},
% ColorFunction -> "Rainbow",
% PlotStyle -> Thickness[0.02],
% Boxed -> False,
% Axes -> False
% ]

Siedem węzłów z tabeli na końcu książki to węzły torusowe.
Są to niewęzeł, $3_1 = T_{3,2}$, $5_1 = T_{5,2}$, $7_1 = T_{7,2}$, $8_{19} = T_{4,3}$, $9_1 = T_{9,2}$ oraz $10_{124} = T_{5, 3}$.

\begin{proposition}
    Ustalmy względnie pierwsze $q, r$ takie, że $|q|, |r| \ge 2$.
    Splot $K(q, r)$ jest równoważny z~odwrotnym do niego splotem $K(-q, -r)$.
\end{proposition}

Sploty $K(q, r)$ oraz $K(r, q)$ również są równoważne.
Murasugi prezentuje w~książce \cite{murasugi96} przyjemny dowód opierający się na następującym lemacie:
Sfera $S^3$ powstaje z~powierzchni dwóch węzłów trywialnych z~wnętrzem ($D^2 \times S^1$) przez wzajemne sklejenie południka i~równoleżnika z~równoleżnikiem i~południkiem.

% {\color{red} Longitude, meridian -- ustalić słownictwo}

Macierze Seiferta $M$ mają nieskomplikowaną blokową budowę, która może posłużyć do znalezienia wielomianu Alexandera (wzorem $\alexander = \det (M - tM^t)$).
Rachunki są nieco uciążliwe.

\begin{proposition}
    Wielomianem Alexandera splotu torusowego $K(q, r) \neq K(0,0)$ o~$d$ składowych jest
    \[
        \alexander_{q, r}(t) = (-1)^{d-1} \frac{(1-t)(1 - t^{qr/d})^d}{(1-t^q)(1-t^r)} \left/t^{(q-1)(r-1)/2}\right. .
    \]
\end{proposition}

\begin{proof}
    Macierzą Seiferta węzła torusowego $K(q,r)$ jest macierz
    \[
        M = \begin{bmatrix}
            B & & & & \\
            -B & B & & & \\
            & \ddots & \ddots & & \\
            & & \ddots & B & \\
            & & & -B & B
        \end{bmatrix}
    \]
    złożona z~$(r-1)^2$ bloków o~wymiarach $(q-1) \times (q-1)$:
    \[
        B \begin{bmatrix}
            -1 & & & & \\
            1 & -1 & & & \\
            & 1 & \ddots & & \\
            & & \ddots & -1 & \\
            & & & 1 & -1
        \end{bmatrix} \qedhere
    \]
\end{proof}

Znajomość wielomianu Alexandera wystarcza na szczęście do podania pełnej klasyfikacji węzłów torusowych bez uciążliwego dowodu.

\begin{proposition}
    Węzły torusowe $K(q, r)$, $K(p, s)$ są równoważne wtedy i~tylko wtedy, gdy $\{q, r\} = \{p, s\}$ lub $\{q, r\} = \{-p, -s\}$.
\end{proposition}

\begin{proof}
    Ograniczymy się do przypadku, gdy $p, q, r, s \ge 2$.
    Tylko jedna implikacja wymaga dowodu, w~prawo.
    Bez straty ogólności załóżmy więc, że $q > r$, $p > s$.
    Skoro węzły $K(q, r)$ i~$K(p,s)$ są równoważne, to porównanie najwyższych współczynników w~ich wielomianach Alexandera daje równość $(q-1)(r-1) = (p-1)(s-1)$.
    Wymnożenie wszystkiego prowadzi do czterech przypadków: $s = r$, $s = ps$, $qr = r$, $qr = ps$, z~których dwa środkowe nie mogą zachodzić (gdyż $p, q > 1$).
    Z czwartego wynika, że $qr \le s < ps$, czyli sprzeczność.
\end{proof}

Podamy teraz wartości całkowitoliczbowych niezmienników dla węzłów torusowych przy założeniu, że $q$ lub $r$ nie jest zerem.
Nietrywialne węzły torusowe są pierwsze i~odwracalne, ale mają niezerową sygnaturę, więc nie są chiralne (wiedział to Schreier w 1924: ,,Über die Gruppen $A^a B^b = 1$'').

\begin{proposition}
    Węzeł torusowy $K(q, r)$ ma okres $|q|$ oraz $|r|$.
\end{proposition}

\begin{proposition}
    Dla $q, r > 0$ zdefiniujmy wielkość $\sigma(q, r) = - \sigma(K_{q, r})$.
    Wtedy, jeśli
    \begin{itemize}[leftmargin=*]
    \itemsep0em
        \item $2r < q$ i~$r$ jest parzyste, to $\sigma(q, r) = \sigma(q-2r, r) + r^2$.
        \item $2r < q$ i~$r$ jest nieparzyste, to $\sigma(q, r) = \sigma(q-2r, r) + r^2 - 1$.
        \item $\sigma(2r, r) = r^2 - 1$.
        \item jeśli $r \le q < 2r$ i~$r$ jest parzyste, to $\sigma(q, r) + \sigma(2r-q, r) = r^2-1$.
        \item jeśli $r \le q < 2r$ i~$r$ jest nieparzyste, to $\sigma(q, r) + \sigma(2r-q, r) = r^2-2$.
    \end{itemize}
    Co więcej, $\sigma(q, r) = \sigma(r, q)$, $\sigma(q, 1) = 0$, $\sigma(q, 2) = q-1$.
\end{proposition}

Dowód zawiera praca \cite{litherland81}.
Borodzik niedawno przyjrzał się dokładniej sygnaturom węzłów torusowych.
W pracy \cite{borodzik10} napisanej z K. Oleszkiewiczem pokazał, że nie istnieje wymierna funkcja $R(p, q)$, która pokrywałaby się z sygnaturą węzła torusowego $T_{p, q}$ dla wszystkich względnie pierwszych i nieparzystych $p$ oraz $q$.
Uwaga: definicja funkcji $s$ z \cite{borodzik10} zawiera złośliwą literówkę.

\begin{proposition}
    Niech $p, q$ będą względnie pierwszymi liczbami, zaś $C \in [0, 1)$ stałą taką, że $Cpq$ nie jest liczbą całkowitą.
    Przyjmijmy $z = \exp (2 \pi i C)$ i zdefinujmy pomocnicze funkcje: niech $\{x\} = x - \lfloor x \rfloor$ oznacza część ułamkową, zaś
    \begin{equation}
        \langle x \rangle = \begin{cases}
            0 & \text{dla } x \in \Z \\
            \{x\} - 1/2 & \text{dla } x \not \in \Z
        \end{cases}
    \end{equation}
    funkcję piłę.
    Dalej, określmy sumę Dedekinda
    \begin{equation}
        s(p, q, x) = \sum_{j = 0}^{q-1} \left\langle \frac {j}{q} \right\rangle \left\langle \frac {jp}{q} + x \right\rangle.
    \end{equation}
    Przy tych oznaczeniach, sygnatura węzła $(p, q)$-torusowego wyznacza się wzorem
    \begin{align}
        \sigma(z) & = \frac{1}{3pq} \left (p^2 + q^2 + 6 \langle Cpq \rangle^2 - \frac {1}{2} \right)  + 2(C^2 - C) pq + (2-4C) \langle Cpq \rangle + {} \\
        & - 2s(p, q, Cp) - 2s(q, p, Cq) - 2s(p, q, p-pC) - 2s(q, p, q-qC). \nonumber
    \end{align}
\end{proposition}

\begin{corollary}
    Jeśli $p, q$ są nieparzyste i względnie pierwsze, to
    \begin{equation}
        \sigma(T_{p,q}) = \frac{1}{6pq} + \frac{2p}{3q} + \frac{2q}{3p} - \frac{pq}{2} - 4(s(2p, q, 0) + s(2q, p, 0)) - 1.
    \end{equation}
\end{corollary}

\begin{corollary}
    Jeśli $p$ jest nieparzyste, zaś $q > 2$ parzyste, to
    \begin{equation}
        \sigma(T_{p,q}) = - \frac{pq}{2} + 4s(2p, q, 0) - 8s(p, q, 0) + 1.
    \end{equation}
\end{corollary}

\begin{proposition}[Murasugi, 1991]
    Mamy $cr(K_{q,r}) = \min\{|q|(|r| -1), |r|(|q|-1)\}$.
\end{proposition}

Wyznaczenie indeksu rozwiązującego było dużo trudniejsze.
Murasugi pisze w~książce \cite{murasugi96}, że mamy nierówność
\[
    u(K(q, r)) \le \frac 12 (q-1)(r-1),
\]
z równością dla względnie pierwszych $q, r > 0$.
Hipoteza Milnora głosiła, że w~rzeczywistości równość zachodzi zawsze.
Dowód został odnaleziony w~latach 1993-1995 przy użyciu tzw. \emph{gauge theory} (działu teorii pola, gdzie lagranżjan jest niezmienniczy względem grup Liego lokalnych transformacji...).
Patrz prace \cite{kronheimer93} oraz \cite{kronheimer95}.

\begin{proposition}[Kronheimer, Mrówka] \label{torus_unknotting}
    Dla względnie pierwszych $q, r > 0$ mamy
    \[
        u(K(q, r)) = \frac 12 (q-1)(r-1),
    \]
\end{proposition}

Genus pokrywa się z~liczbą gordyjską dla węzłów (czyli względnie pierwszych $q, r$).

\begin{proposition} \label{torus_bridge}
    Indeksem mostowym węzła $K_{q,r}$ jest mniejsza z~liczb $|q|, |r|$.
\end{proposition}

\begin{corollary}
    \label{braid-for-forus}
    Indeksem warkoczowym węzła torusowego $T_{p, q}$, $p, q \neq 0$, jest $\min\{|p|, |q|\}$.
\end{corollary}

\begin{proof}
    Niech $K$ będzie węzłem torusowym typu $(p,q)$ z~minimalnym przedstawieniem jako warkocz $\beta$.
    Z konstrukcji domknięcia (czyli dołączenia rozłącznych półokręgów) wynika,
    że diagram $K$ ma dokładnie $b(K)$ lokalnych maksimów.
    Definicja indeksu mostowego orzeka, iż $br(K) \le b(K)$.
    Bez straty ogólności niech $p > q > 0$.
    Skoro węzeł $K$ powstaje z~$q$-warkocza $(\sigma_{q-1} \ldots \sigma_2\sigma_1)^p$,
    indeks $b(K)$ nie przekracza $q = br(K)$.
\end{proof}

\begin{proposition}
    Mamy $\bracket{K_{2, n}} = A \bracket{K_{2,n-1}} + (-1)^{n-1} A^{2-3n}$
    oraz $\bracket{K_{2,1}} = -A^3$.
\end{proposition}

\begin{proposition}
    Wielomianem Jonesa węzła $(m, n)$-torusowego jest
    \[
        V(t) = \frac {t^{(m-1)(n-1):2}}{1-t^2} \cdot (1 - t^{m+1} - t^{n+1} + t^{m+n}).
    \]
\end{proposition}

Murasugi i~Neuwirth w~1961 dla węzłów alternujących,
zaś Burde z~Zieschangiem w~1965 pokazali, że nietrywialny węzeł,
którego grupa ma nietrywialne centrum, jest torusowy.

\begin{proposition}
    Wielomianem Alexandera węzła $(p,q)$-torusowego jest
    \[
         \alexander(T_{p,q}) = \frac{(t^{pq}-1)(t-1)}{(t^p-1)(t^q-1)}.
    \]
\end{proposition}

\begin{proof}
    Przypadek $p = 2$ wymaga prostego rozumowania indukcyjnego.
    Samo ćwiczenie pojawia się w~wielu podręcznikach topologii.
    Pełny dowód można znaleźć w~przykładzie 9.15 książki ,,Knots'' Burdego oraz Zieschanga.
    Inne podejście, tak zwaną formułę Seiferta-Torresa, prezentuje przeglądowa praca Turaewa ,,Reidemeister torsion in knot theory'', 119-182.
\end{proof}

\begin{corollary}
    Wielomian Alexandera odróżnia od siebie węzły $(2,n)$-torusowe.
\end{corollary}

\begin{proof}
    Mamy $\alexander(T_{2,n}) = (t^n+1) / (t+1)$, więc $\deg \alexander (T_{2,n}) = n - 1$.
\end{proof}

% Koniec sekcji Węzły torusowe

\section{Węzły satelitarne} % (fold)
\label{sec:satellite}
Wyobraźmy sobie węzeł $K'$ leżący wewnątrz trywialnego torusa $V$ tak, by nie zawierał się w żadnej mieszczącej się w tym torusie 3-kuli.
Zawiążmy następnie torus $V$: ustalmy włożenie $f \colon V \to S^3$.
Splot $K = f(K')$ orbituje wokół swojego kompana, tj. obrazu rdzenia torusa przez włożenie $f$ i nie opuszcza jego małego rurowego otoczenia.
Dość nieprzypadkowo splot $K$ nazywamy satelitarnym, zaś obraz rdzenia -- węzłem towarzyszącym.
Konstrukcja satelity jest, w porównaniu z sumą spójną, dość zawiła.
Pożądanym byłoby mieć do dyspozycji więcej operacji rozkładających węzły na prostsze, i badać nierozkładalne obiekty.
Ale ich nie ma.

Torus $f(\partial V)$ nie jest równoległy do brzegu ani ściśliwy.
Odwrotnie, jeśli w dopełnieniu węzła mamy torus, którego południk lub równoleżnik ogranicza dysk, to węzeł nie biegnie wzdłuż torusa lub ten jest niezawęźlony.
Przypadek, gdy torus jest otoczeniem rurowym węzła, też nie jest ciekawy.
To motywuje następującą definicję.

\begin{definition}
	Węzeł nazywamy satelitarnym, jeśli zawiera nieściśliwy, nierównoległy do brzegu torus we własnym dopełnieniu. \index{Węzeł!satelitarny}
\end{definition}

Klasa węzłów satelitarnych obejmuje węzły złożone.
W ich przypadku można wskazać pewien szczególny torus nieściśliwy -- połykający pierwszy składnik, a potem podążający za drugim.
Świetnie przedstawione jest to na stronie 82 książki \cite{cromwell04} Cromwella.
Niektóre węzły przedstawiają się jako satelity w dokładnie jeden sposób, inne nie są jednoznaczne.
W \cite{jaco79} ulepszono definicję satelitarności do tzw. \emph{splicing}u i opisano jednoznaczny rozkład Jaco-Shalena-Johannsona, czego prawdziwość przypuszczał wcześniej Waldhausen.
Żaden węzeł torusowy ani trywialny nie jest satelitą, ale są nimi kable oraz duble Whiteheada.

\begin{definition}
	Jeżeli $K' \subseteq V$ jest skręconym jednokrotnie niewęzłem, to węzeł $K$ nazywamy dublem Whiteheada.
\end{definition}

Każdy węzeł posiada nieskończenie wiele dubli Whiteheada: wystarczy rozciąć torus $V$, skręcić jedną końcówkę i ponownie zszyć, żaden z nich nie jest odróżniany od niewęzła przez wielomian Alexandera.

\begin{definition}
	Jeżeli $K' \subseteq \partial V$ jest węzłem torusowym,	to $K$ nazywamy węzłem kablowym.
\end{definition}

Satelita, którego indeks zawijający przekracza dwa, ma co najmniej 27 skrzyżowań.
Jeśli jego kompan to ósemka -- co najmniej siedemnaście.
Oprócz tego istnieją cztery satelity o ponad dwunastu skrzyżowaniach.
Podejrzewa się, że satelita, który wykonuje $m$ pełnych obrotów wokół kompana o indeksie skrzyżowaniowym $k$, nie posiada diagramu o mniej niż $km^2$ skrzyżowaniach.

\begin{proposition}
	Każdy kabel wyznacza jednoznacznie węzeł, z którego powstał.
\end{proposition}

\begin{proof}
	Wniosek 2 z pracy \cite{feustel78} Feustela, Whittena pokazuje, że na podstawie kabla można wyznaczyć parametry węzła torusowego $K'_{p,q}$ oraz topologię dopełnienia oryginalnego węzła.
	Wiemy jednak z twierdzenia Gordona-Lueckego, że różne węzły mają różne dopełnienia.
\end{proof}

Schubert pokazał, że zorientowane klasy izotopii węzłów w $S^3$ tworzą wolny przemienny monoid na przeliczalnie wielu generatorach.
Krótko po tym odkrył, że może podać nowy dowód tego twierdzenia przez uważną analizę nieściśliwych torusów obecnych w dopełnieniu sumy spójnej.
To doprowadziło go do definicji węzłów satelitarnych i towarzyszących w przełomowej pracy \cite{schubert53} oraz zunifikowało teorię 3-rozmaitości oraz węzłów.
Patrz też \cite{motegi97}.

% Koniec sekcji Węzły satelitarne

\section{Węzły hiperboliczne} % (fold)
\label{sec:hyperbolic}
\begin{definition}
    Węzeł nazywamy hiperbolicznym, jeżeli na jego dopełnieniu można zadać metrykę o stałej krzywiźnie $-1$. \index{Węzeł!hiperboliczny}
\end{definition}

\begin{theorem}[Thurston, 1978]
    Jeżeli węzeł nie jest hiperboliczny, to należy do jednej z dwóch nieskończonych rodzin: torusowych oraz satelitarnych.\index{Twierdzenie!Thurstona}
\end{theorem}

Stwierdzenie to nazywa się czasem trychotomią Thurstona, gdyż dzieli węzły na trzy rodzaje.

\begin{proposition}
    Żaden węzeł nie ma mniejszej objętości hiperbolicznej od ósemki.
\end{proposition}

Dowód tego faktu podał Cao z Meyerhoffem w 2001 roku.
Opierali się oni na działaniu komputerowego programu, który wyeliminował inne możliwości.

\begin{proposition}
    Objętość hiperboliczna nie odróżnia hiperbolicznych mutantów.
\end{proposition}

Nie jestem w stanie podać odnośnika do dowodu w literaturze -- przypominam, że chodzi o mutanty z definicji \ref{mutants}.
Stwierdzenie to można znaleźć na przykład w książce Adamsa (strona 124).

\begin{proposition}
    Grupa symetrii węzła hiperbolicznego jest skończona: cykliczna lub diedralna.
\end{proposition}

\begin{proof}
    Praca \cite{kodama92}.
\end{proof}

Objętość hiperboliczna bardzo dobrze odróżnia od siebie węzły.
Pewien węzeł o dwunastu skrzyżowaniach i $5_2$ mają jednak tę samą objętość.

\begin{proposition}
    Każdy węzeł hiperboliczny jest pierwszy. 
\end{proposition}

Spośród wszystkich węzłów pierwszych o mniej niż 17 skrzyżowaniach, prawie wszystkie są hiperboliczne: 12 z nich to węzły torusowe, 20 to satelity trójlistnika (te ostatnie mają ponad 10 skrzyżowań).
Jak pokazuje poniższa tabela (oparta na bazie ciągów OEIS, numery 51764, 51765, 52408), węzły pierwsze nie przypominają licznością liczb pierwszych.

\renewcommand*{\arraystretch}{1.4}
\footnotesize
\begin{longtable}{lcccccccccccccc}
\hline
    \textbf{rodzaj} & 3 & 4 & 5 & 6 & 7 & 8  & 9  & 10  & 11  & 12   & 13   & 14    & 15     \\ \hline \endhead
    torusowe        & 1 & 0 & 1 & 0 & 1 & 1  & 1  & 1   & 1   & 0    & 1    & 1     & 2      \\
    satelitarne     & 0 & 0 & 0 & 0 & 0 & 0  & 0  & 0   & 0   & 0    & 2    & 2     & 6      \\
    hiperboliczne   & 0 & 1 & 1 & 3 & 6 & 20 & 48 & 164 & 551 & 2176 & 9985 & 46969 & 253285 \\
    \hline
\end{longtable}
\normalsize

% Every non-split, prime, alternating link that is not a torus link is hyperbolic by a result of William Menasco.

Z twierdzenia o sztywności (\emph{rigidity theorem}) Mostowa i Prasada (1973), jeśli na dopełnieniu węzła można zadać strukturę hiperbolkiczną, to w tylko jeden sposób.
Co więcej, Mostow pokazał, że jeśli istnieje izomorfizm grup podstawowych domkniętych, hiperboliczynch 3-rozmaitości, to są one izometryczne.
Wiedzę o węzłach hiperbolicznych można czerpać z prac: \cite{weeks05} (poprawiona wersja dostępna w serwisie ArXiv), "Hyperbolic Knot Theory" J. Purcell.

Kryterium Thurstona: splot $L$ z dopełnieniem $X$ i grupą podstawową $\pi$ który spełnia poniższe warunki (a każdy warunek zakłada poprzednie): $L$ nie rozszczepia się ($X$ nie zawiera właściwej 2-sfery, $\pi$ nie jest wolnym produktem), $L$ nie jest niewęzłem ($X$ nie zawiera właściwego dysku, $\pi$ nie jest cykliczna), żadna składowa $L$ nie jest niezakłóconym węzłem satelitarnym (?) ($X$ nie zawiera właściwego torusa), $L$ nie jest węzłem torusowym ($X$ nie zawiera właściwego pierścienia, $\pi$ nie zawiera kopii $\Z^2$) daje zadać na swoim dopełnieniu strukturę hiperboliczną.
% https://arxiv.org/abs/math/0309466
% https://arxiv.org/abs/math/0311380
% Koniec sekcji Węzły hiperboliczne
\section{Węzły plastrowe i taśmowe} % (fold)
\label{sec:slice}
Węzły plastrowe i taśmowe oraz pojęcie kobordyzmu, które wkrótce opiszemy, należą do świata 4-wymiarowej teorii węzłów.
Nie zapoznamy się z nią bliżej oraz nie podamy naszego ulubionego odniesienia do tego tematu w~literaturze, ponieważ sami nie rozumiemy go zbyt dobrze.
Wszystko zaczęło się od artykułu \cite{fox66} Foxa, Milnora.

\begin{definition}[płaski]
    Niech $D \subseteq B^4$ będzie dyskiem posiadającym otoczenie $N$, kopię zbioru $D \times I
    ^2$, która przecina sferę $S^3$ dokładnie w $\partial D \times I^2$.
    Mówimy wtedy, że dysk $D$ jest płaski.
\end{definition}

\begin{definition}[węzeł plastrowy]
    \index{węzeł!plastrowy}
    Niech $K \subseteq S^3$ będzie takim węzłem, że w kuli $B^4$ istnieje płaski dysk $D$ taki, że $K = \partial D = D \cap S^3$.
\end{definition}

Następujące węzły o~mniej niż jedenastu skrzyżowaniach są plastrowe: $6_1$, $8_{20}$, $8_{8}$, $8_{9}$, $9_{27}$, $9_{41}$, $9_{46}$, $10_{22}$, $10_{35}$, $10_{3}$, $10_{42}$, $10_{48}$, $10_{75}$, $10_{87}$, $10_{99}$, $10_{123}$, $10_{129}$, $10_{137}$, $10_{140}$, $10_{153}$, $10_{155}$, $10_{155}$.
Wśród pierwszych węzłów do dwunastu skrzyżowań najdłużej opierał się węzeł Conwaya, aż Piccirillo pokazała w~\cite{piccirillo20}, że nie jest plastrowy.

\begin{proposition}
    Niech $K$ będzie węzłem.
    Wtedy $K \shrap mr K$ jest węzłem plastrowym.
\end{proposition}

\begin{proof}[Niedowód]
    Pierwszy był Fox z Milnorem \cite{fox66}, patrz także lemat 12.1.2.2 w \cite{kawauchi96}.
\end{proof}

Istnieje konkurencyjna definicja węzłów plastrowych.
Dwa sploty $K, L \subseteq S^n$ nazywamy zgodnymi (z angielskiego \emph{concordant}), jeśli istnieje włożenie $f \colon K \times [0,1] \to S^n \times [0,1]$ spełniające dwa warunki: $f(K \times 0) = K \times 0$ oraz $f(K \times 1) = L \times 1$.

\begin{definition}
    Węzeł zgodny z~niewęzłem nazywamy plastrowym.
\end{definition}

Zgodność jest relacją równoważności, słabszą od izotopii, ale mocniejszą od homotopii.
W zbiorze jej klas abstrakcji, oznaczanym przez $C^1$, można zadać strukturę grupy abelowej.
%izomorficznej z~$\Z^\infty \oplus (\Z/2)^\infty \oplus (\Z/4)^\infty$.
\index{grupa!zgodności}
Działanie dane jest wzorem $[K] + [L] = [K \shrap L]$; niewęzeł stanowi element neutralny.
Elementem odwrotnym do $[K]$ jest $[mrK]$.

% TODO: \textbf{Livingston: A SURVEY OF CLASSICAL KNOT CONCORDANCE}. The application of abelian knot invariants (those determined by the cohomology of abelian covers or, equivalently, by the Seifert form) to concordance culminated in 1969 with Levine’s classification of higher dimensional knot concordance, [62, 63], which applied in the classical dimension to give a surjective homomorphism $\varphi \colon C \to \Z^\infty \oplus \Z_2^\infty \oplus \Z_4^\infty$. In 1975 Casson and Gordon [8, 9] proved that Levine’s homomorphism is not an isomorphism, constructing nontrivial elements in the kernel, and Jiang expanded on this to show that the kernel contains a subgroup isomorphic to $\Z_2^\infty$. More significant, Freedman proved that all knots with trivial Alexander polynomial are in fact slice in the topological locally flat category.

\begin{proposition}
    Albo wszystkie trzy węzły $K, L, K \shrap L$ są plastrowe, albo co najwyżej jeden z~nich.
\end{proposition}

\begin{proof}[Niedowód]
    Lemat 12.1.2.3 w \cite{kawauchi96}.
\end{proof}

Pierwszym poważnym wynikiem z dziedziny teorii węzłów plastrowych, pochodzącym jeszcze z pracy \cite{fox66}, był:

\begin{proposition}[warunek Foxa-Milnora]
    \index{warunek!Foxa-Milnora}
    Niech $K$ będzie węzłem plastrowym.
    Wtedy jego wielomian Alexandera jest postaci $\alexander(t) = f(t) f(1/t)$ dla pewnego wielomianu Laurenta $f \in \Z[t, 1/t]$.
\end{proposition}

\begin{corollary}
    Wyznacznik węzła plastrowego jest kwadratem.
\end{corollary}

\begin{proof}
    Mamy $\det K = |\alexander(-1)| = f(-1) f(-1)$.
\end{proof}

Ten prosty test stwierdza, że 2743 spośród 2977 węzłów o mniej niż 13 skrzyżowaniach nie jest plastrowych.

\begin{proposition}
    Niech $K$ będzie węzłem plastrowym.
    Wtedy $\operatorname{Arf} K = 0$.
\end{proposition}

\begin{proof}
    Ustalmy węzeł $K$, wiemy już, że jego wyznacznik jest kwadratem, a na mocy faktu \ref{cor:knot_determinant_odd} także tyle, że jest liczbą nieparzystą.
    Wynika stąd przystawanie $\det K \equiv 1 \mod 8$, które w~połączeniu z warunkiem Murasugiego (fakt \ref{prp:arf_murasugi}) daje $\operatorname{Arf} K = 0$.
\end{proof}

\begin{proposition}
    \label{prp:slice_signature}
    Niech $K$ będzie węzłem plastrowym.
    Wtedy $\sigma(K) = 0$.
\end{proposition}

\begin{proof}[Szkic dowodu]
    Ustalmy odwzorowanie $f$, które jest niesingularne, symetryczne i~dwuliniowe, z~przestrzeni $V$ o~wymiarze $2n$ oraz wyznaczoną przez nie formę kwadratową.
    Jeśli znika ona na podprzestrzeni wymiaru $n$, to ma zerową sygnaturę.
    % TODO: \textbf{(dowód znaleziony w~podręczniku Lickorisha). Patrz też twierdzenie 8.8 z~artykułu \cite{murasugi65}. Praca "Infinite Order Amphicheiral Knots". (Charles Livingston, 2001) -- chyba nie?}
\end{proof}

Test ten eliminuje kolejne 45 węzłów poniżej 13 skrzyżowań.

\begin{definition}
    \index{węzeł!taśmowy}
    Węzeł $K = f(S^1)$ będący brzegiem singularnego dysku $f \colon D \to S^3$ posiadającego następującą własność: każda przecinająca siebie składowa jest łukiem $A \subseteq f(D^2)$, dla którego $f^{-1}(A)$ składa się z~dwóch łuków w~$D^2$ (jeden z~nich jest wewnętrzny), nazywamy taśmowym.
\end{definition}

Jak pisze Kawauchi, mamy oczywiste wynikanie:

\begin{proposition}
    Każdy węzeł taśmowy jest plastrowy.
\end{proposition}

Dawno temu Fox zapytał, czy implikacja odwrotna także jest prawdziwa:

\begin{conjecture}[slice-ribbon problem]
    Czy każdy węzeł plastrowy jest taśmowy?
\end{conjecture}

Nie wiemy do dzisiaj.
Lisca pokazał prawdziwość hipotezy dla węzłów 2-mostowych \cite{lisca07}, Greene oraz Jabuka zrobili to dla precli o trzech pasmach w \cite{greene11}.
P. Teichner myśli o niej jako o~życzeniu, które uprościłoby pewne 4-wymiarowe problemy, gdyby było prawdziwe, ale Gompf, Scharlemann i Thompson zasugerowali w~\cite{gompf10} potencjalny kontrprzykład.


\begin{proposition}
    Każda macierz Seiferta $V$ (całkowitoliczbowa, kwadratowa, taka że $\det (V - V^t) = 1$), która jest unimodularnie sprzężona: istnieje całkowitoliczbowa macierz $P$ o~wyznaczniku równym $\pm 1$, że
    \begin{equation}
        V = P \begin{pmatrix} 0 & V_{21} \\ V_{12} & V_{22} \end{pmatrix} P^{-1}
    \end{equation}
    stanowi macierz Seiferta pewnego węzła plastrowego.
\end{proposition}

Takie węzły nazywamy plastrowymi algebraicznie.
Węzeł $K$ w~$S^3$ jest algebraicznie plastrowy dokładnie wtedy, gdy ogranicza izotropiczną powierzchnię w~kuli $B^4$.
Więcej informacji w~podręczniku Kawauchiego.

% Theorem 1.3[Long 1984].A strongly positive amphicheiral knot is algebraicallyslice.
% Theorem 1.4[Hartley and Kawauchi 1979].If K is strongly positive amphicheiral,the Alexander polynomial1Kis the square of a symmetric polynomial.

\subsection{Węzły skręcone}
\begin{definition}
    \index{węzeł!skręcony}
    \label{def:twist_knot}
    Węzeł powstały przez $n$-krotne półskręcanie domkniętej pętli oraz splecienie końców nazywamy węzłem skręconym.
\end{definition}

Węzły skręcone to dokładnie towarzyszące niewęzłowi w~węzłach satelitarnych, tak zwane whiteheadowskie duble niewęzła.
Wszystkie są odwracalne (ale tylko niewęzeł oraz ósemka są amfichiralne) i~mają liczbę gordyjską $1$, ponieważ wystarczy rozwiązać skrzyżowanie, które plotło końce.
Każdy jest $2$-mostowy i~posiada zerową sygnaturę.
Dalsze własności węzłów skręconych zależą od $n$, ilości półskrętów.
Indeks skrzyżowaniowy wynosi $n + 2$.

\begin{proposition}
    Wielomianowymi niezmiennikami węzłów skręconych są:
    \begin{align*}
    (q+1)\jones(q) & = \begin{cases}
        1+q^{-2}+q^{-n}-q^{-n-3} & n \mbox{ nieparzyste} \\
        q^{3}+q-q^{3-n}+q^{-n} & n \mbox{ parzyste}
    \end{cases} \\
    2 \conway (z) & = \begin{cases}
        (n+1) z^{2} + 2 & n \mbox{ nieparzyste} \\
        2 - nz^2 & n \mbox{ parzyste}
    \end{cases}
    \end{align*}
\end{proposition}

\begin{proposition}
    Niewęzeł oraz węzeł dokerski $6_1$ są jedynymi skręconymi węzłami plastrowymi.
\end{proposition}

\begin{proof}
    \cite{casson86}.
\end{proof}

% Koniec sekcji Węzły plastrowe

\section{Warkocze} % (fold)
\label{sec:braid}
Podam teraz opis grupy warkoczy, rozważanej po raz pierwszy niejawnie przez A. Hurwitza w~1885 roku i~jawnie przez E. Artina czterdzieści lat później.
O dwóch punktach $(d_1, t_1)$, $(d_2, t_2)$ w~$B^2 \times [0, 1] \subseteq \R^3$ powiemy, że łączący je odcinek jest malejący, jeśli $t_1 > t_2$.
Łamana malejąca to taka, która jest złożona z~malejących odcinków.

\begin{definition}[warkocz]
    \label{braid_def}
    \index{warkocz}
    Teoriomnogościową sumę parami rozłącznych łamanych malejących, które łączą zbiory $\{x_1, \ldots, x_n\} \times \{1\}$ oraz $\{x_1, \ldots, x_n\} \times \{0\}$, nazywamy warkoczem o~$n$ pasmach.
\end{definition}

Poszczególne pasma warkocza możemy utożsamiać z~wykresami pewnych (gładkich) funkcji $f_i \colon [0, 1] \to \R^2$, jeśli zbiory $\{f_i(0) : 1 \le i \le n\} = \{f_i(1) : 1 \le i \le n\}$ są równe.
Wtedy dwa warkocze uznajemy za równoważne, jeśli istnieje między nimi izotopia: funkcje ciągłe dwóch zmiennych $F_i(t, s)$ określone na zbiorze $[0,1] \times [0,1]$ takie, że $F_i(t,0)= f_i(t)$ oraz $F_i(t, 1) = g_i(t)$.
Przez analogię do węzłów można zdefiniować diagramy warkoczy jako cienie bez katastrof.
Najczęściej rzutujemy prostopadle do odcinka $\{0\} \times [0, 1]$.

\begin{definition}
    \index{grupa!warkoczy}
    Określmy pomocniczo dwie kontrakcje $B^2 \times [0,1] \to B^2 \times [0,1]$:
    \begin{align*}
        \psi_1(d, t)&  = (d, t/2) \\
        \psi_2(d, t)&  = (d, \frac12 (t+1))
    \end{align*}
    Klasy abstrakcji warkoczy z~mnożeniem danym wzorem $z_1z_2 = \psi_1(z_1) \cup \psi_2(z_2)$ tworzą grupę warkoczy $B_n$.
    Jej elementem neutralnym jest warkocz $1_n = \bigcup_{i = 1}^n \{x_1\} \times [0,1]$.
\end{definition}

Sprawdzenie aksjomatów grupy pozostawiamy Czytelnikowi,
pozostawiając mu małą wskazówkę graficzną:
\[
    \begin{tikzpicture}[baseline=-0.65ex, scale=0.2]
    \begin{knot}[clip width=5, end tolerance=1pt]
        \strand[semithick] (-6, 0) .. controls (-4, 0) and (-5, 2) .. (-3, 2);
        \strand[semithick] (-6, 2) .. controls (-4, 2) and (-5, 0) .. (-3, 0);
        \strand[semithick] (-6, -2) to (-3, -2);
        \strand[semithick] (-3, 0) .. controls (-1, 0) and (-2, -2) .. (0, -2);
        \strand[semithick] (-3, -2) .. controls (-1, -2) and (-2, 0) .. (0, 0);
        \strand[semithick] (-3, 2) to (0, 2);
        \strand[semithick] (+6, 0) .. controls (+4, 0) and (+5, 2) .. (+3, 2);
        \strand[semithick] (+6, 2) .. controls (+4, 2) and (+5, 0) .. (+3, 0);
        \strand[semithick] (+6, -2) to (+3, -2);
        \strand[semithick] (+3, 0) .. controls (+1, 0) and (+2, -2) .. (0, -2);
        \strand[semithick] (+3, -2) .. controls (+1, -2) and (+2, 0) .. (0, 0);
        \strand[semithick] (+3, 2) to (0, 2);
        \draw (+6, -3) rectangle (0, 3);
        \draw (-6, -3) rectangle (0, 3);
        \draw[semithick, decoration={brace,mirror,raise=3pt},decorate]  (-5.75, -3) -- node[below=6pt] {$\beta$} (-0.25, -3);
        \draw[semithick, decoration={brace,mirror,raise=3pt},decorate]  (0.25, -3) -- node[below=6pt] {$\beta^{-1}$} (5.75, -3);
    \end{knot}
    \end{tikzpicture}
    \cong
    \begin{tikzpicture}[baseline=-0.65ex, scale=0.2]
        \draw[semithick] (-3, -2) to (3, -2);
        \draw[semithick] (-3, 0) to (3, 0);
        \draw[semithick] (-3, 2) to (3, 2);
        \draw (-3, -3) rectangle (3, 3);
        \draw[semithick, decoration={brace,mirror,raise=3pt},decorate]  (-2.75, -3) -- node[below=6pt] {$1_3$} (2.75, -3);
    \end{tikzpicture}
    \quad\quad\quad
    \begin{tikzpicture}[baseline=-0.65ex, scale=0.2]
        \useasboundingbox (-6, -3) rectangle (12, 5);
\begin{knot}[clip width=5, end tolerance=1pt]
        \strand[semithick] (-6, 0) .. controls (-4, 0) and (-5, 2) .. (-3, 2);
        \strand[semithick] (-6, 2) .. controls (-4, 2) and (-5, 0) .. (-3, 0);
        \strand[semithick] (-6, -2) to (-3, -2);
        \strand[semithick] (-3, 0) .. controls (-1, 0) and (-2, -2) .. (0, -2);
        \strand[semithick] (-3, -2) .. controls (-1, -2) and (-2, 0) .. (0, 0);
        \strand[semithick] (-3, 2) to (0, 2);
        \draw (-6, -3) rectangle (0, 3);
        \draw[semithick, decoration={brace,mirror,raise=3pt},decorate]  (-5.75, -3) -- node[below=6pt] {$\beta_1$} (-0.25, -3);
        \strand[semithick] (+6, 0) .. controls (+4, 0) and (+5, 2) .. (+3, 2);
        \strand[semithick] (+6, 2) .. controls (+4, 2) and (+5, 0) .. (+3, 0);
        \strand[semithick] (+6, -2) to (+3, -2);
        \strand[semithick] (+3, 0) .. controls (+1, 0) and (+2, -2) .. (0, -2);
        \strand[semithick] (+3, -2) .. controls (+1, -2) and (+2, 0) .. (0, 0);
        \strand[semithick] (+3, 2) to (0, 2);
        \draw (+6, -3) rectangle (0, 3);
        \strand[semithick] (6+6, 0) .. controls (6+4, 0) and (6+5, 2) .. (6+3, 2);
        \strand[semithick] (6+6, 2) .. controls (6+4, 2) and (6+5, 0) .. (6+3, 0);
        \strand[semithick] (6+6, -2) to (6+3, -2);
        \strand[semithick] (6+3, 0) .. controls (6+1, 0) and (6+2, -2) .. (6+0, -2);
        \strand[semithick] (6+3, -2) .. controls (6+1, -2) and (6+2, 0) .. (6+0, 0);
        \strand[semithick] (6+3, 2) to (6+0, 2);
        \draw (6+6, -3) rectangle (6+0, 3);
        \draw[semithick, decoration={brace,mirror,raise=3pt},decorate]  (0.25, -3) -- node[below=6pt] {$\beta_2\beta_3$} (11.75, -3);
        \draw[semithick, decoration={brace,raise=3pt},decorate]  (6.25, 3) -- node[above=6pt] {$\beta_3$} (11.75, 3);
        \draw[semithick, decoration={brace,raise=3pt},decorate]  (-5.75, 3) -- node[above=6pt] {$\beta_1\beta_2$} (5.75, 3);
    \end{knot}
    \end{tikzpicture}
\]

\begin{proposition}
    Grupa warkoczy jest izomorficzna z~grupę prezentowaną przez generatory $\sigma_1, \ldots, \sigma_{n-1}$ oraz relacje:
    $\sigma_i \sigma_j = \sigma_j \sigma_i$ dla $|i - j| \neq 1$,
    $\sigma_i\sigma_{i+1} \sigma_i = \sigma_{i+1} \sigma_i \sigma_{i+1}$ dla $1 \le i \le n-2$.
\end{proposition}

Generatory $\sigma_i$ posiadają prostą interpretację graficzną:
\[
    \begin{tikzpicture}[baseline=-0.65ex, scale=0.05]
    \useasboundingbox (-15, -10) rectangle (15, 15);
    \begin{knot}[clip width=5, end tolerance=1pt]
        \strand[semithick] (-15, -10) to (-15, 10);
        \strand[semithick] ( 15, -10) to ( 15, 10);
        \strand[semithick] (-5, -10) to (-5, -5) .. controls (-5, 1) and (5, -1) .. (5, 5) to (5, 10);
        \strand[semithick] (-5, 10) to (-5, 5) .. controls (-5, -1) and (5, 1) .. (5, -5) to (5, -10);
        \node  at (-10, 0) {\ldots};
        \node at ( 10, 0) {\ldots};
        \node [above] at (-15, 12) {$1$};
        \node [above] at ( -5, 12) {$i$};
        \node [above] at (  5, 12) {$i+1$};
        \node [above] at ( 15, 12) {$n$};
    \end{knot}
    \end{tikzpicture}
\]

\begin{proposition}
    Jeśli $n \ge 3$, to centrum grupy $B_n$ jest generowane
    przez warkocz $(\prod_{i = 1}^{n-1} \sigma_i)^n$.
\end{proposition}

Grupa $B_1$ jest trywialna, $B_2$ cykliczna, zaś $B_3$ to grupa podstawowa trójlistnika.
Nie istnieje węzeł, którego grupą podstawową byłaby jednak $B_n$ dla $n \ge 4$: tam elementy $\sigma_1$, $\sigma_n$ oraz generator centrum rozpinają grupę izomorficzną z~$\Z^3$.
Natomiast asferyczna, niezwarta 3-rozmaitość nie może mieć grupy podstawowej $\Z^3$.
Musimy pominąć czysto kohomologiczny dowód faktu, ale zaiste prowadzi to do sprzeczności.

Każdy warkocz można domknąć do węzła, łącząc punkty $(x_i, 1)$ oraz $(x_i, 0)$
łamanymi, których rzuty do płaszczyzny diagramu nie przecinają się.
Jeden węzeł może być przy tym domknięciem różnych warkoczy.
Żaden węzeł nie jest pomijany.
Co ciekawe, domknięcia warkoczy były rozważane przed samymi warkoczami!

\begin{theorem}[Alexander, 1923] \label{alex_thm}
     Każdy splot powstaje przez domknięcie pewnego warkocza.
     \index{twierdzenie!Alexandera}
\end{theorem}

Niech $b \in B_n$ będzie słowem zapisanym na standardowych generatorach.
Oznaczmy przez $b_+$, $b_-$ nieznakowaną sumę dodatnich, ujemnych wykładników.
Jeśli $b_+ - 3b_- \ge n$, to domknięcie warkocza $b$ nie jest achiralne (twierdzenie 5 z~\cite{jones85}).

\begin{theorem}[Markow, 1936]
    % Każdy splot jest domknięciem pewnego warkocza. -- Alexander
    Dwa domknięte warkocze są równoważne jako sploty wtedy i~tylko wtedy,
    gdy jeden powstaje z~drugiego przez ciąg
    sprzężeń: $z_1 \mapsto z_2 z_1 z_2^{-1}$ oraz procesów Markowa,
    które zastępują $n$-warkocz $\beta$ przez $(n+1)$-warkocz $\beta\sigma_n^{\pm 1}$.
\end{theorem}

\begin{proof}
    Kompletny i~godny naśladowania dowód znajduje się w~książce \cite{birman74} Birman.
    Jest ona trudno dostępna, więc warto sprawdzić też \cite{birman02}, artykuł napisany przez Birman i~Menasco.
\end{proof}

Problem słowa (czy dane słowo przedstawia element neutralny?) oraz sprzężoności (czy dwa słowa są sprzeżone?) są rozwiązalne w~grupach warkoczowych.
Problem, czy dwa słowa prezentują równoważne węzły -- nie.

\index{reprezentacja Burau}
Na zakończenie sekcji wspomnijmy o~macierzowej reprezentacji Burau.
Wyznaczona jest ona przez obrazy generatorów:
\[
    \varphi(\sigma_i) = I_{i-1} \oplus \begin{pmatrix}
        1-t & t \\
        1   & 0
    \end{pmatrix} \oplus I_{n-i-1}
\]
Reprezentacja $\varphi$ jest wierna dla $n = 2, 3$ i~niewierna dla $n \ge 5$.
Czy reprezentacja Burau dla $B_4$ jest wierna?
Negatywna odpowiedź na to pytanie prawie na pewno prowadziłaby do
nietrywialnego węzła, którego wielomianem HOMFLY jest $1$,
natomiast odpowiedź pozytywna raczej nie ma aż tak dramatycznych następstw.
Bigelow w~1999 roku znalazł nietrywialne elementy jądra zadane komutatorem $[\psi_1^{{-1}}\sigma_4\psi_1,\psi_2^{{-1}}\sigma_4\sigma_3\sigma_2\sigma_1^2\sigma_2\sigma_3\sigma_4\psi_2]$, gdzie
    \begin{align*}
        \psi_1 & = \sigma_3^{{-1}}\sigma_2\sigma_1^2\sigma_2\sigma_4^3\sigma_3\sigma_2, \\
\psi_2 & = \sigma_4^{{-1}}\sigma_3\sigma_2\sigma_1^{{-2}}\sigma_2\sigma_1^2\sigma_2^2\sigma_1\sigma_4^5.
    \end{align*}

Grupy $B_n$ mogą być obiektem badań algebry bez związku z~teorią węzłów.

\begin{proposition}
    Grupa warkoczy $B_n$ jest beztorsyjna dla każdego $n \ge 1$.
\end{proposition}

Istnieje wiele dowodów tego faktu: pierwszy korzystał z~krótkich ciągów dokładnych (Fadell, Neuwirth 1965), później podano oparty o~struktury Garside'a (Garside 1969), czysto teoriogrupowy pochodzi od Dyera (1980).
My przedstawimy inne rozumowanie, opisując przy tym ciekawy sam w sobie porządek Dehornoya.

\begin{proof}
    Mówimy, że grupa $G$ jest lewo-porządkowalna, jeśli można wyposażyć ją w~zupełny porządek $<$, niezmienniczy na mnożenie z lewej strony.
    To znaczy, dla każdych $a, b, c \in G$, z~nierówności $a < b$ wynika $ca < cb$.
    Wtedy zbiór $P = \{g \in G \mid e < g\}$ nazywamy półgrupą elementów dodatnich.
    Łatwo widać, że $G$ jest sumą rozłączną $P \sqcup \{e\} \sqcup P^{-1}$.
    Odwrotnie, każde takie rozbicie wyznacza porządek: wystarczy zdefiniować $a < b \iff a^{-1}b \in P$.

    Dehornoy znalazł taki porządek dla grupy warkoczowej $B_n$ w~\cite{dehornoy94}.
    Za zbiór $P$ elementów dodatnich wziął te słowa na standardowych generatorach, które dla pewnego $i$ zawierają $\sigma_i$, ale nie $\sigma_i^{-1}$ ani $\sigma_j^{\pm 1}$ dla $j < i$.
    Pokazanie, że $P$ jest półgrupą nie sprawia trudności, ale tego, że jest dobrze określonym zbiorem stanowi bardzo nietrywialne zadanie.

    Lewo-porządkowalna grupa jest beztorsyjna.
    Istotnie, ustalmy element $g \in G$ różny od elementu neutralnego.
    Bez straty ogólności niech $e < g$, przemnóżmy tę nierówność stronami przez $g$.
    Dostaniemy tak nową nierówność $g < g^2$.
    Powtarzając proces otrzymujemy łańcuch $e < g < g^2 < g^3 < \ldots$.
    Skoro $<$ jest porządkiem, nie jest możliwe by któryś z elementów $g^n$ był neutralny.
\end{proof}

\begin{proposition}
    Grupa warkoczy $B_n$ jest grupą Hopfa dla każdego $n \ge 1$: nie jest izomorficzna z żadnym ze swoich właściwych ilorazów.
\end{proposition}

\begin{proof}
    Podręcznik \cite{magnus66} dobrze wyjaśnia różne idee stojące za dowodem, który podamy.

    Mówimy, że grupa $G$ jest rezydualnie skończona, jeśli przekrój jej podgrup skończonego indeksu jest trywialny.
    Łatwo widać, że własność ta przenosi się na wszystkie podgrupy grupy $G$.
    Baumslag zauważył, że jeśli grupa $G$ jest skończenie generowana i~rezydualnie skończona, to grupa jej automorfizmów $\operatorname{Aut} G$ jest rezydualnie skończona.
    Grupa $G = \Z^2$ spełnia te założenia.
    Wolna grupa $F_2$ jest podgrupą grupy automorfizmów $\Z^2$, na przykład
    \begin{equation}
        F_2 \simeq \left\langle
        \begin{pmatrix}
            1 & 2 \\
            0 & 1
        \end{pmatrix},
        \begin{pmatrix}
            1 & 0 \\
            2 & 1
        \end{pmatrix}
        \right\rangle \subseteq \operatorname{Aut} \Z^2.
    \end{equation}
    Wszystkie grupy wolne $F_n$, $n \in \N$, są podgrupami grupy $F_2$, dlatego także są rezydualnie skończone, a z nimi grupa warkoczy, gdyż $B_n \subseteq \operatorname{Aut} F_n$.

    Malcew pokazał, że skończenie generowana i~rezydualnie skończona grupa jest grupą Hopfa.
    Krótki dowód tego faktu można znaleźć w~sekcji 6.5 książki \cite{magnus66}.
\end{proof}

\subsection{Liczba warkoczowa} % (fold)
\label{sub:braid_number}
\index{liczba!warkoczowa}
Z angielskiego \emph{braid number}.

\begin{definition}
    Liczba warkoczowa to minimalna liczba pasm, na których można zbudować warkocz, którego domknięciem jest wyjściowy splot.
\end{definition}

Tylko jeden węzeł ma liczbę warkoczową $1$, jest to niewęzeł.
Dwuwarkoczowe są dokładnie węzły torusowe typu $(2, n)$ dla $|n| \ge 3$.
Węzły spełniające $\operatorname{br} (K) = 3$ nie zostały jeszcze sklasyfikowane.
Liczba warkoczowa splotu zależy od orientacji ogniw i~trudno wyznacza się w~ogólnym przypadku.

\begin{proposition}
    Węzeł o~$n$ skrzyżowaniach można zapleść na $n - 1$ pasmach.
\end{proposition}

Powyższe ograniczenie ($\operatorname{b} \le \operatorname{cr} - 1$) nie jest zbyt użyteczne, równość mamy jedynie dla trójlistnika i ósemki.
Dokładną wartość liczby warkoczowej znamy między innymi dla węzłów torusowych (fakt \ref{braid-for-forus}).

Wielomian Alexandera wykrywa czasami węzły, których nie otrzyma się przez domykanie ,,małych'' warkoczy.
Przytoczone tu wyniki pochodzą z pracy \cite{jones85} Jonesa, gdzie nie ma jednak ich dowodów.
Jeśli $|\alexander(i)| > 3$, to węzeł nie jest domknięciem 3-warkocza (wniosek 23).
Ta implikacja jest skuteczna przy 43 z 59 węzłów o mniej niż 10 skrzyżowaniach.
Jeśli zaś spełniona jest nierówność $\alexander (\exp (2\pi i / 5)) > 13/2$, nie jest on domknięciem 4-warkocza (wniosek 24).
Prawdopodobnie nie istnieją podobne warunki dla 5-warkoczy.

% Koniec podsekcji Liczba warkoczowa

% Koniec sekcji warkocze
\section{Supły} % (fold)
\label{sec:tangle}

Na przełomie lat sześćdziesiątych i~siedemdziesiątych Conway szukał sposobu na zbudowanie kompletnej tablicy węzłów.
Niezmienniki znane w~tym czasie nie były dostatecznie mocne, by sprostać temu wyzwaniu.
Conway wprowadził pojęcie supła i~chociaż wszystkich węzłów nie można z~nich uzyskać, teoria została pchnięta do przodu.
Supły stanowią budulec splotów takich jak na przykład precle z~definicji \ref{def:pretzel}.

Sekcja oparta jest na podręczniku Murasugiego \cite{murasugi96} i~pracach \cite{conway70}, \cite{kauffman97}, \cite{kauffman04}, a~także \cite{schubert56}.
Po raz pierwszy z~supłami zetknęliśmy się w~pracy \cite{janiak04}, także czerpiącej inspirację z~\cite{murasugi96}, choć podejrzewamy, że konwencje nazewnicze w tych pracach wzajemnie wykluczają się.

\begin{definition}[supeł]
    \label{def:tangle}
    \index{supeł}
    Zawarty w~kole fragment diagramu splotu o~dwóch łukach wyjściowych oraz dwóch wejściowych, nazywamy supłem.
\end{definition}

Istnieją dwa rodzaje supłów -- naprzemienne i~sąsiadujące.
\begin{comment}
\begin{center}
    \begin{tikzpicture}[baseline=-0.65ex, scale=0.1]
    \useasboundingbox (-5, -9) rectangle (5, 5);
        \node [left] at (-5, -5) {SW};
        \draw[semithick,latex-] (-3, -3) to (-5,-5);
        \draw[semithick] (-3, -3) to (-1,-1);
        \node [right] at (5, 5) {NE};
        \draw[semithick,latex-] (3, 3) to (5,5);
        \draw[semithick] (3, 3) to (1,1);
        \node [right] at (5, -5) {SE};
        \draw[semithick,-latex] (3, -3) to (5,-5);
        \draw[semithick] (3, -3) to (1,-1);
        \node [left] at (-5, 5) {NW};
        \draw[semithick,-latex] (-3, 3) to (-5,5);
        \draw[semithick] (-3, 3) to (-1,1);
        \draw[semithick, densely dotted] (-0, 0) circle (3);
        \node at (0, -5) [below] {\small naprzemienny};
    \end{tikzpicture}
    \quad\quad\quad\quad\quad\quad
    \begin{tikzpicture}[baseline=-0.65ex, scale=0.1]
    \useasboundingbox (-5, -9) rectangle (5, 5);
        \node [left] at (-5, -5) {SW};
        \draw[semithick,latex-] (-3, -3) to (-5,-5);
        \draw[semithick] (-3, -3) to (-1,-1);
        \node [right] at (5, 5) {NE};
        \draw[semithick,-latex] (3, 3) to (5,5);
        \draw[semithick] (3, 3) to (1,1);
        \node [right] at (5, -5) {SE};
        \draw[semithick,latex-] (3, -3) to (5,-5);
        \draw[semithick] (3, -3) to (1,-1);
        \node [left] at (-5, 5) {NW};
        \draw[semithick,-latex] (-3, 3) to (-5,5);
        \draw[semithick] (-3, 3) to (-1,1);
        \draw[semithick, densely dotted] (-0, 0) circle (3);
        \node at (0, -5) [below] {\small sąsiadujący};
    \end{tikzpicture}
\end{center}
\end{comment}


Podobnie jak dla węzłów, pojawia się naturalne pytanie o~równoważność dwóch supłów.
Jest tak wtedy, gdy istnieje homeomorfizm* kuli na siebie, który przekształca jeden supeł na drugi, ale nie rusza sfery otaczającej.
Dla diagramów odpowiada to ruchom Reidemeistera, nie mamy jednak prawa opuszczać kuli zawierającej supeł.

Wszystkich supłów jest bardzo dużo, więc ograniczymy się do końca rozdziału do pewnej ich regularnej rodziny.
Oto cztery podstawowe supły:
\[
	\begin{tikzpicture}[baseline=-0.65ex, scale=0.1]
	\useasboundingbox (-5, -9) rectangle (5, 5);
		\draw[semithick] (-5 / 1.4142, -5 / 1.4142) [in=-45, out=45] to (-5 / 1.4142, 5 / 1.4142);
		\draw[semithick] (5 / 1.4142, -5 / 1.4142) [in=-135, out=135]  to (5 / 1.4142, 5 / 1.4142);
		\draw[semithick, densely dotted] (-0, 0) circle (5);
		\node at (0, -5) [below] {$(0)$};
	\end{tikzpicture}
	\quad\quad\quad
	\begin{tikzpicture}[baseline=-0.65ex, scale=0.1]
	\useasboundingbox (-5, -9) rectangle (5, 5);
		\draw[semithick] (-5 / 1.4142, -5 / 1.4142) [in=135, out=45] to (5 / 1.4142, -5 / 1.4142);
		\draw[semithick] (-5 / 1.4142, 5 / 1.4142) [in=-135, out=-45] to (5 / 1.4142, 5 / 1.4142);
		\draw[semithick, densely dotted] (-0, 0) circle (5);
		\node at (0, -5) [below] {$(\infty) = (0, 0)$};
	\end{tikzpicture}
	\quad\quad\quad
	\begin{tikzpicture}[baseline=-0.65ex, scale=0.1]
	\useasboundingbox (-5, -9) rectangle (5, 5);
	\begin{knot}[clip width=5, end tolerance=1pt]
		\strand[semithick] (-5 / 1.4142, -5 / 1.4142) to (5 / 1.4142, 5 / 1.4142);
		\strand[semithick] (5 / 1.4142, -5 / 1.4142) to (-5 / 1.4142, 5 / 1.4142);
		\strand[semithick, densely dotted] (-0, 0) circle (5);
		\node at (0, -5) [below] {$(-1)$};
	\end{knot}
	\end{tikzpicture}
	\quad\quad\quad
	\begin{tikzpicture}[baseline=-0.65ex, scale=0.1]
	\useasboundingbox (-5, -9) rectangle (5, 5);
	\begin{knot}[clip width=5, end tolerance=1pt, flip crossing/.list={1}]
		\strand[semithick] (-5 / 1.4142, -5 / 1.4142) to (5 / 1.4142, 5 / 1.4142);
		\strand[semithick] (-5 / 1.4142, 5 / 1.4142) to (5 / 1.4142, -5 / 1.4142);
		\strand[semithick, densely dotted] (-0, 0) circle (5);
		\node at (0, -5) [below] {$(1)$};
	\end{knot}
	\end{tikzpicture}
\]

\begin{definition}
    \label{def:rational_tangle}
    Supły powstające z~$(0)$ lub $(\infty)$ przez homeomorfizm kuli na siebie permutujący wejścia i~wyjścia nazywamy wymiernymi.
\end{definition}

Wystarczy się przy tym ograniczyć do ciągu obrotów półsfery dolnej (SW--SE) oraz lewej (SW--NW)
Z obrotami prawoskrętnymi wiążemy liczby dodatnie, natomiast z~lewoskrętnymi -- ujemne.
Otrzymujemy tak pewną krotkę $(a_1, \ldots, a_n)$.
Jeśli $n$ jest parzyste, zaczynamy od supła $(\infty)$, jeśli nie -- od $(0)$.
Ostatni obrót dotyczy półsfery dolnej.
Krotka nie jest jeszcze niezmiennikiem supłów (gdyż $(-2,3,3) \cong (3, -2, 3)$), ale nic straconego.
Potrzebować będziemy prostej definicji, uzasadniającej określenie ,,supły wymierne''.
% Conway pokazał, że dla dowolnego supła przy pomocy wielomianu Alexandera zdefiniować można  pewien ułamek.

\begin{proposition}
\label{prp:continued_fractions}
    Istnieje bijekcja między klasami supłów wymiernych oraz łańcuchowymi ułamkami, które je przedstawiają:
    \[
        \frac \alpha \beta = a_n + \frac{1}{a_{n-1} + 1 / (a_{n-2} +  \ldots + 1/a_1)} \in \Q \cup \{\infty\}.
    \]
\end{proposition}

\begin{proof}
    Praca \cite{conway70} Conwaya.
\end{proof}

\begin{proposition}
\label{prp:continued_fractions_2}
    Supły o~łańcuchowych ułamkach różnych od $0$ i~$\infty$ można kodować ciągami liczb całkowitych tego samego znaku.
\end{proposition}

Z każdym supłem $T$ związane jest jego odbicie $\overline T$, obraz wyjściowego przez symetrię względem prostej $y = -x$.
Mając dwa supły obok siebie, można dokonać ich sklejenia wzdłuż połówek kul, w~których leżą:
\[
	\begin{tikzpicture}[baseline=-0.65ex, scale=0.1]
	\useasboundingbox (-5, -9) rectangle (5, 5);
		\draw[semithick] (-2, -2) to (-5,-5);
		\draw[semithick] (2, 2) to (5,5);
		\draw[semithick] (2, -2) to (5,-5);
		\draw[semithick] (-2, 2) to (-5,5);		%
		\draw[semithick, densely dotted] (-0, 0) circle (5);
		\node at (0, 0) {$T_1$};
		\node [below] at (0, -5) {supeł};
	\end{tikzpicture}
	\quad \quad
	\begin{tikzpicture}[baseline=-0.65ex, scale=0.1]
	\useasboundingbox (-5, -9) rectangle (5, 5);
		\draw[semithick] (-2, -2) to (-5,-5);
		\draw[semithick] (2, 2) to (5,5);
		\draw[semithick] (2, -2) to (5,-5);
		\draw[semithick] (-2, 2) to (-5,5);		%
		\draw[semithick, densely dotted] (-0, 0) circle (5);
		\node at (0, 0) {$T_2$};
		\node [below] at (0, -5) {supeł};
	\end{tikzpicture}
	\quad \quad
	% \begin{tikzpicture}[baseline=-0.65ex, scale=0.1]
	% \useasboundingbox (-15, -9) rectangle (15, 5);
	% 	\draw[semithick] (-12, -2) to (-15,-5);
	% 	\draw[semithick] (-12, 2) to (-15,5);		%
	% 	\draw[semithick, densely dotted] (-10, 0) circle (5);
	% 	\node at (-10, 0) {$T_1$};
	% 	\draw[semithick] (12, -2) to (15,-5);
	% 	\draw[semithick] (12, 2) to (15,5);		%
	% 	\draw[semithick, densely dotted] (10, 0) circle (5);
	% 	\node at (10, 0) {$T_2$};
	% 	\draw[semithick] (-8, 2) [in=135, out=45] to (8, 2);
	% 	\draw[semithick] (-8, -2) [in=-135, out=-45] to (8, -2);
	% 	\node [below] at (0, -5) {produkt};
	% \end{tikzpicture}
	% \quad \quad
	\begin{tikzpicture}[baseline=-0.65ex, scale=0.1]
	\useasboundingbox (-15, -9) rectangle (15, 5);
		\draw[semithick] (-12, -2) to (-15,-5);
		\draw[semithick] (-12, 2) to (-15,5);		%
		\draw[semithick, densely dotted] (-10, 0) circle (5);
		\node at (-10, 0) {$T_1$};
		\draw[semithick] (12, -2) to (15,-5);
		\draw[semithick] (12, 2) to (15,5);		%
		\draw[semithick, densely dotted] (10, 0) circle (5);
		\node at (10, 0) {$T_2$};
		\draw[semithick] (-8, 2) [in=135, out=45] to (8, 2);
		\draw[semithick] (-8, -2) [in=-135, out=-45] to (8, -2);
		\node [below] at (0, -5) {suma};
	\end{tikzpicture}
\]

Oznaczmy tak otrzymany węzeł przez $T_1 + T_2$.
Niektórzy definiują dalsze działania, jak produkt: $T_1 \cdot T_2 = \overline T_1 + T_2$ czy rozgałęzienie, $\overline T_1 + \overline T_2$.
Rodzina supłów wymiernych jest zamknięta na branie produktów, ale nie sum.
Wprowadzamy więc następującą, ogólniejszą definicję.
Supeł będący skończoną sumą supłów wymiernych, ich luster, odbić lub odbić luster nazywamy algebraicznym.

Przez zszycie par łuków wejściowych (lub wyjściowych) zamieniamy supły w~węzły:
\[
    \begin{tikzpicture}[baseline=-0.65ex, scale=0.1]
    \useasboundingbox (-5, -11) rectangle (5, 7);
        \draw[semithick] (-2, -2) to (-5,-5);
        \draw[semithick] (2, 2) to (5,5);
        \draw[semithick] (2, -2) to (5,-5);
        \draw[semithick] (-2, 2) to (-5,5);        %
        \draw[semithick, densely dotted] (-0, 0) circle (5);
        \node at (0, 0) {$T$};
        \node [below] at (0, -8) {$N(T)$};
        \draw[semithick] (-5, -5) [in=-45, out=-135] to (5, -5);        %
        \draw[semithick] (-5,    5) [in=45, out=135] to (5, 5);        %
    \end{tikzpicture}
    \quad \quad
    \begin{tikzpicture}[baseline=-0.65ex, scale=0.1]
    \useasboundingbox (-5, -11) rectangle (5, 7);
        \draw[semithick] (-2, -2) to (-5,-5);
        \draw[semithick] (2, 2) to (5,5);
        \draw[semithick] (2, -2) to (5,-5);
        \draw[semithick] (-2, 2) to (-5,5);        %
        \draw[semithick, densely dotted] (-0, 0) circle (5);
        \node at (0, 0) {$T$};
        \node [below] at (0, -8) {supeł $T$};
    \end{tikzpicture}
    \quad \quad
    \begin{tikzpicture}[baseline=-0.65ex, scale=0.1]
    \useasboundingbox (-5, -11) rectangle (5, 7);
        \draw[semithick] (-2, -2) to (-5,-5);
        \draw[semithick] (2, 2) to (5,5);
        \draw[semithick] (2, -2) to (5,-5);
        \draw[semithick] (-2, 2) to (-5,5);        %
        \draw[semithick, densely dotted] (-0, 0) circle (5);
        \node at (0, 0) {$T$};
        \node [below] at (0, -8) {$D(T)$};
        \draw[semithick] (-5, -5) [in=135, out=-135] to (-5, 5);        %
        \draw[semithick] (5, -5) [in=45, out=-45] to (5, 5);        %
    \end{tikzpicture}
    \quad \quad
\]


Oznaczenia $N(T)$ oraz $D(T)$ pochodzą od angielskich słów \emph{numerator}, \emph{denominator}.
Być może nie jest jasne, dlaczego terminy stosowane zazwyczaj do opisu ułamków stosujemy wobec diagramów splotów.
Nazewnictwo nie jest przypadkowe. %, wrócimy do tego tematu wkrótce.
\begin{proposition}
\label{prp:knot_fraction}
    Ułamek supła zadany wzorem
    \[
        F(A) = \frac{\conway_{N(A)}(z)}{\conway_{D(A)}(z)}
    \]
    spełnia zależność $F(A+B) = F(A) + F(B)$.
\end{proposition}

\begin{proof}
    Praca \cite{conway70} Conwaya.
\end{proof}

Istnieją supły $T_1$, $T_2$ takie, że węzły $N(T_i)$ są nietrywialne, ale $N(T_1 + T_2)$ to niewęzeł.
Co gorsza, dla każdego wymiernego supła $A$ istnieje taki supeł $B$, że $N(A+B)$ jest niewęzłem.

Praca \cite{conway70} zawiera jeszcze jeden ciekawy rezultat, uogólniony przez Lickorisha i~Milletta w~\cite{lickorish87}.
Przyjmijmy następujące skróty: niech $A_n = P_{N(A)}(x,y)$, $A_d = P_{D(A)}(x,y)$.

\begin{proposition}
    Dla dowolnych supłów $A, B$ mamy
    \[
    (1 - (x+y)^2)(A+B)_n = (A_nB_d + A_dB_n) - (x+y)(A_nB_n+  A_dB_d)
    \]
    oraz
    \[
        (A+B)_d = A_dB_d.
    \]
\end{proposition}

Uwaga, zastosowano tu nieco inną parametryzację wielomianu HOMFLY.
Na zakończenie wspomnimy o~mutacjach.

\begin{definition}[mutacja]
\label{def:mutant}
    Półobrót supła względem osi poziomej, pionowej albo też prostopadłej do płaszczyzny, w~jakiej leży diagram, nazywamy mutacją, zaś otrzymany tak splot -- mutantem.
    W razie potrzeby zmieniamy orientację supła na przeciwną.
\end{definition}

Mutacja węzła o~co najwyżej dziesięciu skrzyżowaniach nie zmienia jego klasy abstrakcji.
Najsłynniejszą parę mutantów stanowią węzeł Conwaya $11n_{34}$ oraz Kinoshity-Terasakiego $11n_{42}$.
Conway zauważył podczas klasyfikacji niealternujących węzłów, że tylko one posiadają trywialny wielomian Alexandera.
Mają też taki sam wielomian Jonesa,
\begin{equation}
    \jones(t) = t^{6} -2t^5 +2t^4 -2t^3 +t^2 +2t^{-1} -2t^{-2} +2t^{-3} -t^{-4}.
\end{equation}
Kinoshita, Terasaki zdefiniowali nieskończoną rodzinę węzłów o trywialnym wielomianie Alexandera, której pierwszym wyrazem jest węzeł $11n_{42}$ (w~\cite{kinoshita57}).
Dowód tego, że $11n_{34}$ oraz $11n_{42}$ są różne, jako pierwszy podał prawdopodobnie Riley w~1971 roku \cite{riley71}: wykorzystał on homomorfizmy z~grupy węzła w~$PSL(2, 7)$.
Genusy, odpowiednio: $3$ i~$2$, wyznaczył Gabai piętnaście lat później w~\cite{gabai86}, używał foliacji.

Niedawno Stojmenow podjął się systematycznie znalezienia mutantów wśród węzłów o~mniej niż dziewiętnastu skrzyżowaniach (praca \cite{stoimenow10} z~2010 roku).
Mutanty nie dają się łatwo odróżniać niezmiennikami.

\begin{proposition}
    Mutacja węzłów nie zmienia następujących niezmiennikow:
    kablowego wielomianu Jonesa, % menasco91
    2-kablowego wielomianu HOMFLY, % przytycki89
    kablowego wielomianu Kauffmana, % lipson87
    sygnatury Tristrama-Levine'a, % cooper99
    symplicjalnEj objętości Gromowa, % ruberman87
    instanton homologii Floera, % ruberman99
    niezmienników Wittena % rong94
    ani Cassona. % kirk89
\end{proposition}

\begin{proof}
    Prace \cite{menasco91}, \cite{przytycki89}, \cite{lipson87}, \cite{cooper99}, \cite{ruberman87}, \cite{ruberman99}, \cite{rong94} oraz \cite{kirk89}.
\end{proof}

Stojemow twierdzi w~\cite{stoimenow10}, że praca \cite{chmutov94} zawiera dowód takiego stwierdzenia: ,,niezmienniki Wasiljewa stopnia co najwyżej ósmego nie rozróżniają mutantów''; ja tego nie widzę.
Wynik poprawił Murakami sześć lat później do dziesiątego stopnia w dostępnej na swojej stronie internetowej pracy ,,Finite type invariants detecting the mutant knots''.
Potwierdził też, że pewien niezmiennik stopnia 11 używany przez Mortona i~Cromwella odróżnia węzeł Conwaya od węzła Kinoshity-Terasakiego.

Jeśli wyjściowy diagram był alternujący, to mutant też jest alternujący.
Istnieje podejrzenie, że mutacja nie zmienia liczby gordyjskiej.
Gordon i~Luecke w~2006 pokazali to dla klasy węzłów $1$-gordyjskich (\cite{gordon06}), dużo wcześniej wiedzieliśmy tylko, że jedynym mutantem niewęzła jest niewęzeł (Rolfsen w~\cite{rolfsen93}?).

\begin{proposition}
    Niech $m, n$ będą nieujemnymi liczbami całkowitymi.
    Wtedy istnieje węzeł $K$ o genusie plastrowym równym $m$, którego pewien mutant ma genus plastrowy równy $n$.
\end{proposition}

\begin{proof}
    Kim, Livingston w \cite{kim05}.
    Wcześniej Livingston pokazał istnienie mutantów o~różnym genusie plastrowym (\cite{livingston83}).
\end{proof}

\subsection{Sploty o~dwóch mostach} % (fold)
\label{sub:twobridge}
Zajmiemy się teraz związkiem supłów z indeksem mostowym.
Wiemy, że węzeł trywialny jest jednomostowy, następne w hierarchii są sploty dwumostowe.
Nazywa się je także wymiernymi, po angielsku czasami \emph{4-plats}.
Jako pierwszy studiował je Bankwitz z~Schumannem w~1934 roku.
% Kawauchi: as 4-plat presentations, which is just Conway's normal form.
Mają co najwyżej dwie składowe i~są odwracalne.

\begin{proposition}
    Sploty dwumostowe są pierwsze.
\end{proposition}

\begin{proof}
    Prosty wniosek z~tego, że liczba mostowa prawie jest addytywna (fakt \ref{prp:bridge_additive}).
\end{proof}

\begin{corollary}
    Węzły dwumostowe są $(\pm 2, n)$-torusowe albo hiperboliczne.
\end{corollary}

\begin{tobedone}
    % Skorzystajmy z trychotomii Thurstona: węzły dwumostowe są pierwsze, zatem nie są satelitarne.
    % Pozostało rozpatrzyć przypadek, kiedy nie są hiperboliczne.
    Teza wynika wtedy bezpośrednio z faktu \ref{prp:torus_bridge_number}, który głosi, że $br(T_{p, q}) = \min\{|p|, |q|\}$.
\end{tobedone}

%\todo[inline]{Murasugi Theorem 9.3.3 (138) lub Janiak-Osajca, Pogoda (34).}
% Aus der unten stehenden Klassifikation ergibt sich, dass man jede Verschlingung mit 2 Brücken wie im Bild rechts darstellen kann, wobei {\displaystyle a_{i}\in \mathbb {Z} } a_{i}\in \mathbb{Z }  die Anzahl der Halbtwists in der jeweiligen Box bezeichnet und für gerade bzw. ungerade {\displaystyle i} i~positive {\displaystyle a_{i}} a_{i} links- bzw. rechtshändigen Halbtwists entsprechen.
% Diese Darstellung wird als Conway-Normalform bezeichnet.
% Man kann stets erreichen, dass alle {\displaystyle a_{i}} a_{i} dasselbe Vorzeichen haben.[1] Insbesondere gibt die Conway-Normalform dann ein alternierendes Knotendiagramm.[2]
%Insbesondere ist ein 2-Brücken-Knoten genau dann amphichiral, wenn {\displaystyle q^{2}\equiv -1\ mod\ p} q^{2}\equiv -1\ mod\ p ist.

\begin{proposition}
    Sploty z~dwoma mostami to dokładnie sploty typu $D(T)$ dla pewnego supła wymiernego $T$.
\end{proposition}

Dowód tego stwierdzenia znaleźć można na przykład w książce \cite{murasugi96}, strony 183-187.
Wynika z niego, że każdy splot dwumostowy można przedstawić następującym diagramem:
\begin{comment}
\[
    \begin{tikzpicture}[baseline=-0.65ex, xscale=0.14, yscale=0.07]
    \useasboundingbox (-40, -20) rectangle (40, 20);
        %%% A1
        \draw[semithick] (-36, -5) .. controls (-34,-5) and (-35, 5) .. (-33, 5);
        \draw[semithick] (-36,  5) .. controls (-34, 5) and (-35,-5) .. (-33,-5);
        \node at (-31.5, 0) {$\ldots$};
        \draw[semithick] (-30, -5) .. controls (-28,-5) and (-29, 5) .. (-27, 5);
        \draw[semithick] (-30,  5) .. controls (-28, 5) and (-29,-5) .. (-27,-5);
        %%% A2
        \draw[semithick] (-24,  5) .. controls (-22,  5) and (-23, 15) .. (-21, 15);
        \draw[semithick] (-24, 15) .. controls (-22, 15) and (-23,  5) .. (-21,  5);
        \node at (-19.5, 10) {$\ldots$};
        \draw[semithick] (-18,   5) .. controls (-16,  5) and (-17, 15) .. (-15, 15);
        \draw[semithick] (-18,  15) .. controls (-16, 15) and (-17,  5) .. (-15,  5);
        %%% A3
        \draw[semithick] (-12, -5) .. controls (-10,-5) and (-11, 5) .. (-9, 5);
        \draw[semithick] (-12,  5) .. controls (-10, 5) and (-11,-5) .. (-9,-5);
        \node at (-7.5, 0) {$\ldots$};
        \draw[semithick] (-6, -5) .. controls (-4,-5) and (-5, 5) .. (-3, 5);
        \draw[semithick] (-6,  5) .. controls (-4, 5) and (-5,-5) .. (-3,-5);
        %%% A4
        \draw[semithick] (0,  5) .. controls (2,  5) and (1, 15) .. (3, 15);
        \draw[semithick] (0, 15) .. controls (2, 15) and (1,  5) .. (3,  5);
        \node at (4.5, 10) {$\ldots$};
        \draw[semithick] (6,   5) .. controls (8,  5) and (7, 15) .. (9, 15);
        \draw[semithick] (6,  15) .. controls (8, 15) and (7,  5) .. (9,  5);
        %%% A 2k+1
        \draw[semithick] (27,  -5) .. controls (29,  -5) and (28, 5) .. (30, 5);
        \draw[semithick] (27, 5) .. controls (29, 5) and (28,  -5) .. (30,  -5);
        \node at (31.5, 0) {$\ldots$};
        \draw[semithick] (33,   -5) .. controls (35,  -5) and (34, 5) .. (36, 5);
        \draw[semithick] (33,  5) .. controls (35, 5) and (34,  -5) .. (36,  -5);
        %%%    - A3
        \draw[semithick] (-36, 15) to (-24, 15);
        %%% A1 - A3
        \draw[semithick] (-27, -5) to (-12, -5);
        %%% A1 - A2
        \draw[semithick] (-27,  5) to (-24,  5);
        %%% A2 - A3
        \draw[semithick] (-15,  5) to (-12,  5);
        %%% A2 - A4
        \draw[semithick] (-15, 15) to (0, 15);
        %%% A3 - A4
        \draw[semithick] (-3, 5) to (0, 5);
        %%%
        \draw[semithick] ( 9, 15) to (14, 15);
        \draw[semithick] ( 9,  5) to (14,  5);
        \draw[semithick] (-3, -5) to (14, -5);
        \node at (18,  15) {$\ldots$};
        \node at (18,   5) {$\ldots$};
        \node at (18,  -5) {$\ldots$};
        \node at (18, -15) {$\ldots$};
        \draw[semithick] (22, 15) to (36, 15);
        \draw[semithick] (22,  5) to (27,  5);
        \draw[semithick] (22, -5) to (27, -5);
        \draw[semithick] (-36, -15) to (36, -15);
        \draw[semithick] (-36, -15) [in=left,  out=left]  to (-36, -5);
        \draw[semithick] (-36,   5) [in=left,  out=left]  to (-36, 15);
        \draw[semithick] ( 36, -15) [in=right, out=right] to ( 36, -5);
        \draw[semithick] ( 36,   5) [in=right, out=right] to ( 36, 15);
        %
        \draw[semithick, decoration={brace,mirror,raise=3pt},decorate]  (-36, -5) -- node[below=4pt] {$a_1$}      (-27, -5);
        \draw[semithick, decoration={brace,mirror,raise=3pt},decorate]  (-24,  5) -- node[below=4pt] {$a_2$}      (-15,  5);
        \draw[semithick, decoration={brace,mirror,raise=3pt},decorate]  (-12, -5) -- node[below=4pt] {$a_3$}      ( -3, -5);
        \draw[semithick, decoration={brace,mirror,raise=3pt},decorate]  (  0,  5) -- node[below=4pt] {$a_4$}      (  9,  5);
        \draw[semithick, decoration={brace,mirror,raise=3pt},decorate]  ( 27, -5) -- node[below=4pt] {$a_{2k+1}$} ( 36, -5);
    \end{tikzpicture}
\]
\end{comment}


Oto reguła, zgodnie z~którą wybieramy znaki liczb $a_i$:
jeśli $i$ jest nieparzyste, prawy skręt jest dodatni, jeśli parzyste -- lewy jest dodatni.
Sam diagram oznaczamy $C(a_1, \ldots, a_{2k+1})$ i~nazywamy postacią normalną Conwaya.

\begin{proposition}
    % Murasugi proposition 9.3.2
    Sploty dwumostowe są alternujące.
\end{proposition}

\begin{proof}
    Goodrick w~\cite{goodrick72} podał diagramatyczny dowód, gdzie ciąg ruchów zmienia diagram splotu dwumostowego w~alternujący.
    Wynika to też z faktu \ref{prp:continued_fractions}.
\end{proof}

Przez analogię do supłów, definiujemy ułamek łańcuchowy
\begin{equation}
    C(a_1, \ldots, a_{2k+1}) \mapsto a_1 + \frac{1}{a_2 + 1/\ldots} = \frac \alpha \beta.
\end{equation}

\begin{tobedone}
    To jest postać normalna Conwaya, ale mamy jeszcze postać Schuberta - \cite[s. 21]{kawauchi96}.
\end{tobedone}

Zauważmy, że wartość bezwzględna ułamka $\alpha/\beta$ zawsze przekracza $1$ i~odwrotnie, każdy taki ułamek pochodzi od pewnego węzła dwumostowego.
Parę względnie pierwszych liczb $(\alpha, \beta)$ nazywamy typem węzła dwumostowego.

\begin{proposition}
    \label{prp:tangle_equivalence}
    Dwumostowe sploty typów $(\alpha, \beta)$ oraz $(\alpha', \beta')$ są, pomijając orientację, równoważne wtedy i~tylko wtedy, gdy spełniony jest jeden z warunków:
    \begin{itemize}
        \item $\alpha = \alpha'$ oraz $\beta \equiv \beta' \pmod \alpha$,
        \item $\alpha = \alpha'$ oraz $\beta \beta' \equiv 1 \pmod \alpha$.
    \end{itemize}
\end{proposition}

\begin{tobedone}
    \cite[s. 23]{kawauchi96}: czasem $2\alpha$ zamiast $\alpha$.
\end{tobedone}

\begin{proof}
    Dowód opiera się na tym, że podwójnie cykliczna przestrzeń nakrywająca rozcięta wzdłuż splotu jest przestrzenią soczewkową typu $(\alpha, \beta)$.
    Nie definiowaliśmy nawet tych przestrzeni, szczegóły można znaleźć w~podręczniku \cite{murasugi96} albo \cite{schubert56}.
\end{proof}

\begin{proposition}

    Splot typu $(\alpha, \beta)$ jest achiralny dokładnie, gdy $\beta^2 \equiv -1 \mod \alpha$.
\end{proposition}

\begin{proof}
    Wynika to z tego, że lustrem splotu typu $(\alpha, \beta)$ jest splot typu $(\alpha, -\beta)$ oraz faktu \ref{prp:tangle_equivalence}.
\end{proof}

Cztery kolejne stwierdzenia to fakt 9.3.4 w \cite{murasugi96}:

\begin{proposition}
    Niech $b$ będzie dowolną liczbą całkowitą.
    Wtedy następujące sploty są tego samego typu:
    \begin{align}
        N(T(a_1, a_2, \ldots, a_{2k+1})) & \approx N(T(a_1, a_2, \ldots, a_{2k+1}, b, 0)) \\
                                         & \approx D(T(-a_1, -a_2, \ldots, -a_{2k+1}, b)) \\
                                         & \approx C(a_1, a_2, \ldots, a_{2k}-1, 1). \\
        N(T(a_1, a_2, \ldots, a_{2k}))   & \approx D(T(-a_1, -a_2, \ldots, -a_{2k}, b)) \\
                                         & \approx C(a_1, a_2, \ldots, a_{2k}-1, 1). \\
        D(T(a_1, a_2, \ldots, a_{2k+1})) & \approx D(T(a_1, a_2, \ldots, a_{2k}, 0)) \\
                                         & \approx C(1, a_1-1, a_2, \ldots, a_{2k}). \\
        D(T(a_1, a_2, \ldots, a_{2k}))   & \approx D(T(a_1, a_2, \ldots, a_{2k-1}, 0)) \\
                                         & \approx C(a_1, a_2, \ldots, a_{2k-1}).
    \end{align}
\end{proposition}

\begin{proposition}

    Wyznacznikiem dwumostowego splotu typu $(\alpha, \beta)$ jest $\alpha$.
\end{proposition}

Wynika stąd, że wyznacznik nie wystarcza do odróżniania splotów dwumostowych.

\begin{proof}
    % Chcąc oszczędzić niektórym Czytelnikom cierpień odsyłamy po prostu do \cite{schubert56}.
    \url{https://math.stackexchange.com/questions/3327846/}.
\end{proof}

Niech $A, B$ będą supłami.
Wiemy, że suma $A+B$ nie musi być supłem, zaś $D(A+B)$ niekoniecznie jest splotem dwumostowym.
Pomimo to, splot $N(A+B)$ jest dwumostowy, potrafimy nawet powiedzieć, jaki ma wyznacznik:

\begin{proposition}
    % Theorem 9.3.5 Murasugi

    Niech $A, B$ będą supłami, którym odpowiadają skrócone ułamki $p/q$ oraz $r/s$.
    Wtedy splot $L = N(A+B)$ jest dwumostowy, typu $(\alpha, \beta)$ i ma wyznacznik $\alpha = |ps + qr|$.
\end{proposition}

Murasugi (twierdzenie 9.3.5) twierdzi, że dowód znajduje się w \cite{ernst90}.

\begin{proposition}

    Rozpatrzmy węzeł dwumostowy typu $(\alpha, \beta)$, gdzie $0 < \beta < \alpha$ i~$\beta$ jest nieparzyste.
    Niech $r_k$ będzie resztą z~dzielenia $k\beta$ przez $2\alpha$ leżącą w~przedziale $(-\alpha, \alpha)$ dla $k = 0, 1, \ldots, \alpha - 1$.
    Różnica między ilością dodatnich reszt i~ujemnych reszt to sygnatura węzła.
\end{proposition}

Wygląda na to, że jedynym niewyznaczonym do końca klasycznym niezmiennikiem jest liczba gordyjska.

% Koniec podsekcji Sploty o~dwóch mostach


% Koniec sekcji Supły

\section{Precle} % (fold)
\label{sec:pretzel}
\begin{definition}
	Splotem Montesinosa nazywamy splot o poniższym diagramie 
	kodowanym wymiernymi liczbami $\alpha_i/\beta_i$ oraz całkowitą $e$.
	Odpowiadają one supłom.
	\[
	\begin{tikzpicture}[baseline=-0.65ex, scale=0.1]
	%\useasboundingbox (-5, -9) rectangle (5, 5);
		\draw[semithick] (-5, 5) rectangle (5, 15);
		\foreach \x in {0,1,3,4} {
			\draw[semithick] (15*\x-35, -15) rectangle (15*\x-25, -5);
		}
		\foreach \x in {0,1,2,3,4,5} {
			\draw[semithick] (15*\x-35, -8) to (15*\x-40, -8);
			\draw[semithick] (15*\x-35, -12) to (15*\x-40, -12);
		}
		\draw[semithick] (-40, -8) [in=down, out=left] to (-45, -3);
		\draw[semithick] (-40, -12) [in=down, out=left] to (-49, -3);

		\draw[semithick] (-40, 8) [in=up, out=left] to (-45, 3);
		\draw[semithick] (-40, 12) [in=up, out=left] to (-49, 3);

		\draw[semithick] (40, 8) [in=up, out=right] to (45, 3);
		\draw[semithick] (40, 12) [in=up, out=right] to (49, 3);

		\draw[semithick] (40, -8) [in=down, out=right] to (45, -3);
		\draw[semithick] (40, -12) [in=down, out=right] to (49, -3);

		\draw[semithick] (-45, -3)  to (-45, 3);
		\draw[semithick] (-49, -3)  to (-49, 3);
		\draw[semithick] (45, -3)  to (45, 3);
		\draw[semithick] (49, -3)  to (49, 3);

		\draw[semithick] (-5, 8)  to (-40, 8);
		\draw[semithick] ( 5, 8)  to ( 40, 8);
		\draw[semithick] (-5, 12)  to (-40, 12);
		\draw[semithick] ( 5, 12)  to ( 40, 12);

		\node at (0, 10) {e};
		\node at (0, -10) {\ldots};
		\node at (-15, -10) {$\displaystyle \frac{\alpha_2}{\beta_2}$};
		\node at (-30, -10) {$\displaystyle \frac{\alpha_1}{\beta_1}$};
		\node at (15, -10) {$\displaystyle \frac{\alpha_{n-1}}{\beta_{n-1}}$};
		\node at (30, -10) {$\displaystyle \frac{\alpha_n}{\beta_n}$};
	\end{tikzpicture}
	\]	
\end{definition}

\begin{definition}
	\label{def:pretzel}
	Splot Montesinosa o całkowitych współczynnikach nazywamy preclem.
\end{definition}

Na standardowym diagramie precla $(p_1, p_2, \ldots, p_n)$ występuje $p_1$ lewych skrzyżowań w pierwszym suple, $p_2$ w drugim, i tak dalej.
Taki precel jest węzłęm dokładnie wtedy, gdy $n$ oraz $p_i$ są nieparzyste lub dokładnie jedna z liczb $p_i$ jest parzysta.

\begin{proposition}
	Jeśli co najmniej dwa współczynniki $p_i$ zerują się, splot jest rozdzielczy.
\end{proposition}

Nie jest prawdziwa implikacja odwrotna.

Precel $(1,1,1)$ to prawy trójlistnik, $(5, -1, -1)$ to węzeł dokerski $6_1$, $(-3, 0, -3)$ to splot dwóch trójlistników, zaś $(2p, 2q, 2r)$ (zapewne dla nieparzystych, różnych $p, q, r$) jest splotem trzech niewęzłów.
Precle $(-2, 3, 2n+1)$ są szczególnie użyteczne jako narzędzie do badania 3-rozmaitości.
Wiele twierdzeń, które dotyczą takich rozmaitości, opiera się na przykład na chirurgii Dehna precla $(-2, 3, 7)$.

Patrz także twierdzenie \ref{trotter}.
% Koniec sekcji Precle

\todo[inline]{\url{https://en.wikipedia.org/wiki/Lissajous_knot}}

\chapter{Tablice węzłów pierwszych}
\section{Wartości niezmienników}
\label{sec:table_of_invariants}
Tabela przedstawia węzły pierwsze o~co najwyżej dziesięciu skrzyżowaniach oraz wartości ich niezmienników (całkowitoliczbowych lub wielomianowych).
Stosowane oznaczenia:
\begin{itemize}
    \item $u$ liczba gordyjska, $b$ indeks warkoczowy, $br$ indeks mostowy,
    \item $\det$ wyznacznik, sygn. -- sygnatura, Arf -- niezmiennik Arfa,
    \item $\nabla(z)$ -- wielomian Conwaya (jako ciąg współczynników, dla $4_1$: $1-z^2$)
    \item sym. -- typ symetrii, alt. -- czy alternujący.
\end{itemize}

Dane zawarte w~tej tabeli pochodzą ze strony \url{http://www.indiana.edu/~knotinfo/}.
Jej autorzy, Chuck Livingston z~uniwersytetu Indiany  oraz Jae Choon Cha (z koreańskiego Pohangu) prezentują tam wiele innych niezmienników.

\renewcommand*{\arraystretch}{1.4}
\footnotesize
\begin{longtable}{lccccccllc}
\hline
nazwa & u~& b & br & $\det$ & sygn. & Arf & $\nabla(z)$ & sym. & alt. \\ \hline
\endhead % all the lines above this will be repeated on every page
$3_{1}$    & $1$   & $2$ & $2$ & $3$   & $-2$ & $1$ & $1+1$         & odwracalny & tak \\
$4_{1}$    & $1$   & $3$ & $2$ & $5$   & $0$  & $1$ & $1-1$         & całkowicie & tak \\
$5_{1}$    & $2$   & $2$ & $2$ & $5$   & $-4$ & $1$ & $1+3+1$       & odwracalny & tak \\
$5_{2}$    & $1$   & $3$ & $2$ & $7$   & $-2$ & $0$ & $1+2$         & odwracalny & tak \\
$6_{1}$    & $1$   & $4$ & $2$ & $9$   & $0$  & $0$ & $1-2$         & odwracalny & tak \\
$6_{2}$    & $1$   & $3$ & $2$ & $11$  & $-2$ & $1$ & $1-1-1$       & odwracalny & tak \\
$6_{3}$    & $1$   & $3$ & $2$ & $13$  & $0$  & $1$ & $1+1+1$       & całkowicie & tak \\
$7_{1}$    & $3$   & $2$ & $2$ & $7$   & $-6$ & $0$ & $1+6+5+1$     & odwracalny & tak \\
$7_{2}$    & $1$   & $4$ & $2$ & $11$  & $-2$ & $1$ & $1+3$         & odwracalny & tak \\
$7_{3}$    & $2$   & $3$ & $2$ & $13$  & $-4$ & $1$ & $1+5+2$       & odwracalny & tak \\
$7_{4}$    & $2$   & $4$ & $2$ & $15$  & $-2$ & $0$ & $1+4$         & odwracalny & tak \\
$7_{5}$    & $2$   & $3$ & $2$ & $17$  & $-4$ & $0$ & $1+4+2$       & odwracalny & tak \\
$7_{6}$    & $1$   & $4$ & $2$ & $19$  & $-2$ & $1$ & $1+1-1$       & odwracalny & tak \\
$7_{7}$    & $1$   & $4$ & $2$ & $21$  & $0$  & $1$ & $1-1+1$       & odwracalny & tak \\
$8_{1}$    & $1$   & $5$ & $2$ & $13$  & $0$  & $1$ & $1-3$         & odwracalny & tak \\
$8_{2}$    & $2$   & $3$ & $2$ & $17$  & $-4$ & $0$ & $1+0-3-1$     & odwracalny & tak \\
$8_{3}$    & $2$   & $5$ & $2$ & $17$  & $0$  & $0$ & $1-4$         & całkowicie & tak \\
$8_{4}$    & $2$   & $4$ & $2$ & $19$  & $2$  & $1$ & $1-3-2$       & odwracalny & tak \\
$8_{5}$    & $2$   & $3$ & $3$ & $21$  & $-4$ & $1$ & $1-1-3-1$     & odwracalny & tak \\
$8_{6}$    & $2$   & $4$ & $2$ & $23$  & $-2$ & $0$ & $1-2-2$       & odwracalny & tak \\
$8_{7}$    & $1$   & $3$ & $2$ & $23$  & $2$  & $0$ & $1+2+3+1$     & odwracalny & tak \\
$8_{8}$    & $2$   & $4$ & $2$ & $25$  & $0$  & $0$ & $1+2+2$       & odwracalny & tak \\
$8_{9}$    & $1$   & $3$ & $2$ & $25$  & $0$  & $0$ & $1-2-3-1$     & całkowicie & tak \\
$8_{10}$   & $2$   & $3$ & $3$ & $27$  & $2$  & $1$ & $1+3+3+1$     & odwracalny & tak \\
$8_{11}$   & $1$   & $4$ & $2$ & $27$  & $-2$ & $1$ & $1-1-2$       & odwracalny & tak \\
$8_{12}$   & $2$   & $5$ & $2$ & $29$  & $0$  & $1$ & $1-3+1$       & całkowicie & tak \\
$8_{13}$   & $1$   & $4$ & $2$ & $29$  & $0$  & $1$ & $1+1+2$       & odwracalny & tak \\
$8_{14}$   & $1$   & $4$ & $2$ & $31$  & $-2$ & $0$ & $1+0-2$       & odwracalny & tak \\
$8_{15}$   & $2$   & $4$ & $3$ & $33$  & $-4$ & $0$ & $1+4+3$       & odwracalny & tak \\
$8_{16}$   & $2$   & $3$ & $3$ & $35$  & $2$  & $1$ & $1+1+2+1$     & odwracalny & tak \\
$8_{17}$   & $1$   & $3$ & $3$ & $37$  & $0$  & $1$ & $1-1-2-1$     & ujemny     & tak \\
$8_{18}$   & $2$   & $3$ & $3$ & $45$  & $0$  & $1$ & $1+1-1-1$     & całkowicie & tak \\
$8_{19}$   & $3$   & $3$ & $3$ & $3$   & $-6$ & $1$ & $1+5+5+1$     & odwracalny & nie \\
$8_{20}$   & $1$   & $3$ & $3$ & $9$   & $0$  & $0$ & $1+2+1$       & odwracalny & nie \\
$8_{21}$   & $1$   & $3$ & $3$ & $15$  & $-2$ & $0$ & $1+0-1$       & odwracalny & nie \\
$9_{1}$    & $4$   & $2$ & $2$ & $9$   & $-8$ & $0$ & $1+10+15+7+1$ & odwracalny & tak \\
$9_{2}$    & $1$   & $5$ & $2$ & $15$  & $-2$ & $0$ & $1+4$         & odwracalny & tak \\
$9_{3}$    & $3$   & $3$ & $2$ & $19$  & $-6$ & $1$ & $1+9+9+2$     & odwracalny & tak \\
$9_{4}$    & $2$   & $4$ & $2$ & $21$  & $-4$ & $1$ & $1+7+3$       & odwracalny & tak \\
$9_{5}$    & $2$   & $5$ & $2$ & $23$  & $-2$ & $0$ & $1+6$         & odwracalny & tak \\
$9_{6}$    & $3$   & $3$ & $2$ & $27$  & $-6$ & $1$ & $1+7+8+2$     & odwracalny & tak \\
$9_{7}$    & $2$   & $4$ & $2$ & $29$  & $-4$ & $1$ & $1+5+3$       & odwracalny & tak \\
$9_{8}$    & $2$   & $5$ & $2$ & $31$  & $-2$ & $0$ & $1+0-2$       & odwracalny & tak \\
$9_{9}$    & $3$   & $3$ & $2$ & $31$  & $-6$ & $0$ & $1+8+8+2$     & odwracalny & tak \\
$9_{10}$   & $3$   & $4$ & $2$ & $33$  & $-4$ & $0$ & $1+8+4$       & odwracalny & tak \\
$9_{11}$   & $2$   & $4$ & $2$ & $33$  & $4$  & $0$ & $1+4-1-1$     & odwracalny & tak \\
$9_{12}$   & $1$   & $5$ & $2$ & $35$  & $-2$ & $1$ & $1+1-2$       & odwracalny & tak \\
$9_{13}$   & $3$   & $4$ & $2$ & $37$  & $-4$ & $1$ & $1+7+4$       & odwracalny & tak \\
$9_{14}$   & $1$   & $5$ & $2$ & $37$  & $0$  & $1$ & $1-1+2$       & odwracalny & tak \\
$9_{15}$   & $2$   & $5$ & $2$ & $39$  & $2$  & $0$ & $1+2-2$       & odwracalny & tak \\
$9_{16}$   & $3$   & $3$ & $3$ & $39$  & $-6$ & $0$ & $1+6+7+2$     & odwracalny & tak \\
$9_{17}$   & $2$   & $4$ & $2$ & $39$  & $-2$ & $0$ & $1-2+1+1$     & odwracalny & tak \\
$9_{18}$   & $2$   & $4$ & $2$ & $41$  & $-4$ & $0$ & $1+6+4$       & odwracalny & tak \\
$9_{19}$   & $1$   & $5$ & $2$ & $41$  & $0$  & $0$ & $1-2+2$       & odwracalny & tak \\
$9_{20}$   & $2$   & $4$ & $2$ & $41$  & $-4$ & $0$ & $1+2-1-1$     & odwracalny & tak \\
$9_{21}$   & $1$   & $5$ & $2$ & $43$  & $2$  & $1$ & $1+3-2$       & odwracalny & tak \\
$9_{22}$   & $1$   & $4$ & $3$ & $43$  & $-2$ & $1$ & $1-1+1+1$     & odwracalny & tak \\
$9_{23}$   & $2$   & $4$ & $2$ & $45$  & $-4$ & $1$ & $1+5+4$       & odwracalny & tak \\
$9_{24}$   & $1$   & $4$ & $3$ & $45$  & $0$  & $1$ & $1+1-1-1$     & odwracalny & tak \\
$9_{25}$   & $2$   & $5$ & $3$ & $47$  & $-2$ & $0$ & $1+0-3$       & odwracalny & tak \\
$9_{26}$   & $1$   & $4$ & $2$ & $47$  & $2$  & $0$ & $1+0+1+1$     & odwracalny & tak \\
$9_{27}$   & $1$   & $4$ & $2$ & $49$  & $0$  & $0$ & $1+0-1-1$     & odwracalny & tak \\
$9_{28}$   & $1$   & $4$ & $3$ & $51$  & $-2$ & $1$ & $1+1+1+1$     & odwracalny & tak \\
$9_{29}$   & $2$   & $4$ & $3$ & $51$  & $2$  & $1$ & $1+1+1+1$     & odwracalny & tak \\
$9_{30}$   & $1$   & $4$ & $3$ & $53$  & $0$  & $1$ & $1-1-1-1$     & odwracalny & tak \\
$9_{31}$   & $2$   & $4$ & $2$ & $55$  & $-2$ & $0$ & $1+2+1+1$     & odwracalny & tak \\
$9_{32}$   & $2$   & $4$ & $3$ & $59$  & $2$  & $1$ & $1-1+0+1$     & chiralny   & tak \\
$9_{33}$   & $1$   & $4$ & $3$ & $61$  & $0$  & $1$ & $1+1+0-1$     & chiralny   & tak \\
$9_{34}$   & $1$   & $4$ & $3$ & $69$  & $0$  & $1$ & $1-1+0-1$     & odwracalny & tak \\
$9_{35}$   & $3$   & $5$ & $3$ & $27$  & $-2$ & $1$ & $1+7$         & odwracalny & tak \\
$9_{36}$   & $2$   & $4$ & $3$ & $37$  & $4$  & $1$ & $1+3-1-1$     & odwracalny & tak \\
$9_{37}$   & $2$   & $5$ & $3$ & $45$  & $0$  & $1$ & $1-3+2$       & odwracalny & tak \\
$9_{38}$   & $3$   & $4$ & $3$ & $57$  & $-4$ & $0$ & $1+6+5$       & odwracalny & tak \\
$9_{39}$   & $1$   & $5$ & $3$ & $55$  & $2$  & $0$ & $1+2-3$       & odwracalny & tak \\
$9_{40}$   & $2$   & $4$ & $3$ & $75$  & $-2$ & $1$ & $1-1-1+1$     & odwracalny & tak \\
$9_{41}$   & $2$   & $5$ & $3$ & $49$  & $0$  & $0$ & $1+0+3$       & odwracalny & tak \\
$9_{42}$   & $1$   & $4$ & $3$ & $7$   & $2$  & $0$ & $1-2-1$       & odwracalny & nie \\
$9_{43}$   & $2$   & $4$ & $3$ & $13$  & $-4$ & $1$ & $1+1-3-1$     & odwracalny & nie \\
$9_{44}$   & $1$   & $4$ & $3$ & $17$  & $0$  & $0$ & $1+0+1$       & odwracalny & nie \\
$9_{45}$   & $1$   & $4$ & $3$ & $23$  & $2$  & $0$ & $1+2-1$       & odwracalny & nie \\
$9_{46}$   & $2$   & $4$ & $3$ & $9$   & $0$  & $0$ & $1-2$         & odwracalny & nie \\
$9_{47}$   & $2$   & $4$ & $3$ & $27$  & $-2$ & $1$ & $1-1+2+1$     & odwracalny & nie \\
$9_{48}$   & $2$   & $4$ & $3$ & $27$  & $2$  & $1$ & $1+3-1$       & odwracalny & nie \\
$9_{49}$   & $3$   & $4$ & $3$ & $25$  & $-4$ & $0$ & $1+6+3$       & odwracalny & nie \\
$10_{1}$   & $1$   & $6$ & $2$ & $17$  & $0$  & $0$ & $1-4$         & odwracalny & tak \\
$10_{2}$   & $3$   & $3$ & $2$ & $23$  & $-6$ & $0$ & $1+2-5-5-1$   & odwracalny & tak \\
$10_{3}$   & $2$   & $6$ & $2$ & $25$  & $0$  & $0$ & $1-6$         & odwracalny & tak \\
$10_{4}$   & $2$   & $5$ & $2$ & $27$  & $2$  & $1$ & $1-5-3$       & odwracalny & tak \\
$10_{5}$   & $2$   & $3$ & $2$ & $33$  & $4$  & $0$ & $1+4+7+5+1$   & odwracalny & tak \\
$10_{6}$   & $3$   & $4$ & $2$ & $37$  & $-4$ & $1$ & $1-1-6-2$     & odwracalny & tak \\
$10_{7}$   & $1$   & $5$ & $2$ & $43$  & $-2$ & $1$ & $1-1-3$       & odwracalny & tak \\
$10_{8}$   & $2$   & $4$ & $2$ & $29$  & $-4$ & $1$ & $1-3-7-2$     & odwracalny & tak \\
$10_{9}$   & $1$   & $3$ & $2$ & $39$  & $-2$ & $0$ & $1-2-7-5-1$   & odwracalny & tak \\
$10_{10}$  & $1$   & $5$ & $2$ & $45$  & $0$  & $1$ & $1+1+3$       & odwracalny & tak \\
$10_{11}$  & $2/3$ & $5$ & $2$ & $43$  & $-2$ & $1$ & $1-5-4$       & odwracalny & tak \\
$10_{12}$  & $2$   & $4$ & $2$ & $47$  & $2$  & $0$ & $1+4+6+2$     & odwracalny & tak \\
$10_{13}$  & $2$   & $6$ & $2$ & $53$  & $0$  & $1$ & $1-5+2$       & odwracalny & tak \\
$10_{14}$  & $2$   & $4$ & $2$ & $57$  & $-4$ & $0$ & $1+2-4-2$     & odwracalny & tak \\
$10_{15}$  & $2$   & $4$ & $2$ & $43$  & $2$  & $1$ & $1+3+6+2$     & odwracalny & tak \\
$10_{16}$  & $2$   & $5$ & $2$ & $47$  & $-2$ & $0$ & $1-4-4$       & odwracalny & tak \\
$10_{17}$  & $1$   & $3$ & $2$ & $41$  & $0$  & $0$ & $1+2+7+5+1$   & całkowicie & tak \\
$10_{18}$  & $1$   & $5$ & $2$ & $55$  & $-2$ & $0$ & $1-2-4$       & odwracalny & tak \\
$10_{19}$  & $2$   & $4$ & $2$ & $51$  & $-2$ & $1$ & $1+1+5+2$     & odwracalny & tak \\
$10_{20}$  & $2$   & $5$ & $2$ & $35$  & $-2$ & $1$ & $1-3-3$       & odwracalny & tak \\
$10_{21}$  & $2$   & $4$ & $2$ & $45$  & $-4$ & $1$ & $1+1-5-2$     & odwracalny & tak \\
$10_{22}$  & $2$   & $4$ & $2$ & $49$  & $0$  & $0$ & $1-4-6-2$     & odwracalny & tak \\
$10_{23}$  & $1$   & $4$ & $2$ & $59$  & $2$  & $1$ & $1+3+5+2$     & odwracalny & tak \\
$10_{24}$  & $2$   & $5$ & $2$ & $55$  & $-2$ & $0$ & $1-2-4$       & odwracalny & tak \\
$10_{25}$  & $2$   & $4$ & $2$ & $65$  & $-4$ & $0$ & $1+0-4-2$     & odwracalny & tak \\
$10_{26}$  & $1$   & $4$ & $2$ & $61$  & $0$  & $1$ & $1-3-5-2$     & odwracalny & tak \\
$10_{27}$  & $1$   & $4$ & $2$ & $71$  & $2$  & $0$ & $1+2+4+2$     & odwracalny & tak \\
$10_{28}$  & $2$   & $5$ & $2$ & $53$  & $0$  & $1$ & $1+3+4$       & odwracalny & tak \\
$10_{29}$  & $2$   & $5$ & $2$ & $63$  & $-2$ & $0$ & $1-4-1+1$     & odwracalny & tak \\
$10_{30}$  & $1$   & $5$ & $2$ & $67$  & $-2$ & $1$ & $1+1-4$       & odwracalny & tak \\
$10_{31}$  & $1$   & $5$ & $2$ & $57$  & $0$  & $0$ & $1+2+4$       & odwracalny & tak \\
$10_{32}$  & $1$   & $4$ & $2$ & $69$  & $0$  & $1$ & $1-1-4-2$     & odwracalny & tak \\
$10_{33}$  & $1$   & $5$ & $2$ & $65$  & $0$  & $0$ & $1+0+4$       & całkowicie & tak \\
$10_{34}$  & $2$   & $5$ & $2$ & $37$  & $0$  & $1$ & $1+3+3$       & odwracalny & tak \\
$10_{35}$  & $2$   & $6$ & $2$ & $49$  & $0$  & $0$ & $1-4+2$       & odwracalny & tak \\
$10_{36}$  & $2$   & $5$ & $2$ & $51$  & $-2$ & $1$ & $1+1-3$       & odwracalny & tak \\
$10_{37}$  & $2$   & $5$ & $2$ & $53$  & $0$  & $1$ & $1+3+4$       & całkowicie & tak \\
$10_{38}$  & $2$   & $5$ & $2$ & $59$  & $-2$ & $1$ & $1-1-4$       & odwracalny & tak \\
$10_{39}$  & $2$   & $4$ & $2$ & $61$  & $-4$ & $1$ & $1+1-4-2$     & odwracalny & tak \\
$10_{40}$  & $2$   & $4$ & $2$ & $75$  & $2$  & $1$ & $1+3+4+2$     & odwracalny & tak \\
$10_{41}$  & $2$   & $5$ & $2$ & $71$  & $-2$ & $0$ & $1-2-1+1$     & odwracalny & tak \\
$10_{42}$  & $1$   & $5$ & $2$ & $81$  & $0$  & $0$ & $1+0+1-1$     & odwracalny & tak \\
$10_{43}$  & $2$   & $5$ & $2$ & $73$  & $0$  & $0$ & $1+2+1-1$     & całkowicie & tak \\
$10_{44}$  & $1$   & $5$ & $2$ & $79$  & $-2$ & $0$ & $1+0-1+1$     & odwracalny & tak \\
$10_{45}$  & $2$   & $5$ & $2$ & $89$  & $0$  & $0$ & $1-2+1-1$     & całkowicie & tak \\
$10_{46}$  & $3$   & $3$ & $3$ & $31$  & $-6$ & $0$ & $1+0-6-5-1$   & odwracalny & tak \\
$10_{47}$  & $2/3$ & $3$ & $3$ & $41$  & $4$  & $0$ & $1+6+8+5+1$   & odwracalny & tak \\
$10_{48}$  & $2$   & $3$ & $3$ & $49$  & $0$  & $0$ & $1+4+8+5+1$   & odwracalny & tak \\
$10_{49}$  & $3$   & $4$ & $3$ & $59$  & $-6$ & $1$ & $1+7+10+3$    & odwracalny & tak \\
$10_{50}$  & $2$   & $4$ & $3$ & $53$  & $-4$ & $1$ & $1-1-5-2$     & odwracalny & tak \\
$10_{51}$  & $2/3$ & $4$ & $3$ & $67$  & $2$  & $1$ & $1+5+5+2$     & odwracalny & tak \\
$10_{52}$  & $2$   & $4$ & $3$ & $59$  & $-2$ & $1$ & $1+3+5+2$     & odwracalny & tak \\
$10_{53}$  & $3$   & $5$ & $3$ & $73$  & $-4$ & $0$ & $1+6+6$       & odwracalny & tak \\
$10_{54}$  & $2/3$ & $4$ & $3$ & $47$  & $2$  & $0$ & $1+4+6+2$     & odwracalny & tak \\
$10_{55}$  & $2$   & $5$ & $3$ & $61$  & $-4$ & $1$ & $1+5+5$       & odwracalny & tak \\
$10_{56}$  & $2$   & $4$ & $3$ & $65$  & $-4$ & $0$ & $1+0-4-2$     & odwracalny & tak \\
$10_{57}$  & $2$   & $4$ & $3$ & $79$  & $2$  & $0$ & $1+4+4+2$     & odwracalny & tak \\
$10_{58}$  & $2$   & $6$ & $3$ & $65$  & $0$  & $0$ & $1-4+3$       & odwracalny & tak \\
$10_{59}$  & $1$   & $5$ & $3$ & $75$  & $-2$ & $1$ & $1-1-1+1$     & odwracalny & tak \\
$10_{60}$  & $1$   & $5$ & $3$ & $85$  & $0$  & $1$ & $1-1+1-1$     & odwracalny & tak \\
$10_{61}$  & $2/3$ & $4$ & $3$ & $33$  & $-4$ & $0$ & $1-4-7-2$     & odwracalny & tak \\
$10_{62}$  & $2$   & $3$ & $3$ & $45$  & $4$  & $1$ & $1+5+8+5+1$   & odwracalny & tak \\
$10_{63}$  & $2$   & $5$ & $3$ & $57$  & $-4$ & $0$ & $1+6+5$       & odwracalny & tak \\
$10_{64}$  & $2$   & $3$ & $3$ & $51$  & $-2$ & $1$ & $1-3-8-5-1$   & odwracalny & tak \\
$10_{65}$  & $2$   & $4$ & $3$ & $63$  & $2$  & $0$ & $1+4+5+2$     & odwracalny & tak \\
$10_{66}$  & $3$   & $4$ & $3$ & $75$  & $-6$ & $1$ & $1+7+9+3$     & odwracalny & tak \\
$10_{67}$  & $2$   & $5$ & $3$ & $63$  & $-2$ & $0$ & $1+0-4$       & chiralny   & tak \\
$10_{68}$  & $2$   & $5$ & $3$ & $57$  & $0$  & $0$ & $1+2+4$       & odwracalny & tak \\
$10_{69}$  & $2$   & $5$ & $3$ & $87$  & $2$  & $0$ & $1+2-1+1$     & odwracalny & tak \\
$10_{70}$  & $2$   & $5$ & $3$ & $67$  & $2$  & $1$ & $1-3-1+1$     & odwracalny & tak \\
$10_{71}$  & $1$   & $5$ & $3$ & $77$  & $0$  & $1$ & $1+1+1-1$     & odwracalny & tak \\
$10_{72}$  & $2$   & $4$ & $3$ & $73$  & $-4$ & $0$ & $1+2-3-2$     & odwracalny & tak \\
$10_{73}$  & $1$   & $5$ & $3$ & $83$  & $2$  & $1$ & $1+1-1+1$     & odwracalny & tak \\
$10_{74}$  & $2$   & $5$ & $3$ & $63$  & $-2$ & $0$ & $1+0-4$       & odwracalny & tak \\
$10_{75}$  & $2$   & $5$ & $3$ & $81$  & $0$  & $0$ & $1+0+1-1$     & odwracalny & tak \\
$10_{76}$  & $2/3$ & $4$ & $3$ & $57$  & $-4$ & $0$ & $1-2-5-2$     & odwracalny & tak \\
$10_{77}$  & $2/3$ & $4$ & $3$ & $63$  & $2$  & $0$ & $1+4+5+2$     & odwracalny & tak \\
$10_{78}$  & $2$   & $5$ & $3$ & $69$  & $-4$ & $1$ & $1+3+1-1$     & odwracalny & tak \\
$10_{79}$  & $2/3$ & $3$ & $3$ & $61$  & $0$  & $1$ & $1+5+9+5+1$   & ujemny     & tak \\
$10_{80}$  & $3$   & $4$ & $3$ & $71$  & $-6$ & $0$ & $1+6+9+3$     & chiralny   & tak \\
$10_{81}$  & $2$   & $5$ & $3$ & $85$  & $0$  & $1$ & $1+3+2-1$     & ujemny     & tak \\
$10_{82}$  & $1$   & $3$ & $3$ & $63$  & $-2$ & $0$ & $1+0-4-4-1$   & chiralny   & tak \\
$10_{83}$  & $2$   & $4$ & $3$ & $83$  & $2$  & $1$ & $1+1+3+2$     & chiralny   & tak \\
$10_{84}$  & $1$   & $4$ & $3$ & $87$  & $-2$ & $0$ & $1+2+3+2$     & chiralny   & tak \\
$10_{85}$  & $2$   & $3$ & $3$ & $57$  & $4$  & $0$ & $1+2+4+4+1$   & chiralny   & tak \\
$10_{86}$  & $2$   & $4$ & $3$ & $85$  & $0$  & $1$ & $1-1-3-2$     & chiralny   & tak \\
$10_{87}$  & $2$   & $4$ & $3$ & $81$  & $0$  & $0$ & $1+0-3-2$     & chiralny   & tak \\
$10_{88}$  & $1$   & $5$ & $3$ & $101$ & $0$  & $1$ & $1-1+2-1$     & ujemny     & tak \\
$10_{89}$  & $2$   & $5$ & $3$ & $99$  & $2$  & $1$ & $1+1-2+1$     & odwracalny & tak \\
$10_{90}$  & $2$   & $4$ & $3$ & $77$  & $0$  & $1$ & $1-3-4-2$     & chiralny   & tak \\
$10_{91}$  & $1$   & $3$ & $3$ & $73$  & $0$  & $0$ & $1+2+5+4+1$   & chiralny   & tak \\
$10_{92}$  & $2$   & $4$ & $3$ & $89$  & $-4$ & $0$ & $1+2-2-2$     & chiralny   & tak \\
$10_{93}$  & $2$   & $4$ & $3$ & $67$  & $2$  & $1$ & $1+1+4+2$     & chiralny   & tak \\
$10_{94}$  & $2$   & $3$ & $3$ & $71$  & $-2$ & $0$ & $1-2-5-4-1$   & chiralny   & tak \\
$10_{95}$  & $1$   & $4$ & $3$ & $91$  & $2$  & $1$ & $1+3+3+2$     & chiralny   & tak \\
$10_{96}$  & $2$   & $5$ & $3$ & $93$  & $0$  & $1$ & $1-3+1-1$     & odwracalny & tak \\
$10_{97}$  & $2$   & $5$ & $3$ & $87$  & $-2$ & $0$ & $1+2-5$       & odwracalny & tak \\
$10_{98}$  & $2$   & $4$ & $3$ & $81$  & $-4$ & $0$ & $1+0-3-2$     & chiralny   & tak \\
$10_{99}$  & $2$   & $3$ & $3$ & $81$  & $0$  & $0$ & $1+4+6+4+1$   & całkowicie & tak \\
$10_{100}$ & $2/3$ & $3$ & $3$ & $65$  & $4$  & $0$ & $1+4+5+4+1$   & odwracalny & tak \\
$10_{101}$ & $3$   & $5$ & $3$ & $85$  & $-4$ & $1$ & $1+7+7$       & odwracalny & tak \\
$10_{102}$ & $1$   & $4$ & $3$ & $73$  & $0$  & $0$ & $1-2-4-2$     & chiralny   & tak \\
$10_{103}$ & $3$   & $4$ & $3$ & $75$  & $2$  & $1$ & $1+3+4+2$     & odwracalny & tak \\
$10_{104}$ & $1$   & $3$ & $3$ & $77$  & $0$  & $1$ & $1+1+5+4+1$   & odwracalny & tak \\
$10_{105}$ & $2$   & $5$ & $3$ & $91$  & $-2$ & $1$ & $1-1-2+1$     & odwracalny & tak \\
$10_{106}$ & $2$   & $3$ & $3$ & $75$  & $-2$ & $1$ & $1-1-5-4-1$   & chiralny   & tak \\
$10_{107}$ & $1$   & $5$ & $3$ & $93$  & $0$  & $1$ & $1+1+2-1$     & chiralny   & tak \\
$10_{108}$ & $2$   & $4$ & $3$ & $63$  & $-2$ & $0$ & $1+0+4+2$     & odwracalny & tak \\
$10_{109}$ & $2$   & $3$ & $3$ & $85$  & $0$  & $1$ & $1+3+6+4+1$   & ujemny     & tak \\
$10_{110}$ & $2$   & $5$ & $3$ & $83$  & $-2$ & $1$ & $1-3-2+1$     & chiralny   & tak \\
$10_{111}$ & $2$   & $4$ & $3$ & $77$  & $-4$ & $1$ & $1+1-3-2$     & odwracalny & tak \\
$10_{112}$ & $2$   & $3$ & $3$ & $87$  & $2$  & $0$ & $1+2-1-3-1$   & odwracalny & tak \\
$10_{113}$ & $1$   & $4$ & $3$ & $111$ & $-2$ & $0$ & $1+0+1+2$     & odwracalny & tak \\
$10_{114}$ & $1$   & $4$ & $3$ & $93$  & $0$  & $1$ & $1+1-2-2$     & odwracalny & tak \\
$10_{115}$ & $2$   & $5$ & $3$ & $109$ & $0$  & $1$ & $1+1+3-1$     & ujemny     & tak \\
$10_{116}$ & $2$   & $3$ & $3$ & $95$  & $2$  & $0$ & $1+0-2-3-1$   & odwracalny & tak \\
$10_{117}$ & $2$   & $4$ & $3$ & $103$ & $2$  & $0$ & $1+2+2+2$     & chiralny   & tak \\
$10_{118}$ & $1$   & $3$ & $3$ & $97$  & $0$  & $0$ & $1+0+2+3+1$   & ujemny     & tak \\
$10_{119}$ & $1$   & $4$ & $3$ & $101$ & $0$  & $1$ & $1-1-2-2$     & chiralny   & tak \\
$10_{120}$ & $3$   & $5$ & $3$ & $105$ & $-4$ & $0$ & $1+6+8$       & odwracalny & tak \\
$10_{121}$ & $2$   & $4$ & $3$ & $115$ & $-2$ & $1$ & $1+1+1+2$     & odwracalny & tak \\
$10_{122}$ & $2$   & $4$ & $3$ & $105$ & $0$  & $0$ & $1+2-1-2$     & odwracalny & tak \\
$10_{123}$ & $2$   & $3$ & $3$ & $121$ & $0$  & $0$ & $1-2-1+2+1$   & całkowicie & tak \\
$10_{124}$ & $4$   & $3$ & $3$ & $1$   & $-8$ & $0$ & $1+8+14+7+1$  & odwracalny & nie \\
$10_{125}$ & $2$   & $3$ & $3$ & $11$  & $2$  & $1$ & $1+3+4+1$     & odwracalny & nie \\
$10_{126}$ & $2$   & $3$ & $3$ & $19$  & $2$  & $1$ & $1+5+4+1$     & odwracalny & nie \\
$10_{127}$ & $2$   & $3$ & $3$ & $29$  & $-4$ & $1$ & $1+1-2-1$     & odwracalny & nie \\
$10_{128}$ & $3$   & $4$ & $3$ & $11$  & $-6$ & $1$ & $1+7+9+2$     & odwracalny & nie \\
$10_{129}$ & $1$   & $4$ & $3$ & $25$  & $0$  & $0$ & $1+2+2$       & odwracalny & nie \\
$10_{130}$ & $2$   & $4$ & $3$ & $17$  & $0$  & $0$ & $1+4+2$       & odwracalny & nie \\
$10_{131}$ & $1$   & $4$ & $3$ & $31$  & $-2$ & $0$ & $1+0-2$       & odwracalny & nie \\
$10_{132}$ & $1$   & $4$ & $3$ & $5$   & $0$  & $1$ & $1+3+1$       & odwracalny & nie \\
$10_{133}$ & $1$   & $4$ & $3$ & $19$  & $-2$ & $1$ & $1+1-1$       & odwracalny & nie \\
$10_{134}$ & $3$   & $4$ & $3$ & $23$  & $-6$ & $0$ & $1+6+8+2$     & odwracalny & nie \\
$10_{135}$ & $2$   & $4$ & $3$ & $37$  & $0$  & $1$ & $1+3+3$       & odwracalny & nie \\
$10_{136}$ & $1$   & $4$ & $3$ & $15$  & $2$  & $0$ & $1+0-1$       & odwracalny & nie \\
$10_{137}$ & $1$   & $5$ & $3$ & $25$  & $0$  & $0$ & $1-2+1$       & odwracalny & nie \\
$10_{138}$ & $2$   & $5$ & $3$ & $35$  & $-2$ & $1$ & $1-3+1+1$     & odwracalny & nie \\
$10_{139}$ & $4$   & $3$ & $3$ & $3$   & $-6$ & $1$ & $1+9+14+7+1$  & odwracalny & nie \\
$10_{140}$ & $2$   & $4$ & $3$ & $9$   & $0$  & $0$ & $1+2+1$       & odwracalny & nie \\
$10_{141}$ & $1$   & $3$ & $3$ & $21$  & $0$  & $1$ & $1-1-3-1$     & odwracalny & nie \\
$10_{142}$ & $3$   & $4$ & $3$ & $15$  & $-6$ & $0$ & $1+8+9+2$     & odwracalny & nie \\
$10_{143}$ & $1$   & $3$ & $3$ & $27$  & $2$  & $1$ & $1+3+3+1$     & odwracalny & nie \\
$10_{144}$ & $2$   & $4$ & $3$ & $39$  & $-2$ & $0$ & $1-2-3$       & odwracalny & nie \\
$10_{145}$ & $2$   & $4$ & $3$ & $3$   & $2$  & $1$ & $1+5+1$       & odwracalny & nie \\
$10_{146}$ & $1$   & $4$ & $3$ & $33$  & $0$  & $0$ & $1+0+2$       & odwracalny & nie \\
$10_{147}$ & $1$   & $4$ & $3$ & $27$  & $-2$ & $1$ & $1-1-2$       & chiralny   & nie \\
$10_{148}$ & $2$   & $3$ & $3$ & $31$  & $2$  & $0$ & $1+4+3+1$     & chiralny   & nie \\
$10_{149}$ & $2$   & $3$ & $3$ & $41$  & $-4$ & $0$ & $1+2-1-1$     & chiralny   & nie \\
$10_{150}$ & $2$   & $4$ & $3$ & $29$  & $-4$ & $1$ & $1+1-2-1$     & chiralny   & nie \\
$10_{151}$ & $2$   & $4$ & $3$ & $43$  & $2$  & $1$ & $1+3+2+1$     & chiralny   & nie \\
$10_{152}$ & $4$   & $3$ & $3$ & $11$  & $-6$ & $1$ & $1+7+13+7+1$  & odwracalny & nie \\
$10_{153}$ & $2$   & $4$ & $3$ & $1$   & $0$  & $0$ & $1+4+5+1$     & chiralny   & nie \\
$10_{154}$ & $3$   & $4$ & $3$ & $13$  & $-4$ & $1$ & $1+5+6+1$     & odwracalny & nie \\
$10_{155}$ & $2$   & $3$ & $3$ & $25$  & $0$  & $0$ & $1-2-3-1$     & odwracalny & nie \\
$10_{156}$ & $1$   & $4$ & $3$ & $35$  & $2$  & $1$ & $1+1+2+1$     & odwracalny & nie \\
$10_{157}$ & $2$   & $3$ & $3$ & $49$  & $4$  & $0$ & $1+4+0-1$     & odwracalny & nie \\
$10_{158}$ & $2$   & $4$ & $3$ & $45$  & $0$  & $1$ & $1-3-2-1$     & odwracalny & nie \\
$10_{159}$ & $1$   & $3$ & $3$ & $39$  & $-2$ & $0$ & $1+2+2+1$     & odwracalny & nie \\
$10_{160}$ & $2$   & $4$ & $3$ & $21$  & $-4$ & $1$ & $1+3-2-1$     & odwracalny & nie \\
$10_{161}$ & $3$   & $3$ & $3$ & $5$   & $-4$ & $1$ & $1+7+6+1$     & odwracalny & nie \\
$10_{162}$ & $2$   & $4$ & $3$ & $35$  & $2$  & $1$ & $1-3-3$       & odwracalny & nie \\
$10_{163}$ & $2$   & $4$ & $3$ & $51$  & $-2$ & $1$ & $1+1+1+1$     & odwracalny & nie \\
$10_{164}$ & $1$   & $4$ & $3$ & $45$  & $0$  & $1$ & $1+1+3$       & odwracalny & nie \\
$10_{165}$ & $2$   & $4$ & $3$ & $39$  & $2$  & $0$ & $1+2-2$       & odwracalny & nie \\
\hline
\end{longtable}
\normalsize
\section{Diagramy węzłów pierwszych}
\label{sec:table_of_prime_knots}
Poniżej znajdują się diagramy węzłów pierwszych, które realizują liczbę gordyjską, jeśli ta nie przekracza dziesięciu.
One także pochodzą ze strony \url{http://www.indiana.edu/~knotinfo/}, o której mowa na początku rozdziału.

\begin{figure}[H]
	\begin{minipage}[b]{.18\linewidth}
		\centering
		\includegraphics[width=\linewidth]{../data/3_1.png}
		\subcaption{$3_{1}$}
	\end{minipage}
	\begin{minipage}[b]{.18\linewidth}
		\centering
		\includegraphics[width=\linewidth]{../data/4_1.png}
		\subcaption{$4_{1}$}
	\end{minipage}
	\begin{minipage}[b]{.18\linewidth}
		\centering
		\includegraphics[width=\linewidth]{../data/5_1.png}
		\subcaption{$5_{1}$}
	\end{minipage}
	\begin{minipage}[b]{.18\linewidth}
		\centering
		\includegraphics[width=\linewidth]{../data/5_2.png}
		\subcaption{$5_{2}$}
	\end{minipage}
	\begin{minipage}[b]{.18\linewidth}
		\centering
		\includegraphics[width=\linewidth]{../data/6_1.png}
		\subcaption{$6_{1}$}
	\end{minipage}
\end{figure}
\begin{figure}[H]
	\begin{minipage}[b]{.18\linewidth}
		\centering
		\includegraphics[width=\linewidth]{../data/6_2.png}
		\subcaption{$6_{2}$}
	\end{minipage}
	\begin{minipage}[b]{.18\linewidth}
		\centering
		\includegraphics[width=\linewidth]{../data/6_3.png}
		\subcaption{$6_{3}$}
	\end{minipage}
	\begin{minipage}[b]{.18\linewidth}
		\centering
		\includegraphics[width=\linewidth]{../data/7_1.png}
		\subcaption{$7_{1}$}
	\end{minipage}
	\begin{minipage}[b]{.18\linewidth}
		\centering
		\includegraphics[width=\linewidth]{../data/7_2.png}
		\subcaption{$7_{2}$}
	\end{minipage}
	\begin{minipage}[b]{.18\linewidth}
		\centering
		\includegraphics[width=\linewidth]{../data/7_3.png}
		\subcaption{$7_{3}$}
	\end{minipage}
\end{figure}
\begin{figure}[H]
	\begin{minipage}[b]{.18\linewidth}
		\centering
		\includegraphics[width=\linewidth]{../data/7_4.png}
		\subcaption{$7_{4}$}
	\end{minipage}
	\begin{minipage}[b]{.18\linewidth}
		\centering
		\includegraphics[width=\linewidth]{../data/7_5.png}
		\subcaption{$7_{5}$}
	\end{minipage}
	\begin{minipage}[b]{.18\linewidth}
		\centering
		\includegraphics[width=\linewidth]{../data/7_6.png}
		\subcaption{$7_{6}$}
	\end{minipage}
	\begin{minipage}[b]{.18\linewidth}
		\centering
		\includegraphics[width=\linewidth]{../data/7_7.png}
		\subcaption{$7_{7}$}
	\end{minipage}
	\begin{minipage}[b]{.18\linewidth}
		\centering
		\includegraphics[width=\linewidth]{../data/8_1.png}
		\subcaption{$8_{1}$}
	\end{minipage}
\end{figure}
\begin{figure}[H]
	\begin{minipage}[b]{.18\linewidth}
		\centering
		\includegraphics[width=\linewidth]{../data/8_2.png}
		\subcaption{$8_{2}$}
	\end{minipage}
	\begin{minipage}[b]{.18\linewidth}
		\centering
		\includegraphics[width=\linewidth]{../data/8_3.png}
		\subcaption{$8_{3}$}
	\end{minipage}
	\begin{minipage}[b]{.18\linewidth}
		\centering
		\includegraphics[width=\linewidth]{../data/8_4.png}
		\subcaption{$8_{4}$}
	\end{minipage}
	\begin{minipage}[b]{.18\linewidth}
		\centering
		\includegraphics[width=\linewidth]{../data/8_5.png}
		\subcaption{$8_{5}$}
	\end{minipage}
	\begin{minipage}[b]{.18\linewidth}
		\centering
		\includegraphics[width=\linewidth]{../data/8_6.png}
		\subcaption{$8_{6}$}
	\end{minipage}
\end{figure}
\begin{figure}[H]
	\begin{minipage}[b]{.18\linewidth}
		\centering
		\includegraphics[width=\linewidth]{../data/8_7.png}
		\subcaption{$8_{7}$}
	\end{minipage}
	\begin{minipage}[b]{.18\linewidth}
		\centering
		\includegraphics[width=\linewidth]{../data/8_8.png}
		\subcaption{$8_{8}$}
	\end{minipage}
	\begin{minipage}[b]{.18\linewidth}
		\centering
		\includegraphics[width=\linewidth]{../data/8_9.png}
		\subcaption{$8_{9}$}
	\end{minipage}
	\begin{minipage}[b]{.18\linewidth}
		\centering
		\includegraphics[width=\linewidth]{../data/8_10.png}
		\subcaption{$8_{10}$}
	\end{minipage}
	\begin{minipage}[b]{.18\linewidth}
		\centering
		\includegraphics[width=\linewidth]{../data/8_11.png}
		\subcaption{$8_{11}$}
	\end{minipage}
\end{figure}
\begin{figure}[H]
	\begin{minipage}[b]{.18\linewidth}
		\centering
		\includegraphics[width=\linewidth]{../data/8_12.png}
		\subcaption{$8_{12}$}
	\end{minipage}
	\begin{minipage}[b]{.18\linewidth}
		\centering
		\includegraphics[width=\linewidth]{../data/8_13.png}
		\subcaption{$8_{13}$}
	\end{minipage}
	\begin{minipage}[b]{.18\linewidth}
		\centering
		\includegraphics[width=\linewidth]{../data/8_14.png}
		\subcaption{$8_{14}$}
	\end{minipage}
	\begin{minipage}[b]{.18\linewidth}
		\centering
		\includegraphics[width=\linewidth]{../data/8_15.png}
		\subcaption{$8_{15}$}
	\end{minipage}
	\begin{minipage}[b]{.18\linewidth}
		\centering
		\includegraphics[width=\linewidth]{../data/8_16.png}
		\subcaption{$8_{16}$}
	\end{minipage}
\end{figure}
\begin{figure}[H]
	\begin{minipage}[b]{.18\linewidth}
		\centering
		\includegraphics[width=\linewidth]{../data/8_17.png}
		\subcaption{$8_{17}$}
	\end{minipage}
	\begin{minipage}[b]{.18\linewidth}
		\centering
		\includegraphics[width=\linewidth]{../data/8_18.png}
		\subcaption{$8_{18}$}
	\end{minipage}
	\begin{minipage}[b]{.18\linewidth}
		\centering
		\includegraphics[width=\linewidth]{../data/8_19.png}
		\subcaption{$8_{19}$}
	\end{minipage}
	\begin{minipage}[b]{.18\linewidth}
		\centering
		\includegraphics[width=\linewidth]{../data/8_20.png}
		\subcaption{$8_{20}$}
	\end{minipage}
	\begin{minipage}[b]{.18\linewidth}
		\centering
		\includegraphics[width=\linewidth]{../data/8_21.png}
		\subcaption{$8_{21}$}
	\end{minipage}
\end{figure}
\begin{figure}[H]
	\begin{minipage}[b]{.18\linewidth}
		\centering
		\includegraphics[width=\linewidth]{../data/9_1.png}
		\subcaption{$9_{1}$}
	\end{minipage}
	\begin{minipage}[b]{.18\linewidth}
		\centering
		\includegraphics[width=\linewidth]{../data/9_2.png}
		\subcaption{$9_{2}$}
	\end{minipage}
	\begin{minipage}[b]{.18\linewidth}
		\centering
		\includegraphics[width=\linewidth]{../data/9_3.png}
		\subcaption{$9_{3}$}
	\end{minipage}
	\begin{minipage}[b]{.18\linewidth}
		\centering
		\includegraphics[width=\linewidth]{../data/9_4.png}
		\subcaption{$9_{4}$}
	\end{minipage}
	\begin{minipage}[b]{.18\linewidth}
		\centering
		\includegraphics[width=\linewidth]{../data/9_5.png}
		\subcaption{$9_{5}$}
	\end{minipage}
\end{figure}
\begin{figure}[H]
	\begin{minipage}[b]{.18\linewidth}
		\centering
		\includegraphics[width=\linewidth]{../data/9_6.png}
		\subcaption{$9_{6}$}
	\end{minipage}
	\begin{minipage}[b]{.18\linewidth}
		\centering
		\includegraphics[width=\linewidth]{../data/9_7.png}
		\subcaption{$9_{7}$}
	\end{minipage}
	\begin{minipage}[b]{.18\linewidth}
		\centering
		\includegraphics[width=\linewidth]{../data/9_8.png}
		\subcaption{$9_{8}$}
	\end{minipage}
	\begin{minipage}[b]{.18\linewidth}
		\centering
		\includegraphics[width=\linewidth]{../data/9_9.png}
		\subcaption{$9_{9}$}
	\end{minipage}
	\begin{minipage}[b]{.18\linewidth}
		\centering
		\includegraphics[width=\linewidth]{../data/9_10.png}
		\subcaption{$9_{10}$}
	\end{minipage}
\end{figure}
\begin{figure}[H]
	\begin{minipage}[b]{.18\linewidth}
		\centering
		\includegraphics[width=\linewidth]{../data/9_11.png}
		\subcaption{$9_{11}$}
	\end{minipage}
	\begin{minipage}[b]{.18\linewidth}
		\centering
		\includegraphics[width=\linewidth]{../data/9_12.png}
		\subcaption{$9_{12}$}
	\end{minipage}
	\begin{minipage}[b]{.18\linewidth}
		\centering
		\includegraphics[width=\linewidth]{../data/9_13.png}
		\subcaption{$9_{13}$}
	\end{minipage}
	\begin{minipage}[b]{.18\linewidth}
		\centering
		\includegraphics[width=\linewidth]{../data/9_14.png}
		\subcaption{$9_{14}$}
	\end{minipage}
	\begin{minipage}[b]{.18\linewidth}
		\centering
		\includegraphics[width=\linewidth]{../data/9_15.png}
		\subcaption{$9_{15}$}
	\end{minipage}
\end{figure}
\begin{figure}[H]
	\begin{minipage}[b]{.18\linewidth}
		\centering
		\includegraphics[width=\linewidth]{../data/9_16.png}
		\subcaption{$9_{16}$}
	\end{minipage}
	\begin{minipage}[b]{.18\linewidth}
		\centering
		\includegraphics[width=\linewidth]{../data/9_17.png}
		\subcaption{$9_{17}$}
	\end{minipage}
	\begin{minipage}[b]{.18\linewidth}
		\centering
		\includegraphics[width=\linewidth]{../data/9_18.png}
		\subcaption{$9_{18}$}
	\end{minipage}
	\begin{minipage}[b]{.18\linewidth}
		\centering
		\includegraphics[width=\linewidth]{../data/9_19.png}
		\subcaption{$9_{19}$}
	\end{minipage}
	\begin{minipage}[b]{.18\linewidth}
		\centering
		\includegraphics[width=\linewidth]{../data/9_20.png}
		\subcaption{$9_{20}$}
	\end{minipage}
\end{figure}
\begin{figure}[H]
	\begin{minipage}[b]{.18\linewidth}
		\centering
		\includegraphics[width=\linewidth]{../data/9_21.png}
		\subcaption{$9_{21}$}
	\end{minipage}
	\begin{minipage}[b]{.18\linewidth}
		\centering
		\includegraphics[width=\linewidth]{../data/9_22.png}
		\subcaption{$9_{22}$}
	\end{minipage}
	\begin{minipage}[b]{.18\linewidth}
		\centering
		\includegraphics[width=\linewidth]{../data/9_23.png}
		\subcaption{$9_{23}$}
	\end{minipage}
	\begin{minipage}[b]{.18\linewidth}
		\centering
		\includegraphics[width=\linewidth]{../data/9_24.png}
		\subcaption{$9_{24}$}
	\end{minipage}
	\begin{minipage}[b]{.18\linewidth}
		\centering
		\includegraphics[width=\linewidth]{../data/9_25.png}
		\subcaption{$9_{25}$}
	\end{minipage}
\end{figure}
\begin{figure}[H]
	\begin{minipage}[b]{.18\linewidth}
		\centering
		\includegraphics[width=\linewidth]{../data/9_26.png}
		\subcaption{$9_{26}$}
	\end{minipage}
	\begin{minipage}[b]{.18\linewidth}
		\centering
		\includegraphics[width=\linewidth]{../data/9_27.png}
		\subcaption{$9_{27}$}
	\end{minipage}
	\begin{minipage}[b]{.18\linewidth}
		\centering
		\includegraphics[width=\linewidth]{../data/9_28.png}
		\subcaption{$9_{28}$}
	\end{minipage}
	\begin{minipage}[b]{.18\linewidth}
		\centering
		\includegraphics[width=\linewidth]{../data/9_29.png}
		\subcaption{$9_{29}$}
	\end{minipage}
	\begin{minipage}[b]{.18\linewidth}
		\centering
		\includegraphics[width=\linewidth]{../data/9_30.png}
		\subcaption{$9_{30}$}
	\end{minipage}
\end{figure}
\begin{figure}[H]
	\begin{minipage}[b]{.18\linewidth}
		\centering
		\includegraphics[width=\linewidth]{../data/9_31.png}
		\subcaption{$9_{31}$}
	\end{minipage}
	\begin{minipage}[b]{.18\linewidth}
		\centering
		\includegraphics[width=\linewidth]{../data/9_32.png}
		\subcaption{$9_{32}$}
	\end{minipage}
	\begin{minipage}[b]{.18\linewidth}
		\centering
		\includegraphics[width=\linewidth]{../data/9_33.png}
		\subcaption{$9_{33}$}
	\end{minipage}
	\begin{minipage}[b]{.18\linewidth}
		\centering
		\includegraphics[width=\linewidth]{../data/9_34.png}
		\subcaption{$9_{34}$}
	\end{minipage}
	\begin{minipage}[b]{.18\linewidth}
		\centering
		\includegraphics[width=\linewidth]{../data/9_35.png}
		\subcaption{$9_{35}$}
	\end{minipage}
\end{figure}
\begin{figure}[H]
	\begin{minipage}[b]{.18\linewidth}
		\centering
		\includegraphics[width=\linewidth]{../data/9_36.png}
		\subcaption{$9_{36}$}
	\end{minipage}
	\begin{minipage}[b]{.18\linewidth}
		\centering
		\includegraphics[width=\linewidth]{../data/9_37.png}
		\subcaption{$9_{37}$}
	\end{minipage}
	\begin{minipage}[b]{.18\linewidth}
		\centering
		\includegraphics[width=\linewidth]{../data/9_38.png}
		\subcaption{$9_{38}$}
	\end{minipage}
	\begin{minipage}[b]{.18\linewidth}
		\centering
		\includegraphics[width=\linewidth]{../data/9_39.png}
		\subcaption{$9_{39}$}
	\end{minipage}
	\begin{minipage}[b]{.18\linewidth}
		\centering
		\includegraphics[width=\linewidth]{../data/9_40.png}
		\subcaption{$9_{40}$}
	\end{minipage}
\end{figure}
\begin{figure}[H]
	\begin{minipage}[b]{.18\linewidth}
		\centering
		\includegraphics[width=\linewidth]{../data/9_41.png}
		\subcaption{$9_{41}$}
	\end{minipage}
	\begin{minipage}[b]{.18\linewidth}
		\centering
		\includegraphics[width=\linewidth]{../data/9_42.png}
		\subcaption{$9_{42}$}
	\end{minipage}
	\begin{minipage}[b]{.18\linewidth}
		\centering
		\includegraphics[width=\linewidth]{../data/9_43.png}
		\subcaption{$9_{43}$}
	\end{minipage}
	\begin{minipage}[b]{.18\linewidth}
		\centering
		\includegraphics[width=\linewidth]{../data/9_44.png}
		\subcaption{$9_{44}$}
	\end{minipage}
	\begin{minipage}[b]{.18\linewidth}
		\centering
		\includegraphics[width=\linewidth]{../data/9_45.png}
		\subcaption{$9_{45}$}
	\end{minipage}
\end{figure}
\begin{figure}[H]
	\begin{minipage}[b]{.18\linewidth}
		\centering
		\includegraphics[width=\linewidth]{../data/9_46.png}
		\subcaption{$9_{46}$}
	\end{minipage}
	\begin{minipage}[b]{.18\linewidth}
		\centering
		\includegraphics[width=\linewidth]{../data/9_47.png}
		\subcaption{$9_{47}$}
	\end{minipage}
	\begin{minipage}[b]{.18\linewidth}
		\centering
		\includegraphics[width=\linewidth]{../data/9_48.png}
		\subcaption{$9_{48}$}
	\end{minipage}
	\begin{minipage}[b]{.18\linewidth}
		\centering
		\includegraphics[width=\linewidth]{../data/9_49.png}
		\subcaption{$9_{49}$}
	\end{minipage}
	\begin{minipage}[b]{.18\linewidth}
		\centering
		\includegraphics[width=\linewidth]{../data/10_1.png}
		\subcaption{$10_{1}$}
	\end{minipage}
\end{figure}
\begin{figure}[H]
	\begin{minipage}[b]{.18\linewidth}
		\centering
		\includegraphics[width=\linewidth]{../data/10_2.png}
		\subcaption{$10_{2}$}
	\end{minipage}
	\begin{minipage}[b]{.18\linewidth}
		\centering
		\includegraphics[width=\linewidth]{../data/10_3.png}
		\subcaption{$10_{3}$}
	\end{minipage}
	\begin{minipage}[b]{.18\linewidth}
		\centering
		\includegraphics[width=\linewidth]{../data/10_4.png}
		\subcaption{$10_{4}$}
	\end{minipage}
	\begin{minipage}[b]{.18\linewidth}
		\centering
		\includegraphics[width=\linewidth]{../data/10_5.png}
		\subcaption{$10_{5}$}
	\end{minipage}
	\begin{minipage}[b]{.18\linewidth}
		\centering
		\includegraphics[width=\linewidth]{../data/10_6.png}
		\subcaption{$10_{6}$}
	\end{minipage}
\end{figure}
\begin{figure}[H]
	\begin{minipage}[b]{.18\linewidth}
		\centering
		\includegraphics[width=\linewidth]{../data/10_7.png}
		\subcaption{$10_{7}$}
	\end{minipage}
	\begin{minipage}[b]{.18\linewidth}
		\centering
		\includegraphics[width=\linewidth]{../data/10_8.png}
		\subcaption{$10_{8}$}
	\end{minipage}
	\begin{minipage}[b]{.18\linewidth}
		\centering
		\includegraphics[width=\linewidth]{../data/10_9.png}
		\subcaption{$10_{9}$}
	\end{minipage}
	\begin{minipage}[b]{.18\linewidth}
		\centering
		\includegraphics[width=\linewidth]{../data/10_10.png}
		\subcaption{$10_{10}$}
	\end{minipage}
	\begin{minipage}[b]{.18\linewidth}
		\centering
		\includegraphics[width=\linewidth]{../data/10_11.png}
		\subcaption{$10_{11}$}
	\end{minipage}
\end{figure}
\begin{figure}[H]
	\begin{minipage}[b]{.18\linewidth}
		\centering
		\includegraphics[width=\linewidth]{../data/10_12.png}
		\subcaption{$10_{12}$}
	\end{minipage}
	\begin{minipage}[b]{.18\linewidth}
		\centering
		\includegraphics[width=\linewidth]{../data/10_13.png}
		\subcaption{$10_{13}$}
	\end{minipage}
	\begin{minipage}[b]{.18\linewidth}
		\centering
		\includegraphics[width=\linewidth]{../data/10_14.png}
		\subcaption{$10_{14}$}
	\end{minipage}
	\begin{minipage}[b]{.18\linewidth}
		\centering
		\includegraphics[width=\linewidth]{../data/10_15.png}
		\subcaption{$10_{15}$}
	\end{minipage}
	\begin{minipage}[b]{.18\linewidth}
		\centering
		\includegraphics[width=\linewidth]{../data/10_16.png}
		\subcaption{$10_{16}$}
	\end{minipage}
\end{figure}
\begin{figure}[H]
	\begin{minipage}[b]{.18\linewidth}
		\centering
		\includegraphics[width=\linewidth]{../data/10_17.png}
		\subcaption{$10_{17}$}
	\end{minipage}
	\begin{minipage}[b]{.18\linewidth}
		\centering
		\includegraphics[width=\linewidth]{../data/10_18.png}
		\subcaption{$10_{18}$}
	\end{minipage}
	\begin{minipage}[b]{.18\linewidth}
		\centering
		\includegraphics[width=\linewidth]{../data/10_19.png}
		\subcaption{$10_{19}$}
	\end{minipage}
	\begin{minipage}[b]{.18\linewidth}
		\centering
		\includegraphics[width=\linewidth]{../data/10_20.png}
		\subcaption{$10_{20}$}
	\end{minipage}
	\begin{minipage}[b]{.18\linewidth}
		\centering
		\includegraphics[width=\linewidth]{../data/10_21.png}
		\subcaption{$10_{21}$}
	\end{minipage}
\end{figure}
\begin{figure}[H]
	\begin{minipage}[b]{.18\linewidth}
		\centering
		\includegraphics[width=\linewidth]{../data/10_22.png}
		\subcaption{$10_{22}$}
	\end{minipage}
	\begin{minipage}[b]{.18\linewidth}
		\centering
		\includegraphics[width=\linewidth]{../data/10_23.png}
		\subcaption{$10_{23}$}
	\end{minipage}
	\begin{minipage}[b]{.18\linewidth}
		\centering
		\includegraphics[width=\linewidth]{../data/10_24.png}
		\subcaption{$10_{24}$}
	\end{minipage}
	\begin{minipage}[b]{.18\linewidth}
		\centering
		\includegraphics[width=\linewidth]{../data/10_25.png}
		\subcaption{$10_{25}$}
	\end{minipage}
	\begin{minipage}[b]{.18\linewidth}
		\centering
		\includegraphics[width=\linewidth]{../data/10_26.png}
		\subcaption{$10_{26}$}
	\end{minipage}
\end{figure}
\begin{figure}[H]
	\begin{minipage}[b]{.18\linewidth}
		\centering
		\includegraphics[width=\linewidth]{../data/10_27.png}
		\subcaption{$10_{27}$}
	\end{minipage}
	\begin{minipage}[b]{.18\linewidth}
		\centering
		\includegraphics[width=\linewidth]{../data/10_28.png}
		\subcaption{$10_{28}$}
	\end{minipage}
	\begin{minipage}[b]{.18\linewidth}
		\centering
		\includegraphics[width=\linewidth]{../data/10_29.png}
		\subcaption{$10_{29}$}
	\end{minipage}
	\begin{minipage}[b]{.18\linewidth}
		\centering
		\includegraphics[width=\linewidth]{../data/10_30.png}
		\subcaption{$10_{30}$}
	\end{minipage}
	\begin{minipage}[b]{.18\linewidth}
		\centering
		\includegraphics[width=\linewidth]{../data/10_31.png}
		\subcaption{$10_{31}$}
	\end{minipage}
\end{figure}
\begin{figure}[H]
	\begin{minipage}[b]{.18\linewidth}
		\centering
		\includegraphics[width=\linewidth]{../data/10_32.png}
		\subcaption{$10_{32}$}
	\end{minipage}
	\begin{minipage}[b]{.18\linewidth}
		\centering
		\includegraphics[width=\linewidth]{../data/10_33.png}
		\subcaption{$10_{33}$}
	\end{minipage}
	\begin{minipage}[b]{.18\linewidth}
		\centering
		\includegraphics[width=\linewidth]{../data/10_34.png}
		\subcaption{$10_{34}$}
	\end{minipage}
	\begin{minipage}[b]{.18\linewidth}
		\centering
		\includegraphics[width=\linewidth]{../data/10_35.png}
		\subcaption{$10_{35}$}
	\end{minipage}
	\begin{minipage}[b]{.18\linewidth}
		\centering
		\includegraphics[width=\linewidth]{../data/10_36.png}
		\subcaption{$10_{36}$}
	\end{minipage}
\end{figure}
\begin{figure}[H]
	\begin{minipage}[b]{.18\linewidth}
		\centering
		\includegraphics[width=\linewidth]{../data/10_37.png}
		\subcaption{$10_{37}$}
	\end{minipage}
	\begin{minipage}[b]{.18\linewidth}
		\centering
		\includegraphics[width=\linewidth]{../data/10_38.png}
		\subcaption{$10_{38}$}
	\end{minipage}
	\begin{minipage}[b]{.18\linewidth}
		\centering
		\includegraphics[width=\linewidth]{../data/10_39.png}
		\subcaption{$10_{39}$}
	\end{minipage}
	\begin{minipage}[b]{.18\linewidth}
		\centering
		\includegraphics[width=\linewidth]{../data/10_40.png}
		\subcaption{$10_{40}$}
	\end{minipage}
	\begin{minipage}[b]{.18\linewidth}
		\centering
		\includegraphics[width=\linewidth]{../data/10_41.png}
		\subcaption{$10_{41}$}
	\end{minipage}
\end{figure}
\begin{figure}[H]
	\begin{minipage}[b]{.18\linewidth}
		\centering
		\includegraphics[width=\linewidth]{../data/10_42.png}
		\subcaption{$10_{42}$}
	\end{minipage}
	\begin{minipage}[b]{.18\linewidth}
		\centering
		\includegraphics[width=\linewidth]{../data/10_43.png}
		\subcaption{$10_{43}$}
	\end{minipage}
	\begin{minipage}[b]{.18\linewidth}
		\centering
		\includegraphics[width=\linewidth]{../data/10_44.png}
		\subcaption{$10_{44}$}
	\end{minipage}
	\begin{minipage}[b]{.18\linewidth}
		\centering
		\includegraphics[width=\linewidth]{../data/10_45.png}
		\subcaption{$10_{45}$}
	\end{minipage}
	\begin{minipage}[b]{.18\linewidth}
		\centering
		\includegraphics[width=\linewidth]{../data/10_46.png}
		\subcaption{$10_{46}$}
	\end{minipage}
\end{figure}
\begin{figure}[H]
	\begin{minipage}[b]{.18\linewidth}
		\centering
		\includegraphics[width=\linewidth]{../data/10_47.png}
		\subcaption{$10_{47}$}
	\end{minipage}
	\begin{minipage}[b]{.18\linewidth}
		\centering
		\includegraphics[width=\linewidth]{../data/10_48.png}
		\subcaption{$10_{48}$}
	\end{minipage}
	\begin{minipage}[b]{.18\linewidth}
		\centering
		\includegraphics[width=\linewidth]{../data/10_49.png}
		\subcaption{$10_{49}$}
	\end{minipage}
	\begin{minipage}[b]{.18\linewidth}
		\centering
		\includegraphics[width=\linewidth]{../data/10_50.png}
		\subcaption{$10_{50}$}
	\end{minipage}
	\begin{minipage}[b]{.18\linewidth}
		\centering
		\includegraphics[width=\linewidth]{../data/10_51.png}
		\subcaption{$10_{51}$}
	\end{minipage}
\end{figure}
\begin{figure}[H]
	\begin{minipage}[b]{.18\linewidth}
		\centering
		\includegraphics[width=\linewidth]{../data/10_52.png}
		\subcaption{$10_{52}$}
	\end{minipage}
	\begin{minipage}[b]{.18\linewidth}
		\centering
		\includegraphics[width=\linewidth]{../data/10_53.png}
		\subcaption{$10_{53}$}
	\end{minipage}
	\begin{minipage}[b]{.18\linewidth}
		\centering
		\includegraphics[width=\linewidth]{../data/10_54.png}
		\subcaption{$10_{54}$}
	\end{minipage}
	\begin{minipage}[b]{.18\linewidth}
		\centering
		\includegraphics[width=\linewidth]{../data/10_55.png}
		\subcaption{$10_{55}$}
	\end{minipage}
	\begin{minipage}[b]{.18\linewidth}
		\centering
		\includegraphics[width=\linewidth]{../data/10_56.png}
		\subcaption{$10_{56}$}
	\end{minipage}
\end{figure}
\begin{figure}[H]
	\begin{minipage}[b]{.18\linewidth}
		\centering
		\includegraphics[width=\linewidth]{../data/10_57.png}
		\subcaption{$10_{57}$}
	\end{minipage}
	\begin{minipage}[b]{.18\linewidth}
		\centering
		\includegraphics[width=\linewidth]{../data/10_58.png}
		\subcaption{$10_{58}$}
	\end{minipage}
	\begin{minipage}[b]{.18\linewidth}
		\centering
		\includegraphics[width=\linewidth]{../data/10_59.png}
		\subcaption{$10_{59}$}
	\end{minipage}
	\begin{minipage}[b]{.18\linewidth}
		\centering
		\includegraphics[width=\linewidth]{../data/10_60.png}
		\subcaption{$10_{60}$}
	\end{minipage}
	\begin{minipage}[b]{.18\linewidth}
		\centering
		\includegraphics[width=\linewidth]{../data/10_61.png}
		\subcaption{$10_{61}$}
	\end{minipage}
\end{figure}
\begin{figure}[H]
	\begin{minipage}[b]{.18\linewidth}
		\centering
		\includegraphics[width=\linewidth]{../data/10_62.png}
		\subcaption{$10_{62}$}
	\end{minipage}
	\begin{minipage}[b]{.18\linewidth}
		\centering
		\includegraphics[width=\linewidth]{../data/10_63.png}
		\subcaption{$10_{63}$}
	\end{minipage}
	\begin{minipage}[b]{.18\linewidth}
		\centering
		\includegraphics[width=\linewidth]{../data/10_64.png}
		\subcaption{$10_{64}$}
	\end{minipage}
	\begin{minipage}[b]{.18\linewidth}
		\centering
		\includegraphics[width=\linewidth]{../data/10_65.png}
		\subcaption{$10_{65}$}
	\end{minipage}
	\begin{minipage}[b]{.18\linewidth}
		\centering
		\includegraphics[width=\linewidth]{../data/10_66.png}
		\subcaption{$10_{66}$}
	\end{minipage}
\end{figure}
\begin{figure}[H]
	\begin{minipage}[b]{.18\linewidth}
		\centering
		\includegraphics[width=\linewidth]{../data/10_67.png}
		\subcaption{$10_{67}$}
	\end{minipage}
	\begin{minipage}[b]{.18\linewidth}
		\centering
		\includegraphics[width=\linewidth]{../data/10_68.png}
		\subcaption{$10_{68}$}
	\end{minipage}
	\begin{minipage}[b]{.18\linewidth}
		\centering
		\includegraphics[width=\linewidth]{../data/10_69.png}
		\subcaption{$10_{69}$}
	\end{minipage}
	\begin{minipage}[b]{.18\linewidth}
		\centering
		\includegraphics[width=\linewidth]{../data/10_70.png}
		\subcaption{$10_{70}$}
	\end{minipage}
	\begin{minipage}[b]{.18\linewidth}
		\centering
		\includegraphics[width=\linewidth]{../data/10_71.png}
		\subcaption{$10_{71}$}
	\end{minipage}
\end{figure}
\begin{figure}[H]
	\begin{minipage}[b]{.18\linewidth}
		\centering
		\includegraphics[width=\linewidth]{../data/10_72.png}
		\subcaption{$10_{72}$}
	\end{minipage}
	\begin{minipage}[b]{.18\linewidth}
		\centering
		\includegraphics[width=\linewidth]{../data/10_73.png}
		\subcaption{$10_{73}$}
	\end{minipage}
	\begin{minipage}[b]{.18\linewidth}
		\centering
		\includegraphics[width=\linewidth]{../data/10_74.png}
		\subcaption{$10_{74}$}
	\end{minipage}
	\begin{minipage}[b]{.18\linewidth}
		\centering
		\includegraphics[width=\linewidth]{../data/10_75.png}
		\subcaption{$10_{75}$}
	\end{minipage}
	\begin{minipage}[b]{.18\linewidth}
		\centering
		\includegraphics[width=\linewidth]{../data/10_76.png}
		\subcaption{$10_{76}$}
	\end{minipage}
\end{figure}
\begin{figure}[H]
	\begin{minipage}[b]{.18\linewidth}
		\centering
		\includegraphics[width=\linewidth]{../data/10_77.png}
		\subcaption{$10_{77}$}
	\end{minipage}
	\begin{minipage}[b]{.18\linewidth}
		\centering
		\includegraphics[width=\linewidth]{../data/10_78.png}
		\subcaption{$10_{78}$}
	\end{minipage}
	\begin{minipage}[b]{.18\linewidth}
		\centering
		\includegraphics[width=\linewidth]{../data/10_79.png}
		\subcaption{$10_{79}$}
	\end{minipage}
	\begin{minipage}[b]{.18\linewidth}
		\centering
		\includegraphics[width=\linewidth]{../data/10_80.png}
		\subcaption{$10_{80}$}
	\end{minipage}
	\begin{minipage}[b]{.18\linewidth}
		\centering
		\includegraphics[width=\linewidth]{../data/10_81.png}
		\subcaption{$10_{81}$}
	\end{minipage}
\end{figure}
\begin{figure}[H]
	\begin{minipage}[b]{.18\linewidth}
		\centering
		\includegraphics[width=\linewidth]{../data/10_82.png}
		\subcaption{$10_{82}$}
	\end{minipage}
	\begin{minipage}[b]{.18\linewidth}
		\centering
		\includegraphics[width=\linewidth]{../data/10_83.png}
		\subcaption{$10_{83}$}
	\end{minipage}
	\begin{minipage}[b]{.18\linewidth}
		\centering
		\includegraphics[width=\linewidth]{../data/10_84.png}
		\subcaption{$10_{84}$}
	\end{minipage}
	\begin{minipage}[b]{.18\linewidth}
		\centering
		\includegraphics[width=\linewidth]{../data/10_85.png}
		\subcaption{$10_{85}$}
	\end{minipage}
	\begin{minipage}[b]{.18\linewidth}
		\centering
		\includegraphics[width=\linewidth]{../data/10_86.png}
		\subcaption{$10_{86}$}
	\end{minipage}
\end{figure}
\begin{figure}[H]
	\begin{minipage}[b]{.18\linewidth}
		\centering
		\includegraphics[width=\linewidth]{../data/10_87.png}
		\subcaption{$10_{87}$}
	\end{minipage}
	\begin{minipage}[b]{.18\linewidth}
		\centering
		\includegraphics[width=\linewidth]{../data/10_88.png}
		\subcaption{$10_{88}$}
	\end{minipage}
	\begin{minipage}[b]{.18\linewidth}
		\centering
		\includegraphics[width=\linewidth]{../data/10_89.png}
		\subcaption{$10_{89}$}
	\end{minipage}
	\begin{minipage}[b]{.18\linewidth}
		\centering
		\includegraphics[width=\linewidth]{../data/10_90.png}
		\subcaption{$10_{90}$}
	\end{minipage}
	\begin{minipage}[b]{.18\linewidth}
		\centering
		\includegraphics[width=\linewidth]{../data/10_91.png}
		\subcaption{$10_{91}$}
	\end{minipage}
\end{figure}
\begin{figure}[H]
	\begin{minipage}[b]{.18\linewidth}
		\centering
		\includegraphics[width=\linewidth]{../data/10_92.png}
		\subcaption{$10_{92}$}
	\end{minipage}
	\begin{minipage}[b]{.18\linewidth}
		\centering
		\includegraphics[width=\linewidth]{../data/10_93.png}
		\subcaption{$10_{93}$}
	\end{minipage}
	\begin{minipage}[b]{.18\linewidth}
		\centering
		\includegraphics[width=\linewidth]{../data/10_94.png}
		\subcaption{$10_{94}$}
	\end{minipage}
	\begin{minipage}[b]{.18\linewidth}
		\centering
		\includegraphics[width=\linewidth]{../data/10_95.png}
		\subcaption{$10_{95}$}
	\end{minipage}
	\begin{minipage}[b]{.18\linewidth}
		\centering
		\includegraphics[width=\linewidth]{../data/10_96.png}
		\subcaption{$10_{96}$}
	\end{minipage}
\end{figure}
\begin{figure}[H]
	\begin{minipage}[b]{.18\linewidth}
		\centering
		\includegraphics[width=\linewidth]{../data/10_97.png}
		\subcaption{$10_{97}$}
	\end{minipage}
	\begin{minipage}[b]{.18\linewidth}
		\centering
		\includegraphics[width=\linewidth]{../data/10_98.png}
		\subcaption{$10_{98}$}
	\end{minipage}
	\begin{minipage}[b]{.18\linewidth}
		\centering
		\includegraphics[width=\linewidth]{../data/10_99.png}
		\subcaption{$10_{99}$}
	\end{minipage}
	\begin{minipage}[b]{.18\linewidth}
		\centering
		\includegraphics[width=\linewidth]{../data/10_100.png}
		\subcaption{$10_{100}$}
	\end{minipage}
	\begin{minipage}[b]{.18\linewidth}
		\centering
		\includegraphics[width=\linewidth]{../data/10_101.png}
		\subcaption{$10_{101}$}
	\end{minipage}
\end{figure}
\begin{figure}[H]
	\begin{minipage}[b]{.18\linewidth}
		\centering
		\includegraphics[width=\linewidth]{../data/10_102.png}
		\subcaption{$10_{102}$}
	\end{minipage}
	\begin{minipage}[b]{.18\linewidth}
		\centering
		\includegraphics[width=\linewidth]{../data/10_103.png}
		\subcaption{$10_{103}$}
	\end{minipage}
	\begin{minipage}[b]{.18\linewidth}
		\centering
		\includegraphics[width=\linewidth]{../data/10_104.png}
		\subcaption{$10_{104}$}
	\end{minipage}
	\begin{minipage}[b]{.18\linewidth}
		\centering
		\includegraphics[width=\linewidth]{../data/10_105.png}
		\subcaption{$10_{105}$}
	\end{minipage}
	\begin{minipage}[b]{.18\linewidth}
		\centering
		\includegraphics[width=\linewidth]{../data/10_106.png}
		\subcaption{$10_{106}$}
	\end{minipage}
\end{figure}
\begin{figure}[H]
	\begin{minipage}[b]{.18\linewidth}
		\centering
		\includegraphics[width=\linewidth]{../data/10_107.png}
		\subcaption{$10_{107}$}
	\end{minipage}
	\begin{minipage}[b]{.18\linewidth}
		\centering
		\includegraphics[width=\linewidth]{../data/10_108.png}
		\subcaption{$10_{108}$}
	\end{minipage}
	\begin{minipage}[b]{.18\linewidth}
		\centering
		\includegraphics[width=\linewidth]{../data/10_109.png}
		\subcaption{$10_{109}$}
	\end{minipage}
	\begin{minipage}[b]{.18\linewidth}
		\centering
		\includegraphics[width=\linewidth]{../data/10_110.png}
		\subcaption{$10_{110}$}
	\end{minipage}
	\begin{minipage}[b]{.18\linewidth}
		\centering
		\includegraphics[width=\linewidth]{../data/10_111.png}
		\subcaption{$10_{111}$}
	\end{minipage}
\end{figure}
\begin{figure}[H]
	\begin{minipage}[b]{.18\linewidth}
		\centering
		\includegraphics[width=\linewidth]{../data/10_112.png}
		\subcaption{$10_{112}$}
	\end{minipage}
	\begin{minipage}[b]{.18\linewidth}
		\centering
		\includegraphics[width=\linewidth]{../data/10_113.png}
		\subcaption{$10_{113}$}
	\end{minipage}
	\begin{minipage}[b]{.18\linewidth}
		\centering
		\includegraphics[width=\linewidth]{../data/10_114.png}
		\subcaption{$10_{114}$}
	\end{minipage}
	\begin{minipage}[b]{.18\linewidth}
		\centering
		\includegraphics[width=\linewidth]{../data/10_115.png}
		\subcaption{$10_{115}$}
	\end{minipage}
	\begin{minipage}[b]{.18\linewidth}
		\centering
		\includegraphics[width=\linewidth]{../data/10_116.png}
		\subcaption{$10_{116}$}
	\end{minipage}
\end{figure}
\begin{figure}[H]
	\begin{minipage}[b]{.18\linewidth}
		\centering
		\includegraphics[width=\linewidth]{../data/10_117.png}
		\subcaption{$10_{117}$}
	\end{minipage}
	\begin{minipage}[b]{.18\linewidth}
		\centering
		\includegraphics[width=\linewidth]{../data/10_118.png}
		\subcaption{$10_{118}$}
	\end{minipage}
	\begin{minipage}[b]{.18\linewidth}
		\centering
		\includegraphics[width=\linewidth]{../data/10_119.png}
		\subcaption{$10_{119}$}
	\end{minipage}
	\begin{minipage}[b]{.18\linewidth}
		\centering
		\includegraphics[width=\linewidth]{../data/10_120.png}
		\subcaption{$10_{120}$}
	\end{minipage}
	\begin{minipage}[b]{.18\linewidth}
		\centering
		\includegraphics[width=\linewidth]{../data/10_121.png}
		\subcaption{$10_{121}$}
	\end{minipage}
\end{figure}
\begin{figure}[H]
	\begin{minipage}[b]{.18\linewidth}
		\centering
		\includegraphics[width=\linewidth]{../data/10_122.png}
		\subcaption{$10_{122}$}
	\end{minipage}
	\begin{minipage}[b]{.18\linewidth}
		\centering
		\includegraphics[width=\linewidth]{../data/10_123.png}
		\subcaption{$10_{123}$}
	\end{minipage}
	\begin{minipage}[b]{.18\linewidth}
		\centering
		\includegraphics[width=\linewidth]{../data/10_124.png}
		\subcaption{$10_{124}$}
	\end{minipage}
	\begin{minipage}[b]{.18\linewidth}
		\centering
		\includegraphics[width=\linewidth]{../data/10_125.png}
		\subcaption{$10_{125}$}
	\end{minipage}
	\begin{minipage}[b]{.18\linewidth}
		\centering
		\includegraphics[width=\linewidth]{../data/10_126.png}
		\subcaption{$10_{126}$}
	\end{minipage}
\end{figure}
\begin{figure}[H]
	\begin{minipage}[b]{.18\linewidth}
		\centering
		\includegraphics[width=\linewidth]{../data/10_127.png}
		\subcaption{$10_{127}$}
	\end{minipage}
	\begin{minipage}[b]{.18\linewidth}
		\centering
		\includegraphics[width=\linewidth]{../data/10_128.png}
		\subcaption{$10_{128}$}
	\end{minipage}
	\begin{minipage}[b]{.18\linewidth}
		\centering
		\includegraphics[width=\linewidth]{../data/10_129.png}
		\subcaption{$10_{129}$}
	\end{minipage}
	\begin{minipage}[b]{.18\linewidth}
		\centering
		\includegraphics[width=\linewidth]{../data/10_130.png}
		\subcaption{$10_{130}$}
	\end{minipage}
	\begin{minipage}[b]{.18\linewidth}
		\centering
		\includegraphics[width=\linewidth]{../data/10_131.png}
		\subcaption{$10_{131}$}
	\end{minipage}
\end{figure}
\begin{figure}[H]
	\begin{minipage}[b]{.18\linewidth}
		\centering
		\includegraphics[width=\linewidth]{../data/10_132.png}
		\subcaption{$10_{132}$}
	\end{minipage}
	\begin{minipage}[b]{.18\linewidth}
		\centering
		\includegraphics[width=\linewidth]{../data/10_133.png}
		\subcaption{$10_{133}$}
	\end{minipage}
	\begin{minipage}[b]{.18\linewidth}
		\centering
		\includegraphics[width=\linewidth]{../data/10_134.png}
		\subcaption{$10_{134}$}
	\end{minipage}
	\begin{minipage}[b]{.18\linewidth}
		\centering
		\includegraphics[width=\linewidth]{../data/10_135.png}
		\subcaption{$10_{135}$}
	\end{minipage}
	\begin{minipage}[b]{.18\linewidth}
		\centering
		\includegraphics[width=\linewidth]{../data/10_136.png}
		\subcaption{$10_{136}$}
	\end{minipage}
\end{figure}
\begin{figure}[H]
	\begin{minipage}[b]{.18\linewidth}
		\centering
		\includegraphics[width=\linewidth]{../data/10_137.png}
		\subcaption{$10_{137}$}
	\end{minipage}
	\begin{minipage}[b]{.18\linewidth}
		\centering
		\includegraphics[width=\linewidth]{../data/10_138.png}
		\subcaption{$10_{138}$}
	\end{minipage}
	\begin{minipage}[b]{.18\linewidth}
		\centering
		\includegraphics[width=\linewidth]{../data/10_139.png}
		\subcaption{$10_{139}$}
	\end{minipage}
	\begin{minipage}[b]{.18\linewidth}
		\centering
		\includegraphics[width=\linewidth]{../data/10_140.png}
		\subcaption{$10_{140}$}
	\end{minipage}
	\begin{minipage}[b]{.18\linewidth}
		\centering
		\includegraphics[width=\linewidth]{../data/10_141.png}
		\subcaption{$10_{141}$}
	\end{minipage}
\end{figure}
\begin{figure}[H]
	\begin{minipage}[b]{.18\linewidth}
		\centering
		\includegraphics[width=\linewidth]{../data/10_142.png}
		\subcaption{$10_{142}$}
	\end{minipage}
	\begin{minipage}[b]{.18\linewidth}
		\centering
		\includegraphics[width=\linewidth]{../data/10_143.png}
		\subcaption{$10_{143}$}
	\end{minipage}
	\begin{minipage}[b]{.18\linewidth}
		\centering
		\includegraphics[width=\linewidth]{../data/10_144.png}
		\subcaption{$10_{144}$}
	\end{minipage}
	\begin{minipage}[b]{.18\linewidth}
		\centering
		\includegraphics[width=\linewidth]{../data/10_145.png}
		\subcaption{$10_{145}$}
	\end{minipage}
	\begin{minipage}[b]{.18\linewidth}
		\centering
		\includegraphics[width=\linewidth]{../data/10_146.png}
		\subcaption{$10_{146}$}
	\end{minipage}
\end{figure}
\begin{figure}[H]
	\begin{minipage}[b]{.18\linewidth}
		\centering
		\includegraphics[width=\linewidth]{../data/10_147.png}
		\subcaption{$10_{147}$}
	\end{minipage}
	\begin{minipage}[b]{.18\linewidth}
		\centering
		\includegraphics[width=\linewidth]{../data/10_148.png}
		\subcaption{$10_{148}$}
	\end{minipage}
	\begin{minipage}[b]{.18\linewidth}
		\centering
		\includegraphics[width=\linewidth]{../data/10_149.png}
		\subcaption{$10_{149}$}
	\end{minipage}
	\begin{minipage}[b]{.18\linewidth}
		\centering
		\includegraphics[width=\linewidth]{../data/10_150.png}
		\subcaption{$10_{150}$}
	\end{minipage}
	\begin{minipage}[b]{.18\linewidth}
		\centering
		\includegraphics[width=\linewidth]{../data/10_151.png}
		\subcaption{$10_{151}$}
	\end{minipage}
\end{figure}
\begin{figure}[H]
	\begin{minipage}[b]{.18\linewidth}
		\centering
		\includegraphics[width=\linewidth]{../data/10_152.png}
		\subcaption{$10_{152}$}
	\end{minipage}
	\begin{minipage}[b]{.18\linewidth}
		\centering
		\includegraphics[width=\linewidth]{../data/10_153.png}
		\subcaption{$10_{153}$}
	\end{minipage}
	\begin{minipage}[b]{.18\linewidth}
		\centering
		\includegraphics[width=\linewidth]{../data/10_154.png}
		\subcaption{$10_{154}$}
	\end{minipage}
	\begin{minipage}[b]{.18\linewidth}
		\centering
		\includegraphics[width=\linewidth]{../data/10_155.png}
		\subcaption{$10_{155}$}
	\end{minipage}
	\begin{minipage}[b]{.18\linewidth}
		\centering
		\includegraphics[width=\linewidth]{../data/10_156.png}
		\subcaption{$10_{156}$}
	\end{minipage}
\end{figure}
\begin{figure}[H]
	\begin{minipage}[b]{.18\linewidth}
		\centering
		\includegraphics[width=\linewidth]{../data/10_157.png}
		\subcaption{$10_{157}$}
	\end{minipage}
	\begin{minipage}[b]{.18\linewidth}
		\centering
		\includegraphics[width=\linewidth]{../data/10_158.png}
		\subcaption{$10_{158}$}
	\end{minipage}
	\begin{minipage}[b]{.18\linewidth}
		\centering
		\includegraphics[width=\linewidth]{../data/10_159.png}
		\subcaption{$10_{159}$}
	\end{minipage}
	\begin{minipage}[b]{.18\linewidth}
		\centering
		\includegraphics[width=\linewidth]{../data/10_160.png}
		\subcaption{$10_{160}$}
	\end{minipage}
	\begin{minipage}[b]{.18\linewidth}
		\centering
		\includegraphics[width=\linewidth]{../data/10_161.png}
		\subcaption{$10_{161}$}
	\end{minipage}
\end{figure}
\begin{figure}[H]
	\begin{minipage}[b]{.18\linewidth}
		\centering
		\includegraphics[width=\linewidth]{../data/10_162.png}
		\subcaption{$10_{162}$}
	\end{minipage}
	\begin{minipage}[b]{.18\linewidth}
		\centering
		\includegraphics[width=\linewidth]{../data/10_163.png}
		\subcaption{$10_{163}$}
	\end{minipage}
	\begin{minipage}[b]{.18\linewidth}
		\centering
		\includegraphics[width=\linewidth]{../data/10_164.png}
		\subcaption{$10_{164}$}
	\end{minipage}
	\begin{minipage}[b]{.18\linewidth}
		\centering
		\includegraphics[width=\linewidth]{../data/10_165.png}
		\subcaption{$10_{165}$}
	\end{minipage}
\end{figure}

\raggedright
\bibliographystyle{plain}
%\bibliographystyle{plunsrt}
\bibliography{knot_theory}

\newpage
\listoftodos[Lista rzeczy do poprawienia]
\end{document}