% \subsection{Odwrotności, lustra i sumy}
\begin{proposition}
    Niech $L$ będzie zorientowanym splotem.
    $V(rL)=V(L)$, $V(mL)(t)=V(L)(t^{-1})$.
\end{proposition}

\begin{proof}
    Aby obliczyć wielomian rewersu, wykorzystujemy te same diagramy kłębiaste,
    jak dla zwykłego, a przy tym nie zmieniamy znaku żadnego skrzyżowania.

    Druga część: Florian Gellert, Kombinatorische Invarianten, strona 12.
\end{proof}

\begin{corollary}
    Wielomian Jonesa nie zależy od orientacji węzła (ale nie splotu!).
\end{corollary}

\begin{proof}
    Każdy węzeł ma tylko dwie orientacje, splot może mieć ich $2^n$, gdzie $n$ to liczba składowych.
\end{proof}

\begin{proposition}
    Niech $L, M$ będą zorientowanymi splotami, zaś $J, K$: zorientowanymi węzłami.
    \begin{enumerate}
        \item $V(L \sqcup M) = (-t^{1/2} - t^{-1/2}) V(L) V(M)$,
        \item $V(J \# K) = V(J) V(K)$.
    \end{enumerate}
\end{proposition}

\begin{proof}
    Wybierzmy diagramy $D, E$ dla (odpowiednio) $L, M$.
    Po podstawieniu $t^{1/2}=A^{-2}$ widzimy, że chcemy pokazać
    $(-A)^{-3w(D\sqcup E)}\langle D\sqcup E\rangle =(-A^2-A^{-2})(-A)^{-3(w(D)+w(E))}\langle D\rangle  \langle E\rangle$.

    Oczywiście $w(D\sqcup E)=w(D)+w(E)$, więc wystarczy udowodnić, że
    \[
        \langle D\sqcup E\rangle = (-A^2-A^{-2})\langle D\rangle\langle E\rangle.
    \]

    Oznaczmy przez $f_1(D)$, $f_2(D)$ lewą i prawą stronę ostatniego równania.
    Są to wielomiany Laurenta, które zależą tylko od $D$.
    Aksjomaty Kauffmana pozwalają na pokazanie, że obie funkcje mają następujące własności:
    $f_i(\LittleUnknot)=(-A^2-A^{-2})\langle E\rangle$,
    $f_i(D\sqcup\LittleUnknot)=(-A^2-A^{-2})f_i(D)$,
    $f_i(\LittleRightCrossing)=Af_i(\LittleRightSmoothing) + A^{-1}f_i(\LittleLeftSmoothing)$.
    To pozwala na wyznaczenie ich wartości dla dowolnego $D$, zatem $f_1 \equiv f_2$, co kończy dowód.
\end{proof}

\begin{proof}
    Rozpatrzmy sploty
    \[
        \begin{tikzpicture}[baseline=-0.65ex,scale=0.07]
        \begin{knot}[clip width=5, flip crossing/.list={1}]
            \strand[semithick] (-17, -5) rectangle (-7, 5);
            \strand[semithick] (17, -5) rectangle (7, 5);

            \strand[semithick,Latex-] (-7, 3) [in=left, out=right] to (7, -3);
            \strand[semithick,Latex-] (7, 3) [in=right, out=left] to (-7, -3);

            \node at (-12, 0) {$J$};
            \node at (12, 0) {$K$};
        \end{knot}
        \end{tikzpicture}
        \quad\quad
        \begin{tikzpicture}[baseline=-0.65ex,scale=0.07]
        \begin{knot}[clip width=5]
            \strand[semithick] (-17, -5) rectangle (-7, 5);
            \strand[semithick] (17, -5) rectangle (7, 5);

            \strand[semithick,Latex-] (-7, 3) [in=left, out=right] to (7, -3);
            \strand[semithick,Latex-] (7, 3) [in=right, out=left] to (-7, -3);

            \node at (-12, 0) {$J$};
            \node at (12, 0) {$K$};
        \end{knot}
        \end{tikzpicture}
        \quad\quad
        \begin{tikzpicture}[baseline=-0.65ex,scale=0.07]
        \begin{knot}[clip width=5]
            \strand[semithick] (-17, -5) rectangle (-7, 5);
            \strand[semithick] (-7, -3) [in=down, out=right] to (-2, 0);
            \strand[semithick,Latex-] (-7, 3) [in=up, out=right] to (-2, 0);

            \strand[semithick] (17, -5) rectangle (7, 5);
            \strand[semithick] (7, -3) [in=down, out=left] to (2, 0);
            \strand[semithick,Latex-] (7, 3) [in=up, out=left] to (2, 0);

            \node at (-12, 0) {$J$};
            \node at (12, 0) {$K$};
        \end{knot}
        \end{tikzpicture}
    \]
    Relacja kłębiasta może zostać użyta do pokazania, że
    \[
    t^{-1}V(J\#K)-tV(J\#K)+(t^{-1/2}-t^{1/2})V(J\sqcup K)=0.
    \]
    Ale $V(J\sqcup K)=(-t^{1/2}-t^{-1/2})V(J)V(K)$, co upraszcza się do $V(J\#K)=V(J)V(K)$ i kończy dowód.
\end{proof}


