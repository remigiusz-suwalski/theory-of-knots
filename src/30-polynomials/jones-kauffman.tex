\subsection{Nawias Kauffmana} % (fold)
\label{sub:kauffman_bracket}
Zaczniemy od zdefiniowania nawiasu Kauffmana.
Przypomnijmy, wielomian Laurenta zmiennej $X$ to formalny symbol $f=a_r X^r + \ldots + a_s X^s$,
gdzie $r, s, a_r, \ldots, a_s$ są całkowite i $r \le s$.

Poszukujemy niezmiennika dla splotów o kilku prostych własnościach.
Przede wszystkim żądamy,
by niewęzłowi przypisany był wielomian $1$: $\bracket{\LittleUnknot} = 1$.
Po drugie chcemy wyznaczać nawiasy znając je dla prostszych splotów,
co zapiszemy symbolicznie $\bracket{\LittleRightCrossing} = A \bracket{\LittleRightSmoothing} + B \bracket{\LittleLeftSmoothing}$.
Zależy nam wreszcie na tym, by móc dodać do splotu trywialną składową:
$\langle L \cup \LittleUnknot \rangle = C \langle L \rangle$.

Prosty rachunek pokazuje wpływ drugiego ruchu Reidemeistera na nawias:
\[
    \bracket{\reidemeisterIIaa}
    = (A^2 + ABC + B^2) \bracket{\LittleLeftSmoothing} + BA \bracket{\LittleRightSmoothing}
    \stackrel{?}{=} \bracket{\LittleRightSmoothing}.
\]

Aby zachodziła ostatnia równość wystarczy (chociaż wcale nie trzeba) przyjąć
$B = A^{-1}$, co wymusza na nas $C = -A^2 - A^{-2}$.
W ten sposób odkryliśmy następującą definicję.

\begin{definition}[nawias Kauffmana]
    \index{nawias!Kauffmana}
    Wielomian Laurenta $\bracket{D}$ dla diagramu splotu $D$ zmiennej $A$,
    który jest niezmienniczy ze względu na gładkie deformacje diagramu,
    a przy tym spełnia trzy poniższe aksjomaty, to nawias Kauffmana.
    \begin{align}
        \bracket{\LittleUnknot} & = 1 \\
        \bracket{D \sqcup \LittleUnknot} & = (-A^{-2} - A^2) \bracket{D} \\
        \bracket{\LittleRightCrossing} & = A \bracket{\LittleRightSmoothing} + A^{-1} \bracket{\LittleLeftSmoothing}
    \end{align}
\end{definition}

Tutaj $\LittleUnknot$ oznacza standardowy diagram dla niewęzła,
zaś trzy symbole $\LittleRightCrossing$, $\LittleRightSmoothing$ oraz $\LittleLeftSmoothing$ odnoszą się do diagramów,
które są identyczne wszędzie poza małym obszarem (tak jak w relacji kłębiastej).

Diagramy $\LittleRightSmoothing$ oraz $\LittleLeftSmoothing$ nazywa się odpowiednio
dodatnim (prawym) i ujemnym (lewym) wygładzeniem $\LittleRightCrossing$.

\begin{lemma}
    Nawias Kauffmana każdego diagramu wyznacza się w skończenie wielu krokach.
\end{lemma}

\begin{proof}
    Jeżeli diagram $D$ ma $n$ skrzyżowań, to nieustanne stosowanie aksjomatu trzeciego pozwala na zapisanie $\bracket{D}$ jako sumy $2^n$ składników,
    z których każdy jest po prostu zamkniętą krzywą i ma trywialny nawias ($\bracket{\LittleUnknot} = 1$).
    Nawias sumy wyznacza się korzystając z drugiego aksjomatu.
\end{proof}

Przedstawimy teraz wpływ ruchów Reidemeistera na nawias Kauffmana.

\begin{lemma}
    Drugi i trzeci ruch Reidemeistera nie ma wpływu na klamrę Kauffmana,
    pierwszy ruch zmienia ją zgodnie z regułą:
    \[
        \bracket{\reidemeisterIa} = -A^{-3} \bracket{\,\reidemeisterIb\,}.
    \]
\end{lemma}

\begin{proof}
Pierwszy ruch Reidemeistera:
\[
    \bracket{\reidemeisterIa} \stackrel{K3}{=} A \bracket{
    \begin{tikzpicture}[baseline=-0.65ex,scale=0.07]
    \useasboundingbox (-4, -5) rectangle (3, 5);
    \begin{knot}[clip width=5, end tolerance=1pt]
        \strand[semithick]
            (-3, 5) [in=left, out=down] to (-1,1) [in=left, out=right]
                                        to (1,3)
                                        to [in=up, out=right] (3,0);
        \strand[semithick]
            (-3, -5) [in=left, out=up] to (-1,-1) [in=left, out=right]
                                       to (1, -3)
                                       to [in=down, out=right] (3,0);
    \end{knot}
    \end{tikzpicture}}
    + A^{-1} \bracket{\,
    \begin{tikzpicture}[baseline=-0.65ex,scale=0.07]
    \begin{knot}[clip width=5]
        \strand[semithick] (0,-5) [in=down, out=up] to (1, -2) to (1, 2) to (0, 5);
        \strand[semithick] (4,0) circle (1.5);
    \end{knot}
    \end{tikzpicture}}
    \stackrel{K2}{=} A \bracket{\,\reidemeisterIb\,} + A^{-1}(-A^{-2}-A^2) \bracket{\,\reidemeisterIb\,}
    = -A^{-3}\bracket{\,\reidemeisterIb\,}
\]

Pierwsza równość wynika z $K3$, druga z $K2$, trzecia jest oczywista.
Dla drugiego ruchu:
\begin{align*}
    \bracket{\reidemeisterIIa} &\stackrel{K3}{=} A
    \bracket{\reidemeisterIab}
    + A^{-1} \bracket{\begin{tikzpicture}[baseline=-0.65ex,scale=0.07]
    \useasboundingbox (-5, -6) rectangle (5, 6);
    \begin{knot}[clip width=5, end tolerance=1pt]
        \strand[semithick] (4,-5) .. controls (4,-2) and (-4,-2) .. (-4,0);
        \strand[semithick] (4,5) to (4,0);
        \strand[semithick] (-4,-5) .. controls (-4,-2) and (4,-2) .. (4,0);
        \strand[semithick] (-4,5) to (-4,0);
    \end{knot}
    \end{tikzpicture}}
    \stackrel{K1}{=} -A^{-2} \bracket{\LittleLeftSmoothing} + A^{-1}
    \bracket{\begin{tikzpicture}[baseline=-0.65ex,scale=0.07]
    \useasboundingbox (-5, -6) rectangle (5, 6);
    \begin{knot}[clip width=5, end tolerance=1pt]
        \strand[semithick] (4,-5) .. controls (4,-2) and (-4,-2) .. (-4,0);
        \strand[semithick] (4,5) to (4,0);
        \strand[semithick] (-4,-5) .. controls (-4,-2) and (4,-2) .. (4,0);
        \strand[semithick] (-4,5) to (-4,0);
    \end{knot}
    \end{tikzpicture}}
    \\ & \stackrel{K3}{=} -A^{-2} \bracket{\LittleLeftSmoothing}
    + A^{-1}A \bracket{\LittleRightSmoothing} + A^{-1}A^{-1} \bracket{\LittleLeftSmoothing}
    = \bracket{\LittleRightSmoothing}
\end{align*}

Dla trzeciego ruchu:
\begin{align*}
\bracket{\,\reidemeisterIIIa\,} &\stackrel{K3}{=} A
\bracket{\,\begin{tikzpicture}[baseline=-0.65ex,yscale=0.07, xscale=0.1]
    \useasboundingbox (-5, -6) rectangle (5, 6);
    \begin{knot}[clip width=5, flip crossing/.list={1,2,3}, end tolerance=1pt]
        \strand[semithick] (-5, 5) [in=-135, out=-45] to (5,5);
        \strand[semithick] (-5, -5) [in=135, out=45] to (5,-5);
        \strand[semithick] (-5, 0) .. controls (-2, 0) and (-2,5) .. (0,5) .. controls (2, 5) and (2, 0) .. (5, 0);
    \end{knot}
    \end{tikzpicture}\,}
+A^{-1} \bracket{\RightCrossSmoothing} \\
%\stackrel{R2}{=} A \bracket{\,\LeftCrossSmoothing\,} +A^{-1} \bracket{\RightCrossSmoothing} \\
& \stackrel{R2}{=} A \bracket{\,\LeftCrossSmoothing\,} +A^{-1} \bracket{\RightCrossSmoothing}
\stackrel{K3}{=} \bracket{\,\reidemeisterIIIb\,}
\end{align*}
korzystaliśmy tu z własności drugiego ruchu.
\end{proof}

Okazało się, że użycie najprostszego, I ruchu Reidemeistera, ,,psuje'' nawias!
W akcie desperacji moglibyśmy zmienić definicję,
zaniechamy tego i przejdziemy do kolejnego składnika w przepisie na wielomian Jonesa.

Gregor Schaumann w notatkach \cite{schaumann16} wprowadza rachunek schematyczny,
który pozwala spojrzeć na nawias Kauffmana z nowej perspektywy.
Definiuje kwantowe niezmienniki oraz plątaniny (równoważne wtedy,
kiedy związane są przez ciąg tożsamości wężowych, ruchów Turaewa oraz Reidemeistera).
Wspomniane są też dokonania Wasiljewa.
% Koniec podsekcji Nawias Kauffmana
