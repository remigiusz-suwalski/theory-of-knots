\section{Niezmienniki Wasiljewa} % (fold)
\label{sec:vassiliev}
Będziemy teraz rozważać węzły z~samoprzecięciami.

\begin{definition}
    \index{węzeł!singularny}
    Obraz kawałkami liniowej funkcji $f \colon S^1 \to \R^3$ nazywamy węzłem singularnym, jeżeli funkcja $f$ jest różnowartościowa poza skończenie wieloma punktami.
    Jeśli dwa punkty na okręgu mają ten sam obraz, to $K$ przecina się tam pod kątem prostym.
    Dodatkowo wymagamy, by w~żadnych trzech punktach funkcja $f$ nie przybierała tej samej wartości.
\end{definition}

Gdy używamy słowa węzeł, nigdy nie mamy na myśli węzła singularnego.
Punkt, gdzie węzeł tnie samego siebie, nazywamy wierzchołkiem.

Do każdego wierzchołka $A$ przypiszmy małe otoczenie domknięte $U_A$ oraz płaszczyznę $P_A$, która zawiera $U_A \cap K$, mały fragment węzła wokół punktu $A$. Dysk $P_A \cap U_A$ nazywamy płaskim dyskiem wokół wierzchołka $A$.

\begin{definition}
    Dwa singularne węzły $K, L$ traktujemy jako równoważne: $K \approx L$, jeśli istnieje zachowujący orientację homeomorfizm $\varphi$ przestrzeni $\R^3$ w~siebie, który przenosi jeden węzeł na drugi: $\varphi(K) = L$ oraz indukuje bijekcję między rodzinami płaskich dysków $K$ i~$L$.
\end{definition}

Ten drugi warunek gwarantuje nam, że skrzyżowania wokół podwójnych punktów nie ulegną zniszczeniu.
Istnieje podobne kryterium dla diagramów.
Dwa singularne diagramy są równoważne dokładnie wtedy, kiedy można między nimi przejść przy użyciu ciągu operacji $\Omega$:

\todo[inline]{Dye definiuje zupełnie inne ruchy!}

oraz ruchów Reidemeistera (poza małymi otoczeniami wierzchołków).

Załóżmy teraz, że $v$ jest jakimś niezmiennikiem węzłów o~wymiernych wartościach.
Można indukcyjnie próbować go przedłużyć do węzłów singularnych.
Znając jego wartości dla węzłów o~co najwyżej $n - 1$ wierzchołkach,  wybieramy węzeł $K$ o~$n$ wierzchołkach i~kładziemy $v(K) = v(K_+) - v(K_-)$.
Tak otrzymany byt nazywamy niezmiennikiem Wasiljewa rzędu co najwyżej $m$, gdy znika na wszystkich singularnych węzłach o~$m + 1$ wierzchołkach.

Istnieje jeden niezmiennik Wasiljewa rzędu zero i~żadnych niezmienników rzędu jeden.

\begin{example}
    Niech $a_m$ będzie współczynnikiem stojącym przy $z^{2m}$ w~wielomianie Conwaya.
    Jest to niezmiennik Wasiljewa rzędu $2m$.
\end{example}

\begin{example}
    Niech $b_m$ będzie współczynnikiem stojącym przy $q^m$ w~rozwinięciu Taylora dla funkcji $V(e^q)$, gdzie $V$ oznacza wielomian Jonesa.
    Jest to niezmiennik Wasiljewa rzędu co najwyżej $m$.
\end{example}

\begin{example}
    Niech $c_m$ będzie $m$-tym współczynnikiem rozwinięcia Taylora dla wielomianu Jonesa wokół punktu $t = 1$.
    Jest to niezmiennik Wasiljewa rzędu co najwyżej $m$.
\end{example}

% Niezmiennik Wasiljewa węzła z~pętelką to zero.

Pójdźmy w ślad za Murasugim i zdefiniujmy nieskończoną rodzinę węzłów wirtualnych $K[p, q]$, gdzie $p$ jest liczbą wierzchołków, zaś $|q|$ liczbą klasycznych skrzyżowań.
Jeśli $q < 0$, wszystkie skrzyżowania odwracamy:
\[
\begin{tikzpicture}[baseline=-0.65ex, scale=0.1]
\begin{knot}[clip width=5, end tolerance=1pt, flip crossing/.list={2}]
    % left part
    \draw[semithick] (5, 0) [in=-60, out=-120] to (-5, 0) [in=60, out=120] to (-15, 0) [in=-60, out=-120] to (-25, 0) [in=60, out=120] to (-35, 0) [in=180, out=-120] to (-35, -10);
    \draw[semithick] (5, 0) [in=60, out=120] to (-5, 0) [in=-60, out=-120] to (-15, 0) [in=60, out=120] to (-25, 0) [in=-60, out=-120] to (-35, 0) [in=-180, out=120] to (-35, 10);
    % right part
    \strand[semithick] (5, 0) [in=120, out=60] to (15, 0) [in=-120, out=-60] to (25, 0) [in=120, out=60] to (35, 0) [in=0, out=-60] to (35, -10);
    \strand[semithick] (5, 0) [in=-120, out=-60] to (15, 0) [in=120, out=60] to (25, 0) [in=-120, out=-60] to (35, 0) [in=0, out=60] to (35, 10);
    % external lines
    \draw[semithick] (-35, 10) to (35, 10);
    \draw[semithick] (-35, -10) to (35, -10);
    \draw[black,fill=black] (5,0) circle (0.5);
    \draw[black,fill=black] (-5,0) circle (0.5);
    \draw[black,fill=black] (-15,0) circle (0.5);
    \draw[black,fill=black] (-25,0) circle (0.5);
    \draw[black,fill=black] (-35,0) circle (0.5);
\end{knot}
\end{tikzpicture}\]

Następujące nie są niezmiennikami Wasiljewa: sygnatura, indeks skrrzyżowaniowy, liczba gordyjska, indeks mostowy, indeks warkoczowy, genus.
Świadkiem jest za każdym razem węzeł $K[n+1, n]$.

Ohyama pokazał w~\cite{ohyama95}, że dla każdego węzła $K$ i~liczby naturalnej $n$ istnieje nieskończona rodzina węzłów, których niezmienniki rzędu co najwyżej $n$ nie odróżniają od $K$.
Istnieje przypuszczenie, że wszystkie niezmienniki Wasiljewa (jednocześnie) są już niezmiennikiem zupełnym.