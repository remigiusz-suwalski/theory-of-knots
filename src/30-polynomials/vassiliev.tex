\section{Niezmienniki Wasiljewa} % (fold)
\label{sec:vassiliev}
Niezmienniki Wasiljewa (znane także jako niezmienniki skończonego typu) umożliwiają radykalnie nowe podejście do węzłów.
Około 1989 roku odkryli je niezależnie Wiktor Wasiljew, w~oparciu o~teorię osobliwości, oraz Michaił Goussarow, metodami kombinatorycznymi.
My przedstawimy to drugie podejście.
Niezmienniki Wasiljewa najlepiej zrozumieć, jeśli osłabimy trochę definicję węzła i będziemy rozważać także krzywe z~samoprzecięciami.

Uważam, że przyjemnym wprowadzeniem do niezmienników Wasiljewa jest artykuł \cite{chmutov12} Chmutowa.
Za najlepsze uważa się pracę \cite{barnatan_95}: Bar-Natan, jej autor, pozostaje do dziś jedną z~najważniejszych osób dla rozwoju tego działu matematyki.
Oprócz tego jest jeszcze pięknie zilustrowana książka \cite{duzhin12} trzech rosyjskich matematyków, będąca obszernym kompendium wiedzy o~węzłach wirtualnych oraz ciekawy podręcznik \cite{dye16}, który rozwija od podstaw teorię węzłów wirtualnych i~klasycznych równocześnie; wydaje się jednak, że nie jest do końca zgodny z~przedstawionymi tu definicjami, dlatego nie odnosimy się do niego dalej.
% Kenneth A. Perko, Jr. (amazon): Not too happy about the reference at page 49 to "old Perko's notation," but that's just me. (Also, Fermat, mentioned at page 10, was not much of a lawyer, having used his father's money to buy a life-long provincial judgeship.) Clearly the book is an essential acquisition for anyone in the field and should provide great recreational value for reading on long airplane flights.
Szkielet sekcji oparliśmy natomiast o piętnasty rozdział podręcznika Murasugiego, \cite{murasugi96}.

\begin{definition}[węzeł singularny]
    \index{węzeł!singularny}
    Niech $f \colon S^1 \to \R^3$ będzie kawałkami liniową funkcją, która jest różnowartościowa poza skończenie wieloma punktami.
    Załóżmy dodatkowo, że
    \begin{itemize}
        \item włókna funkcji $f$ (przeciwobrazy punktów) są co najwyżej dwuelementowe
        \item jeżeli dwa różne punkty mają ten sam obraz, to $K = f(S^1)$ przecina się tam pod kątem prostym.
    \end{itemize}
    Obraz funkcji $f$ nazywamy węzłem singularnym.
\end{definition}

Gdy używamy słowa węzeł, nigdy nie mamy na myśli węzła singularnego.
Punkt, gdzie węzeł singularny tnie samego siebie, nazywamy wierzchołkiem i~oznaczamy pogrubioną kropką na diagramie.
Podamy teraz definicję równoważności dla węzłów singularnych przez analogię do zwykłych węzłów (definicja \ref{equivalent_knots_2}).

\begin{definition}[płaski dysk]
    Niech $A$ będzie wierzchołkiem węzła singularnego $K$, $B_A$ jego małym domkniętym otoczeniem, zaś $P_A$ płaszczyzną, która zawiera $B_A \cap K$, mały fragment węzła wokół wierzchołka.
    Dysk $P_A \cap B_A$ nazywamy płaskim dyskiem wokół wierzchołka $A$.
\end{definition}

\begin{definition}
    Dwa singularne węzły $K, L$ są równoważne, co zapisujemy jako $K = L$, jeśli istnieje zachowujący orientację homeomorfizm $\varphi$ przestrzeni $\R^3$ w~siebie, który przenosi jeden węzeł na drugi: $\varphi(K) = L$ oraz indukuje bijekcję między rodzinami płaskich dysków $K$ i~$L$.
\end{definition}

Ten drugi warunek gwarantuje nam, że skrzyżowania wokół podwójnych punktów nie ulegną zniszczeniu.
Istnieje podobne kryterium dla diagramów, odpowiednik twierdzenia Reidemeistera (\ref{thm:reidemeister}).

\begin{proposition}
    Dwa singularne diagramy są równoważne dokładnie wtedy, kiedy można między nimi przejść przy użyciu ciągu ruchów Reidemeistera oraz operacji $\Omega$:
\begin{comment}
    \[
    \begin{tikzpicture}[baseline=-0.65ex, scale=0.1]
    \begin{knot}[clip width=5, end tolerance=1pt, flip crossing/.list={1,2,3,4}]
        \strand[semithick] (-15, 5) to (0, 5) to (5, 0) to (0, -5) to (-15, -5);
        \strand[semithick] (-2, -2) to (-10, -10);
        \strand[semithick] (-2, 2) to (-10, 10);
        \strand[semithick] (2, -2) to (10, -10);
        \strand[semithick] (2, 2) to (10, 10);
        \draw[semithick] (2, 2) to (-2, -2);
        \draw[semithick] (2, -2) to (-2, 2);
        \draw[black,fill=black] (0,0) circle (.5);
    \end{knot}
    \end{tikzpicture}
    \quad\cong_\Omega\quad
    \begin{tikzpicture}[baseline=-0.65ex, scale=0.1]
    \begin{knot}[clip width=5, end tolerance=1pt, flip crossing/.list={1,2,3,4}]
        \strand[semithick] (-15, 5) to (-10, 5) to (-5, 0) to (-10, -5) to (-15, -5);
        \strand[semithick] (-2, -2) to (-10, -10);
        \strand[semithick] (-2, 2) to (-10, 10);
        \strand[semithick] (2, -2) to (10, -10);
        \strand[semithick] (2, 2) to (10, 10);
        \draw[semithick] (2, 2) to (-2, -2);
        \draw[semithick] (2, -2) to (-2, 2);
        \draw[black,fill=black] (0,0) circle (.5);
    \end{knot}
    \end{tikzpicture}
    \quad\cong_\Omega\quad
    \begin{tikzpicture}[baseline=-0.65ex, scale=0.1]
    \begin{knot}[clip width=5, end tolerance=1pt]
        \strand[semithick] (-15, 5) to (0, 5) to (5, 0) to (0, -5) to (-15, -5);
        \strand[semithick] (-2, -2) to (-10, -10);
        \strand[semithick] (-2, 2) to (-10, 10);
        \strand[semithick] (2, -2) to (10, -10);
        \strand[semithick] (2, 2) to (10, 10);
        \draw[semithick] (2, 2) to (-2, -2);
        \draw[semithick] (2, -2) to (-2, 2);
        \draw[black,fill=black] (0,0) circle (.5);
    \end{knot}
    \end{tikzpicture}
    \]
\end{comment}
\end{proposition}

Załóżmy teraz, że mamy jakiś niezmiennik węzłów $v$ o~wymiernych wartościach i~chcemy przedłużyć go do niezmiennika $\hat v$ węzłów singularnych.
Najprościej zrobić to rekurencyjnie.
Niech $\hat v$ będzie już określony dla węzłów singularnych o co najwyżej $n - 1$ wierzchołkach i~wybierzmy dowolny węzeł o~$n$ wierzchołkach.
Okazuje się, że jeżeli położymy
\begin{comment}
\begin{equation}
    \hat v(K) = \hat v(\LittleLeftCrossing) - \hat v(\LittleRightCrossing),
\end{equation}
\end{comment}
to dostaniemy dobrze określoną funkcję: jeśli $L = K$ jest innym węzłem singularnym, to $\hat v(L) = \hat v(K)$.
Nazywamy ją niezmiennikiem singularnym indukowanym przez niezmiennik węzłów $v_0$.

\begin{definition}[rząd niezmiennika]
    \label{def:vassiliev_order}
    Niech $v$ będzie niezmiennikiem singularnym.
    Mówimy, że $v$ jest niezmiennikiem Wasiljewa rzędu co najwyżej $n$, jeśli dla dowolnego singularnego węzła $K$ o $n + 1$ wierzchołkach zachodzi $v(K) = 0$.
    Jeśli dodatkowo $v$ nie jest rzędu co najwyżej $n - 1$, to mówimy, że jest rzędu dokładnie $n$.
\end{definition}

Czas na garść przykładów.

\begin{example}
    Niech $K$ będzie węzłem, zaś $\conway_K(t) = \sum_k \conway_{2k} z^{2k}$ jego wielomianem Conwaya.
    Współczynnik $\conway_{2k}$ indukuje niezmiennik Wasiljewa rzędu dokładnie $2k$.
\end{example}

Pokazał to Bar-Natan w roku 1991.
Lin oraz Wang w~1994 roku na podstawie niezmienników małych rzędów, to jest $v_2$ oraz $v_3$, wysunęli następującą hipotezę: istnieje uniwersalna stała $C$ taka, że
\begin{equation}
    |v_k(K)| \le C (\operatorname{cr} K)^k.
\end{equation}

Hipotezę wkrótce udowodniono, najpierw dla węzłów (Bar-Natan, \cite{barnatan95}), nieco później także dla splotów (Stojmenow, \cite{stoimenow_01}).
Wartość stałej $C$ trudno obliczyć, dlatego Stojmenow zaproponował ograniczenie się do przypadku $v_k = \conway_k$.

\begin{conjecture}
    Niech $L$ będzie splotem.
    Wtedy
    \begin{equation}
        |\conway_k(L)| \le \frac{1}{2^kk!} \cdot c^k.
    \end{equation}
\end{conjecture}

Nierówność jest nietrywialna tylko dla splotu $L$ z~$k+1, k-1, \ldots$ składowymi; trywialna dla $k = 0$, łatwa dla $k=1$ (wtedy $\conway_1$ jest indeksem zaczepienia splotów o~dwóch składowych) oraz udowodniona dla węzłów i~$k=2$ przez Polyaka, Viro w~2001 (\cite{polyak01}).
% The Casson knot invariant (to be distinguished from the better-known Casson invariant) is defined to be the Vassiliev knot invariant v_2, which turns out to be \alexander_k''(1) / 2 , where \alexander_k is the Alexander polynomial of k. It can be characterized as the unique Vassiliev invariant of degree 2 that takes value 0 on the trivial knot and value 1 on the trefoil knot.

\begin{example}
    Niech $K$ będzie węzłem, zaś $\jones_K(t)$ jego wielomianem Jonesa.
    Podstawmy za $t$ w wielomianie $\jones_K(t)$ formalny szereg potęgowy $e^x = 1 + x + \frac12 x^2 + \ldots$ i rozwińmy go w szereg Taylora:
    \begin{equation}
        \jones_K(e^x) = \sum_{k = 0}^\infty b_k x^k.
    \end{equation}
    Współczynnik $b_{k}$ indukuje niezmiennik Wasiljewa rzędu co najwyżej $k$.
\end{example}

Ten i podobne wyniki dla wielomianów HOMFLY, Kauffmana uzyskała Birman z~Linem w \cite{birman93}.
Praca ta znacznie uprościła oryginalne techniki Wasiljewa.

\begin{example}
    Niech $K$ będzie węzłem, zaś $f(t)$ rozwinięciem Taylora wokół $t = 1$ dla wielomianu Jonesa:
    \begin{equation}
        f(t) = \sum_{k = 0}^\infty c_k (t-1)^k.
    \end{equation}
    Współczynnik $c_{k}$ indukuje niezmiennik Wasiljewa rzędu co najwyżej $k$.
\end{example}

Pójdźmy w ślad za Murasugim i zdefiniujmy nieskończoną rodzinę węzłów wirtualnych $K[p, q]$, gdzie $p$ jest liczbą wierzchołków, zaś $|q|$ liczbą klasycznych skrzyżowań.
Jeśli $q < 0$, wszystkie skrzyżowania odwracamy:
\begin{comment}
\[
\begin{tikzpicture}[baseline=-0.65ex, scale=0.1]
\begin{knot}[clip width=5, end tolerance=1pt, flip crossing/.list={2}]
    % left part
    \draw[semithick] (5, 0) [in=-60, out=-120] to (-5, 0) [in=60, out=120] to (-15, 0) [in=-60, out=-120] to (-25, 0) [in=60, out=120] to (-35, 0) [in=180, out=-120] to (-35, -10);
    \draw[semithick] (5, 0) [in=60, out=120] to (-5, 0) [in=-60, out=-120] to (-15, 0) [in=60, out=120] to (-25, 0) [in=-60, out=-120] to (-35, 0) [in=-180, out=120] to (-35, 10);
    % right part
    \strand[semithick] (5, 0) [in=120, out=60] to (15, 0) [in=-120, out=-60] to (25, 0) [in=120, out=60] to (35, 0) [in=0, out=-60] to (35, -10);
    \strand[semithick] (5, 0) [in=-120, out=-60] to (15, 0) [in=120, out=60] to (25, 0) [in=-120, out=-60] to (35, 0) [in=0, out=60] to (35, 10);
    % external lines
    \draw[semithick,Latex-] (-35, 10) to (35, 10);
    \draw[semithick,Latex-] (-35, -10) to (35, -10);
    \draw[black,fill=black] (5,0) circle (0.5);
    \draw[black,fill=black] (-5,0) circle (0.5);
    \draw[black,fill=black] (-15,0) circle (0.5);
    \draw[black,fill=black] (-25,0) circle (0.5);
    \draw[black,fill=black] (-35,0) circle (0.5);
\end{knot}
\end{tikzpicture}\]
\end{comment}

\begin{proposition}
    Następujące funkcje nie są niezmiennikami Wasiljewa: indeks skrzyżowaniowy $\operatorname{cr}$, liczba gordyjska $\operatorname{u}$, indeks mostowy $\operatorname{br}$, indeks warkoczowy $\operatorname{b}$, genus $g$, sygnatura $\sigma$.
\end{proposition}

Wynik był znany już w latach 90., na przykład Birman w \cite{birman93} pokazała, że liczba gordyjska nie jest niezmiennikiem Wasiljewa.

\begin{proof}
    Żaden z tych niezmienników nie znika na singularnym węźle $K[n+1, n]$.
\end{proof}

Udowodnimy kilka najprostszych własności niezmienników Wasiljewa.

% Niezmiennik Wasiljewa węzła z~pętelką to zero.

\begin{proposition}
    Każdy niezmiennik Wasiljewa rzędu zero jest funkcją stałą.
\end{proposition}

\begin{proof}
    Niech $v$ będzie niezmiennikiem rzędu zero i~znika na każdym singularnym węźle o~jednym wierzchołku.
    Relacja kłębiasta mówi wtedy, że $v(\LittleLeftCrossing) = v(\LittleRightCrossing)$, to znaczy odwrócenie dowolnego skrzyżowania nie zmienia wartości niezmiennika.
    Z lematu \ref{lem:gordian_number} wiemy jednak, że każdy węzeł można zmienić w niewęzeł odwracając niektóre skrzyżowania.
    Wynika stąd, że $v(K) = v(\LittleUnknot)$, co należało udowodnić.
\end{proof}

\begin{proposition}
    Nie istnieje niezmiennik Wasiljewa rzędu jeden.
\end{proposition}

\begin{tobedone}
    Chord diagrams. Ich produkt jest dobrze określony modulo relacja 4T.
\end{tobedone}

\begin{tobedone}
    Relacje AS, IHX, STU, FI.
\end{tobedone}

\begin{tobedone}
    Actuality tables.
\end{tobedone}

\begin{tobedone}
    Wartość niezmienników Wasiljewa zależy tylko od chord diagrams. (\cite{duzhin12}, prop. 3.4.2)
\end{tobedone}

Oznaczmy przez $\mathcal V_n$ zbiór niezmienników Wasiljewa rzędu co najwyżej $n$, o~wartościach w zbiorze liczb zespolonych $\C$.
Z definicji \ref{def:vassiliev_order} wynika, że $\mathcal V_n$ jest przestrzenią wektorową nad ciałem $\C$ oraz $\mathcal V_n \subseteq \mathcal V_{n+1}$, zatem mamy rosnącą filtrację
\begin{equation}
    \mathcal V_0 \subseteq \mathcal V_1 \subseteq \mathcal V_2 \subseteq \ldots \subseteq \mathcal V := \bigcup_{n=0}^\infty \mathcal V_n.
\end{equation}

Dokładny wymiar przestrzeni $\mathcal V_n$ jest znany tylko dla $n \le 12$.
Poniższa tabela ma dość ciekawą historię.
Wasiljew znalazł ręcznie wartości w kolumnach dla $n \le 4$ w 1990 roku.
Potem Bar-Natan napisał komputerowy program rozwiazujący pewne równania liniowe i~znalazł tak wymiary przestrzeni $\mathcal V_n$ dla $n \le 9$, miało to miejsce w roku 1993.
Wreszcie Kneissler cztery lata później znalazł dolne oraz górne ograniczenia: dolne oparte o znaczone powierzchnie, górne pochodzące od algebry Vogela (\cite{kneissler97}).
Dla $n \le 12$ ograniczenia te pokrywają się!

\renewcommand*{\arraystretch}{1.4}
\footnotesize
\begin{longtable}{lcccccccccccccc}
\hline
    $n$ & $0$ & $1$ & $2$ & $3$ & $4$ & $5$ & $6$ & $7$ & $8$ & $9$ & $10$ & $11$ & $12$ \\ \hline \endhead
    $\dim \mathcal V_n$ & $1$ & $1$ & $2$ & $3$ & $6$ & $10$ & $19$ & $33$ & $60$ & $104$ & $184$ & $316$ & $548$ \\
    $\dim \mathcal V_n / \mathcal V_{n-1}$ & $1$ & $0$ & $1$ & $1$ & $3$ & $4$ & $9$ & $14$ & $27$ & $44$ & $80$ & $132$ & $232$ \\
    \hline
\end{longtable}
\normalsize

Oznaczmy wymiar przestrzeni $\mathcal V_n / \mathcal V_{n-1}$ przez $d_n$.
Dla wyższych rzędów nie dość, że nie znamy dokładnych wartości ciągu $d_n$, to dolne i górne ograniczenia asymptotyczne są od siebie bardzo różne: górne jest niemalże silnią, dolne natomiast jest podwykładnicze.

\begin{proposition}
    $d_n < (2n-1)!!$.
\end{proposition}

\begin{proposition}[Chmutov i Duzhin, 1993]
    $d_n < (n-1)!$.
\end{proposition}

\begin{proposition}[Ng, 1995]
    $d_n < \frac 12 (n-2)!$.
\end{proposition}

\begin{proposition}[Stojmenow, 1996]
    Ciąg $d_n$ rośnie wolniej niż $n! \cdot (11/10)^n$.
\end{proposition}

\begin{proposition}[Bollobas i  Riordan, 2000]
    $d_n \lesssim n! / (2 \log 2 + O(1))^n$.
\end{proposition}

\begin{proposition}[Zagier, 2001]
    Niech $a < \frac 1 6 \pi^2$ będzie stałą.
    Wtedy
    \begin{equation}
        \dim \mathcal V_n / \mathcal V_{n-1} \lesssim \frac{n!}{a^n}.
    \end{equation}
\end{proposition}

\begin{proof}
    Zagier znalazł to ograniczenie przy użyciu szeregów Dirichleta w \cite{zagier01}.
\end{proof}

Zanim przejdziemy do ograniczeń z dołu, zdefinujmy jeszcze jedną przestrzeń, $\mathcal P_n \subseteq \mathcal V_n$.
Składa się z~tych niezmienników Wasiljewa, które są jednocześnie morfizmami, to znaczy spełniają równość $v(K_1 \shrap K_2) = v(K_1) + v(K_2)$.
Każdy niezmiennik jest wielomianową kombinacją niezmienników pierwotnch.

\begin{proposition}[Chmutov, Duzhin i Lando, 1994]
    $\dim \mathcal P_n \ge 1$.
\end{proposition}

\begin{proposition}[Melvin i Morton, Chmutov i Varchenko, 1995]
    $\dim \mathcal P_n \ge [n/2]$.
\end{proposition}

\begin{proposition}[Duzhin, 1996]
    $\dim \mathcal P_n \gtrsim \frac{1}{96} n^2$.
\end{proposition}

\begin{proposition}[Chmutov i Duzhin, 1997]
    $\dim \mathcal P_n \gtrsim n^{\log_b n}$ dla $b > 4$.
\end{proposition}

\begin{proposition}[Koncewicz, 1997]
    $\dim \mathcal P_n \gtrsim \exp (\pi \sqrt{n/3})$.
\end{proposition}

\begin{proposition}[Dasbach, 2000]
    $\dim \mathcal P_n \gtrsim \exp (c \sqrt{n})$ dla każdej stałej $c < \pi \sqrt{2/3}$.
\end{proposition}

Ograniczenie Dasbacha pozostaje najlepsze (stan na 2011 rok).

\begin{corollary}
    Niech $a < \frac 1 6 \pi^2$ będzie stałą.
    Wtedy
    \begin{equation}
        \exp \left(\frac {n}{\log_a n} \right) \lesssim \dim \mathcal V_n / \mathcal V_{n-1}.
    \end{equation}
\end{corollary}

\begin{proof}
    Dasbach w \cite{dasbach00}.
\end{proof}

Stojmenow w \cite{stoimenow_01} pokazał, że niezmienniki Wasiljewa każdego rzędu można wyznaczyć w~skończonym czasie.
Dokładniej:

\begin{proposition}
    Każdy niezmiennik Wasiljewa rzędu co najwyżej $k$ jest jednoznacznie określony przez swoje wartości na alternujących węzłach o co najwyżej $2k^2 + k$ skrzyżowaniach.
\end{proposition}

Niezmienniki Wasiljewa nie są zupełne.
Ohyama dla każdego węzła $K$ i~liczby naturalnej $n$ wskazał jawnie nieskończoną rodzinę złożonych węzłów, których niezmienniki rzędu co najwyżej $n$ nie odróżniają od $K$ (\cite{ohyama95}).
Stanford rozszerzył ten wynik: w~\cite{stanford96} udowodnił, że dla każdego splotu $L$ istnieje nieskończona rodzina pierwszych, nierozszczepialnych, alternujących splotów nieodróżnialnych takimi niezmiennikami.

Z drugiej strony, Chmutow i inni piszą w \cite{duzhin12}, że sześć niezmienników rzędu co najwyżej 4 wystarcza do odróżnienia dowolnych dwóch węzłów pierwszych do 8 skrzyżowań.

W 1993 roku Maxim Koncewicz pokazał, że dla każdego węzła można policzyć pewną całkę (teraz nazywaną całką Koncewicza), która jest uniwersalnym niezmiennikiem Wasiljewa.
Oznacza to, że z jej wartości można odtworzyć wszystkie inne niezmienniki skończonego typu.
Bar-Natan w 1995 roku znalazł wartość tej całki dla niewęzła:
\begin{equation}
    I (\LittleUnknot) = \exp \left(\sum_{n=0}^\infty b_{2n} w_{2n}\right),
\end{equation}
gdzie $b_{2n}$ to zmodyfikowane liczby Bernoulliego o funkcji tworzącej
\begin{equation}
    \sum_{n=0}^\infty b_{2n} x^{2n} = \frac 12 \log \frac {e^{x/2} - e^{-x/2}}{x/2},
\end{equation}
zaś $w_{2n}$ to ,,koła'': diagramy okręgu z doczepionymi $2n$ promieniami.
Liniową kombinację należy rozumieć jako element algebry chińskich znaków, opisanej w następnej wersji tej książki.
Następnie Marché w~2003 roku znalazł wartości całki dla węzłów torusowych (\cite{marche04}).
Wygląda na to, że nikt nie odważył się dokonać tego dla innych węzłów (stan na 2019).

\begin{conjecture}
    Uniwersalny niezmiennik Wasiljewa jest zupełny.
\end{conjecture}

To jedna z najtrudniejszych hipotez teorii węzłów -- ponieważ całka Koncewicza jest mocniejsza od każdego wielomianowego niezmiennika, jaki dotąd poznaliśmy (wielomiany Alexandera, Jonesa, HOMFLY, Kauffmana), można o~niej myśleć jako o uogólnieniu hipotezy \ref{jones_conjecture}.
