\subsection{Definicja algebraiczna -- algebra Temperleya-Lieba} % (fold)
\label{sub:jones_paper}
Jones otrzymał swój wielomian jako efekt uboczny badań nad algebrami operatorowymi: wziął ślad pewnej reprezentacji warkoczy w~algebrę, która miała ważne znaczenie w~mechanice statystycznej.
Dalszy opis pochodzi z Wikipedii.
Zaletą tego podejścia jest możliwość wyboru algebry, która reprezentuje grupę warkoczy.

\begin{definition}[algebra Temperleya-Lieba]
    Niech $R$ będzie przemiennym pierścieniem, w~którym ustalono element $\delta \in R$.
    Wtedy $R$-algebrę $TL_n(\delta)$ generowaną przez elementy $e_1, \ldots, e_{n-1}$, które związane są relacjami
    \begin{align}
        e_i^2 & = \delta e_i, \\
        e_i e_{i \pm 1} e_i & = e_i, \\
        e_i e_j & = e_j e_i
    \end{align}
    dla $|i-j| \ge 2$, nazywamy algebrą Temperleya-Lieba.
    % Algebra Temperleya-Lieba $A_n$ to wolna addytywna algebra na multiplikatywnych generatorach $e_1, \ldots, e_{n-1}$ traktowana jako $\C[\tau, \tau^{-1}]$-moduł.
    % Zmienna $\tau$ komutuje ze wszystkimi generatorami, generatory zaś spełniają relacje ($j$ jest różne od $i - 1, i, i+1$):
\end{definition}

$TL_n(\delta)$ można przedstawić przy użyciu diagramów: prostokątów, których przeciwległe boki zawierają po $n$ punktów połączonych w~pary tak, by uniknąć samoprzecięć. Mnożenie elementów algebry odpowiada sklejaniu dwóch diagramów, przy czym każdą zamkniętą pętlę zamieniamy na dodatkowy czynnik $\delta$.
To w~gruncie rzeczy są warkocze.

\begin{definition}[ślad Markowa]
    Niech $K \in TL_n(\delta)$ będzie elementem algebry Temperleya-Lieba będącym iloczynem generatorów\footnote{Czyli element $K$ utożsamia się z~pewnym warkoczem o~$n$ pasmach.} $e_1, \ldots, e_{n-1}$, którego domknięcie rozpada się na $m$ składowych spójności.
    Śladem Markowa elementu $K$ nazywamy wielkość $\operatorname{tr} K = \delta^{m-n}$.
\end{definition}

Na mocy twierdzenia Alexandera, każdy splot $L$ jest domknięciem warkocza zaplecionego na pewnej liczbie pasm.
Zdefiniujmy reprezentację $\rho \colon B_n \to TL_n$ grupy warkoczy w~algebrę Temperleya-Lieba związaną z pierścieniem $R = \Z[A, 1/A]$ oraz elementem $\delta = -A^2 - A^{-2}$ wzorem
\begin{equation}
    \rho(\sigma_i) = A \cdot e_i + \frac{1}{A} \cdot 1.
\end{equation}
Wtedy $\langle K \rangle = \delta^{n-1} \operatorname{tr} \rho (\sigma)$ jest klamrą Kauffmana.
Pozostaje sprawdzić wpływ ruchów Markowa na złożenie $\operatorname{tr} \circ \rho$, ponieważ sploty nie przedstawiają się jako domknięcia warkoczy jednoznacznie.

% Koniec podsekcji Oryginalna praca Jonesa
